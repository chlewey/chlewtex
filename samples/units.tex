\documentclass[twocolumn]{article}
\makeatletter
  %%% Modified from xspace
  \DeclareRobustCommand\uspace{\futurelet\@let@token\@uspace}
  \def\@uspace{%
    \ifx\@let@token\bgroup\else
    \ifx\@let@token\egroup\else
    \ifx\@let@token\/\else
    \ifx\@let@token\ \else
    \ifx\@let@token~\else
    \ifx\@let@token.\,\else
    \ifx\@let@token!\,\else
    \ifx\@let@token,\,\else
    \ifx\@let@token:\,\else
    \ifx\@let@token;\,\else
    \ifx\@let@token?\,\else
    \ifx\@let@token/\else
    \ifx\@let@token'\else
    \ifx\@let@token)\,\else
    \ifx\@let@token-\else
    \ifx\@let@token\@xobeysp\else
    \ifx\@let@token\space\else
    \ifx\@let@token\@sptoken\else
     \space
     \fi\fi\fi\fi\fi\fi\fi\fi\fi\fi\fi\fi\fi\fi\fi\fi\fi\fi}
  \newcommand\unit[1]{\def\@tempa{#1}\unit@}
  \newcommand\unit@[1][\@empty]{%
    \ensuremath{\,%
      \textrm{\@tempa}%
      \ifx#1\@empty\else
       ^{#1}\fi}%
    \uspace}
  \newcommand\newunit[2]{%
    \@namedef{#1}{\unit{#2}}}
  \newcommand\degree{%
    \ensuremath{^\circ}}
\makeatother
% SI official and common derivated units
\newunit{m}{m}
\newunit{kg}{kg}
\newunit{sec}{s}
\newunit{amp}{A}
\newunit{ohm}{$\Omega$}
\newunit{kelvin}{\degree K}
\newunit{celcius}{\degree C}

\begin{document}
\newunit{cm}{cm}
One inch is 2.54\cm and one square inch is 6.4516\cm[2].
\[(2.54\cm)^2=6.4516\cm[2]\]

\newunit{farenheit}{\degree F}
Water freezes at 32\farenheit and boils at 212\farenheit.
\[100\celcius=212\farenheit-32\farenheit\]

International unit system defines the Ampere (\amp) and the
second (\sec), however it does ot define the Culomb (\unit{C})
prefering the compund Ampere-second (\amp\sec).
(Although it should look as \unit{As}.)
\[1\unit{C} = 1\amp\sec = 1\unit{As}.\]

If a tension of 12\unit{V} on a resistor of 15\ohm will produce
a current of 0.8\amp.
\[\frac{12\unit{V}}{15\ohm}=0.8\amp\]

The gravitational field on Eath's surface is 1\unit{g}, and this
is roughly equal to 9.8\unit{N}/\kg or 32\unit{ft}/\sec[2].
\[1\unit{g} \simeq 9.8\frac{\unit{N}}{\kg}
 = 9.8\m\sec[-2] \simeq 32\frac{\unit{ft}}{\sec[2]}\]
 
One Ampere-second, or $1\mathrel{\mathrm A}\mathrel{\mathrm s}.$

\end{document}
