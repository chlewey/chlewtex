\documentclass[spanish,utf8]{oaprop}
\usepackage{chlewid}
\title{Propuesta de Oruga Amarilla}
\date{Diciembre de 2013}
\begin{document}
\frontmatter
\maketitle
\tableofcontents
\begin{abstract}
Propuesta para la Oruga Amarilla y los servicios que la Oruga Amarilla ofrece
en los campos de monitoreo, control y automatización remoto;
gestión de incidentes; compra, venta y alquiler de maquinaria;
y demás servicios relacionados.
\end{abstract}
\mainmatter
\part{Descripción de la Oruga Amarilla}
\chapter{Los servicios}
La Oruga Amarilla ofrece servicios de monitoreo, control y automatización remoto
incluyendo el monitoreo y control de invernaderos y maquinaria, automatización
de invernaderos y monitoreo de estaciones climáticas.

También incluye servicios de gestión de incidentes y casos,
y servicios de compra, venta y alquiler de maquinarias y repuestos.

\section{Monitoreo de estaciones remotas}
Por una estación remota se entiende un bien inmueble o una maquinaria móvil
la cual quiere ser monitoreada en sus parámetros de operación y parámetros
ambientales internos y externos.

Las estaciones remotas incluyen:
\begin{enumerate}
\item invernaderos,
\item maquinaria amarilla (bulldozzers, retroexcavadoras, etc.),
\item casas (domótica), casas de campo y piscinas,
\item plantas de procesamiento,
\item etc.
\end{enumerate}

El monitoreo incluye reportes en tiempo real de los parámetros configurables,
envío de alarmas y resúmenes automáticos de operación diarios, semanales o mensuales.

Dentro de los servicios \emph{prémium} se incluyen resúmenes analizados sobre
la operación y la atención personal de alarmas.

\section{Control y automatización de estaciones remotas}
El control y automatización de estaciones remotas incluye la configuración de
parámetros y acciones tanto remoto (control manual remoto) como automático.

Las acciones automáticas o manuales remotas son igualmente monitoreadas.

\section{Estaciones climáticas}
Las estaciones climáticas son un caso especial de estaciones remotas destinados
a monitorear variables ambientales públicas, sin servicios \emph{prémium}, y
sin opciones de control y automatización; aunque bien, los datos obtenidos por una
estación climática pueden ser usados para la automatización de una estación remota.

\section{Compra y venta de maquinaria}
\section{Alquiler de maquinaria}
\section{Compra y venta de repuestos}
\section{Gestión de incidencias}

\chapter{La página}
La página Oruga Amarilla \emph{http://www.orugaamarilla.com/} es el punto
de entrada para los servicios de la Oruga Amarilla.
\section{Monitoreo de estaciones remotas}
\section{Estaciones climáticas}
\section{Compra y venta de maquinaria}
\section{Alquiler de maquinaria}
\section{Compra y venta de repuestos}
\section{Gestión de incidencias}

\chapter{Aplicaciones móviles}

\section{Reporte de incidencias}
La aplicación de reporte de incidencias es una aplicación que permite reportar de forma georreferenciada incidencias de plagas y enfermedades
en cultivos.

\subsection{Versión libre}
La versión libre de reporte de incidencias permite crear grupos semiprivados para el reporte de incidencias.
Los detalles de un grupo semiprivado serán conocidos sólo por los miembros del grupo y los administradores del sistema pero no serán reportados
a posibles competidores.  Los incidentes reportados podrán ser públicos dentro de un margen de error inducido.

\subsection{Versión \emph{prémium}}
La versión \emph{prémium} permite crear grupos privados donde toda la iformación es confinada al grupo, incluyendo detalles y reportes.
Adicionalmente la versión \emph{prémium} permite la creación de pantallas personalizadas para un reporte más rápido de incidentes más comúnes.

\section{Monitoreo remoto}
La aplicación de monitoreo remoto es una aplicación para Android y iOS
la cual permite ver la iformación de monitoreo remoto a la cual tiene
acceso un usuario web de la Oruga Amarilla.

La aplicación se ofrece bajo un esquema \emph{freemium} por medio de
la cual existe una versión de libre descarga por medio de iTunes y Google Play (anteriormente Android Market)
que puede ser utilizada para evaluación,
y una versión \emph{prémium} la cuál es paga y ofrece servicios adicionales.

\subsection{Versión libre}
La versión libre permite ver la informacón de monitoreo, incluyendo los resúmenes automáticos.

No tiene limitaciones en cuanto al número de días u horas en uso, aunque después de 30 días enviará recordatorios para el uso de la versión \emph{premium}.

\subsection{Versión \emph{prémium}}
Además de los servicios de la versión libre (y del no envío de recordatorios para usar esta versión), la versión \emph{prémium} permite
el acceso a los resúmenes personalizados, crear resúmenes automáticos filtrados, bajar (copiar) la información
o enviarla por correo electrónico u otros servicios de mensajería configurados en el dispositivo móvil y
sincronizar los cambios efectuados en el dispositivo móvil con la sesión web del usuario registrado.

\chapter{Hardware}
\section{Automatización de invernaderos}
\section{Estaciones climáticas}
\section{Automatización industrial}
\section{Geolocalización de maquinaria}
\part{Propuesta de trabajo}
\backmatter
\end{document}
