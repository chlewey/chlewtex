\documentclass[spanish,utf8,letterpaper,oneside,12pt]{oaprop}
\usepackage{chlewid}
\title{Propuesta de Oruga Amarilla}
\date{Diciembre de 2013}
\begin{document}
\frontmatter
\maketitle
\begin{abstract}
Propuesta para Oruga Amarilla y los servicios que Oruga Amarilla ofrece
en los campos de monitoreo, control y automatización remoto;
gestión de incidentes; compra, venta y alquiler de maquinaria;
y demás servicios relacionados.
\end{abstract}
\tableofcontents
\mainmatter
\part{Descripción de Oruga Amarilla}
\chapter{¿Qué es Oruga Amarilla?}
Oruga Amarilla es un servicio y una marca soportado por la empresa Invermeq y la marca Interlecto.
Oruga Amarilla ofrece una serie de servicios orientados a la gestión de maquinaria y estaciones
de control y monitoreo.

\section{Invermeq}
Invermeq Ortiz-Forero E.\,U. es la empresa y razón social que da soporte y apoyo a Oruga Amarilla.

Invermeq es una empresa unipersonal registrada en la Cámara de Comercio de Bogotá.

\section{Interlecto}
Interlecto es una marca no registrada de Carlos Thompson, para el desarrollo de los servicios de
desarrollo y presencia en web y en redes sociales.

\chapter{Los servicios}
Oruga Amarilla ofrece servicios de monitoreo, control y automatización remoto
incluyendo el monitoreo y control de invernaderos y maquinaria, automatización
de invernaderos y monitoreo de estaciones climáticas.

También incluye servicios de gestión de incidentes y casos,
y servicios de compra, venta y alquiler de maquinarias y repuestos.

\section{Monitoreo de estaciones remotas}
Por una estación remota se entiende un bien inmueble o una maquinaria móvil
la cual quiere ser monitoreada en sus parámetros de operación y parámetros
ambientales internos y externos.

Las estaciones remotas incluyen:
\begin{enumerate}
\item invernaderos,
\item maquinaria amarilla (bulldozzers, retroexcavadoras, etc.),
\item casas (domótica), casas de campo y piscinas,
\item plantas de procesamiento,
\item etc.
\end{enumerate}

El monitoreo incluye reportes en tiempo real de los parámetros configurables,
envío de alarmas y resúmenes automáticos de operación diarios, semanales o mensuales.

Dentro de los servicios \emph{prémium} se incluyen resúmenes analizados sobre
la operación y la atención personal de alarmas.

\section{Control y automatización de estaciones remotas}
El control y automatización de estaciones remotas incluye la configuración de
parámetros y acciones tanto remoto (control manual remoto) como automático.

Las acciones automáticas o manuales remotas son igualmente monitoreadas.

\section{Estaciones climáticas}
Las estaciones climáticas son un caso especial de estaciones remotas destinados
a monitorear variables ambientales públicas, sin servicios \emph{prémium}, y
sin opciones de control y automatización; aunque bien, los datos obtenidos por una
estación climática pueden ser usados para la automatización de una estación remota.

\section{Compra y venta de maquinaria}
\section{Alquiler de maquinaria}
\section{Compra y venta de repuestos}
\section{Gestión de incidencias}

\chapter{La página web}
La página Oruga Amarilla \emph{http://www.orugaamarilla.com/} es el punto
de entrada para los servicios de Oruga Amarilla.

La página web ofrece servicios públicos y personalizados.  Entre los servicios
públicos se encuentran blogs, foros públicos, visión de estaciones climáticas,
revisión de maquinaria para compra, venta y alquiler.

Los servicios personalizados ofrecen también información privada de acuerdo
al perfil y preferencias del usuario.

\section{Monitoreo de estaciones remotas}
El monitoreo de estaciones remotas es un servicio privado y personalizado.

\section{Estaciones climáticas}
Los datos generales de las estaciones climáticas son públicos pero
los datos detallados son privados.

\section{Compra y venta de maquinaria}
El directorio general de compra y venta de maquinaria es información pública.

Los datos y formularios de contacto exigen que los usuarios estén registrados.
Igualmente los usuarios registrados pueden establecer un perfil de preferencias
para que se destaquen ofertas de interés.

\section{Alquiler de maquinaria}
El directorio general de alquiler de maquinaria es información pública.

Los datos y formularios de contacto exigen que los usuarios estén registrados.
Igualmente los usuarios registrados pueden establecer un perfil de preferencias
para que se destaquen ofertas de interés.

\section{Compra y venta de repuestos}
El directorio general de compra y venta de repuestos es información pública.

Los datos y formularios de contacto exigen que los usuarios estén registrados.
Igualmente los usuarios registrados pueden establecer un perfil de preferencias
para que se destaquen ofertas de interés.

\section{Gestión de incidencias}
La gestión de incidencias es un producto personalizado que permite
a los usuarios de un cliente empresarial abrir casos, gestionar recursos
y hacer seguimiento de tiquetes.

\chapter{Aplicaciones móviles}

\section{Reporte de incidencias}
La aplicación de reporte de incidencias es una aplicación para Android y iOS
la cual permite reportar de forma georreferenciada incidencias de plagas y enfermedades
en cultivos.

La aplicación se ofrece bajo un esquema \emph{freemium} por medio de
la cual existe una versión de libre descarga por medio de iTunes y Google Play (anteriormente Android Market)
que puede ser utilizada para evaluación,
y una versión \emph{prémium} la cuál es paga y ofrece servicios adicionales.

Si la demanda amerita pueden desarrollarse versiones \emph{prémium} o libres
para otros sistemas operativos tales como Windows Phone, FirefoxOS o Blackberry.

\subsection{Versión libre}
La versión libre de reporte de incidencias permite crear grupos semiprivados para el reporte de incidencias.
Los detalles de un grupo semiprivado serán conocidos sólo por los miembros del grupo y los administradores del sistema pero no serán reportados
a posibles competidores.  Los incidentes reportados podrán ser públicos dentro de un margen de error inducido.

\subsection{Versión \emph{prémium}}
La versión \emph{prémium} permite crear grupos privados donde toda la iformación es confinada al grupo, incluyendo detalles y reportes.
Adicionalmente la versión \emph{prémium} permite la creación de pantallas personalizadas para un reporte más rápido de incidentes más comúnes.

La versión \emph{prémium} ofrece también soporte $5\times 8$
(o $7\times24$ si el cliente ya tiene este SLA en sus productos web)
y un ciclo corto de corrección de bugs y solicitud de nuevas funcionalidades.

\section{Monitoreo remoto}
La aplicación de monitoreo remoto es una aplicación para Android y iOS
la cual permite ver la iformación de monitoreo remoto a la cual tiene
acceso un usuario web de Oruga Amarilla.

La aplicación se ofrece bajo un esquema \emph{freemium} por medio de
la cual existe una versión de libre descarga por medio de iTunes y Google Play (anteriormente Android Market)
que puede ser utilizada para evaluación,
y una versión \emph{prémium} la cuál es paga y ofrece servicios adicionales.

Si la demanda amerita pueden desarrollarse versiones \emph{prémium} o libres
para otros sistemas operativos tales como Windows Phone, FirefoxOS o Blackberry.

\subsection{Versión libre}
La versión libre permite ver la informacón de monitoreo, incluyendo los resúmenes automáticos.

No tiene limitaciones en cuanto al número de días u horas en uso, aunque después de 30 días enviará recordatorios para el uso de la versión \emph{premium}.

\subsection{Versión \emph{prémium}}
Además de los servicios de la versión libre (y del no envío de recordatorios para usar esta versión), la versión \emph{prémium} permite
el acceso a los resúmenes personalizados, crear resúmenes automáticos filtrados, bajar (copiar) la información
o enviarla por correo electrónico u otros servicios de mensajería configurados en el dispositivo móvil y
sincronizar los cambios efectuados en el dispositivo móvil con la sesión web del usuario registrado.

Adicionalmente la versión \emph{prémium} ofrece soporte $5\times 8$
(o $7\times24$ si el cliente ya tiene este SLA en sus productos web)
y un ciclo corto de corrección de bugs y solicitud de nuevas funcionalidades.

\chapter{Hardware}
\section{Automatización de invernaderos}
El equipo de automatización de invernaderos incluye:
\begin{enumerate}
\item el módulo de control y comunicaciones,
\item sensores,
\item accionadores,
\item interfaz local (opcional),
\item módulo externo de comunicación (opcional)
\end{enumerate}

\subsection{módulo de control}
El módulo de control es una caja a la que se conectan los sensores, accionadores y la interfaz local
y que graba la información de los sensores, toma acciones automáticas programadas, comanda acciones
automáticas o reportadas manualmente por medio de la interfaz local o una interfaz remota y envía
los datos al servidor de Oruga Amarilla.

El módulo de control permite la conexión vía Ethernet, WiFi o por medio de un módem 3.5G o 4G (externo).

\subsection{interfaz local}
La interfaz local consiste o bien en un tablero de control, una pantalla con mouse y teclado, una pantalla táctil
o una combinación de estos.

La interfaz local permite informar del estado de los sensores y del sistema y permite que un usuario local
pueda comandar acciones, las cuales son monitoreadas.

\subsection{módulo externo de comunicaciones}
El módulo externo de comunicaciones es un repetidor WiFi al cual pueden conectarse varios
módulos de control cercanos y permite la comunicación a Internet vía una conexión cableada
o inalámbrica.

\subsection{sensores}

\subsection{accionadores}

\section{Estaciones climáticas}
\section{Automatización industrial}
\section{Geolocalización de maquinaria}

\part{Propuesta de trabajo}
\chapter[Interfaz web]{Interfaz web: marco general}
\section{Servidores}
\subsection{Estado actual}
Actualmente la página se encuentra alojada en un servidor web en California bajo un esquema de servidor compartido.
Este servidor compartido aplica tanto al servidor web (Apache) como al servidor de bases de datos (MySQL).

El servidor web es un servidor Linux (CentOS) con sistema operativo Apache y corriendo PHP como programación de \emph{backend}
en una configuración conocida como LAMP (Linux, Apache, MySQL, PHP).

\subsection{Estado propuesto}
Con prioridad para el servidor de base de datos, es importante tener un servidor dedicado.

Idealmente sería un servidor dedicado en Colombia (conectado al NAT Colombia), bien alquilado o bien propio, aunque
en el caso de un servidor propio sería importante tener redundancia en los proveedores de servicio de Internet y de potencia eléctrica.
No se descartaría un servidor dedicado en otro país, sobre todo en Estados Unidos, de acuerdo a la mejor relación
costo/SLA/ancho de banda efectivo con Colombia.

Como sistema operativo se sugiere seguir con Linux (por ahora sin preferencia), y como servidor web se prefiere Apache.

No hay preferencia por ahora sobre el motor de base de datos.

Se está explorando la posibilidad de migrar la programación en PHP a otro lenguaje como C (probablemente Qt) o Python.
No se descarta el uso de un framework como Ruby-on-rails o Django.

Por ahora el desarrollo primario se seguirá haciendo en PHP y MySQL y se espera, por lo pronto, que el nuevo servidor tenga una configuración LAMP.

\section{Gestor de contenidos}

\subsection{Estado actual}
En la actualidad el gestor de contenidos de Oruga Amarilla es un gestor propio.  Realmente son dos gestores de contenidos, uno para
\emph{www.orugaamarilla.com} (www) y otro para \emph{test.orugaamarilla.com} (test).

Sobre el gestor de test corre la aplicación de monitoreo.  Sobre el gestor de www corre la aplicación de compra y venta.

\subsection{Estado propuesto}
Se propone que el gestor de contenidos sea un gestor Interlecto, sobre el cual se desarrollen las aplicaciones
específicas como módulos.

\section{Diseño}

\chapter[Monitoreo y control por web]{Interfaz web: monitoreo, automatización y control}
\chapter[Compra y venta por web]{Interfaz web: compra, venta y alquiler de maquinaria y repuestos}
\chapter[Gestión de incidencias por web]{Interfaz web: gestión de incidencias y tiquetes}
\chapter[Reporte de incidentes (móvil)]{Aplicación móvil de reporte de incidencias}
\chapter[Monitoreo y control (móvil)]{Aplicación móvil de monitoreo, automatización y control}
\chapter[Módulo de control - inmuebles]{Hardware: módulo de control y comunicaciones para estaciones inmuebles}
\chapter[Módulo de control - móvil]{Hardware: módulo de control y comunicaciones para estaciones móviles}
\chapter[Hub de comunicaciones]{Hardware: módulo externo de comunicaciones}
\chapter[Tableros]{Hardware: interfaz local tipo tablero}
\chapter[Interfaz local táctil]{Hardware: interfaz local tipo pantalla táctil}
\chapter[Interfaz local de teclado y pantalla]{Hardware: interfaz local tipo pantalla, mouse y teclado}

\backmatter
\end{document}
