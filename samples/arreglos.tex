\documentclass[spanish,utf8,letterpaper,oneside,12pt]{oaprop}
\title{Propuesta de \oamarilla}
\author{Carlos Thompson}
\date{Diciembre de 2013}
\begin{document}
\frontmatter
\maketitle
\begin{abstract}
Plan de trabajo para lograr mejoras substanciales en
la página \emph{http://test.orugaamarilla.com} de \oamarilla,
la cual, en la actualidad, se cuelga frente a algunas
consultas.

Este plan de trabajo incluye optimizar la base de datos,
crear sistemas de control de procesos en la base de datos
y sugerir algunos pasos para una solución más permanente.
\end{abstract}
\tableofcontents
\mainmatter

\chapter{El problema actual}
\section{Observación}
Cuando se intentan cargar páginas que muestran los resultados
de algunos sitios, principalmente el sitio \textsf{San Pedro} de \textsf{Elite Flowers}
la página tarda un tiempo largo sin presentar ningún tipo de
actividad tras el cual aparece un error
\textsf{500 Internal Server Error}.

Consultas subsiguientes, incluyendo aquellas que no pidan información
de \textsf{San Pedro}, fallan igualmente.

\section{Diagnóstico}
Las consultas por últimas actividades de algunos de los sitios,
y en particular de \textsf{San Pedro}, ejecutan lentamente en el servidor
de bases de datos MySQL \emph{mysql.orugaamarilla.com}\,.

Estas consultas tardan significativamente más que el tiempo configurado
por el servidor web para presentar resultados.  Como el programa no termina
de ejecutar en el tiempo estipulado, el servidor emite un error
\textsf{500 Internal Server Error} y detiene la ejecución del script
pero la consulta en el servidor de base de datos sigue ejecutando.

Las subsiguientes consultas, incluyendo aquellas que son generadas por el usuario
cuando ejecuta F5 o cambia manualmente el URL de la aplicación frente a la frustración
por no recibir información.
Este comportamiento justificado del usuario agrava la situación pues encola nuevas
consultas.

Este bloqueo en el servidor de bases de datos podría estar causando también pérdidas
en la información que la base de datos recibe automáticamente de los sitios.
Esto último no ha sido verificado, pero es una conclusión razonable.

\section{Posibles soluciones}
Entre las soluciones planteadas se incluyen:
\begin{enumerate}
\item Optimizar la base de datos con el objeto de reducir el tamaño de las tablas
  y, en consecuencia, agilizar las consultas individuales.
  
  (ya se han adelantado pasos en este sentido pero no se han realizado pruebas.)
 
\item Agregar sistemas de limpieza de consultas abandonadas.
  Esto es llevar un control de las consultas que se inician pero que no terminan
  porque el script fue suspendido bien por \emph{timeout} del servidor,
  bien por acción del usuario (detener carga de la página, relanzar la consulta,
  cambiar la consulta, etc.)
  
  (Se ha probado con la inclusión de conexiones persistentes, el cual incluye
  automáticamente un módulo de limpieza, pero no se han observado resultados positivos,
  igualmente esto sólo serviría para consultas relanzadas o cambio de consulta y no
  para otro tipo de abandonos.)
  
\item Agregar rutinas en el diseño de la aplicación web que permita la carga
  de la página antes de terminar la consulta y que el resultado de la consulta
  se agregue después, independientemente del tiempo que tarde la consulta.
  
  Esto tiene un efecto más psicológico pues el usuario no \emph{ve} una página
  bloqueada sino que es informado que la consulta está tardando y evita la
  aparición de errores \textsf{500 Internal Server Error}, disminuyendo por ende
  la sobrecarga del servidor por parte de usuarios frustrados.
  
  (La técnica es conocida a través de AJAX, pero aun no se ha buscado cómo
  incluirla en la página de \oamarilla.)
 
\item Utilizar un servidor dedicado.

  Varios de los problemas de desempeño son causados por el hecho de que los servidores
  actuales de \oamarilla\ son un servidor web compartido y un servidor MySQL compartido.
  
  Además del desempeño, no se tiene acceso de súper-usuario al servidor lo que impide
  hacer algunas maniobras de corrección rápidas.
\end{enumerate}

\chapter{Optimización de la base de datos}
Aquí se tratarán las estrategias de optimización de la base de datos de Oruga Amarilla.

\section{Situación actual}
Actualmente la aplicación de Oruga Amarilla utiliza dos tablas en la base de datos
\texttt{orugadata} en el servidor \emph{mysql.orugaamarilla.com}.  Estas tablas son:
\texttt{dl\_status\_i} y \texttt{dl\_status\_p}.  Actualmente estas tablas poseen
respectivamente más de 3,4 y más de 22 millones de registros, ocupando respectivamente
más de 400\,MB y 1,5\,GB.

Toda consulta sobre estados debe hacer un cruce entre estas dos tablas.  Desplegar información
además requiere consultar otras tablas en las que se almacena la información \emph{no cambiante}.

El tiempo de proceso que se requiere para hacer el cruce de estas tablas, debido al tamaño, es
grande.  Ya hay ciertas optimizaciones al filtrar los resultados pero esto requiere un ordenamiento
el cual también tarda.

Sólo \textsf{San Pedro} compone () registros en \texttt{dl\_status\_i} y un número proporcional
de datos en \texttt{dl\_status\_p}.

\section{Sugerencia de optimización}
En lugar de dos tablas que incluyan todas las estaciones en una consulta cruzada
se sugiere crear cuatro o cinco tablas por cada estación, evitando también las consultas
cruzadas.

Cada estación tendrá una tabla de datos activos, en los cuales se registren en una sola
tabla todas las actualizaciones de los últimos 60 días. (Podrían ser dos tablas, una para
estado de instrumentos y otra para alarmas.)

Adicionalmente habría tres tablas de resultados históricos: una tabla con todos los datos
anteriores a 60 días y dos tablas con resultados promediados.  De estas tablas se eliminarían
los datos que se sabe que no son relevantes.

\subsubsection{Ventajas}
\begin{itemize}
\item Las tablas activas (últimos 60 días y promedios) serán más pequeñas y requerirán menos
  recursos y menos tiempo en ser consultadas.
\item Ninguna de estas tablas requerirá cruce de tablas para hacer las consultas pertinentes
  lo que reduce el procesamiento en el servidor con la consiguiente ganancia en tiempo.
\item La mayor parte de los datos históricos que se requieren se requieren promediados.
  Las tablas de resultados promediados agilizan estas consultas en la mayor parte de los casos.
\item Se conservan los datos históricos relevantes.
\item No se guarda información poco relevante.
\end{itemize}

\subsubsection{Desventajas}
\begin{itemize}
\item Aumenta la complejidad de las relaciones en la base de datos.
\item Cada nuevo sitio implica la creación de nuevas tablas.
\item Se duplica información.
\item Cierta información se perdería.
\item Migrar requerirá de consultas largas.
\item Requiere cambios grandes en la aplicación web para responder al nuevo diseño.
\item El rediseño planteado podría no ser el óptimo
\end{itemize}

Ninguna de estas desventajas es crítica.
Las desventajas más críticas afectan sobre todo la fase de pruebas y migración las cuales,
frente a la situación actual, no representan una desventaja adicional.

\subsubsection{Conclusión}
Este u otro rediseño de la base de datos \textbf{debe} hacerse.

Aun si el rediseño aquí propuesto no es el óptimo debe ser una mejora significativa
frente a la situación actual.

\chapter{Optimización en el diseño de la aplicación Web}

\chapter{Sistemas de control de procesos}

\chapter{Sugerencias}

\backmatter
\end{document}
