\documentclass[math,english,utf8]{chlart}
\usepackage{chlewex}
\begin{document}
\section{The \texttt{funcdef} macro}
%Let $\funcdef f\RR[variable=x,definition=x^2]$ be inline in the text.
%While the very same code $$\funcdef f\RR[variable=x,definition=x^2]$$
%will render diferently in display math.

%You can also force display style $\funcdef[display]f\RR[variable=x,definition=x^2]$
%when inline,
%or inline style $$\funcdef[inline]f\RR[variable=x,definition=x^2]$$ in display mode.

The macro \verb|\funcdef| takes to compulsory arguments: the name of the function
and the space.  For example \verb|\funcdef{f}{\mathbb N}| will render as $\funcdef{f}{\mathbb N}$,
meaning that $f$ is a function in the space $\mathbb N$.

The macro has two places for options, before and after the mandatory
arguments: \verb|\funcdef[flags]{function}{space}[options]|.
Currently there is no diference in which of these places the options
and flags are placed, however a good practice is:
\begin{enumerate}
\item flags indicating the overall display go before the mandatory arguments.
\item flags modifying the declaration line, go before the mandatory arguments.
\item when no form line is specified, options affecting the declaration line go before the mandatory arguments.
\item codomine, if different than the space, and anything affecting the form line, go after the mandatory arguments.
\end{enumerate}

\subsection{Flags modifying the overall display}
There are four flags modifying the overal display of the function:
\texttt{common}, \texttt{inline}, \texttt{tabbed}, \texttt{display}
and \texttt{novars}.

Flag \texttt{common} is the default in display math mode, and can be used
to force display math rendering when used in inline math mode.
\typeout{first example}
\begin{example}
You might want to display a two
lined fuction declaration
such as \(\funcdef[common]{f}{A}%
[var=a,def=\sum_{i=1}^{a}r_i]\) inline
in the text.
\end{example}

Flag \texttt{inline} is the default in inline math mode, and can be used
to force one line rendering when used in display math mode.
\begin{example}
You might want to display a two
lined fuction declaration
such as \[\funcdef[inline]{f}{A}%
[var=a,def=\sum_{i=1}^{a}r_i]\] as
one line in display math.
\end{example}

Flag \texttt{tabbed} will render a tabulated version of the macro to
be used inside tabulated environment such as \texttt{cases}, \texttt{array}
or \texttt{align}.
\begin{example}
You might want to align in an array
several function declarations
\[\begin{array}{r@{}lr@{}l}
  \funcdef[tabbed]{f}{A}%
    [var=a,def=\sum_{i=1}^{a}r_i] \\
  \funcdef[tabbed]{g}{A}%
    [var=a,def=\sum_{i=a}^{2a}r_i]
\end{array}.\]
\end{example}

Flag \texttt{display} is used to force display math rendering inside the
definition of the function.
\begin{example}
You might want to display a two
lined fuction declaration
such as \[\funcdef[display]{f}{A}%
[var=a,def=\sum_{i=1}^{a}r_i]\] as one
line in display math, forcing
display style in the definition.

Contrast with \[\funcdef{f}{A}%
[var=a,def=\sum_{i=1}^{a}r_i].\]
\end{example}

Flag \texttt{novars} supress all lines but the first line of the declaration.
This is the default if no variable names are provided.
\begin{example}
You might want to display a simple
declaration (with no second line)
such as \[\funcdef[novars]{f}{A}%
[var=a,def=\sum_{i=1}^{a}r_i]\] even
when second line elements are
provided.
\end{example}

\subsection{Flags indicating the kind of function}
The following flags tell \verb|\funcdef| some information on the type of
function.

Flag \texttt{operator} is used when the function name is a word that should
go in roman typesetting and that normally the variable will not be set
inside parenthes.
\begin{example}
\[\funcdef[operator]{func}{A}%
[var=a,def=\sum_{i=1}^{a}r_i]\]
\end{example}

Flag \texttt{binop} indicates that the function name is a binary operator.
It modifies the default domain from \textit{space} to $\textit{space}\times\textit{space}$.
\begin{example}
\[\funcdef[binop]{o}{A}%
  [variables={a,b},%
  def=\sum_{i=a}^{b}r_i]\]
\end{example}

Flag \texttt{texop} indicate that the first mandatory argument is a \LaTeX\ macro
that accepts one or two arguments.
\begin{example}
\newcommand\myop[1]{\|#1\|}
\[\funcdef[texop]{\myop}{A}%
  [var=a,def=\sum_{i=a}^{b}r_i]\]
\end{example}

\begin{example}
\newcommand\mycom[2]{\binom{#1!}{#2!}}
\[\funcdef[texop,binop]{\mycom}{A}%
  [vara=a,varb=b]\]
\end{example}

Option \texttt{scalar=} is used when the decalred operator or function accepts a
first operand from a second space.  The argument is this second space.
\begin{example}
\[\funcdef[scalar=K]{\cdot}{A}%
  [vara=\kappa,varb=a]\]
\end{example}

Option \texttt{coef=} is used when the decalred operator or function accepts a
secnd operand from a second space.
\begin{example}
\[\funcdef[coef=I]{\uparrow}{A}%
  [vara=a,varb=i]\]
\end{example}

Option \texttt{domain=} is used when the domain is different from the declared
space (which will be set as the codomain as default)
\begin{example}
\[\funcdef[domain=B]{f}{A}%
  [var=a]\]
\end{example}

Option \texttt{codomain=} is used when the codomain is different from the declared
space (which will be set as the domain as default)
\begin{example}
Normal function
\[\funcdef[codomain=B]{f}{A}%
  [var=a]\]
Or binary operator
\[\funcdef[binop,codomain=K]{\cdot}{A}%
  [vara=a,varb=b]\]
\end{example}

Of course, by providing both domain and codomain, the function does not need
an explicitely declared space.  Given that the space is a mandatory argument
it can be left explicetly blank or set to anything:
\begin{example}
Normal function
\[\funcdef[domain=A,codomain=B]{f}{}%
  [var=a]
  \quad
  \funcdef[domain=A,codomain=B]{f}*%
  [var=a]
  \quad
  \funcdef[domain=A,codomain=B]{f}%
  {anything}[vara=a]\]
\end{example}

\subsection{Options affecting the variables}

Option \texttt{variable=} (or \texttt{var=} for short) declares a single
variable.
\begin{example}
\[
 \funcdef{f}{A}[var=a]
\]
\end{example}

Option \texttt{variables=} declares a list of variables.
\begin{example}
\[ \funcdef[binop]{f}{A}%
     [variables={a,b}] \]
\end{example}

Options \texttt{vara=} and \texttt{varb=} declare two variables
when using by infix operators or macro operators.
\begin{example}
\[ \funcdef[binop]{f}{A}%
     [vara=a,varb=b] \]
\newcommand\angbin[2]{%
  \langle{#1},{#2}\rangle}
\[ \funcdef[texop,binop]{\angbin}{A}%
     [vara=a,varb=b] \]
\end{example}

As can be noted defining the variables separately as \texttt{vara=a,varb=b}
in contrast with listed \texttt{variables={a,b}} will assume that the
function is an infixed binary operator rather than a two-variable function.

The option \texttt{notation=} (also aliased as \texttt{as=}) will provide
an alternative notation for the defined function, instead of the default.
\begin{example}
\[ \funcdef{\exp}{\mathbb R}%
     [var=x,as=e^x] \]
\end{example}

The option \texttt{alt=} will provide an alternative notation for the
defined function allowing the default notation as well:
\begin{example}
\[ \funcdef[binop]{C}{\mathbb N}%
    [vara=n,varb=k,alt=\binom{n}{k}] \]
\end{example}

Option \texttt{definition=} (or \texttt{def=} for short) provides
a definition of the function.
\begin{example}
\[
  \funcdef[display]{\exp}{\mathbb R}%
   [var=x,as=e^x,
    def={\lim_{n\to\infty}%
      \left(1+\frac{x}{n}\right)^n}]
\]
\end{example}

Option \texttt{linedef=} provides an extra line (or an extra space in
inline mode) for further definition.
\begin{example}
\[ \funcdef[binop]{+}%
     {\mathbb R^{\mathbb N}}[
     variables={[a_n],[b_n]},
     as={[(a+b)_n]},
     linedef={\forall n\in\mathbb N:
       (a+b)_n = a_n+b_n}]
\]
\end{example}
Note: when using \texttt{linedef=}, a display flag such as \texttt{inline},
  \texttt{common} or \texttt{display} must be provided, as this
  conflicts with the automatic detection of inline or display math.
\section{Aditional macros}
There are two other macros implemented:

Macro \verb/\fdefPH/ is defined as one space: `\verb/\ /' but can be redefined.
It defines which symbol will be used as place holder in functions defined as
\texttt{texop}
\begin{example}
\renewcommand\fdefPH{\Box}
\newcommand\power[2]{{#1}^{#2}}
\[ \funcdef[texop,binop]\power
     {\mathbb N}[vara=n,varb=k]
\]
\end{example}

Macro \verb/\fdefIS/ is defined as one quad space: `\verb/\quad/' but can be redefined.
It defines the separation in inline mode between the formulation and mapping parts of the
function declaration.
\begin{example}
Some function
\(\funcdef{f}{A}[var=a]\)
declared inline.

\renewcommand\fdefIS{\ }
Some function
\(\funcdef{f}{A}[var=a]\)
declared inline.
\end{example}

\section{Implementation}
\verbatiminput{chlewfunc.sty}

\end{document}
