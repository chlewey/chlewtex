\documentclass{book}
\usepackage{blog}
\author{Carlos Thompson}
\title{The Chlewey Blog}
\begin{document}
\frontmatter
\maketitle
\tableofcontents
\mainmatter

\chapter{Paras y guerrilleros.}
\begin{metadata}
	Published by \anchor[chlewey]{chlewey} on \anchor[http://ewey.co/B179]{Mon, 30 Aug 1999 05:00:00 +0000}\\
	\categories{usenet, guerrilla, opinion, paramilitarismo}\\
	Shorthand: \anchor[http://blog.chlewey.net/1999/08/paras-y-guerrilleros/]{paras-y-guerrilleros}
\end{metadata}

\begin{address}
\anchor[http://groups.google.com/group/soc.culture.colombia/msg/6e89e49cde3ac955]{news:7qd05e\$vke\$1@nnrp1.deja.com>}
\end{address}
Una tendencia que nos envuelve en nuestro país es el de rotular personas.  Cuando rotulamos una persona nos olvidamos
 de la persona y del individuo para ver el rótulo y juzgarlo.

Mackartie (no sé si así se escriba), rotulaba a la gente de comunista y les destruía la vida.  Pero a esos puntos
 estamos llegando aquí en Colombia, con un agravante y son la multiplicidad de agentes armados en esta guerra.

Aquí una gente que se hace llamar OCVGC monta una página sobre datos críticos de la guerrilla.  Alguien la lee y los
 tacha de paras.

Un conocido colaborador de este grupo suele colocar artículos de prensa donde militares y paras salen mal librados y
 más de medio grupo tacha a Coreya de guerrillero.

Hay una realidad y es que nos estamos matando entre hermanos porque vivimos en un mundo de intolerancia armada.  Esa
 intolerancia parte de rotular a los demás cuando no piensan como nosotros.

Ahora nos acercamos a un paro cívico.  Hay una escalada de violencia reflejada en petardos a corporaciones, sedes
 sindicales, patrullas de policías, etc.  Y aunque muchas veces la rabia me lleva a pensar que debería empuñar las
 armas para defender lo mío y mis ideas soy conciente que no lo voy a hacer y no por cobardía sino porque la mejor
 forma de defender mis ideas es vivir acorde a ellas.

Antes de rotular a las personas debemos escucharlas.  Es nuestro deber también hacerles caer en sus contradicciones así
 como debemos estar dispuestos a escuchar que nos contradigan.  Yo no creo en la violencia como solución y la violencia
 comienza con la rotulación.

Yo no creo en la violencia como solución.  Ni la violencia de la clase dirigente que sin armas pero con corrupción y
 leyes acomodadas ejerce contra el pueblo.  La violencia de un Banco de la República que pretende bajar la inflación a
 punta de crisis económica.  La violencia de la guerrilla que mata al mismo pueblo que dice defender y acaba con el
 futuro del campo y del país.  La violencia de los paramilitares que se autodefienden matando lo que la guerrilla deja
 en pie.  La violencia de un ejercito que se siente legalmente acorralado y decide así saltarse la legalidad de cuando
 en cuando.  La violencia del raponero que toma por la fuerza lo que cree que la sociedad le negó.  La violencia del
 conductor que sabe que está en su vía y estrella su carro contra quien no sabía que la luz roja era para parar.

Yo no apoyo ninguna de esas violencias ni ningun otro tipo de violencia.  Ni la violencia de andar por ahí rotulando a
 la gente sólo porque no piensa como yo.

tal vez yo soy más pro- que anti-estado y bajo esa premisa no participaré en el próximo paro cívico y no faltará
 entonces quien me rotule de indiferente, o de para, o yo que sé de qué.  Pero creo que si tengo una esperanza de
 cambiar el sistema desde el sistema, puedo hacer de este mi propósito de vida.

\par% p
Por el momento mi meta es vivir acorde a mis principios.

\begin{verbatim}
— Carlos Th
================================O=O=====
Chlewey Thompin  \anchor[http://www.geocities.com/Paris/Rue/9028]{http://www.geocities.com/Paris/Rue/9028}
----------------------------------------
\end{verbatim}

\begin{small}
Sent via Deja.com \anchor[http://www.deja.com/]{http://www.deja.com/} Share what you know. Learn what you don't.
\end{small}

\chapter{FW: Historias desde Paisolandia (que no se deberian contar)}
\begin{metadata}
	Published by \anchor[chlewey]{chlewey} on \anchor[http://ewey.co/B182]{Tue, 16 Jan 2001 02:46:10 +0000}\\
	\categories{usenet, guest, medellin}\\
	Shorthand: \anchor[http://blog.chlewey.net/2001/01/historias-desde-paisolandia/]{historias-desde-paisolandia}
\end{metadata}

\begin{address}
\anchor[http://groups.google.com/group/soc.culture.colombia/msg/2f3f0964a439c753]{news:93vr30\$62d\$1@nnrp1.deja.com}
\end{address}
John Ladino wrote:
\par% div% {'style': 'background:#eee;color:#009;'}
El Tesoro.
7:20pm  Llego al Tesoro. Voy a cine con mi jefe. Quiero ver la Virgen de los Sicarios. Hoy es el último dia
 de la pelicula en cartelera. Parqueo donde siempre, en la puerta mas cercana, aunque muchos prefieren dar toda la
 vuelta y parquear en el primer sotano, abajo y mas cerca de los locales.  Pero por costumbre siempre parqueo en la
 misma parte. Salimos de los sotanos y Mauricio me dice que recuerde la zona donde esta parqueado el carro: por las
 escaleras cercanas a una tienda Carlos Nieto.

7:25 pm . No tengo plata y mi jefe tampoco. La película empieza a las 7:40. Tenemos que apurarnos y buscar un cajero.
 Hay un goofy en el cajero que no sabe usarlo. Tenemos que esperar a que desista y se vaya. Sacamos la plata.  Pasamos
 por el puente sobre la quebrada. Mauricio, mi jefe, me hace notar lo bonito de los arboles, iluminados con luces de
 colores. Yo le digo que si no lo había notado antes.

7:30pm. Paso por la plaza que da acceso a la zona de comidas. Veo el reloj. Como quiero llenarme de palomitas, perro
 caliente y gasesosa, porque me muero de hambre, le digo a Mauricio que apure el paso para tener tiempo de comprar la
 boleta y la comida. El me muestra una parejita muy bonita que habla sentada en la fuente. Luego vemos a un tipo
 rarísimo, alto y flaco que viene en sentido cercano. Nos reimos.

7:35pm. Algo le decía a Mauricio. No me acuerdo que era. Oigo un ruido horrible. Algo me sacude. Pierdo el equilibrio.
 No quiero pensar que algo malo está pasando. Quiero seguir la conversacion que tenía. Quiero dominar mis pasos. Se me
 scapa un grito pero no oigo mi voz.  Siento que algo se me cae encima. El techo se sacude y siento que caen vidrios.
 Creo que es un terremoto y entro en un local, me quedo debajo de la puerta, como he oido tantas veces que hay que
 hacer. Pero me doy cuenta de que no es un terremoto. Hay humo por todas partes. Busco a Mauricio y no lo veo. Salvese
 quien pueda. No pienso mas. Me dejo llevar. Salgo del local y empiezo a correr. Tower Records tiene todos los vidrios
 rotos. Que es esto ? Que pasa ? Que tengo que hacer ?

Llego corriendo al  otro lado del centro comercial. Allí no ha pasado nada. Todo mundo parece seguir en su rutina. Veo
 a algunos viendo vitrinas (pensando luego en esto me doy cuenta que no era así, obviamente, sólo que mi cerebro quería
 hacerme pensar que todo seguía normal, como un mecanismo de defensa). Me siento a salvo y me devuelvo sobre mis pasos.
 Creo que me he asustado más de lo debido. La curiosidad me domina. Quiero saber que está pasando. Alguien me llama. No
 veo quien es hasta que Mauricio se pone a mi lado. Quiere prender un cigarrillo y en medio de semejante alboroto me
 parece el colmo que quiera fumar. Tiembla y no puede hacer que el encendedor funcione. Qué esta pensando, huevón ? le
 pregunto... esta muerto del susto. Veo humo por doquier. Quiero ver que había ocurrido. Le digo que nos acerquemos. Un
 guardia sale de entre el humo, por el corredor donde un minuto antes estaba yo caminando tranquilo. Se rie con
 nerviosismo mientras trata de prender el radio-teléfono y mira a la señora que lo agarra por la camisa, angustiosa por
 saber que pasa. Veo la lujosa tienda de Sony,  los televisores de pantalla plana en el suelo. Una niña tiene un hueco
 en la cabeza. Alguien le limpia la sangre mientras ella se sostiene el pelo. Sale más humo y empiezo a sentir el olor
 de la polvora. Alguien grita algo: hay más en el baño ! Estampida !! ya no pienso nada. Instintivamente me devuelvo de
 nuevo y echo a correr. Arrollan a un niño. Nadie (ni yo) pienso en ayudarlo. Cada quien está por su cuenta y uno no
 piensa nada más en esas situaciones. Quiero irme. Llego a una zona que creo segura y me doy cuenta de que es un baño.
 Me da panico. Mauricio sigue a mi lado, aterrado, con cara de espanto. Un señor lleva a su niña de la mano y le dice
 que se calme. Alguien se sienta a llorar y se soba la pierna.  Dónde esta mi carro ? me quiero ir. Estoy al lado de un
 acceso al parqueo.. veo la tienda de Carlos Nieto ! Bajo sin pensarlo. Mauricio me dice que no, que es por otro lado.
 Lo agarro del brazo y lo hago bajar. Las escaleras eléctricas funcionan, pienso. Me parece raro porque al otro lado, a
 menos de 50m, todo es caos.
Ya para entonces sabía que todo habia pasado en los parqueaderos. De reojo, en el primer
 momento, ví que subía humo por una escalera. Pensaba que mi carro iba a estar destrozado. Pero no, está bien. Cuando
 quiero abrir la puerta, me doy cuenta de que estoy temblando. Me subo y le quito el seguro a la puerta del pasajero
 pero Mauricio no puede abrir y me grita. No sé como puedo manejar. Increible: una familia entera se está bajando de un
 carro. Acaban de parquear y no se han dado cuenta de nada. Se rien. No lo puedo creer. Pienso por instinto que salir
 de ahí va a ser difícil pero el camino esta relativamente libre.  Luego un campero grandísimo, repleto de niños me
 sale de la nada. Casi se me va encima. Cuando salgo del sotano veo gente herida que suben a los carros, todos corren.
 Hay humo. Todos los carros salen al mismo tiempo, pero no hay trancón, por fortuna. No hay guardia en la caseta de
 salida y me quedo ahi, como tonto, esperando que alguien me reciba la  ficha que me dieron cuando entré. Mauricio me
 la quita y la lanza por la ventana, mientras me grita que qué importa la maldita ficha. Me voy. Oigo las sirenas. Le
 pido a Maurico un cigarrillo. Me alejo lo mas que puedo. Pongo el radio en AM pero solo oigo a Pacheco hablando de no
 sé que maldita corroda de toros. Todavía no se que paso. Mauricio se pone a llorar. Yo le digo que fresco huevón, que
 ya pasó todo, que estamos bien... Huele a polvora y todo esta lleno de humo a 20 cuadras a la redonda. Mauricio quiere
 ir a su casa. El apartamento de Mauricio tiene una vista muy bonita, hacia la loma del Tesoro, y se ve el centro
 comercial con su torre amarilla. Cuando entramos huele a humo, que se coló por la ventana de la cocina. Al ver por el
 balcón, no está el Tesoro familiar, brillante y llamativo, si no uno horrible, oscuro y echando humo. Me de mucha
 rabia. Quiero llorar pero no puedo. Que impotente me siento. Nada en el radio. No se todavia que habia pasado
 exactamente. Malditos hijueputas ! grito. Volaron el Tesoro ! grita Mauricio, con la voz cortada. Me paso las manos
 por todo el cuerpo, buscando un rasguño, una esquirla... nada. Dios, gracias, digo en voz alta.

Después de una hora oigo las noticias. Me doy cuenta de lo cerca que había estado de que algo realmente malo me pasara.
 Cinco segundos antes y la bomba me hubiera explotado en los pies. El piso, literalmente, se me habría movido, hundido.
 La bomba había explotado a mis espaldas, 20m atrás.

El Tesoro le hacia honor a su nombre. Era el símbolo de que Medellín había pasado a mejores epocas y que el negro
 pasado había quedado lejos. Era uno de mis sitios favoritos y el de casi todos en esta ciudad. Lo volaron unos
 infelices bastardos, a la hora más concurrida. Se merecen la peor de las suertes en esta vida y la otra. Ojalá se
 pudran en la carcel y luego en el infierno.

Esa noche en la madrugada me desperte pensando si todo esto habia pasado o no. Me acordé de la parejita en la fuente,
 justo debajo del sitio donde explotó la dinamita y del tipo raro de pelo rojo y azul que iba en la direccion opuesta,
 directo a la trampa. No sé que les pasó. Ojalá estén bien.

Iba a ver la Virgen de los Sicarios, pero la verdad es que uno no necesita ver una película para ver y vivir la triste
 realidad de este pobre y desdichado país. Como siempre, la realidad supera a la ficción. Y pensar que hacía poco más
 de una semana andaba lejos en NY y ahora el país me saluda, me dice: Bienvenido ! y me hace ver la triste realidad que
 uno no puede dejar atrás, ni olvidar.

NO HAY DERECHO. NO HAY NINGUNA RAZON.

\par% p

— Carlos Th
================================O=O=====
Chlewey Thompin

\par% p
Sent via Deja.com \anchor[http://www.deja.com/]{http://www.deja.com/}

\chapter{¿Delincuencia enemiga del estado?}
\begin{metadata}
	Published by \anchor[chlewey]{chlewey} on \anchor[http://ewey.co/B184]{Wed, 30 May 2001 23:08:04 +0000}\\
	\categories{usenet, opinion}\\
	Shorthand: \anchor[http://blog.chlewey.net/2001/05/delincuencia-enemiga/]{delincuencia-enemiga}
\end{metadata}

\begin{address}
\anchor[http://groups.google.com/group/soc.culture.colombia/msg/95834e98398f1470]{news:3B157D54.391FB7BB@my-deja.com}
\end{address}
Definamos el estado como ese conjunto de organizaciones que controlan institucionalmente a un país.  El estado está
 compuesto, así, por el gobierno, el congreso, las fuersas armadas, los organismos de control, etc.

El estado no incluye a las grandes empresas ni a la insurgencia, que aunque puedan controlar a su modo un país, no son
 instituciones creadas para ello (no lo controlan institucionalmente).

El otro componente de un país es el pueblo, no como antagonista del estado, ya que las personas que trabajan con el
 estado son parte del pueblo, sino como complemento.

El objetivo del estado debe ser garantizar que los ciudadanos, que el pueblo, puedan vivir bajo ciertas garantías.  Que
 un ciudadano no tenga que temer que los del pueblo de al lado lo linchen, sólo por vivir en otro poblado.  Que un
 ciudadano sepa que puede realizar un negocio, por ejemplo comprar papa, sin que lo estafen, o sin que le peguen un
 tiro porque al otro no le gustó la oferta.  Desde esas pequeñas cosas que damos por hechas, hasta otras que deberíamos
 dar por hechas, como que yo pueda disfrutar de un café al lado de un parque sin que me maten con una bomba, sólo
 porque Menganito había matado a Fulanito y yo no conozco a ninguno de los dos.

Bien.  La sociedad: el conjunto de las personas del pueblo, de los ciudadanos de un país, tiene amenazas.  Una de las
 principales amenazas es la delincuencia, entendiendo por delincuencia los actos que individuos realizan y que afectan
 negativamente a la sociedad.  En otras palabras, un delincuente es el que comete un delito y un delito es algo que se
 ha estipulado como una amenaza a la sociedad.  Y la delincuencia es el conjunto de personas que cometen delitos y el
 conjunto de delitos cometidos.

Bajo este punto de vista, un funcionario corrupto del estado es un delincuente y la corrupción es delincuencia.
 También es delincuente el ratero que deja a la víctima sin plata y sin papeles y sin la cadenita que le regaló la
 mamá.  También es delincuente el que toma un fusil, se va para el monte y se pone a matar policías y campesinos disque
 porque está luchando contra un estado corrupto.  También es delincuente el sargento que envía sus tropas a retener
 civiles sin justificación disque porque está buscando insurgentes y a ejecutar, extrajudicialmente, a todo retenido
 que le huela a rojo.

Bien, si el objetivo del estado es garantizarle seguridad a los ciudadanos, el objetivo del estado debe ser eliminar
 las amenazas a la sociedad, más no, convertirse él mismo en una amenaza a la sociedad.  Lo último que debe hacer el
 estado, y mucho menos la fuerza pública (la parte del estado destinada a controlar la delincuencia, o \_law
 enforcement\_ en inglés), es personalizar la lucha contra la delincuencia o contra ciertas manifestaciones de la
 delincuencia.

En otras palabras, la estado y la fuerza pública no deben considerar ciertas amenazas a la sociedad como ``enemigos del
 estado''.  Y este es un error que ha cometido el estado colombiano durante mucho tiempo.

Y es claro.  Si la lucha contra la insurgencia es tomada por el estado, no como una lucha para proteger a la sociedad
 de una amenaza, sino como una guerra para aniquilar a un enemigo, se ha dejado a la sociedad de lado.  Y así como
 ignoramos a la sociedad, en el afán de combatir a la insurgencia bien puede el estado convertirse en una amenaza a la
 sociedad.

La lucha contra la delincuencia no debe ser nunca una guerra contra los delincuentes.  El papel que debe asumir el
 estado es el de contar con una policía que pueda vigilar y actuar como proyección del estado cuando un delito se
 comete, pero también debe contar con un fiscal que investigue y acuse, un procurador que se encargue de que el estado
 funcione bajo sus propias reglas, un defensor del pueblo que controle que el estado no se convierta en una amenaza a
 la sociedad, un juez que dictamine las culpas y un carcelero que se encargue de evitar que los individuos que amenacen
 a la sociedad lo sigan haciendo.

El estado para cumplir con su labor no debe confiarse exclusivamente de la fuersa pública sino que debe tener una
 presencia permanente, pero a la vez discreta, en todos los ámbitos de la sociedad.  Desde el notario que atestigua por
 los negocios celebrados o el alcalde que decide cual vía se va a pavimentar, es estado debe garantizar a los
 ciudadanos su bienestar.

Si el estado es demasiado omnipresente ahoga; está así la burocracia que exige un montón de papeles sólo para que pueda
 abrir una tienda.  Por otro lado si el estado está ausente, las amenazas a la sociedad crecen y obligan a la larga a
 que los individuos empiecen a tomar, sin preparación ni control, funciones que corresponden al estado garantizar.

Es así como se forman las milicias antisubersivas como reacción a un estado que no está presente, y apoyadas por una
 fuerza pública que, incapaz de ganar una guerra contra la delincuencia, recurre a estos grupos de vigilancia, no como
 una extensión de su labor de proteger, sino como una extensión de su guerra.

Las autodefensas, o paramilitares (según la versión que cada uno acoge), son el resultado de un estado que esta en
 parte ausente y que tan sólo se proyecta, si acaso, como una fuerza pública que ha olvidado su objetivo de proteger a
 la sociedad por el objetivo de ganar una guerra.

La misma insurgencia es también una causa de un estado deficiente.  Un estado que se ha puesto al servicio de una
 oligarquía en lugar de la totalidad del pueblo.  Un estado que no es capaz de controlar la corrupción de sus
 funcionarios.  Un estado que no es capaz de garantizar a los ciudadanos unas mínimas condiciones de igualdad y
 dignidad, es lo que ha generado y que mantiene viva a la insurgencia.

La solución a los problemas de Colombia no consisten en luchar contra el estado, o en crear paraestados.  La solución
 de los problemas de Colombia debe partir del estado mismo y de la responsabilidad que como ciudadanons ejercemos sobre
 el estado que tenemos.  Bien sea que el estado se cambie o que los ciudadanos lo cambiemos, pero el objetivo debe ser
 claro: necesitamos un estado que esté a servicio del pueblo y de la sociedad.

No necesitamos un estado ausente.

No necesitamos un estado paralelo.

No necesitamos un estado sumido en una guerra contra una de las manifestaciones de la falta de estado.

Necesitamos un estado que nos sirva.  Un estado que permita que yo pueda vivir tranquilo y progresando en mi vida
 personal.  Un estado que me permita educar a mis hijos.  Un estado que me permita reír, disfrutar de una película,
 llorar por mi bisabuelita que se murió de vieja, tomarme mi salario en una parranda, ahorrar para tener mi casita
 propia, trabajar mi tierra, comprar la papa y la yuca de la comida de mañana, enamorarme, escribir un poema, disfrutar
 un cielo estrellado, conocer la maravillas naturales de mi país, empaparme en un aguacero porque olvidé el paraguas o
 arruinarme en un juego de pocker.  Pero tambien un estado que no me permita sobrepasarme.  Un estado que no me permita
 pisotear a mis conciudadanos.  Un estado que no me permita olvidarme de mi responsabilidad como miembro de una
 sociedad.

Necesitamos un estado que le permita al jornalero del campo obtener un pago justo por su trabajo.  Que le permita al
 ganadero producir su carne y exportarla.  Que le permita a la multinacional petrolera que pueda entregarle energía al
 mundo, no sin una justa contraprestación a la nación dueña de las reservas.  Que permita opinar a quienes creen que
 ciertas personas tienen demasiados privilegios y a quienes creen que ciertas otras tienen demasiadas ayudas y a
 quienes creen que están en el sandwish de la clase media sin privilegios ni ayudas.  Y que le permita a cualquiera
 protegerse de los abusos del estado.

Necesitamos en fin un estado que no le dé papaya a la guerrilla para justificar su guerra, ni le dé papaya a los
 paramilitares de justificar su guerra.  Y que tenga la fuerza suficiente para que si la guerrilla y los paras
 continúan con su injustificada guerra contra la sociedad, el estado pueda combatir la amenaza en al que estos grupos
 delincuenciales se han convertido.

— Carlos Th

\chapter{Alternativa para acabar con el flagelo del narcotrafico colombiano 2}
\begin{metadata}
	Published by \anchor[chlewey]{chlewey} on \anchor[http://ewey.co/B186]{Wed, 06 Jun 2001 18:50:25 +0000}\\
	\categories{cocaina, usenet, drogas, legalizacion, narcotrafico, opinion}\\
	Shorthand: \anchor[http://blog.chlewey.net/2001/06/alternativa-2/]{alternativa-2}
\end{metadata}

\begin{address}
\anchor[http://groups.google.com/group/soc.culture.colombia/msg/13cabc685b0416be]{news:3B1E7B71.E7462FAF@my-deja.com}
\end{address}
Expuse en una respuesta una alternativa para acabar con el flagelo del narcotráfico colombiano sin fumigar cultivos ni
 legalizar (completamente) la droga: Seguir dejando la droga ilegal como ilegal pero producir droga legal que compita
 con calidad y precio contra la ilegal.

Aquí hay otra:   Se toman varias matas de coca, tanto de variedades selváticas como montañosas y se prueban en
 diferentes condiciones.  La idea es desarrollar una mata que sobreviva y produzca hoja y alcaloide en abundancia en
 condiciones diferentes a la zona tórrida, por ejemplo en materas sembradas en el interior de casas y en la zona
 templada, sobreviviendo a las estaciones.  Estas matas adaptadas son contrabandeadas en el interior de los EE.UU.,
 junto con la tecnología para la extración casera del alcaloide.  Se reparten varios brotes y semillas entre grupos
 dedicados a la distribución de la cocaína, de tal forma que estos grupos puedan producir su propio droga en los EE.UU.
 y no tengan que importarla de Colombia.

Finalmente, como los gringos no son bobos, terminarán creando mejores variedades de las que ofrece el trópico y al no
 tener que importar la droga colombiana a mayor riesgo, pueden ofrecer un producto más económico en la calle.  Los
 precursores son incluso más baratos. Incluso, como ellos mismos hacen el procesamiento, también tendrán los
 subproductos como el bazuco que podrán comercializar.

Esto no soluciona el problema del consumo de drogas en los EE.UU. pero acaba con el problema del tráfico internacional
 de narcóticos que es la principal amenaza a las instituciones colombianas.  Simplemente nos habremos librado del
 problema para convertirlo en un problema de los gringos.

Cuando los distribuidores gringos dejen de comprarle los cargamentos a los traficantes colombianos, estos se dedicarán
 a otros negocios, dejarán de comprarle la coca a los colonos y campesinos y estos últimos terminarán cultivando yuca.

Los gringos consumidores estarán felices: la droga cuesta menos.  Los no consumidores se escandalizarán al principio
 pero al poco tiempo se acostumbrarán: con la droga más barata habrá menos crímenes relacionados con las drogas.
 Pronto aparecerán las bondades medicinales de la coca y habrá quienes propongan legalizarla por los efectos benéficos
 que trae para tolerar drogas químicas utilizadas para el tratamiento del sida y el cancer.

¿Utópico?

¿Quién en Colombia se dedica hoy a exportar mariguana a los EE.UU.?

— Carlos Th

\chapter{Alternativa para acabar con el flagelo del narcotrafico colombiano 1}
\begin{metadata}
	Published by \anchor[chlewey]{chlewey} on \anchor[http://ewey.co/B187]{Wed, 06 Jun 2001 20:09:33 +0000}\\
	\categories{cocaina, usenet, drogas, legalizacion, narcotrafico, opinion}\\
	Shorthand: \anchor[http://blog.chlewey.net/2001/06/alternativa-1/]{alternativa-1}
\end{metadata}

\begin{address}
\anchor[http://groups.google.com/group/soc.culture.colombia/msg/77f6d3a3da0236cf]{news:3B1E8DFD.4E14CF33@my-deja.com}
\end{address}
Imaginemos un ecenario, un tris utópico pero vale la pena analizarlo. Junto con los cultivos ilícitos en la selva y el
 monte, hay 100 hectáreas de coca sembradas en la sabana de Bogotá, cultivados abierta y legalmente por una empresa
 legalmente constituída de capital colombiano y estadinense.  Este cultivo está mecanizado y debidamente fertilizado
 aumentando la calidad y la cantidad de producción por cada hectárea.  Al lado del mismo hay una planta procesadora,
 cuyos insumos son adquiridos legalmente bajo los impuestos que Colombia tenga vigentes a cualquier otra agroindustria
 legal destinada a la exportación.  Esta planta procesadora, con altos estándares higiénicos, produce cocaína de la más
 alta calidad, así como extrae muchas otras substancias presentes en la coca y que tienen fines medicinales... incluso
 hoja calada que pueda ser usada para el mambeo o sobresitos de té de coca.  Esta empresa, localizada cerca del
 aeropuerto Eldorado y conectada por rutas pavimentadas a los puertos de Santa Marta y Buenaventura, puede poner
 grandes cantidades de producto en Miami o Nueva York, donde redes de distribución legales llevan la cocaína a
 droguerías por todos los EE.UU. donde se vende al público con mínimas restricciones.  Bien.  Los procesos se optimizan
 para ofrecer la mayor calidad de producto al menor precio posible para el consumidor final.  La cantidad es, sin
 embargo, todavía demasiado pequeña para que los EE.UU. sea inundados de por esta droga legal.  En este caso, aquellos
 que cultivan en la selva, procesan en cambuches en el monte, sacan el producto por lancha o avioneta, ingresan la
 mercancía ilegalmente a los EE.UU. en cantidades muy pequeñas para ser detectadas o pagando altos sobornos y mantienen
 redes de distribución llenas de intermediarios... estos contrabandistas estarán compitiendo con un producto de calidad
 ordinaria y precio astronómico contra el producto legal.

No es su ecenario, aquí nadie se vuelve rico ni Colombia entra en una época de prosperidad y paz, pero es una
 alternativa para controlar el negocio ilegal de las drogas.

\chapter{La gran conspiración.}
\begin{metadata}
	Published by \anchor[chlewey]{chlewey} on \anchor[http://ewey.co/B196]{Fri, 29 Jun 2001 03:28:41 +0000}\\
	\categories{usenet, fiction}\\
	Shorthand: \anchor[http://blog.chlewey.net/2001/06/conspiracion/]{conspiracion}
\end{metadata}

\begin{address}
\anchor[http://groups.google.com/group/soc.culture.colombia/msg/c42915b636d5a388]{news:9hgsra\$dpn9k\$1@ID-83976.news.dfncis.de}
\end{address}
Según mis últimas investigaciones exhaustivas he venido a descubrir la gran conspiración que hay en contra del pueblo
 colombiano.  Es bastante sencilla por cierto, tan sencilla que pasa por obvia y se remonta a muchísimos años atrás
 pero ha sido tan bien craneada que tan sólo podemos concentrarnos en los aspectos más recientes.

Pero antes de exponer mis descubrimientos hay que hacer unas aclaraciones.  La CIA no está detrás de la gran
 conspiración por el hecho sencillo de que la CIA no existe.  La CIA no es más que la fachada de la NSA que muchos
 creerán que se trata de la agencia de seguridad nacional gringa pero la sigla corresponde realmente al etrusco Nacham
 Sih'oun Agnirk que ha sido mal traducido como Los Sabios de Sión, lo que en realidad no tiene sentido porque los
 etruscos no eran judíos, mucho menos Sionistas, pero tras la gran conspiración judía que se inventaron al cristianismo
 para poder derrotar al imperio romano, los sih'ounitas se aliaron con los judíos para resistir esa rueda suelta en que
 el cristianismo se había convertido.

Durante la edad media, sin embargo, los critianos dejaron de distinguir a los judíos de los sih'ounitas, mientras éstos
 últimos se inmiscuían en los asuntos secretos de los estados, los judíos se dedicaron a crear el sistema bancario.  El
 papel de la NSA a lo largo de la historia se puede evidenciar en el desmembramiento del imperio carolingio, la
 invasión sajona a gran bretaña, la franquización de los normando, la invasión árabe al sur de Italia (la invasión mora
 a Iberia sí fue pura ineptitud de los visigodos), la caída de Gran Zimbawe, la invasión polinesia a Tahiti, la
 creación de Novogorod y la alianza entre Aztecas y Olmecas, porque créanlo o no, la NSA estaba en México desde mucho
 antes de que llegara Colón.  Es más, fue la misma NSA quien trajo a Colón por medio de un plan muy simple el cual
 consistió en convencer a los romanos de la importancia capital que era conocer el verdadero sexo de los ángeles
 mientras otros agentes daban planos detallados de la ciudad de Bizancio a los turcos.  Con esto no pretendían más que
 cortar el comercio por tierra entre las Indias y Europa, de tal forma que su agente especial, Cristóbal Colón, tuviera
 una excusa para que los reyes de España invirtieran en un supuesto viaje a las Indias.  Eso de que Colón murió
 creyendo que había llegado a las Indias es pura paja.  Colón sabía desde el principio que llegarían a América, donde
 los españoles servirían de idiotas útiles para el gran castigo que la NSA quería propiciarle a los Aztecas, quienes ya
 habían traícionado su alianza con los Olmecas.

Así que nosotros, en nuestro país, tenemos el nombre de un agente especial de la NSA.  ¿Pueden creerlo?

Desde luego que la NSA se ha seguido infiltrando en las relaciones internacionales entre los distintos gobiernos.
 Tuvo, sin embargo un revés en las colonias españolas en América y fue la desconfianza de los neogranadinos a la
 propuesta bolivariana de la NSA.  Es por eso que desde que tumbamos al agente Simón Bolívar y asesinamos al agente
 especial Sucre, nos han tenido entre ojos.  Por eso es que no nos dejaron quedarnos con el ferrocarril ni el canal de
 Panamá... claro que hasta no estar seguros de que los panameños no seguían siendo neogranadinos no les dejaron a ellos
 tampoco el canal.

Por eso nos pusieron a cantar un poema larguísimo de 11 estrofas alejandrinas llenas de citas obscuras y una música que
 no se adapta a su ritmo y nos convencieron de que era el segundo himno nacional más bello del mundo.  Y claro,
 nosotros nos lo creímos.  Eso sí se aseguraron de que lo creyéramos el segundo y no el primero, para que nos
 creyéramos pero no demasiado.

Para el colmo de males nos han salvado del honor de no tener una verdadera dictadura militar durante el último siglo.
 ¿¡Pueden imaginarse tanta tortura!?  ¡Más de un siglo gobernados por políticos! Y encima nos ensartan a Andy Rabit
 para terminar.

Y ahora nos tienen rodeados de agentes de la NSA.  Miremos no más a Hugo Chavez a quien le crearon una campaña para la
 presidencia con un supuesto golpe de estado para que las masas lo vieran como aquel capaz de acabar con los políticos
 cuando finalmente se presentara a las elecciones y poco a poco se va afianzando a una posición donde finalmente la NSA
 podrá imponernos a los neogranadinos/colombianos los ideales que no pudo imponernos con el longanizo.

La gran revelación es que Montesinos no pertenece a la NSA y la NSA ha procurado a lo largo de todos estos años
 convencer a Montesinos de que él es realmente la menta macabra detrás de todas las conspiraciones de los Sabios de
 Sih'oun en esta esquina del globo.

Hay otros tres importantes agentes del NSA en nuestra historia reciente. El primero de ellos es Gonzalo Rodríguez
 Gacha, bueno, ese es el nombre con el que conocimos a este agente al que también llamaban El Mexicano. Resulta que El
 Mexicano logró integrarse en la organización delictiva de Pablo Escobar y la convirtió en una de las organizaciones
 criminales más sangrientas de la historia mundial.  Y el hombre sigue vivo, eso de que le explotó una granada en la
 cara cuando el ejército lo acosaba no fue más que una movida de la NSA para que los colombianos enfocáramos nuestros
 esfuerzos en una lucha contra Pablo Escobar mientras el Mexicano se iba a entrenar mercenarios en Sierra Leone.

El otro gran agente de la NSA es Manuel Marulanda, alias Pedro Marín, alias Manuel Marulanda, alias Tirofijo.  Es por
 eso que las FARC sí estuvieron involucradas en toda la campaña de desinformación de que sí y de que nó secuestraron a
 Mejía Campuzano.  La táctica fue bastante astuta, primero convencen al pobre de Andy Rabit para que se valla con dos
 de sus ministros, siete alcaldes, dos directores de la policía y un par de senadores a Asunción para rogar por una
 copa.  Desde luego que para que rogara primero la NSA logró poner unas cuantas bombas en lugares estratégicos, así
 haciía parecer que Colombia necesitaba la tan anhelada copa de la paz que Andy Rabit trajo a besuquear, para que
 luego, nos dieran la estocada final.  Recordemos que la NSA nos tiene tirria porque los neogranadinos no le comieron
 cuento al agente Bolívar.

El otro importante agente de la NSA es Cesar Gaviria Trujillo, el único agente de la NSA que se le ha medido a la
 presidencia de la república y todo para tapar un pequeño error de cálculo y es que cuando convencieron a Escobar de
 asesinar a Galán, no contaron con que Galancito Jr. le enchutara las banderas de su padre a Gaviria, entonces jefe de
 debate. Gaviria ha cumplido cabalmente con su papel como agente de la NSA: aconsejó como ministro a Barco para que
 abriera de piernas los mercados colombianos y continuó con su política como presidente.  Nombró a Ernesto Samper
 ministro para que la opinión pública dejara de hablar de Ernesto el hermano de Daniel y darle así el perfil
 presidencial que necesitaba para que los colombianos pudiéramos rematar con un presidente al que no lo dejaron
 gobernar seguido de un presidente que no puede gobernar.  Es que estos de la NSA son crueles con nosotros los
 colombianos.

Hay muchos más detalles de la conspiración que nos tienen los de la NSA, pero de lo poco que hay claro es que nos
 tienen preparada una jugada bastante fea.  Todo este proceso de paz y las metidas de pata tanto de la guerrilla como
 del gobierno no son más que trucos para alebrestar los ánimos.  Convenzámonos de una cosa, la NSA no es fariana, tan
 sólo ha infiltrado las más altas esferas de las FARC para polarizar al país entre los que creen en la
 institucionalidad del gobierno y los que se oponen.  Ya vamos a ver como al termino de cinco años veremos que las
 conversaciones entre las FARC y las AUC llegan a un acuerdo para tumbar al aún más impopular presidente entonces, que
 aún no me queda claro si será Serpa, Uribe o Sanín.

— Carlos Th

\chapter{La gran conspiración, continuación}
\begin{metadata}
	Published by \anchor[chlewey]{chlewey} on \anchor[http://ewey.co/B198]{Fri, 29 Jun 2001 15:26:34 +0000}\\
	\categories{usenet, fiction}\\
	Shorthand: \anchor[http://blog.chlewey.net/2001/06/conspiracion-2/]{conspiracion-2}
\end{metadata}

\begin{address}
\anchor[http://groups.google.com/group/soc.culture.colombia/msg/0bcbe1a50d868775]{news:3B3C9E2A.43E2DE0D@my-deja.com}
\end{address}
Ya hemos visto como la NSA está conspirando en contra de Colombia, hay sin embargo algunos detalles que vale la pena
 aclarar.

Las FARC no son una conspiración de la NSA.  Los ideólogos de esa organización subversiva tales como Jacobo Arenas e
 Iván Cano han desconocido que han sido utilizados por la NSA para su propósito de vengarse de los colombianos, lo
 mismo que jefes militares como el Jorge Briceño (Mono Jojoy) o Romaña.

Ni Carlos Castaño ni ningún otro de los líderes de las autodefensas son agentes de la NSA, sin embargo acordémonos que
 Gonzalo Rodríguez Gacha síi era un agente y fue quien empezó a crear a los grupos de autodefensas del Magdalena Medio
 y quien en últimas reclutó a Fidel Castaño para formar las autodefensas de Córdoba y Urabá.  Estas autodefensas,
 mentadas por la NSA a través de agentes como Jair Klein (sp?), tomaron vida propia y, aunque ninguno de sus líderes es
 de la NSA, si están infiltradas por la agencia y muchas de sus actuaciones obedecen a los planteamientos
 desestabilizadores de la NSA.

Ningún presidente colombiano, con las excepciones de Bolivar y Gaviria, han pertenecido a la NSA, sin embargo la NSA
 está involucrada en las más altas esferas del poder político y de las oligarquías de Colombia.

Aún no he descubierto cual de los colaboradores de SCC es de la NSA, pero basado en las anteriores investigaciones de
 jcordob y de la forma de actuar de la NSA, estoy casi seguro que cor...@ucs.orst.edu y Firmsteel son parte de los
 caracteres que la NSA ha creado para desinformarnos.  ¿O es jcordob el infiltrado?  Eso sí: juro que yo no soy.

Ni los rosacruci ni los templarios han sido parte de la NSA.  Todo lo contrario, el carácter místico y secreto de estos
 grupos ha sido una forma de contrarrestar el carácter secreto de la NSA sin despertar alarma en el público.  Los
 masones son una logia aparte, infiltrada parcialmente por la NSA, sin embargo los masones han desconocido que la NSA
 existe.  Es así como la NSA logró colar en la logia masónica a personajes como Bolivar y Washington.

En cuanto a los grandes asesinatos políticos en Colombia y la Nueva Granada, la NSA ha estado involucrada como víctima,
 victimario o instigador.

Al mariscal Sucre no lo asesinaron por ser venezolano sino por ser de la NSA.  Lo mató una secta neotemplaria caucana
 que descubrió sus vínculos sih'ounitas.  A Gaitán lo mató el NSA en una conspiración más compleja de cualquiera que
 Paul Wolf pueda imaginarse.  Recordemos que la CIA no es más que una fachada de la NSA por medio de la cual reclutaron
 a Roa Sierra quien se encuentra actualmente en una casa de retiro en Palm Beach bajo otro nombre.  Por otro lado Galán
 no fue asesinado directamente por la NSA, pero si fue instigado por ellos: Rodríguez Gacha fue metiéndole la idea a
 Pablo Escobar que si Galán llegaba a la presidencia lo meterían en un prisión de máxima seguridad en gringolandia así
 que Escobar decidió atajar a Galán.  Su muerte fue tan solo un acto de terrorismo contra el pueblo colombiano, pero
 con lo que no contaban era con que el jefe de debate de Galán, el agente de la NSA César Gaviria, iba a ser señalado
 como el sucesor de Luis Carlos y terminaría en la presidencia de la república.

Hay otros detalles de la historia universal donde la NSA ha estado involucrada.  Recordemos que en europa muchas veces
 confundieron a los Sih'ounitas con los Sionistas, es por eso que cuando un asesor del Zar encontró un documento de
 unas reuniones de la NSA los tradujo como los Protocolos de los Sabios de Sión y en su pésima traducción y movido por
 sus propios prejuicios, atribuyó la gran conspiración mundial sih'ounita como una conspiración mundial judía.

Estos documentos fueron decisivos en el antisemitismo de finales del siglo XIX y la primera mitad del siglo XX y que
 desencadenara en el holocausto judío perpetrado por Hitler y sus secuaces.  No sobra aclarar que los agentes de la NSA
 infiltrados en el partido nacional-socialista de los trabajadores (Nazi) se encargaron de desinformar lo suficiente
 para que los líderes nazis siguieran creyendo que fueron los judíos y no la NSA quienes conspiraban contra la raza
 aria.

En parte la guerra, y principalmente la mundialización de la guerra de 1939, fue instigada por la NSA, quienes
 convencieron a Yamamoto de atacar a Pearl Harbour para que el agente F.D. Roostvel tuviera la excusa perfecta para
 meter a los EE.UU. en la guerra.  Desde luego, fue la misma NSA quien convenció a Hitler de declararle la guerra a los
 EE.UU. una vez estos le habían declarado la guerra al Japón y así podía la NSA borrar los datos de sus actividades en
 Europa antes de que los judíos los descubrieran por traícionar la aliansa que llevaban desde la caída del imperio
 romano de occidente.

Varios presidentes de los EE.UU. han pertenecido a la NSA, comenzando por Jorge Washington y terminando con Jorge Bush
 padre.  El actual presidente no lo es y hasta donde sé desconoce la verdad sobre su padre, quien deliberadamente
 perdió las elecciones de 1992 para no despertar sospechas.

— Carlos Th

\chapter{Un estado fuerte…}
\begin{metadata}
	Published by \anchor[chlewey]{chlewey} on \anchor[http://ewey.co/B202]{Mon, 23 Jul 2001 21:03:38 +0000}\\
	\categories{usenet, opinion}\\
	Shorthand: \anchor[http://blog.chlewey.net/2001/07/un-estado-fuerte/]{un-estado-fuerte}
\end{metadata}

\begin{address}
\anchor[http://groups.google.com/group/soc.culture.colombia/msg/48a56629d4928de5]{news:3B5C912A.A28F5E3F@my-deja.com}
\end{address}
Algunas propuestas:

Fortalecimiento del estado.  El estado debe ser lo suficientemente fuerte para combatir las amenazas que surjan para la
 población.  Un estado fuerte que se evidencie en, por ejemplo, una oficina de atención a desastres que pueda
 reaccionar ante un terremoto, rescatando a las víctimas, dando albergue decente a los damnificados mientras
 reconstruye la infraestructura dañada, canalizando los recursos para que la reconstrucción sea rápida, etc.  Un estado
 fuerte que se refleje en la pronta respuesta a la injusticia, tanto la cometida por los delincuentes comunes como por
 los mismos funcionarios corruptos del estado.

El fortalecimiento del estado significa un fortalecimiento de las fuerzas armadas del estado, tanto de la policía como
 de las fuerzas militares.  La policía debe garantizar a los ciudadanos la protección ante la delincuencia mientras que
 las fuerzas militares deben prestar a los ciudadanos protección ante amenazas externas y, en casos extremos, apoyo a
 la policía cuando un problema delicuencial se agrave.

El fortalecimiento del estado significa también un fortalecimiento de los organizmos de control: de la fiscalía, de la
 procuraduría, de la contraloría, de la defensoría del pueblo y de las cortes y juzgados. También del sistema penal.
 De esta forma podemos garantizarle a la ciudadanía que puede vivir a salvo de la delincuencia y de la corrupción.
 Garantizarles también que el estado mismo no se va a convertir en el opresor.  Buscar que porque el vecino me tiene
 envidia no me vaya a acusar de bruja y me quemen en la hoguera sin que me hayan realizado un juicio justo.

El fortalecimiento del estado no significa un crecimiento desordenado del estado.  No significa que se aumenten sin
 control el número de puestos que devengarán un salario público de debe salir de los bolsillos de los contribuyentes.
 No.  El fortalecimiento del estado significa que los procesos deben hacerse más eficientes y más eficaces y que lo que
 sobra sobra.

El fortalecimeinto del estado no significa que sea el estado el bien último.  Es el pueblo (ricos y pobres), son las
 personas, los individuos, el bien último.  Un estado fuerte no es sinónimo de fascismo... no necesariamente.  El
 estado tiene una función: servir, y esta es también la función de los funcionarios del estado, de los servidores
 públicos: servir.

Y esto lleva a que el estado deba ser transparente.  Las personas no deben trabajar para el estado, sino el estado para
 las personas.  No hay que trabajar para pagar impuestos y llenar papeles, sino para uno mismo, y los mismos
 funcionarios del estado no deben trabajar para el estado sino para el pueblo que legitimiza al estado.

Porque algo importante, la fortaleza del estado debe partir de su legitimación y de su legitimización por el pueblo.
 Cuando el estado no representa al pueblo no es fuerte, o es fuerte sólo por la fuerza física, por el temor que las
 armas producen.  Se tienen así estados débiles como el colombiano, o caemos en el extremo de los estados absolutistas
 como la Alemania nazi o la Unión Soviética.

El estado debe partir del pueblo quien en elecciones libres escoge a sus gobernantes y a sus representantes, quienes se
 encargarán de la cosa pública (res publica \&gt;república), con la legitimidad que el pueblo les da y para servicio del
 pueblo todo, no como una masa sino como la reunión de todos los ciudadanos, de todas las personas, de todos y cada uno
 de los individuos.

Ahora bien: ¿quién señala cuáles de estos puntos son de izquierda y cuáles de derecha?

— Carlos Th

\chapter{La policía por fuera del ministerio de defensa}
\begin{metadata}
	Published by \anchor[chlewey]{chlewey} on \anchor[http://ewey.co/B203]{Tue, 24 Jul 2001 17:15:38 +0000}\\
	\categories{usenet, opinion, policia}\\
	Shorthand: \anchor[http://blog.chlewey.net/2001/07/policia-mindefensa/]{policia-mindefensa}
\end{metadata}

\begin{address}
\anchor[http://groups.google.com/group/soc.culture.colombia/msg/258158ebb663fc87]{news:3B5DAD3A.2DDBD8AB@my-deja.com}
\end{address}
Algunas propuestas:

La policía por fuera del ministerio de defensa.

Es de suponer que el objetivo del ministerio de defensa (antes llamado ministerio de guerra) es el de defender a la
 nación colombiana de un ataque, generalmente externo.  Como tal bajo su cargo se encuentran las fuersas militares que
 son las que ejecutan esta misión.

\par% p
El objetivo de la policía no es propiamente defender a la nación sino proteger a la ciudadanía como una parte del
 sistema de administración de justicia, aquello que en inglés llaman \emph{law enforcement}.

Ahora, hay cosas en las que los papeles de un cuerpo policial y un cierpo militar se encuentran: el uso de las armas.
 Una milicia sin armas carece de los instrumentos para defender una nación se es atacada por un enemigo armado.  Una
 policía sin armas carece de los instrumentos para someter a un criminal si este está armado.  Pero el objetivo de las
 armas es diferente para cada cuerpo.  Un cuerpo militar se prepara para hacer la guerra mientras que la policía se
 prepara para mantener el orden.

Sólo cuando ``el orden'' es igual a la guerra, es que ambas funciones se entremezclan... y es aquí donde toca abrir un
 paréntesis para ver la realidad de nuestra nación.

En Colombia el problema del bandolerismo y de la insurgencia se ha tratado como un problema militar.  Gracias a nuestra
 tradición de guerras civiles durante el siglo XIX, las milicias se han creado, más que para combatir las amenazas
 extranjeras, como un instrumento para imponer una política y de esos ejércitos partidistas es que nace nuestro
 ejército nacional y nuestra policía.

Casi no hemos tenido conflictos externos.  La misma independencia tuvo mucho de guerra civil y eso es claro con las
 denominaciones que se daban en la época: no era una lucha de criollos o americanos contra españoles sino de patriotas
 contra realistas.  Alguna incursión ecuatoriana, la guerra contra el Perú a principios del siglo XX, la declaración
 nominal de guerra a Alemania durante la segunda guerra mundial que no se reflejó en el envío de tropas a luchar en
 Europa y la participación de Colombia en Corea como parte de las Naciones Unidas.

Por eso nuestas fuerzas militares han estado muy entretenidas en la problemática interna y que fácilmente los gobiernos
 han usado para combatir o controlar a los bandoleros, a los del partido de oposición o, más adelante, a las
 autodefensas de Marquetalia.

Las fuerzas militares se preparan para la guerra.  Si decidimos que hay que controlar el crimen como si fuera una
 guerra: la guerra contra el crimen, la guerra contra el narcotráfico, etc. entonces tiene sentido que sean las fuerzas
 militares las que se encargan del control del crimen.

Por otro lado, si se tiene en cuenta que el crimen no es una amenaza externa a la nación, sino que el crimen es un acto
 privado de los ciudadanos contra los ciudadanos, entonces al crimen no se le hace la guerra sino que se le somete a la
 justicia.  Es aquí donde interviene todo el aparato de sometimiento a la justicia: la policía y la fiscalía que
 investigan, la fiscalía que acusa, la policía que detiene, los jueces que juzgan, los defensores del pueblo que se
 encargan de evitar que los diferentes organizmos se extralimiten, las prisiones que contienen a los criminales
 convictos, etc.

En este aparato de sometimiento a la justicia interviene el poder judicial, sin duda, pero también el ejecutivo.  En el
 caso colombiano actual, vemos al ejecutivo a través del ministerio de defensa como jefe de la policía, y el ministerio
 de justicia como el encargado de las prisiones y vemos al poder judicial en los jueces, en los fiscales y en la
 policía judicial (CTI: Cuerpo Técnico de Investigación) adscrita a la fiscalía.

Personalmente creo que el ordenamiento del aparato de sometimiento a al justicia es un tema bastante complejo.  Me
 parece útil la independencia de la fiscalía frente al ejecutivo, pero preferiría ver a la policía judicial y a la
 policía nacional en una sola institución por fuera del ministerio de defensa.

En últimas, preferiría tener una policía más civil y menos militar, que asegure el cumplimiento de la ley tanto en el
 campo como en las ciudades y las cabeceras municipales.  Y, desde luego, que no existan guerrillas ni paramilitares y
 si hay grupos terroristas y criminales que masacran y dezplazan campesinos, la policía esté en capacidad de
 combatirlos, no como una guerra sino como una forma de controlar el orden para el servicio de los ciudadanos.

Aún hay un camino largo para llegar a ese entonces... en la situación actual sacar al ejercito de la lucha contra la
 insurgencia, el paramilitarismo y el narcotráfico es algo medio utópico.  Dejar de llamarlos ``soldados'' por ``policías''
 y decir que ya no son parte del ministerio de defensa sino del de justicia es sólo cambiarle de nombre a la situación.

¿Opiniones?

— Carlos Th

\chapter{El negocio del servicio público}
\begin{metadata}
	Published by \anchor[chlewey]{chlewey} on \anchor[http://ewey.co/B206]{Mon, 06 Aug 2001 15:19:09 +0000}\\
	\categories{usenet, movilidad, opinion, trasnporte-publico}\\
	Shorthand: \anchor[http://blog.chlewey.net/2001/08/el-negocio-del-servicio-publico/]{el-negocio-del-servicio-publico}
\end{metadata}

\begin{address}
\anchor[http://groups.google.com/group/soc.culture.colombia/msg/17dfe5b19b4cca73]{news:3B6EB56D.2CC6696E@my-deja.com}
\end{address}
Tiendo a inclinarme más por la postura del alcalde Antanas Mockus que por la del gremio de transportadores de Bogotá y
 la razón no es sólo el resentimiento de un gremio que ha paralizado a mi ciudad sino que viene además de los
 argumentos que las partes: el gremio, el alcalde, los ciudadanos, etc. han expresado en estos días.

Escribo esto mientras observo a la carrera 15 bastante despejada.  En parte tal vez porque hoy es día de puente, de los
 puentes reales y no de los lunes festivos que hemos optado por llamar puentes.  Ayer fue domingo y mañana es festivo:
 la conmemoración de los 182 años de la batalla del Puente de Boyacá.  Además celebramos hoy los 463 años de fundación
 de nuestra ciudad.  Pero la principal razón por la cual hay pocos automóviles, camiones y busetas por la usualmente
 congestionada carrera 15 es por el paro de transportadores.

En Bogotá sobran buses y taxis para cubrir la demanda de transporte público.  En gran medida esto se da porque en
 Bogotá el transporte público ha sido siempre un negocio: un buen negocio para algunos, un mal negocio para muchos,
 pero, al fin y al cabo, un negocio.

Los taxis han sido una solución al desempleo.  No importa que no se necesiten y que, por lo tanto, la cantidad de
 vehículos amarillos y de personas empleadas en este negocio se canibalicen mutuamente.  Ahora que hay una propuesta
 para que se organicen simplemente se asustan con el cambio y exigen que no les quiten sus puestos de trabajo.

Y esto es porque el esquema del negocio no está diseñado para ofrecer empleo sino para copar, con algo, los huecos del
 sistema de transporte. No importa si somos más los ciudadanos que en búsqueda de una calidad de vida, necesitamos un
 medio de transporte eficiente, el trancón permanente es necesario para que unos pocos tengan empleo.

Y todos los argumentos de los transportadores, de los taxistas principalmente, están centrados en un esquema de
 negocios que es hasta absurdo... y lo peor, un esquema de negocios que en últimas va a ser favorecido por la medida de
 la alcaldía, si es que internamente no logra autoajustarse.

Más complejo es el tema de los buses y busetas.  Más complejo porque es un problema de fondo pero que en últimas está
 basado en un concepto de negocio para algo que debe ser un serivicio público.  Y es que no hay nada de malo en obtener
 ganancias de un servicio público como lo haría una eficiente empresa de teléfonos, energía o agua potable: el problema
 es el enfoque.

Cómo el transporte público de Bogotá ha sido enfocado como un negocio, no hay ningún tipo de ingeniería de transporte
 para crear nuevas rutas. El recorrido promedio de una buseta es absurdamente largo y complicado pero es la forma en
 que puede atravezar una ciudad tan grande como Bogotá exprimiendo las rutas secundarias y aprovechando tramos de
 varias vías principales.  Para el usuario la alternativa sería tomar varios tramos con varios transportadores y
 desplazarse rápido pagando tres o más pasajes o pagar un sólo pasaje para un sightsing de los trancones de Bogotá.

La así llamada guerra del centavo parte de un negocio mal estructurado, donde cada conductor tiene que conseguir al
 máximo número de usuarios a costa de los demás conductores y de los usuarios mismos.  Una señora de edad que necesita
 medio minuto en desencaramarse de lo alto de un chasís de camión con carrocería de bus es un mal negocio para el
 conductor quien no ve inconveniente en arrancar antes de que la señora termine de bajarse.  Ese medio minuto es
 precioso para el conductor que necesita ir a la siguiente cuadra a recoger a otro sufrido usuario.

Pero cuando les montan una competencia que está pensada como un serivicio, y aún así es rentable, la opción de estos
 transportadores no es la de organizarse para prestar un buen servicio sino que, enfocados en un esquema de negocios
 que se ve amenazado, se canibalizan mutuemente prestando un peor servicio que buscan defender a toda costa... y cuando
 se ven amenazados por una nueva legislación buscan paralizar la ciudad como una medida de fuerza para que las cosas
 sigan siendo como antes: como cuando el usuario importaba un pito.

— Carlos Th

\chapter{WI US invaded Colombia}
\begin{metadata}
	Published by \anchor[chlewey]{chlewey} on \anchor[http://ewey.co/B212]{Tue, 11 Dec 2001 17:11:50 +0000}\\
	\categories{usenet, fiction}\\
	Shorthand: \anchor[http://blog.chlewey.net/2001/12/wi-us-invaded-colombia/]{wi-us-invaded-colombia}
\end{metadata}

\begin{address}
\anchor[http://groups.google.com/group/soc.history.what-if/msg/582b6ea603c4d901]{news:3C163E56.C7B6A28@my-deja.com}
\end{address}
A recurrent them in Colombia is a US invasion in our territory, that some people would even want in order to solve our
 problems.  However there is something that stop the US to commit troops in Colombia and is the fear of another Vietnam.

Given the way the Gulf War and the Afgan campaigns have been fought, it seems that repeating Vietnam is not longer
 something that the USA is afraid of.

As I actually think, for the USA it is much better to have a friendly government in Colombia protecting their oil and
 banana interests and controling the guerrillas and the drug trade than actually invading a 40 million inhabitants
 country and dealing directly with those problems.

Which leads to a possible POD.  What if there is not a US friendly government in Colombia.  The problem is that I
 cannot find a plausable way to bring a US hostile regime in Colombia.  Leftist movements have always been to small
 with the possible exception of the massacred Patriotic Union (above 15\% popular vote in their top in the middle of the
 80's) or the demovilized M-19 guerrillas (around 25\% in the constitutional assembly in 1991), however they were still
 too far behind from traditional parties, and not too US hostile to have trigger a US invasion if they happened to have
 win.

BTW, any POD that had a non-masacred Patriotic Union supporting M-19 presidential candidate in 1990 in the absent of
 Cesar Gaviria in representation of a martirized Luis Carlos Galán (most plausible scenario to have a left regime by
 democratic means), I dubt will put a US hostile left regime, not even close to Hugo Chavez in Venezuela.

I also cannot see a plausible way for a US hostile regime to had reached the power by the arms.

So, without a US hostile left that will nationalize oil, legalize drugs and pretend to outcompete Chiquita, I can only
 think on terrorism to have triggered a US invasion.

The two main sources of terrorism against the USA would have been the leftish rebel groups or the drug barons.  Even if
 both these groups had attacked US interests in Colombia, none of them have ever attacked US interests in the USA.  I
 cannot see the FARC nor any other rebel group to be too dumb to have bombed the Twin Towers, nor that fanatical into
 their ideology.

However, giving the modus operandi of people like Pablo Escobar or Gonzalo Rodriguez Gacha, its not hard to me to
 imagine they doing such things.  A dumb thing to do, still, but still not ASB driven.

POD: ``Los Extraditables'', the group formed by drug traffickers Pablo Escobar and Gonzalo Rodríguez Gacha to fight
 extradition to the USA, decide to perform more bold actions when Rodríguez Gacha is killed when persecuted by the
 Colombian Army in 1990.  In a coordinated attack, one ton of TNT is exploted from a truck infront of the US embassy
 and an American Airlines plane is exploted while flying from Bogotá to Miami, in May 1990.

Combined with three presidential candidates assasinated in Colombia, persistent car-bombing, the presedence of drugs as
 a casus belli (Panama), and the leftist rebels; president Bush launches an invasion.

However: Bush is invading a nation governed by a friendly regime, even if that regime seems unable to stop terrorism
 and drug trade.  Also, Bush will be invading just before the scheduled presidential elections and just after the
 parliamentary ones, so interrupting the electoral process of a democratic country.

Most of the Colombian military is very US friendly, so is President Barco.  They would not be willing to fight a US
 army that will help them to combat their enemies: leftist rebels and drug dealers, however they would not like either
 to be bypassed, nor Colombian souveranity violeted by the invaders.  They will face the disjuntive of surrender the
 souveranity of their nation or to fight superior army they actually want to join.

Now what.  What would be the strategy used for the USA to invade Colombia and control the guerrillas thread but mainly
 the drug related terrorism.

How would the international community react?  NATO?  UN?  While these drug dealers are a clear and present danger to
 the USA, they are not a state, nor are they harbored by any state, and the USA is even invading a democratic in the
 midle of the presidential election which is little of a justification.

Is there any more plausable scenario that would have triggered a US invasion on Colombia between 1970 and 2000?

— Carlos Th
[E-mail: if you want to contact me by email, replace my-deja for yahoo.]

\chapter{Los EE.UU. en Vietnam}
\begin{metadata}
	Published by \anchor[chlewey]{chlewey} on \anchor[http://ewey.co/B214]{Wed, 23 Jan 2002 19:05:12 +0000}\\
	\categories{usenet, estados-unidos, information, vietnam}\\
	Shorthand: \anchor[http://blog.chlewey.net/2002/01/los-ee-uu-en-vietnam/]{los-ee-uu-en-vietnam}
\end{metadata}

\begin{address}
\anchor[http://groups.google.com/group/soc.culture.colombia/msg/5e50c42758db7a2d]{news:3C4F0968.B85C445C@my-deja.com}
\end{address}
Hace poco salía a colación aquí las razones por las cuales EE.UU. perdió en Vietnam a pesar de su superioridad militar.

Consultando en otros grupos de noticias he aquí algunos de los elementos que contribuyeron a que los EE.UU. salieran de
 Vietnam con el rabo entre las piernas.

1) Los EE.UU. no fue el actor primario de esa guerra.  La guerra no era una guerra de los EE.UU. contra Vietnam del
 Norte o contra la Unión Soviética.  Ni siquiera era propiamente una guerra de los EE.UU. contra el comunismo.  Fue,
 básicamente, una guerra de Vietnam del Sur contra Vietnam del Norte y el Vietcong, donde los EE.UU. y Australia eran
 un apoyo a Vietnam del Sur.  Esto plantea, de por sí, un enfoque de la guerra diferente al de una guerra abierta.

2) Los estrateguas y generales de los EE.UU. estaban todos en Washington, tratando de controlar todo desde el
 Pentágono.  Westmoreland no era el general más apropiado para conducir el tipo de guerra en la que degeneró Vietnam.
 Por lo demás eran generales que no creían en la capacidad de sus propias tropas.

3) Como el objetivo no era Vietnam del Norte, la estrategia fue una estrategia de contensión.  Los ataques a Vietnam
 del Norte se redujeron a bombardeos estratégicos, bloqueos y un par de operaciones élite bien ejecutadas pero que
 contribuyeron poco al esfuerzo de la guerra.

4) La estrategia contra el Vietcong y las ocacionales tropas norvietnamitas en el sur, consistían usualmente en ir a
 una aldea, limpiarla de guerrilleros y norvietnamitas y salir.  Al poco tiempo el Vietcong regresaba a la aldea así
 que el ciclo tenía que repetirse. Esto contribuía muy poco a la moral de la tropa gringa que tenía que limpiar la
 misma aldea varias veces.     En los sitios donde peleaban los australianos o los survietnamitas había menos
 guerrilleros del vietcong.

Bueno, y hay muchas más causas que contribuyeron.

Lo interesante sería ver, que experiencia se puede concluir de Vietnam dentro del escenario colombiano.

— Carlos Th
[E-mail: if you want to contact me by email, replace my-deja for yahoo.]

\chapter{El derecho de la guerra}
\begin{metadata}
	Published by \anchor[chlewey]{chlewey} on \anchor[http://ewey.co/B217]{Wed, 06 Feb 2002 20:43:10 +0000}\\
	\categories{usenet, information}\\
	Shorthand: \anchor[http://blog.chlewey.net/2002/02/el-derecho-de-la-guerra/]{el-derecho-de-la-guerra}
\end{metadata}

\begin{address}
\anchor[http://groups.google.com/group/soc.culture.colombia/msg/a931f94dea921260]{news:3C61955E.26D6CDF3@my-deja.com}
\end{address}
En el sitio del Comité Internacional de la cruz roja aparece una serie de documentos, entre ellos un cuadernillo donde
 se explica cual es el derecho de la guerra y por qué este existe.

El derecho de la guerra parte de la premisa de que la guerra existe y va a existir y define cómo debe ser la relación
 de las partes y de los no combatientes durante un conflicto bélico.

Parte importante del derecho de la guerra son los Convenios de Ginebra y el derecho de La Haya.

Este vínculo me parece importante para comenzar:

\begin{address}
\anchor[http://www.icrc.org/icrcspa.nsf/22615d8045206c9b41256559002f7de4/8f3bded1e8d112fa412566c500378b60?OpenDocument]{http://www.icrc.org/icrcspa.nsf/22615d8045206c9b41256559002f7de4/8f3bded1e8d112fa412566c500378b60?OpenDocument}
\end{address}
Y en particular copio el texto del Protocolo II:

\begin{address}
\anchor[http://www.icrc.org/icrcspa.nsf/22615d8045206c9b41256559002f7de4/847e60eb673a8a36412565d7003c99a5?OpenDocument]{http://www.icrc.org/icrcspa.nsf/22615d8045206c9b41256559002f7de4/847e60eb673a8a36412565d7003c99a5?OpenDocument}
\end{address}

\par% div% {'style': 'background: #eee; color: #009;'}

\subsection{Protocolo adicional a los Convenios de Ginebra del 12 de agosto de 1949 relativo a la protección de las víctimas de los
 conflictos armados sin carácter internacional (Protocolo II)}
ÍNDICE

PREÁMBULO Comentario del Preámbulo
TÍTULO I - ÁMBITO DEL PRESENTE PROTOCOLO Comentario del Título I
Artículo 1 - Ámbito
 de aplicación material
Artículo 2 - Ámbito de aplicación personal
Artículo 3 - No intervención

TÍTULO II - TRATO HUMANO Comentario del Título II
Artículo 4 - Garantías fundamentales
Artículo 5 - Personas privadas
 de libertad
Artículo 6 - Diligencias penales

TÍTULO III - HERIDOS, ENFERMOS Y NÁUFRAGOS Comentario del Título III
Artículo 7 - Protección y asistencia
Artículo 8 -
 Búsqueda
Artículo 9 - Protección del personal sanitario y religioso
Artículo 10 - Protección general de la misión
 médica
Artículo 11 - Protección de unidades y medios de transporte sanitarios
Artículo 12 - Signo distintivo

TÍTULO IV - POBLACIÓN CIVIL Comentario del Título IV
Artículo 13 - Protección de la población civil
Artículo 14 -
 Protección de los bienes indispensables para la supervivencia de la población civil
Artículo 15 - Protección de las
 obras e instalaciones que contienen fuerzas peligrosas
Artículo 16 - Protección de los bienes culturales y de los
 lugares de culto
Artículo 17 - Prohibición de los desplazamientos forzados
Artículo 18 - Sociedades de socorro y
 acciones de socorro

TÍTULO V - DISPOSICIONES FINALES Comentario del Título V
Artículo 19 - Difusión
Artículo 20 - Firma
Artículo 21 -
 Ratificación
Artículo 22 - Adhesión
Artículo 23 - Entrada en vigor
Artículo 24 - Enmiendas
Artículo 25 -
 Denuncia
Artículo 26 - Notificaciones
Artículo 27 - Registro
Artículo 28 - Textos auténticos

*******

PREÁMBULO
Comentario

Las Altas Partes Contratantes,
Recordando que los principios humanitarios refrendados por el artículo 3 común a los
 Convenios de Ginebra del 12 de agosto de 1949 constituyen el fundamento del respeto a la persona humana en caso de
 conflicto armado sin carácter internacional,

Recordando, asimismo, que los instrumentos internacionales relativos a los derechos humanos ofrecen a la persona humana
 una protección fundamental,

Subrayando la necesidad de garantizar una mejor protección a las víctimas de tales conflictos armados,

Recordando que, en los casos no previstos por el derecho vigente, la persona humana queda bajo la salvaguardia de los
 principios de humanidad y de las exigencias de la conciencia pública,

Convienen en lo siguiente:

TÍTULO I - ÁMBITO DEL PRESENTE PROTOCOLO
Comentario

Artículo 1. Ámbito de aplicación material
1. El presente Protocolo, que desarrolla y completa el artículo 3 común a los
 Convenios de Ginebra del 12 de agosto de 1949, sin modificar sus actuales condiciones de aplicación, se aplicará a
 todos los conflictos armados que no estén cubiertos por el artículo 1 del Protocolo adicional a los Convenios de
 Ginebra del 12 de agosto de 1949 relativo a la protección de las víctimas de los conflictos armados internacionales
 (Protocolo I) y que se desarrollen en el territorio de una Alta Parte contratante entre sus fuerzas armadas y fuerzas
 armadas disidentes o grupos armados organizados que, bajo la dirección de un mando responsable, ejerzan sobre una
 parte de dicho territorio un control tal que les permita realizar operaciones militares sostenidas y concertadas y
 aplicar el presente Protocolo.

2. El presente Protocolo no se aplicará a las situaciones de tensiones internas y de disturbios interiores, tales como
 los motines, los actos esporádicos y aislados de violencia y otros actos análogos, que no son conflictos armados.

Artículo 2. Ámbito de aplicación personal

1. El presente Protocolo se aplicará sin ninguna distinción de carácter desfavorable por motivos de raza, color, sexo,
 idioma, religión o creencia, opiniones políticas o de otra índole, origen nacional o social, fortuna, nacimiento u
 otra condición o cualquier otro criterio análogo (denominada en adelante distinción de carácter desfavorable ), a
 todas las personas afectadas por un conflicto armado en el sentido del artículo 1.

2. Al fin del conflicto armado, todas las personas que hayan sido objeto de una privación o de una restricción de
 libertad por motivos relacionados con aquél, así como las que fuesen objeto de tales medidas después del conflicto por
 los mismos motivos, gozarán de la protección prevista en los artículos 5 y 6 hasta el término de esa privación o
 restricción de libertad.

Artículo 3. No intervención

1. No podrá invocarse disposición alguna del presente Protocolo con objeto de menoscabar la soberanía de un Estado o la
 responsabilidad que incumbe al gobierno de mantener o restablecer la ley y el orden en el Estado o de defender la
 unidad nacional y la integridad territorial del Estado por todos los medios legítimos.

2. No podrá invocarse disposición alguna del presente Protocolo como justificación para intervenir, directa o
 indirectamente, sea cual fuere la razón, en el conflicto armado o en los asuntos internos o externos de la Alta Parte
 contratante en cuyo territorio tenga lugar ese conflicto.

TÍTULO II - TRATO HUMANO
Comentario

Artículo 4. Garantías fundamentales
1. Todas las personas que no participen directamente en las hostilidades, o que
 hayan dejado de participar en ellas, estén o no privadas de libertad, tienen derecho a que se respeten su persona, su
 honor, sus convicciones y sus prácticas religiosas. Serán tratadas con humanidad en toda circunstancia, sin ninguna
 distinción de carácter desfavorable. Queda prohibido ordenar que no haya supervivientes.

2. Sin perjuicio del carácter general de las disposiciones que preceden, están y quedarán prohibidos en todo tiempo y
 lugar con respecto a las personas a que se refiere el párrafo 1:

a) los atentados contra la vida, la salud y la integridad física o mental de las personas, en particular el homicidio y
 los tratos crueles tales como la tortura y las mutilaciones o toda forma de pena corporal;

b) los castigos colectivos;

c) la toma de rehenes;

d) los actos de terrorismo;

e) los atentados contra la dignidad personal, en especial los tratos humillantes y degradantes, la violación, la
 prostitución forzada y cualquier forma de atentado al pudor;

f) la esclavitud y la trata de esclavos en todas sus formas;

g) el pillaje;

h) las amenazas de realizar los actos mencionados.

3. Se proporcionarán a los niños los cuidados y la ayuda que necesiten y, en particular:

a) recibirán una educación, incluida la educación religiosa o moral, conforme a los deseos de los padres o, a falta de
 éstos, de las personas que tengan la guarda de ellos;

b) se tomarán las medidas oportunas para facilitar la reunión de las familias temporalmente separadas;

c) los niños menores de quince años no serán reclutados en las fuerzas o grupos armados y no se permitirá que
 participen en las hostilidades;

d) la protección especial prevista en este artículo para los niños menores de quince años seguirá aplicándose a ellos
 si, no obstante las disposiciones del apartado c), han participado directamente en las hostilidades y han sido
 capturados;

e) se tomarán medidas, si procede, y siempre que sea posible con el consentimiento de los padres o de las personas que,
 en virtud de la ley o la costumbre, tengan en primer lugar la guarda de ellos, para trasladar temporalmente a los
 niños de la zona en que tengan lugar las hostilidades a una zona del país más segura y para que vayan acompañados de
 personas que velen por su seguridad y bienestar.

Artículo 5. Personas privadas de libertad

1. Además de las disposiciones del artículo 4, se respetarán, como mínimo, en lo que se refiere a las personas privadas
 de libertad por motivos relacionados con el conflicto armado, ya estén internadas o detenidas, las siguientes
 disposiciones:

a) los heridos y enfermos serán tratados de conformidad con el artículo 7;

b) las personas a que se refiere el presente párrafo recibirán, en la misma medida que la población local, alimentos y
 agua potable y disfrutarán de garantías de salubridad e higiene y de protección contra los rigores del clima y los
 peligros del conflicto armado;

c) serán autorizadas a recibir socorros individuales o colectivos;

d) podrán practicar su religión y, cuando así lo soliciten y proceda, recibir la asistencia espiritual de personas que
 ejerzan funciones religiosas, tales como los capellanes;

e) en caso de que deban trabajar, gozarán de condiciones de trabajo y garantías análogas a aquellas de que disfrute la
 población civil local.

2. En la medida de sus posibilidades, los responsables del internamiento o la detención de las personas a que se
 refiere el párrafo 1 respetarán también, dentro de los límites de su competencia, las disposiciones siguientes
 relativas a esas personas:

a) salvo cuando hombres y mujeres de una misma familia sean alojados en común, las mujeres estarán custodiadas en
 locales distintos de los destinados a los hombres y se hallarán bajo la vigilancia inmediata de mujeres;

b) dichas personas serán autorizadas para enviar y recibir cartas y tarjetas postales, si bien su número podrá ser
 limitado por la autoridad competente si lo considera necesario;

c) los lugares de internamiento y detención no deberán situarse en la proximidad de la zona de combate. Las personas a
 que se refiere el párrafo 1 serán evacuadas cuando los lugares de internamiento o detención queden particularmente
 expuestos a los peligros resultantes del conflicto armado, siempre que su evacuación pueda efectuarse en condiciones
 suficientes de seguridad;

d) dichas personas serán objeto de exámenes médicos;

e) no se pondrán en peligro su salud ni su integridad física o mental, mediante ninguna acción u omisión
 injustificadas. Por consiguiente, se prohibe someter a las personas a que se refiere el presente artículo a cualquier
 intervención médica que no esté indicada por su estado de salud y que no esté de acuerdo con las normas médicas
 generalmente reconocidas que se aplicarían en análogas circunstancias médicas a las personas no privadas de libertad.

3. Las personas que no estén comprendidas en las disposiciones del párrafo 1 pero cuya libertad se encuentre
 restringida, en cualquier forma que sea, por motivos relacionados con el conflicto armado, serán tratadas humanamente
 conforme a lo dispuesto en el artículo 4 y en los párrafos 1 a), c) y d) y 2 b) del presente artículo.

4. Si se decide liberar a personas que estén privadas de libertad, quienes lo decidan deberán tomar las medidas
 necesarias para garantizar la seguridad de tales personas.

Artículo 6. Diligencias penales

1. El presente artículo se aplicará al enjuiciamiento y a la sanción de infracciones penales cometidas en relación con
 el conflicto armado.

2. No se impondrá condena ni se ejecutará pena alguna respecto de una persona declarada culpable de una infracción,
 sino en virtud de sentencia de un tribunal que ofrezca las garantías esenciales de independencia e imparcialidad. En
 particular:

a) el procedimiento dispondrá que el acusado sea informado sin demora de los detalles de la infracción que se le
 atribuya y garantizará al acusado, en las actuaciones que procedan al juicio y en el curso de éste, todos los derechos
 y medios de defensa necesarios;

b) nadie podrá ser condenado por una infracción si no es sobre la base de su responsabilidad penal individual;

c) nadie será condenado por actos u omisiones que en el momento de cometerse no fueran delictivos según el derecho;
 tampoco se impondrá pena más grave que la aplicable en el momento de cometerse la infracción; si, con posterioridad a
 la comisión de la infracción, la ley dispusiera la imposición de una pena más leve, el delincuente se beneficiará de
 ello;

d) toda persona acusada de una infracción se presumirá inocente mientras no se pruebe su culpabilidad conforme a la ley;

e) toda persona acusada de una infracción tendrá derecho a hallarse presente al ser juzgada;

f) nadie podrá ser obligado a declarar contra sí mismo ni a confesarse culpable.

3. Toda persona condenada será informada, en el momento de su condena, de sus derechos a interponer recurso judicial y
 de otro tipo, así como de los plazos para ejercer esos derechos.

4. No se dictará pena de muerte contra las personas que tuvieren menos de 18 años de edad en el momento de la
 infracción ni se ejecutará en las mujeres encintas ni en las madres de niños de corta edad.

5. A la cesación de las hostilidades, las autoridades en el poder procurarán conceder la amnistía más amplia posible a
 las personas que hayan tomado parte en el conflicto armado o que se encuentren privadas de libertad, internadas o
 detenidas por motivos relacionados con el conflicto armado.

TÍTULO III - HERIDOS, ENFERMOS Y NÁUFRAGOS
Comentario

Artículo 7. Protección y asistencia

1. Todos los heridos, enfermos y náufragos, hayan o no tomado parte en el conflicto armado, serán respetados y
 protegidos.

2. En toda circunstancia serán tratados humanamente y recibirán, en toda la medida de lo posible y en el plazo más
 breve, los cuidados médicos que exija su estado. No se hará entre ellos distinción alguna que no esté basada en
 criterios médicos.

Artículo 8. Búsqueda

Siempre que las circunstancias lo permitan, y en particular después de un combate, se tomarán sin demora todas las
 medidas posibles para buscar y recoger a los heridos, enfermos y náufragos a fin de protegerlos contra el pillaje y
 los malos tratos y asegurarles la asistencia necesaria, y para buscar a los muertos, impedir que sean despojados y dar
 destino decoroso a sus restos.

Artículo 9. Protección del personal sanitario y religioso

1. El personal sanitario y religioso será respetado y protegido. Se le proporcionará toda la ayuda disponible para el
 desempeño de sus funciones y no se le obligará a realizar tareas que no sean compatibles con su misión humanitaria.

2. No se podrá exigir que el personal sanitario, en el cumplimiento de su misión, dé prioridad al tratamiento de
 persona alguna salvo por razones de orden médico.

Artículo 10. Protección general de la misión médica

1. No se castigará a nadie por haber ejercido una actividad médica conforme con la deontología, cualesquiera que
 hubieren sido las circunstancias o los beneficiarios de dicha actividad.

2. No se podrá obligar a las personas que ejerzan una actividad médica a realizar actos ni a efectuar trabajos
 contrarios a la deontología u otras normas médicas destinadas a proteger a los heridos y a los enfermos, o a las
 disposiciones del presente Protocolo, ni a abstenerse de realizar actos exigidos por dichas normas o disposiciones.

3. A reserva de lo dispuesto en la legislación nacional, se respetarán las obligaciones profesionales de las personas
 que ejerzan una actividad médica, en cuanto a la información que puedan adquirir sobre los heridos y los enfermos por
 ellas asistidos.

4. A reserva de lo dispuesto en la legislación nacional, la persona que ejerza una actividad médica no podrá ser
 sancionada de modo alguno por el hecho de no proporcionar o de negarse a proporcionar información sobre los heridos y
 los enfermos a quienes asista o haya asistido.

Artículo 11. Protección de unidades y medios de transporte sanitarios

1. Las unidades sanitarias y los medios de transporte sanitarios serán respetados y protegidos en todo momento y no
 serán objeto de ataques.

2. La protección debida a las unidades y a los medios de transporte sanitarios solamente podrá cesar cuando se haga uso
 de ellos con objeto de realizar actos hostiles al margen de sus tareas humanitarias. Sin embargo, la protección cesará
 únicamente después de una intimación que, habiendo fijado cuando proceda un plazo razonable, no surta efectos.

Artículo 12. Signo distintivo

Bajo la dirección de la autoridad competente de que se trate, el signo distintivo de la cruz roja, de la media luna
 roja o del león y sol rojos sobre fondo blanco será ostentado tanto por el personal sanitario y religioso como por las
 unidades y los medios de transporte sanitarios. Dicho signo deberá respetarse en toda circunstancia. No deberá ser
 utilizado indebidamente.

TÍTULO IV - POBLACIÓN CIVIL
Comentario

Artículo 13. Protección de la población civil

1. La población civil y las personas civiles gozarán de protección general contra los peligros procedentes de
 operaciones militares. Para hacer efectiva esta protección, se observarán en todas las circunstancias las normas
 siguientes.

2. No serán objeto de ataque la población civil como tal, ni las personas civiles. Quedan prohibidos los actos o
 amenazas de violencia cuya finalidad principal sea aterrorizar a la población civil.

3. Las personas civiles gozarán de la protección que confiere este Título, salvo si participan directamente en las
 hostilidades y mientras dure tal participación.

Artículo 14. Protección de los bienes indispensables para la supervivencia de la población civil

Queda prohibido, como método de combate, hacer padecer hambre a las personas civiles. En consecuencia, se prohibe
 atacar, destruir, sustraer o inutilizar con ese fin los bienes indispensables para la supervivencia de la población
 civil, tales como los artículos alimenticios y las zonas agrícolas que los producen, las cosechas, el ganado, las
 instalaciones y reservas de agua potable y las obras de riego.

Artículo 15. Protección de las obras e instalaciones que contienen fuerzas peligrosas

Las obras o instalaciones que contienen fuerzas peligrosas, a saber las presas, los diques y las centrales nucleares de
 energía eléctrica, no serán objeto de ataques, aunque sean objetivos militares, cuando tales ataques puedan producir
 la liberación de aquellas fuerzas y causar, en consecuencia, pérdidas importantes en la población civil.

Artículo 16. Protección de los bienes culturales y de los lugares de culto

Sin perjuicio de las disposiciones de la Convención de La Haya del 14 de mayo de 1954 para la Protección de los Bienes
 Culturales en caso de Conflicto Armado, queda prohibido cometer actos de hostilidad dirigidos contra los monumentos
 históricos, las obras de arte o los lugares de culto que constituyen el patrimonio cultural o espiritual de los
 pueblos, y utilizarlos en apoyo del esfuerzo militar.

Artículo 17. Prohibición de los desplazamientos forzados

1. No se podrá ordenar el desplazamiento de la población civil por razones relacionadas con el conflicto, a no ser que
 así lo exijan la seguridad de las personas civiles o razones militares imperiosas. Si tal desplazamiento tuviera que
 efectuarse, se tomarán todas las medidas posibles para que la población civil sea acogida en condiciones
 satisfactorias de alojamiento, salubridad, higiene, seguridad y alimentación.

2. No se podrá forzar a las personas civiles a abandonar su propio territorio por razones relacionadas con el conflicto.

Artículo 18. Sociedades de socorro y acciones de socorro

1. Las sociedades de socorro establecidas en el territorio de la Alta Parte contratante, tales como las organizaciones
 de la Cruz Roja (Media Luna Roja, León y Sol Rojos), podrán ofrecer sus servicios para el desempeño de sus funciones
 tradicionales en relación con las víctimas del conflicto armado. La población civil puede, incluso por propia
 iniciativa, ofrecerse para recoger y cuidar los heridos, enfermos y náufragos.

2. Cuando la población civil esté padeciendo privaciones extremadas por la falta de abastecimientos indispensables para
 su supervivencia, tales como víveres y suministros sanitarios, se emprenderán, con el consentimiento de la Alta Parte
 contratante interesada, acciones de socorro en favor de la población civil, de carácter exclusivamente humanitario e
 imparcial y realizadas sin distinción alguna de carácter desfavorable.

TÍTULO V - DISPOSICIONES FINALES
Comentario

Artículo 19. Difusión
El presente Protocolo deberá difundirse lo más ampliamente posible.

Artículo 20. Firma

El presente Protocolo quedará abierto a la firma de las Partes en los Convenios seis meses después de la firma del Acta
 Final y seguirá abierto durante un período de doce meses.

Artículo 21. Ratificación

El presente Protocolo será ratificado lo antes posible. Los instrumentos de ratificación se depositarán en poder del
 Consejo Federal Suizo, depositario de los Convenios.

Artículo 22. Adhesión

El presente Protocolo quedará abierto a la adhesión de toda Parte en los Convenios no signataria de este Protocolo. Los
 instrumentos de adhesión se depositarán en poder del depositario.

Artículo 23. Entrada en vigor

1. El presente Protocolo entrará en vigor seis meses después de que se hayan depositado dos instrumentos de
 ratificación o de adhesión.

2. Para cada Parte en los Convenios que lo ratifique o que a él se adhiera ulteriormente, el presente Protocolo entrará
 en vigor seis meses después de que dicha Parte haya depositado su instrumento de ratificación o de adhesión.

Artículo 24. Enmiendas

1. Toda Alta Parte contratante podrá proponer una o varias enmiendas al presente Protocolo. El texto de cualquier
 enmienda propuesta se comunicará al depositario, el cual, tras celebrar consultas con todas las Altas Partes
 contratantes y con el Comité Internacional de la Cruz Roja, decidirá si conviene convocar una conferencia para
 examinar la enmienda propuesta.

2. El depositario invitará a esa conferencia a las Altas Partes contratantes y a las Partes en los Convenios, sean o no
 signatarias del presente Protocolo.

Artículo 25. Denuncia

1. En el caso de que una Alta Parte contratante denuncie el presente Protocolo, la denuncia sólo surtirá efecto seis
 meses después de haberse recibido el instrumento de denuncia. No obstante, si al expirar los seis meses la Parte
 denunciante se halla en la situación prevista en el artículo 1, la denuncia no surtirá efecto antes del fin del
 conflicto armado. Las personas que hayan sido objeto de una privación o de una restricción de libertad por motivos
 relacionados con ese conflicto seguirán no obstante beneficiándose de las disposiciones del presente Protocolo hasta
 su liberación definitiva.

2. La denuncia se notificará por escrito al depositario. Este último la comunicará a todas las Altas Partes
 contratantes.

Artículo 26. Notificaciones

El depositario informará a las Altas Partes contratantes y a las Partes en los Convenios, sean o no signatarias del
 presente Protocolo, sobre:

a) las firmas del presente Protocolo y el depósito de los instrumentos de ratificación y de adhesión, de conformidad
 con los artículos 21 y 22;
b) la fecha en que el presente Protocolo entre en vigor, de conformidad con el artículo 23;
 y

c) las comunicaciones y declaraciones recibidas de conformidad con el artículo 24.

Artículo 27. Registro

1. Una vez haya entrado en vigor el presente Protocolo, el depositario lo transmitirá a la Secretaría de las Naciones
 Unidas con objeto de que se proceda a su registro y publicación, de conformidad con el artículo 102 de la Carta de las
 Naciones Unidas.

2. El depositario informará igualmente a la Secretaría de las Naciones Unidas de todas las ratificaciones y adhesiones
 que reciba en relación con el presente Protocolo.

Artículo 28. Textos auténticos

El original del presente Protocolo, cuyos textos árabe, chino, español, francés, inglés y ruso son igualmente
 auténticos, se depositará en poder del depositario, el cual enviará copias certificadas conformes a todas las Partes
 en los Convenios.

\par% p

— Carlos Th
[E-mail: if you want to contact me by email, replace my-deja for yahoo.]

\chapter{Algunas ideas para mejorar el tránsito en Bogotá}
\begin{metadata}
	Published by \anchor[chlewey]{chlewey} on \anchor[http://ewey.co/B4]{Wed, 22 Aug 2007 09:40:00 +0000}\\
	\categories{movilidad, opinion, restriccion-vehicular}\\
	Shorthand: \anchor[http://blog.chlewey.net/2007/08/mejorar-el-transito/]{mejorar-el-transito}
\end{metadata}

Partamos de un principio: las personas utilizarán el medio de transporte que se les haga más cómodo bajo las
 limitaciones que tienen.

Para mí es más cómodo ir al trabajo en carro a pesar de los trancones, del costo de la gasolina, y de que tengo un bus
 casi puerta a puerta (en realidad como dos o tres rutas distintas). Para otras opciones de movilidad prefiero el taxi,
 el bus tradicional o Transmilenio, dependiendo de las circunstancias.

En mi ciudad ideal, yo preferiré dejar el carro en casa, no porque yo tenga una conciencia ecológica, o porque los
 costos del automóvil particular sean prohibitivos, sino porque consideraré más cómodo el transporte público colectivo
 o masivo que el carro particular.

Y ahí viene la forma de pensar la solución. La solución no consiste en más y mayores restricciones al carro particular.
 La solución consiste en generar más y mejores alternativas de movilidad.

Partamos así de algunos puntos específicos:

1. Un sistema de pago centralizado y unificado para todo el transporte público colectivo y masivo. Eliminar el uso del
 efectivo al subir a un bus o colectivo, o las colas en las taquillas de Transmilenio. Esto también elimina la guerra
 del centavo, los asaltos a las busetas para robarles el producido, etc.

1. bis. Con el sistema de pago centralizado, reorganizar completamente las rutas en esquemas de rutas locales y
 troncales, incluyendo a Transmilenio como una gran troncal. No tiene sentido que una buseta parta de un rincón de
 Suba, pase por Bulevar Niza, Unicentro, la 13, Chapinero, el Simón Bolívar y tras cuatro horas de camino termine en
 algún rincón de Kennedy. Con la tarifa centralizada, el precio al usuario no sería por el número de trasbordos sino
 por el desplazamiento final (o por algún otro esquema).

2. Mejorar y ampliar las vías destinadas al transporte colectivo, incluso a costa del transporte particular.
 Igualmente: mejorar la calidad de las vías, incluso aquellas destinadas al transporte particular. Todo esto sin
 detrimento de la ampliación general de la malla vial.

3. Dotar a la policía de tránsito de las herramientas técnicas y jurídicas que permitan sancionar a los infractores que
 atenten contra la movilidad. Revisar el códico de tránsito para aclarar las faltas que atentan contra la seguridad y
 las que atentan contra la movilidad (cualquier otra cosa es accesoria). Entre las faltas contra la movilidad se
 incluye parquear un camión de 2,5 m de ancho en una vía arteria estrecha (no es un oximoron en Bogotá) en horas pico.
 O parar una buseta a recoger pasajeros sobre la cebra. O esperar 15 minutos parqueado en el cupé familiar mientras la
 esposa hace una vuelta en el banco.

4. Eliminar el pico y placa u otras medidas artificiales destinadas a restringir la libertad de elegir.

El efecto inmediato de todas estas medidas será sin duda más trancones (un 60\% más de carros en las horas pico), pero
 aligerados por el hecho de que hay menos obstáculos (camiones repartidores, huecos, etc.), pero por otro lado se habrá
 mejorado enormemente el transporte colectivo (racionalización de rutas, menos obstáculos, incluyendo particulares y
 taxis dejando y recogiendo pasajeros). Finalmente muchos automovilistas aprenderemos a que es mejor dejar el carro en
 casa y usar un servicio público eficiente, pero que igual, el día en que necesitemos del transporte particular (p.ej.
 una emergencia médica, o transportar una maqueta), no debemos estar condicionados a restricciones artificiales tales
 como el pico y placa.

Y todo sin construir un metro o Transmilenios por la 7a.  El metro vendrá después como un necesario complemento.

\chapter{Placeres culposos y vicios sociales}
\begin{metadata}
	Published by \anchor[chlewey]{chlewey} on \anchor[http://ewey.co/B5]{Wed, 05 Sep 2007 13:11:00 +0000}\\
	\categories{usenet, facebook, hi5, personal}\\
	Shorthand: \anchor[http://blog.chlewey.net/2007/09/placeres-culposos/]{placeres-culposos}
\end{metadata}

Lo confieso: no fumo, no bebo, pero soy un adicto a Facebook.

\par% p
Llevo más de doce años metido en Internet y a lo largo de todo este tiempo he pasado por distintos vicios.  En algún
 momento me encarreté con las listas de correo y con Usenet.  El último par de años se los he dedicado a la \anchor[http://es.wikipedia.org/]{Wikipedia}.  Durante este tiempo Internet y sus servicios han sido una forma de contactar personas con las que comparto algún
 tipo de interés común, no todos plenamente compartidos por las personas de carne y hueso con las que me relaciono en
 la vida diaria.  Para una persona de personalidad introvertida como yo, al mismo tiempo que me aislaba de ciertas
 personas, me permitían desarrollar una actividad social que nunca hubiera alcanzado por medios tradicionales.

\par% p
Pero ninguno de esos vicios se compara al \anchor[http://www.facebook.com]{Facebook}.  El concepto de las redes sociales por Internet es muy interesante.  Es algo que ya venía sucediendo: los chats de
 las emisoras de radio, las listas de correo, Usenet, los blogs, etc. conectaban a personas de distintas partes del
 globo con algún interés común, y a personas conocidas por los medios tradicionales (el compañero del colegio, el
 vecino, el amigo del amigo) para mantener el contacto aún cuando la distancia física los separara.

\par% p
Luego vendrían los sitios dedicados a las redes sociales: \anchor[http://www.myspace.com]{Myspace}, \anchor[http://www.hi5.com]{Hi5}, \anchor[http://www.linkedin.com]{Linked In}, etc.

Estos sitios concentran el aspecto social derivado de los demás espacios.  Se basan en que cada persona tenga un
 espacio (un perfil) donde se presentan a ellos mismos, compartiendo más o menos información (fotos, biografía, hoja de
 vida, bitácoras, etc.) y permitiendo contactar y ser contactado, y permitiendo desarrollar los otros intereses,
 también.

Mi contagio comenzó con Hi5, pero Hi5 era un placer culposo.  El esquema fue pensado para los adolecentes de la era del
 Internet que querían hacer un montón de amigos virtuales y dentro de ese concepto, pasados mis treinta años, no me
 ofrecía mayores expectativas salvo mirar fotos de las amigas de los amigos.  No muy distinto a ver ``Así es la vida'' o
 ``El Gran Hermano'' en televisión, o a camuflar una torta de chocolate cuando debo convencer a mi doctor o a mi esposa
 que estoy cuidando mi dieta.

Pasé por Linked In.  Una red social con un concepto radicalmente diferente: buscar y mantener los contactos
 profesionales con el objetivo de buscar oportunidades de negocio o de empleo.  Las fotos de las fiestas y de los
 paseos se reemplazan por hojas de vida.  El primer nombre o el apodo se reemplaza por el nombre completo.  Es más
 serio... pero por ello mismo le falta el picante.  Linked In se puede manejar abiertamente, sin esconderlo de los
 jefes o de los amigos, pero por otro lado no envicia.

Las redes sociales pululan hoy en día.  Tanto escuché hablar de Myspace que me inscribí hace meses y nunca le encontré
 el chiste.  Mi red de amigos allá es nula.  Igual, con dos redes sociales de conceptos diferentes (hacer amigos /
 mantener contactos) todo era suficiente para mí, así que normalmente ignoraba todos los demás pedidos de entrar a
 otros sitios de contactos.

Hasta que caí en las garras del Facebook.  Supe que varios amigos habrían formado un grupo allá y entré con ese
 propósito específico... y no pude salir.  No he podido salir.  Facebook es un vicio social.  Es como dejarse atrapar
 del alcohol o del cigarrillo: nadie te va a juzgar por consumirlos, particularmente si no lo haces a deshoras, pero es
 un vicio.  Es un Hi5 sin el elemento culposo, es un Linked In con picante, es, en fin, el vicio social que nunca tuve.

Y como un buen bebedor o un fumador diría: no lo condeno.  No les advierto del problema.  Todo lo contrario, los invito
 a compartir este placer.

Venid a mí, sed mis amigos, compartid mi vicio.

\chapter{No me envíen invitaciones a causas por los derechos de los animales}
\begin{metadata}
	Published by \anchor[chlewey]{chlewey} on \anchor[http://ewey.co/B6]{Wed, 19 Sep 2007 21:43:00 +0000}\\
	\categories{derechos-de-los-animales, facebook, opinion, tauromaquia}\\
	Shorthand: \anchor[http://blog.chlewey.net/2007/09/no-me-envien/]{no-me-envien}
\end{metadata}

\par% p
Describiré la muerte del animal: primero sufre una serie de punzadas con púas de 6 cm de largo que desgarran la piel y
 del cual se cuelgan varias cargas de más de 100 kg c/u.  Cuando el animal no puede arrastrar más tales cargas, por
 medio de unas tenazas se le ejerce presión en el cuello hasta asfixiarlo.  Esta no es la descripción de una práctica
 de toreo sádico, sino la forma como un \anchor[http://www.youtube.com/watch?v=CkZ8kOMWTdI]{búfalo muere ante un ataque de una manada de leones}.  Ante esto las banderillas y la espada son una forma bastante \emph{humana} de morir.

¿Por qué como seres humanos debemos tener con la naturaleza consideraciones que la naturaleza no tiene con ella
 misma?
Tal vez... porque podemos.  Si queremos tratar mejor a los animales de lo que ellos se tratan a si mismos no es
 por que ellos sientan igual que nosotros o sean en derecho iguales a nosotros, sino porque nosotros somos moralmente
 superiores para decidir darles un trato más humano.

Pero así como es nuestra elección decidir el trato que les damos, es nuestra elección también dar prioridades a
 nuestras causas y a nuestra acción.  Yo como carne por que me gusta.  Tomo leche, uso cuero, disfruto la miel.  Fumigo
 mi casa si esta se infecta de cucarachas.  Simplemente he decidido que uso de los animales si estos me son útiles y
 cómodos simplemente porque puedo hacerlo.  También lo hago con los seres humanos.  Y a diferencia de con los seres
 humanos, no creo que los animales sean sujetos de derecho.

¿Soy especista?  Bonito término que se inventaron para equiparar el desprecio a los animales de otras especies con el
 desprecio de otras épocas y otras culturas (y no tan otras) a otras razas de seres humanos o a otras nacionalidades,
 sexos o grupos de edad.  Pero sí, creo que es un imperativo moral pensar antes en los seres humanos que en otras
 especies, y si eso me hace especista lo soy.

\par% p
Así que no me envíen más invitaciones por la causa de los \anchor[http://apps.facebook.com/causes/view\_cause/293]{derechos de los animales} o en \anchor[http://apps.facebook.com/causes/view\_cause/11617]{contra de las corridas de toros}.  Soy un especista irredento que no se siente culpable de decirle ``ignorar'' a esas invitaciones.

\chapter{Propositario}
\begin{metadata}
	Published by \anchor[chlewey]{chlewey} on \anchor[http://ewey.co/B22]{Wed, 04 Jan 2006 13:13:00 +0000}\\
	\categories{personal}\\
	Shorthand: \anchor[http://blog.chlewey.net/2006/01/propositario/]{propositario}
\end{metadata}

Inicia un nuevo año y sin duda inicia con él los nuevos propósitos con los que habría de orientar mi vida. Algo muy
 bueno quedó de 2005: Sebastián David, aunque en general no fue un buen año. Laboral y  profesionalmente fue un año
 perdido y aunque no faltó trabajo, de estos no quedaron mayores perspectivas.

2006 ya conezó y ya han transcurrido tres días del mismo. Bueno, el primero igual no cuenta, ya que además de festivo
 coincidió en domingo, y el lunes y martes estuve en una especie de vacaciones; necesarias en parte aunque no del todo
 merecidas.

\par% p
Así que aquí viene una lista, no del todo exhaustiva, de los propósitos para este año (y para el resto de mi vida)

\begin{enumerate}

\item Orientar o reorientar mi carrera. Esto es plantearme si sigo buscando un empleo de acuerdo al perfil que me he estado
 formando: telecomunicaciones de datos, con la certificación Cisco, o plantearme nuevos rumbos:
\begin{itemize}

\item Terminar mi postgrado en matemáticas y orientar a una tesis de matemáticas puras que no tenga nada que ver con la
 ingeniería
\item Terminar mi postgrado en matemáticas y orientar a una tesis relacionada con la ingeniería
\item Hacer un postgrado en un área afín de la ingeniería (sistemas, telecomunicaciones o electrónica)
\item Orientarme a la docencia, claro que esto puede bien implicar terminar un postgrado primero
\item Buscar un empleo en algo diferente pero que me permita crecer en un área afín
\item Irme a aventurar por Australia o Canadá (o donde me reciban)

\end{itemize}

Es claro que cualquier decisión que implique estudiar o irme debe estar tomada para el segundo semestre de 2006 o para
 el primero de 2007, y que lo primero que debo hacer es culminar mi certificación de Cisco.
\item Definir mis prioridades familiares. Sebastián David necesita de mi tiempo y de mi dedicación, pero podría ser mejor
 para él buscar una buena actividad profesional por fuera en lugar de quedarme a su lado sin progresar (y sin dedicarle
 tiempo, siquiera)
\item Tomarme con juicio mis wikivacaciones y, cuando tenga definidas mis prioridades profesionales y familiares regresar a
 la misma y con propósitos bien definidos:
\begin{itemize}

\item Arrancar dij Schdzjarvk para los propósitos expuestos: generar y homogenizar la presentación de los artículos
 relacionados con los wikiproyectos en los que participo. (sub propósito: aprender bien Python)
\item Honrar mi dignidad de bibliotecario buscando mecanismos para mejorar Wikipedia y participando en las labores de
 mantenimiento (con ayuda de chlewbot y personalmente)
\item Seguir ampliando la información en los wikiproyectos que he arrancado o con los que me he comprometido. Si es
 necesario, replantear los wikiproyectos para que sean manejables.

\end{itemize}

\item Arrancar finalmente mi página personal, (desde luego, una vez tenga definidas las prioridades profesionales y
 familiares).

\end{enumerate}

Releyendo los propósitos expuestos me doy cuenta que no hay acciones concretas. Lo primero que tengo que hacer es
 tomar desiciones y muchas desiciones que puedan concretarse en acciones específicas. ¿estudiar o no estudiar?
 ¿emplearme en lo primero que me ofrezcan o plantear una carrera? ¿criar a mi hijo o aportar los medios para su
 desarrollo? (este último «o» no es necesariamente exclusivo) ¿irme o quedarme?

Queda mucho por hacer. Mucho por decidir. La primera acción concreta ha de ser tomar estas desiciones y me planteo una
 fecha para haber tomado desiciones trasendentales: 16 de marzo de 2006. Que sea el primer año de mi hijo la
 celebración también de que he decidido hacer algo con mi vida.

\chapter{Conocimiento libre}
\begin{metadata}
	Published by \anchor[chlewey]{chlewey} on \anchor[http://ewey.co/B23]{Wed, 16 Nov 2005 23:16:00 +0000}\\
	\categories{information, libertad, wikimedia, wikipedia}\\
	Shorthand: \anchor[http://blog.chlewey.net/2005/11/conocimiento-libre/]{conocimiento-libre}
\end{metadata}

\par% p
Sin duda muchos de los que colaboran actualmente en \anchor[http://es.wikipedia.org/]{Wikipedia} lo hacen por el amor al conocimiento libre.  Otros muchos lo hacen por vanidad.  En mi caso particular es más por
 afición o, como dicen por ahí, por amor al arte.

Pero en el proceso me he vuelto medio adicto al conocimiento libre y al legalismo que hay detrás del mismo.  El
 conocimiento libre contrasta enormemente con la libertad de opinión, aunque ambos son pilares de un sistema que,
 muchos de los defensores de la libertad aborrecen: el capitalismo.

Bueno, no estrictamente.  El capitalismo se basa en la acumulación de capital y el conocimiento es parte de ese
 capital.  En esto, la idea del conocimiento libre puede llegar a ser revolucionaria en este sentido.  Pero el
 conocimiento libre, tal como lo pregona wikimedia, no se trata de subvertir al capitalismo y a obligarlo de despojarse
 de sus ataduras, sino que se construye al lado del sistema... de este y de cualquier otro sistema, incluyendo sistemas
 antítesis de lo libre.

La libertad en este caso, no se basa en el uso que yo haga de la propiedad de los demás sino en el uso que yo hago de
 mi propio capital.  Es, por decirlo así, mi libertad de gastar, o más exactamente, mi libertad de invertir.

\par% p
Cuando yo creo una imagen y la subo en \anchor[http://commons.wikimedia.org]{commons}, o cuando escribo un artículo y lo publico en Wikipedia, estoy invirtiendo parte de mi capital en la comunidad.  Es
 una donación de lo mío en los demás, pero es una donación que doy en libertad y no una donación que me exige el
 sistema: no es un impuesto.

Por otro lado, la libertad de expresión y la libertad de opinión es algo que yo impongo a los demás; particularmente
 cuando exijo que se me escuche.  Cuando exijo que mi contribución quede como yo la escribí sin que me sea corregida.
 Cuando exijo que el formato sea como yo lo impongo.  Mi expresión libre riñe con mi libertad de donar mis
 conocimientos en cierta forma; y particularmente con la libertad que quiero dar a los demás de usar mi capital para
 construir a partir de ahí más capital libre para todos.

\chapter{Nuestras tradiciones}
\begin{metadata}
	Published by \anchor[chlewey]{chlewey} on \anchor[http://ewey.co/B24]{Wed, 09 Nov 2005 02:26:00 +0000}\\
	\categories{opinion, tlc}\\
	Shorthand: \anchor[http://blog.chlewey.net/2005/11/nuestras-tradiciones/]{nuestras-tradiciones}
\end{metadata}

88.9 es ahora Radio Uno...

La HJCK será ahora Las 40 Principales, para que las 40 sean ahora La Vallenata...

El fin de semana pasado me encontraba ante una pregunta algo paradójico; por un lado siempre me ha parecido algo
 admirable cómo una música colombiana, como lo es el vallenato, no sólo compite con música extranjera (p.ej. el rock)
 sino que también la exportamos... pero por otro lado ¿por qué el vallenato?  ¿por qué no el porro o el pasillo?
 Luego, ese lunes me entero del fallecimiento de Emiliano Zuleta y no puedo dejar de pensar que no sólo se trata de que
 sea el vallenato la música que triunfa en mi país... sino que además se trata es del vallenato de Diomedes Díaz...

No soy muy amigo del vallenato, partamos por ahí.  Mi gusto musical se acerca más al rock (en español o en inglés... o
 en el idioma que sea) que a los ritmos tropicales.  Por otro lado, la HJCK nunca la escuchaba.  Hace años que 88.9 no
 estaba entre mis emisoras favoritas.  Lo que me parece lamentable es, sin embargo, esa cultura uniformadora: ``como el
 vallenato es lo que más escuchan los colombianos, pues sacrifiquemos la música de la inmensa minoría y pongamos más
 vallenato en la FM.''

Es lo mismo que ha pasado con el yogurt y las gaseosas.  Es casi imposible conseguir en un supermercado un yogurt
 grande que no sea de fresa, mora o melocotón... a duras penas kiwi o guanábana, pero estas no en Alpina.  La Fanta
 Naranja, la Pera Postobón o la Link Limón son ya imposibles de conseguir... y si lo que quiero es lata, Panamco sólo
 me ofrece Coca Cola o Coca Cola light.  Todo se uniforma y el mercado ``mainstream'' no ofrece alternativas para las
 inmensas minorías que podemos querer un yogurt de café, una lata de Fanta (naranja) o escuchar la HJCK desde el radio
 del carro (un radio FM convencional, no un radio satelital IP).

Y luego nos preocupa que el TLC, con su cuota de pantalla, vaya a arruinar a nuestra TV nacional.  Como si con el 35\%
 de televisión extranjera que hoy se puede transmitir en el horario Triple A (o Prime Time), no fuera suficiente para
 tener 100\% de telenovelas y realities nacionales y 0\% de buenas series o películas extranjeras (o malas: 0\% es 0\%).

Si Pablo Laserna, presidente del Canal Caracol, es el primero en protestar por el TLC, lo que veo es que él es el
 primero en querer cambiar nuestro uniforme de telenovelas y realities nacionales para meter también viejas películas
 holliwoodenses...  porque ya estamos uniformados.

\chapter{Fascismo y lo que no nos gusta}
\begin{metadata}
	Published by \anchor[chlewey]{chlewey} on \anchor[http://ewey.co/B25]{Thu, 06 Oct 2005 17:05:00 +0000}\\
	\categories{derecha, izquierda, opinion, tlc}\\
	Shorthand: \anchor[http://blog.chlewey.net/2005/10/fascismo/]{fascismo}
\end{metadata}

Bien... vuelvo a encontrarme por estos días con aquella concepción que relaciona al TLC y sus proponentes colombianos,
 empezando por el gobierno, y el fascismo.

Si no fuera por el drama que hay detrás de todo el asunto sería hasta risible.

Para empezar, por el respeto de las víctimas del fascismo italiano y del nacional-socialismo alemán, junto con las
 víctimas de los demás regímenes verdaderamente fascistas no podemos, no debemos darnos el lujo de ir llamando fascista
 o nazi a todo aquello que no nos gusta.

Hay muchas cosas malas que no son fascistas... tal vez, en últimas, el verdadero fascismo (italiano, no alemán) es
 preferible a muchos otros males... como el Gran Salto Adelante.  Por lo tanto no necesitamos deshonrar a las víctimas
 del fascismo, llamando fascista a todo aquellos que nos desagrada de la voluntad de nuestros líderes.

\par% p
El fascismo es una forma de nacionalismo.  Musolini y sus colaboradores querían una Italia grande, que retomara la
 grandeza de la época de Roma y que no estuviera influenciada por el pérfido albión (léase: \textbf{por fuera de la influencia anglosajona}).

\par% p
Para lograr esa Italia grande necesitaba de la colaboración de todos los estamentos de la sociedad, los gremios, las \emph{fasces}.  Quienes colaborarían en conjunto para la grandeza de la patria.  Pensar que alguien vendiera la patria, o la
 cultura, o lo que fuera a la influencia extranjera, principalmente la influencia británica o estadounidense, era
 inaceptable.  Era un crímen a la patria.  Como crimen era igualmente ceder al impulso globalizador de la revolución
 internacional: de los bolcheviques.

Como vemos, en esa época el fascismo es lo que se oponía a la globalización, mientras el comunismo (y la versión
 anglosajona de la democracia) eran los entes globalizantes que querían acabar con la cultura italiana... o alemana, o
 japonesa, o española.

Desde luego, en su afán de ver enemigos de la patria entre aquellos que escuchaban Jazz o leían a Marx, el fascismo
 reprimió muchos movimientos o individuos que disentían de los postulados del régimen.  El régimen daba todo lo que los
 trabajadores pedían, con tal de que no armaran un sindicato.  Pero cualquier intento de sindicato era fuertemente
 sancionado.

¡Qué diferencia con Colombia y con Uribe!

Si bien no es fácil ser sindicalista, mal que bien no es el estado el que persigue a los sindicalistas.  ¿y cuádo hace
 algo el estado para mejorar el bienestar de los trabajadores previniendo así que los sindicatos sean necesarios?

\par% p
¿Cuándo hace algo el estado para proteger la \emph{colombianidad} por encima de las influencias anglosajonas o bolivarianas, o en general de las globalizaciones de izquierda o derecha
 (o de arriba, o de abajo, etc.)?

Fascista es más Chávez y muchas de nuestras izquierdas, que la misma derecha colombiana.

Reprimir la disención no es fascismo: eso lo han hecho todos los regímenes no liberales (y muchos liberales) desde que
 los estados existen.

Bueno, eso es todo por ahora.

\chapter{Parlamentarismo}
\begin{metadata}
	Published by \anchor[chlewey]{chlewey} on \anchor[http://ewey.co/B27]{Thu, 22 Apr 2004 15:26:00 +0000}\\
	\categories{usenet, opinion}\\
	Shorthand: \anchor[http://blog.chlewey.net/2004/04/parlamentarismo/]{parlamentarismo}
\end{metadata}

\begin{address}
\anchor[http://groups.google.com/group/soc.culture.colombia/msg/7d90bb691c08b33c]{news:c68o7b\$95enl\$1@ID-83976.news.uni-berlin.de}
\end{address}
Otra de las ideas que se andan discutiendo en estos momentos en Colombia, encabezada por Alfonso López Michelsen, es
 que en Colombia debería plantearse un régimen parlamentario.

Gran parte de la discusión que hoy gira al rededor de la reelección (y/o de la reelección de Álvaro Uribe Vélez), tiene
 su origen en la necesidad percibida o el temor del caudillismo.  La figura presidencial es tan importante, tan
 poderosa y tan personalista en Colombia que se confunde el proyecto institucional de un gobierno con la persona que lo
 encabeza.

La mayor parte de los regímenes parlamentarios (¿todos?) distinguen en dos personas distintas al jefe de estado y al
 jefe de gobierno.

Un jefe de estado, como el Rey Juan Carlos de España, o el Presidente Chirac de Francia, o la Reina Isabel de Canadá,
 son la máxima autoridad en su país, representa al estado y es símbolo de su pueblo.

Por otro lado un jefe de gobierno, como el Presidente de Gobierno en España o los Primeros Ministros de Francia y
 Canadá, representa al gobierno político; y es a su vez el representante de la política (i.e. de la coalición
 mayoritaria en el parlamento.)

La división exacta de funciones varía.  Chirac tiene mucho más poder político en Francia de aquel poder que Isabel II
 ejercita en Canadá.  Pero esa separación personal de ámbitos permite separar igualmente lo que son políticas de
 gobierno con políticas de estado.

Una política de estado es aquella que encausa a una nación a largo plazo y que debe contar con el concenso del pueblo.
 Una política de gobierno es aquella que define el estilo y las prioridades que una determinada doctrina política
 impone.

El otro punto a favor del régimen parlamentario es que hace irrelevante la discusión sobre reelección presidencial.  En
 un parlamento que funcione existirán varios partidos políticos que tienen que estructurarse en base a proyectos.  Hoy
 en día los partidos en Colombia están estructurados alrededor de personas.

La continuidad de un proyecto político está garantizada en la medida en la
que el pueblo siga confiando, y por lo tanto
 eligiendo, a un partido
político y no en la medida de la popularidad de un gobernante o su estilo.

La gran duda que se presentaría en Colombia es si el parlamentarismo generará la disciplina partidaria suficiente; o si
 por el contrario, se necesita primero que los partidos funciones para que el experimento parlamentarista tenga algún
 tipo de éxito.

Pero esto apenas lo están discutiendo columnistas como el ex presidente López.  No hay, a la fecha, un proyecto serio
 de reforma constitucional sobre el tema.

Así que más bien nos preocupemos de si vamos o no a reelegir a Uribe.

\chapter{Reeligiendo 2}
\begin{metadata}
	Published by \anchor[chlewey]{chlewey} on \anchor[http://ewey.co/B28]{Wed, 21 Apr 2004 22:12:00 +0000}\\
	\categories{usenet, opinion}\\
	Shorthand: \anchor[http://blog.chlewey.net/2004/04/reeligiendo-2/]{reeligiendo-2}
\end{metadata}

\begin{address}
\anchor[http://groups.google.com/group/soc.culture.colombia/msg/a9b5fdd9c1f5632e]{news:c66rjk\$7s7n5\$1@ID-83976.news.uni-berlin.de}
\end{address}
Uno de los argumentos más esgrimido por los partidarios de la reelección es que debe ser el pueblo el que elija si
 quieren seguir o no seguir con un gobernante.  Incluso algunos de los contradictores de Uribe acogen esta tesis.

Hay que tener en cuenta que la reelección de un presidente implica varias cosas:

1) la posibilidad de quien haya sido presidente de volverse a presentar.

2) la posibilidad de quien es presidente de extender su mandato por otro período.

3) el hecho de que quien es presidente esté en campaña política.

4) la posibilidad de que el pueblo no reelija a quien es el presidente actual.

5) el hecho de que las propuestas tengan nombre propio o no.

6) si se trata sólo del presidente o si incluye también a alcaldes y gobernadores.

Veamos.  Antes de la constitución de 1991, era posible reelegir a un ex presidente.  No estoy seguro si la norma estaba
 en la constitución original de 1886, pero tras la inminente reelección de Rafael Reyes, que acabó con su carrera, en
 1910 se reformó la constitución para prohibir la reelección inmediata.

Entre los presidentes reelegidos siendo ex presidentes se encuentran Rafael Nuñez y Alfonso López Pumarejo.  Otros ex
 presidentes elegidos incluyen a Alberto Lleras (pero el no fue reelegido, el fue presidente designado tras la renuncia
 de López Pumarejo.)

En el caso de Nuñez, el tuvo que renunciar a su presidencia para poder ser reelegido, ya que como presidente no podía
 ser candidato.

Nuestros ex presidentes acutales incluyen a César Gaviria, Ernesto Samper, Andrés Pastrana, Belisario Betancurt, Julio
 César Turbay y Alfonso López Michelsen.  Samper ha dicho que si se lo permiten se lanza de nuevo de candidato.
 Supongo que Gaviria lo haría igualmente.  Betancurt, Turbay y López lo dudo.

Uribe tendría en este caso que esperar hasta 2010 para que lo vuelvan a elegir.

Si se permite la reelección inmediata se puede plantear que sea incluyendo o no al actual gobierno.  El problema es que
 cualquiera de las dos posturas tiene nombre propio: Álvaro Uribe Vélez.  Y es muy difícil hacer el debate sin
 desligarnos de, bien, nuestro presidente.

Si cursa el proyecto, y se puede reelegir al actual presidente, surge entonces un problema: desde hace muchos años
 (desde antes de 1886, incluso) se ha prohibido que los funcionarios públicos administrativos participen en política
 (proselitismo político) y el presidente es, bien, el primer cargo público administrativo.

La idea de fondo es que no utilice los recursos del estado para una campaña política.

Personalmente me parece algo absurda la prohibición y ésta no hace más que convertirse en un método de persecución
 política.  La ley debería ser clara sobre lo que constituye peculado y cualquier alcalde, gobernador o presidente es
 libre de participar en proselitismo político en la medida en la que no cometa peculado (desvíe fondos públicos para
 una campaña política).

Pero la prohibición está y no creo que la levanten así no más.

Así que el Congreso tendrá que hacer una serie de reformas constitucionales más que permitan que el presidente pueda
 ser candidato sin hacer proselitismo político, o algo así.

Medio absurdo, pero para eso estamos en el ex país del sagrado corazón.

Ahora, quienes quieren reelegir a Uribe como presidente, no quieren reelegir a Lucho como alcalde de Bogotá.  Lucho
 Garzón, quien se opone a la reelección presidencial, se opone igualmente a la reelección de él mismo como alcalde
 (nada insalvable, en dado caso: sólo es cuestión que en 2007 diga que no quiere volverse a lanzar a la alcaldía).

Pero regresemos al punto.  La reelección inmediata nos da al pueblo la oportunidad de premiar o castigar a nuestros
 gobernantes: si nos gustó lo reelegimos y si no no lo reelegimos.

Si parto de ahí no tengo por qué preocuparme de que la propuesta de reelección se la estemos acomodando con nombre
 propio a Uribe o no: Uribe debe mostrar en estos dos años que faltan para las elecciones de 2006, que merece que lo
 reelijamos, o el pueblo votará por otro candidato.

La otra vez en Semana hacían un análisis: hoy las Autodefensas están más o menos apresuradas para buscar algún tipo de
 perdón en este gobierno.  Hoy las FARC están replegadas, posiblemente en un repliegue táctico, esperando a un gobierno
 más favorable para la negociación.  Si ambos grupos creen que Uribe se quedará hasta 2010, las AUC dejarán de estar
 apresuradas en ser perdonadas, y las FARC podrían salir de su repliegue táctico.  Esto no sólo implicaría la tradición
 de que las segundas presidencias no son tan buenas como las primeras, sino que harían de la segunda mitad de la
 presidencia de Uribe algo difícil de manejar.

Eso por no hablar del subempleo rampante, el incontrolable gasto público y la escalada de impuestos que nos afecta a
 pobres y no tan pobres, empleados, subempleados y desempleados por igual, que son fenómenos que ya se están dando.

Es decir que Uribe podría quemarse de aquí a 2006.

Bueno, Rafael Reyes no tuvo chance de que lo reeligieran, así que tenemos un precedente.

Tal vez la posibilidad de que podamos reelegir a Uribe en 2006 (o elegir a otro candidato enfrentado a Uribe), sea la
 verdadera prueba de fuego de la presidencia de Álvaro Uribe Vélez.

Pero ahí surge tambien mi mayor temor: a Uribe lo están idolatrando.  Así que aunque al país le vaya mal en estos
 próximos dos años, Uribe seguirá teniendo una muy alta aceptación y lo reelegirán a pesar de todo, para ahí si
 mostrarnos nuevamente que las segundas presidencias nunca son tan buenas como las primeras.

\chapter{Un grupo de noticias de Colombia}
\begin{metadata}
	Published by \anchor[chlewey]{chlewey} on \anchor[http://ewey.co/B29]{Sun, 09 Feb 2003 17:49:00 +0000}\\
	\categories{usenet, usenet, opinion}\\
	Shorthand: \anchor[http://blog.chlewey.net/2003/02/un-grupo-de-noticias-de-colombia/]{un-grupo-de-noticias-de-colombia}
\end{metadata}

\begin{address}
\anchor[http://groups.google.com/group/soc.culture.colombia/msg/bc2e979f807fe585]{news:b264c3\$18tcii\$1@ID-83976.news.dfncis.de}
\end{address}
El pasado viernes se presentaron cuatro noticias de alto impacto en Colombia.  No propiamente ``importantes'' como lo
 mensionaba un medio de comunicación, sino más bien trágicas.

Entre esos cuatro ítemes noticiosos sin duda se generaron más hechos, entre ese montón de hechos trágicos que suceden
 pero no registra el país, hechos positivos y hechos más o menos importantes en los mundos deportivos, de farándula o
 de carnaval en el que vivimos.

Mientras tanto veo que en un sitio destinado para que los colombianos hablemos de lo que nos pasa y nos interesa, es
 muy poco lo que se habla de Colombia.  No digo que tangamos que dedicar el 90\% del tiempo a la explosión en el Club El
 Nogal de Bogotá con sus treintaypucho de muertos y cientos de heridos, tal como han hecho los noticieros aquí.  Ni
 digo que tengamos que dedicar el 80\% de los mensajes a hablar del incendio de decenas de casas de invasión construídas
 sobre el antiguo basurero de Medellín, tal y como no han hecho los noticieros aquí.

El punto es que salvo algunos mensajes de Vladimir, que son básicamente copia de las denuncias de organismos de
 derechos humanos en Colombia, entre bananas alcohólicas y coca cola, Iraq y Venezuela, casi todas por mensajes
 crosposteados por todos los grupos soc.culture latinoamericanos, es muy poco lo que vemos aquí del país.

¿Tanto nos hastía ver lo bueno o malo que tenemos?

\chapter{Castaño en El Mundo}
\begin{metadata}
	Published by \anchor[chlewey]{chlewey} on \anchor[http://ewey.co/B30]{Mon, 27 May 2002 13:38:00 +0000}\\
	\categories{auc, carlos-castano, usenet, elecciones, guest}\\
	Shorthand: \anchor[http://blog.chlewey.net/2002/05/castano-en-el-mundo/]{castano-en-el-mundo}
\end{metadata}

\begin{address}
\anchor[http://groups.google.com/group/soc.culture.colombia/msg/f454324255501a67]{news:3CF242FA.96EB71BB@my-deja.com}
\end{address}
Buscando en las páginas de los grupos ilegales de Colombia (AUC, FARC, etc.) si había un planteaminiento respecto a las
 elecciónes de ayer, todavía no hay tales pero encontré este artículo de El Mundo (España) sobre Carlos Castaño.

No lo niego, la franqueza con la que habla o pretende hablar el hombre es atractiva: defiende su lucha, reconoce la
 ilegalidad (que no ilegitimidad) de la misma, etc.  Sobre la persecución a la campaña de Serpa... sí la hubo aunque
 Castaño lo niegue, aunque es curioso al ver los resultados:

Córdova; zona de amplia influencia de las AUC (y en particular de las ACCU): ganó Serpa.

Chocó; zona actualmente en disputa: ganó Serpa.

Magdalena; donde más quejas hubo sobre intervensión de los paramilitares: único departamento de la costa donde ganó
 Uribe.  En Santa Marta (es de suponerse que la presión es menor en las ciudades): ganó Serpa.

Cundinamarca; donde mayores denuncias hubo de presión de las FARC: ganó Uribe.  (Aún no he visto resultados municipio a
 municipio; los datos excluyen a Bogotá).

Meta, Caquetá, Putumayo, Casanare; sitios donde las FARC más han presionado: ganó Uribe.

Bueno, me gustaría ver un mapa municipio a municipio, así como el mapa de abstención en los mismos.  Mientras tanto los
 dejo con la entrevista a Castaño.

\anchor[http://www.elmundo.es/2002/05/26/mundo/1148306.html]{http://www.elmundo.es/2002/05/26/mundo/1148306.html}
«Las AUC no apoyamos a ningún candidato ni impedimos que la gente vote a su preferido»
SALUD HERNANDEZ-MORA. Especial
 para EL MUNDO
\begin{blockquote}
ANTIOQUIA. La sombra de Carlos Castaño y sus AUC (Autodefensas Unidas de Colombia) ha planeado sobre toda la contienda
 electoral.Primero fueron las denuncias de que su organización estaba haciendo campaña abierta a favor de Alvaro Uribe
 y en contra de Horacio Serpa. Le acusaron de amenazar a periodistas por transmitir mensajes del candidato del Partido
 Liberal y de obligar a la gente en sus áreas de influencia a votar por Uribe. Luego señalaron al hombre que hoy puede
 convertirse en el nuevo presidente de Colombia, como el candidato de las AUC, muy cercano a sus planteamientos.
\end{blockquote}

Castaño sonríe cuando pregunto por sus preferencias, «no las digo porque como grupo ilegal no podemos estar en la lucha
 electoral» y desmiente que haya consignas a favor de Uribe o cualquier otro candidato.

«Las AUC no apoyamos a ningún candidato. Tampoco podemos prohibir que la gente vote por el de su preferencia. En
 nuestras regiones hay congresistas serpistas y uribistas y todos son testigos de nuestro respeto a sus campañas por
 sus respectivos candidatos presidenciales. Las AUC tenemos unidad ideológica en la defensa del Estado, pero en materia
 electoral, no. No hemos saboteado nunca la campaña del doctor Serpa y él lo sabe». Para quien es aún el líder único de
 las AUC, aunque haya abandonado la Jefatura del Estado Mayor para quedarse con la dirección política, es la guerrilla
 la única que está saboteando las elecciones.

«Mire, por el señor Garzón las FARC presionan duro con el fusil para que la gente vote por él en el sur del país, y
 esto no puede afectar a Lucho ni puede llevarnos a creer que él es un guerrillero; es más, en lo único que las FARC y
 yo estamos de acuerdo es en que es el mejor de los candidatos. Quienes sí han declarado la guerra contra la campaña de
 Uribe Velez son las FARC y todo el país lo sabe».

El favorito en los comicios, Alvaro Uribe, propone fortalecer a las Fuerzas Armadas para enfrentar la subversión. Usted
 que los conoce bien, ¿cree que se puede combatir a la guerrilla sólo con las armas?

En el nuevo mundo de la globalización el primer frente de guerra del próximo presidente, cualquiera que sea, será el
 internacional, consiguiendo recursos y preparando la propuesta de paz para la guerrilla, alterna a la confrontación
 armada interna. La guerra en Colombia se acaba en los próximos cuatro años, indistintamente del presidente que llegue
 al poder. El banquete para la inversión está servido y no puede esperar. Los comensales: EEUU, la UE, las FARC, y la
 reacia Administración colombiana como anfitriona, se repartirán el pastel. Ojalá quede algo para el pueblo y no
 precisamente las sobras, ahí estaremos las AUC para exigir inversión social por él.

¿Cree que si gana Alvaro Uribe abrirá una negociación con las AUC?

Cualquiera que sea el presidente de Colombia tendrá la obligación de enfrentar a todos los actores al margen de la ley,
 y de negociar la paz finalmente con los actores políticos, y ahí estaremos las AUC quieran o no las guerrillas.

¿Cómo afecta la pérdida de más de 100 hombres de las AUC en sólo una semana?

Estamos preparados para enfrentar la guerrilla en sus guaridas y es lo que estamos haciendo. Creemos tener lo
 suficiente como para decirles a los colombianos que la guerrilla por la vía de las armas lo único asegurado que tiene
 es su derrota. Sí, lloramos nuestros muertos, pero los sepultamos y cada caído levanta una familia contra la guerrilla.

\par% p
Castaño piensa que, por el contrario a lo que se cree, para ellos Uribe sería la peor opción en la Presidencia.
 Actuaría contra las AUC con mayor dureza porque tendrá que demostrar que no le une ningún lazo con este Ejército de
 10.000 hombres bien armados, a los que ya no sólo podrán derrotar con las armas. La mesa de negociación será la única
 manera de desactivar un movimiento que aún cuenta con el apoyo expreso del 7\% de los colombianos.

\chapter{Para los alvaristas}
\begin{metadata}
	Published by \anchor[chlewey]{chlewey} on \anchor[http://ewey.co/B219]{Mon, 11 Feb 2002 20:59:31 +0000}\\
	\categories{alvaro-uribe, usenet, elecciones, opinion}\\
	Shorthand: \anchor[http://blog.chlewey.net/2002/02/para-los-alvaristas/]{para-los-alvaristas}
\end{metadata}

\begin{address}
\anchor[http://groups.google.com/group/soc.culture.colombia/msg/497183e4782a1e8a]{news:3C6830B3.268C966A@my-deja.com}
\end{address}
Carlos Lemos Simmonds no suele ser de mis columnistas preferidos sobre todo por su Serpo-Samperismo, sin embargo me
 parece que en este artículo suyo en El Tiempo, refleja varias de las dudas que la propuestas de Álvaro Uribe Vélez
 plantean.

Desde luego, si estamos convencidos de que lo único que necesita nuestro país es mano fuerte, no debemos preocuparnos
 de lo demás.

Lemos habla principalmente de la propuesta de revocar el congreso.  Un congreso que aún no hemos elegido y que aún no
 habrá entrado a sesionar cuando votemos por presidente y ya lo vamos a revocar...  No será mejor esperar primero a ver
 que congreso vamos a elegir.

Dejo aquí uno de los parrafos y el vínculo para que los demás puedan leer el artículo completo.

\par% p
— Carlos Th
[E-mail: if you want to contact me by email, replace my-deja for yahoo.]

\par% div% {'style': 'background:#eee;color:#009;'}

\section{El malabarista}
\begin{small}
Por CARLOS LEMOS SIMMONDS
\end{small}

\textbf{“Calma, calma”, como el doctor Álvaro Uribe suele decir.}

\begin{address}
\anchor[http://eltiempo.terra.com.co/11-02-2002/reda169241.html]{http://eltiempo.terra.com.co/11-02-2002/reda169241.html}
\end{address}
[...]
Pero, además, no debe perderse de vista que revocar el Congreso no es el único acto intrépido que el doctor Uribe
 se ha propuesto ejecutar el mismo día en que llegue al poder.  Con claridad ha dicho que el 7 de agosto va a sacar a
 tiros a las Farc del Caguán y al Eln de donde esté; que va a reconstruir la economía que Pastrana y el neoliberalismo
 dejan en la peor condición (33 millones de pobres, 9 de menesterosos y la más pavorosa desigualdad); que va a acometer
 la reforma pensional, lo que lo enfrentará dura y quizás violentamente con el sindicalismo, los maestros y el poder
 judicial, y que, en un país en guerra, va a acabar con el servicio militar para crear un enorme ejército de
 profesionales, es decir de solo jóvenes pobres y campesinos, que no se sabe cómo se va a financiar.
[...]

\chapter{Tratando de pensar con cabeza fria}
\begin{metadata}
	Published by \anchor[chlewey]{chlewey} on \anchor[http://ewey.co/B224]{Mon, 25 Feb 2002 18:24:31 +0000}\\
	\categories{usenet, opinion}\\
	Shorthand: \anchor[http://blog.chlewey.net/2002/02/cabeza-fria/]{cabeza-fria}
\end{metadata}

\begin{address}
\anchor[http://groups.google.com/group/soc.culture.colombia/msg/40631e8889753e4c]{news:3C7A815E.F4D3B1B@my-deja.com}
\end{address}
En el libro sobre Carlos Castaño ``Mi confesión'', el jefe político de las Autodefensas Unidas de Colombia dice que todo
 lo que él dice es la verdad pero que él no está diciendo toda la verdad, que sólo cuando en el país podamos decirnos
 toda la verdad es que las cosas se solucionan pero que mientras tanto ejército, guerrillas y autodefensas no somos más
 que peones de los grandes intereses internacionales.

(No he leído el libro aún y la frase anterior la puedo estar citando mal.)

Ahora, soy consciente que porque Carlos Castaño diga que es la verdad, esto no lo hace automáticamente la verdad.
 Según muchos analistas, de confesión el libro tiene poco y no es más que una justificación de su accionar:
 justificación de por qué mataron a quien mataron o por qué están involucrados en los negocios de drogas en los que
 están involucrados, entre muchas otras cosas.

Pero entra la duda ¿somos simples peones de un juego de intereses internacionales?  ¿podemos conocer la verdad?
 Siempre hay algo atractivo en las teorías de conspiración y a decir verdad la parte más atractiva de las teorías de
 conspiración es que nos libera de la carga de asumir nuestra responsabilidad: ``si los gringos quieren que haya guerra
 tendremos guerra'' y no nos preocupamos realmente por evitar la guerra.

Por otro lado no podemos taparnos los ojos y asumir que no existen tales intereses internacionales.
 Estos existen y son muy reales, lo que no es claro es qué tan poderosos son y donde no estoy de acuerdo (aunque esto
 no es más que una profesión de fe) es que sea omnipotentes.

Estamos entrando en una guerra (bueno, continuando y arreciando la existente) ¿a quién le sirve esta guerra?

¿A las FARC?  En varias declaraciones ellos han expresado que no le temen a la guerra así que podemos creer que no han hecho más que provocarnos para que la guerra continúe.
 ¿Les serviría mejor la paz? Con seguridad que les serviría mejor una paz bajo sus términos, pero el problema es que la paz que el establecimiento colombiano (lo que quiera que esto signifique) está dispuesto a dar no les conviene a los actuales combatientes de las FARC.
 No les conviene dejarse masacrar como a la Unión Patriótica, ni cambiar el poder actual que les otorgan las armas por
 el que les otorgarán las urnas si se desmovilizan hoy mismo.

A muchos les gusta pensar en las FARC como simples delincuentes, pero si nos tomamos el trabajo de contrastar lo que dicen con lo que hacen, las FARC son consistentes con una línea política y un accionar dentro de la misma.
 Yo podré no estar de acuerdo con su línea política ni con sus métodos, pero debo reconocer que son consistentes sin
 necesidad de recurrir a una simplificación de calificar su accionar como netamente criminal.

¿Le conviene la guerra al gobierno y al estado constitucional? Empecemos por darnos cuenta que el gobierno, y mucho menos el estado constitucional, es un ente uniforme con un propósito claro y definido dentro de nuestro país.
 Existe el gobierno nacional, encabezado por el Presidente de la República con su gabinete y sus asesores.
 Existen las fuerzas militares constitucionales, nominalmente comandadas por el Presidente de la República pero operativamente comandada por sus generales.
 Está el congreso, las cortes nacionales, los gobiernos departamentales, distritales y municipales.
 Están las empresas estatales, también; lo ministerios y secretarías, etc.

Así:

¿Le conviene al gobierno nacional en cabeza del Presidente?  Sin duda, como la persona que decretó el fin del proceso de paz en el Caguán, algún interés personal, nacional o internacional debió motivar al presidente, luego sí, lo beneficia o bien como Presidente de la República de Colombia o como Andrés Pastrana Arango.
 Si juzgamos por las encuestas, Pastrana subió ante la opinión rompiendo el proceso de paz; el problema es que no es claro como puede capitalizar esta popularidad cuando en seis meses entrega el mando y como ex presidente haber sido popular o impopular no le aportará ni quitará nada.
 No logra la continuación de sus políticas bandera al pasar su popularidad a un sucesor, porque no hay tal sucesor ni
 los logros de su gobierno serán desbaratados por cualquier presidente medianamente responsable que llegue, y porque su
 mayor política visible que pudo haber heredado a un sucesor: el proceso de paz con las FARC, fue precisamente lo que
 terminó.

Descartado así el interés político directo, significa que Pastrana siguió un interés nacional o internacional, pero
 básicamente un interés de otra persona.

¿A quién le conviene la guerra?

¿A las fuerzas militares?  Mayor guerra significa un mayor presupuesto, que implicará mayores contratos que redundará en mayores oportunidades de lucro personal —si asumimos militares corruptos—.
 O por lo menos mayor guerra significa que las fuerzas armadas de Colombia tienen una razón de ser, así que los oficiales no quedarán desempleados.
 La guerra misma es una opción de empleo para el pueblo de base, bien sea combatiendo con las fuerzas militares constitucionales, las guerrillas o el paramilitarismo ilegal.
 Sin embargo evaluando la guerra como generadora de empleo puede ser una falacia porque la guerra misma, y más en las condiciones actuales en Colombia, mata muchas otras oportunidades de empleo.
 Desde luego que estas otras oportunidades de empleo, como sería la pequeña y mediana empresa, son también amenazadas
 por elementos distintos a la guerra, y como tal, aunque la guerra garantiza una baja en la inversión, la no guerra no
 garantiza que esta inversión se dé.

Pero regresando a las fuerzas militares la guerra significa mayor presupuesto, pero también mayores gastos.
 La guerra es una garantía de que las fuerzas militares tienen una razón de ser pero también convierte a cada militar en un blanco.
 Tanteando la situación, pareciera que las fuerzas militares, al igual que las FARC, perciben como más substanciosa
 para ellas que la guerra continúe por un tiempo, así sea nominalmente.

¿Le conviene nuestros legisladores?  Como grupo me atrevería a decir que no.
 La guerra no le conviene al congreso ni a la clases política.
 Ya vemos como los congresistas son secuestrados por las guerrillas o las autodefensas.
 Pero el Congreso no es una entidad uniforme sino tan sólo un foro donde muchos intereses se discuten y sin duda hay y
 habrán congresistas que representen a casi cualquier cosa en el país, incluyendo a los grupos armados ilegales
 (guerrillas y \_paras\_).

¿Le conviene a las empresas estatales?  No.
 Si bien muchas empresas de servicios públicos pueden usar la guerra como excusa para subir sus tarifas, los altos costos de los servicios públicos no redundan en el interés de las empresas.
 Si acaso en sus dueños, gerentes o sindicatos.
 La reparación de infraestructura es un costo que encarece la operación y encarece los seguros.

Pero en últimas, la guerra le sirve a menos gente que la paz.
 El problema es qué tipo de paz.
 Mi pregunta inicial está entonces mal formulada.
 El problema de la guerra no es a quien beneficia la guerra sino cómo logramos la paz, cómo lograr la paz que sea
 beneficiosa a los actores del conflicto y a los extras que somos todos los demás colombianos.

En cierta forma podríamos pensar que Pastrana no acabó con el proceso de paz porque la guerra le fuera más provechosa
 que la paz, sino porque el proceso en medio de la guerra empezó a ser menos provechoso que una guerra abierta, en
 vista de que la paz no se iba a conseguir.

Como ya mencioné, las FARC quieren una paz que les sea provechosa (políticamente, económicamente, lo que sea), y una simple desmovilización no les parece tan atractiva.
 No con la experiencia de la Unión Patriótica.
 Si las FARC se desmovilizan perderán poder aparentemente porque el pueblo no se volcará a votar por ellos en las circunstancias actuales (culpa de ellos dirán algunos, pero es la situación actual).
 Sus banderas de lucha quedarán igualmente por el suelo, así que la simple desmovilización significa que no lograron ni lucro personal, ni poder político ni mejorar al país de acuerdo a su pensamiento.
 Ante estas perspectivas continuar en la guerra puede ser darse la oportunidad a un largo plazo de obtener cualquiera
 de estas cosas... o rendirse finalmente cuando no puedan físicamente continuar.

Ante esta perspectiva de las FARC, el estado constitucional tiene que enfrentarlas.
 No enfrentarlas sería rendirse.
 Ahora, enfrentarlas puede significar un enfrentamiento netamente defensivo dejando que sean las FARC las que tomen las decisiones (aunque Gregorio crea que esto no es enfrentamiento), o emprender una ofensiva.
 Esto no quiere decir que el estado tenga que limitarse a enfrentar a las guerrillas, sino que también es imperativo
 del estado el buscar una solución integral: negociar con las guerrillas sus banderas de lucha o quitárselas.

Los tres años de negociación en el Caguán fueron el resultado de la ingenuidad de nuestro presidente, y de muchos de los que lo apoyamos. El presidente fue ingenuo porque no entendió la magnitud de la empresa de consiliar los intereses de Colombia ante la comunidad internacional y de esta ante Colombia, del estado con los colombianos y las exigencias de las guerrillas.
 La gran pregunta que le hicieron las FARC al presidente: ``díganos qué es negociable'', el presidente no la pudo contestar porque no dependía de él ni de sus altos comisionados el negociar la deuda externa, el derecho a la propiedad, la libertad de empresa y el interés de
 los votantes gringos, entre otros, con unos alzados en armas.

Sin duda Pastrana no se dio cuenta el pasado miércoles 20 de febrero que la negociación, tal como iba, era menos
 ventajosa que un enfrentamiento abierto... lo que necesitaba el presidente era encontrar una salida viable para
 continuar o una excusa para romper.

La salida viable para continuar era una tregua (y aunque no querramos reconocerlo, una tregua no puede ser concentrar a
 las FARC en el Caguán) pero ni el establecimiento ni los militares estaban dispuestos a pactar esta tregua en esas
 condiciones y las FARC le dieron la excusa: el secuestro del senador Géchem Turbay, a través del asalto a un avión
 comercial.

Termino por ahora.~ Después discuto el papel de la comunidad internacional en la guerra y en la paz de Colombia.

— Carlos Th

\chapter{Tratando de pensar con cabeza fria II}
\begin{metadata}
	Published by \anchor[chlewey]{chlewey} on \anchor[http://ewey.co/B225]{Mon, 25 Feb 2002 22:22:30 +0000}\\
	\categories{usenet, opinion}\\
	Shorthand: \anchor[http://blog.chlewey.net/2002/02/cabeza-fria-ii/]{cabeza-fria-ii}
\end{metadata}

\begin{address}
\anchor[http://groups.google.com/group/soc.culture.colombia/msg/e5c3987a4a524105]{news:3C7AB926.82E39030@my-deja.com}
\end{address}
Colombia produce petroleo, café, banano, flores y muchos otros productos que tienen alguna cotización en el mercado internacional.
 Produce también textiles y confecciones, algo de productos procesados y algo también en el sector de servicios.
 Aún así, el fuerte de las exportaciones colombianas están en el sector primario.

Muchos de estos productos son de importancia estratégica internacional, tal como el petróleo.
 Aunque nuestra producción es bastante baja en comparación a nuestro vecino venezolano, es comparable, sin embargo, con
 la producción ecuatoriana, y mal que bien, de ese petróleo viven algunas multinacionales y algunos colombianos.

Colombia es, igualmente, un país enorme.
 Más grande que cualquier país europeo con la excepción de Rusia y más grande que cualquier estado gringo.
 El segundo país en población en Sudamérica y el cuarto en el nuevo mundo (superado por Brasil, Estados Unidos y México), Colombia es igualmente un consumidor apreciable de hidrocarburos.
 Tenemos la ventaja del trópico que no nos obliga al gasto desmesurado de energía de los países templados, pero carecemos de un sistema ferroviario decente por lo que la mayor parte de nuestra carga se mueve sobre carretera.
 A duras penas nuestra producción de hidrocarburos cubre nuestra demanda y deja algún excedente.

Nuestra posición geográfica no es particularmente privilegiada desde el punto de vista del comercio mundial actual.
 No con el canal en nuestra vecina Panamá y el sistema ferroviario de Estados Unidos que ofrecen una mejor forma de pasar mercancías del Pacífico al Atlántico y viceversa que nuestro país.
 No cuando la única frontera con un país latinoamericano que aspire al primer mundo es la selva amazónica. Podríamos
 ser, si nos lo propusiéramos un importante centro para la integración latinoamericana, pero no un importante centro
 para la integración mundial.

Así que no somos ni los mayores productores de petroleo (ni siquiera entre nuestros vecinos inmediatos), ni de café
 (superados por Brasil y Vietnam), ni de banano (detrás de Ecuador y de Costa Rica), no somos un gran exportador
 industrial ni un centro de generación de servicios. Colombia tiene, sin embargo, muchas riquezas que pesan: el
 conjunto de muchos de estos segundos y terceros lugares, la biodiversidad, la masa humana.

Adicionalmente somos el mayor productor de cocaína en el mundo y uno de los mayores productores de heroína, lo cual,
 con la distorsión del mercado que trae la prohibición de estas substancias nos pone en el ojo del huracán de las
 políticas antidroga estadounidenses y mundiales, dejando pocos beneficios al país.

\par% p
Ante esto, la guerra que libra Colombia no nos favorece ni a los colombianos ni al mundo.
 Nuestras riquezas y en particular la biodiversidad, se pierden cuando los colonos arrazan el monte para cultivar yuca y detrás de ellos llegan los ganaderos desplazando a los colonos.
 Los colonos siguen entonces tumbando monte para sembrar coca y tras de ellos el glifosfato, las guerrillas y los \emph{paras}.
 Si no son los colonos son las petroleras, o los ganaderos, o los narcotraficantes.
 En este escenario la guerra complica la situación: los guerrilleros posando de defensores de los colonos para
 cobrarles luego el gramaje, los \emph{paras} como instrumentos de ciertos latifundistas, la imposibilidad de que alguien pueda ir tranquilo a la selva a explorar
 nuestra riqueza natural entre el fuego cruzado de unos y otros.

Pero sufre también la explotación agrícola y minera.
 Ser cafetero hoy en día significa vivir entre el fuego cruzado para obtener una baja remuneración.
 Las petroleras invierten en seguridad lo que bien podrían invertir en ayuda al país y el contrabando de combustibles
 que necesitan los grupos ilegales para operar se convierten en operaciones que tampoco repercuten en el bienestar que
 el estado colombiano puede ofrecernos.

Colombia es un mal país para invertir.
 Podría ser un gran mercado de cuarenta millones de personas.
 Podría ser una despensa mundial, tanto de alimentos como de energía.
 Podría ser muchas cosas que la guerra no permite.
 Bueno, no sólo que la guerra no permite: la corrupción tampoco ni la inestabilidad legislativa.

Así las cosas, la única razón por la que la comunidad internacional está interesada en la guerra en Colombia es porque
 la guerra, la guerra que estamos librando los colombianos, está afectando a la comunidad internacional.

Lo que debemos hacer los colombianos es acabar con esta guerra y venderles la paz a la comunidad internacional.
 Una paz que sea provechosa para ellos y lucrativa para nosotros.~ ¿No es posible?  Nos falta entonces imaginación.

— Carlos Th

\chapter{Cali, Venezuela, Israel...}
\begin{metadata}
	Published by \anchor[chlewey]{chlewey} on \anchor[http://ewey.co/B228]{Fri, 12 Apr 2002 16:06:51 +0000}\\
	\categories{usenet, information, israel, venezuela}\\
	Shorthand: \anchor[http://blog.chlewey.net/2002/04/cali-venezuela-israel/]{cali-venezuela-israel}
\end{metadata}

\begin{address}
\anchor[http://groups.google.com/group/soc.culture.colombia/msg/c66825c09d67eae6]{news:3CB7061B.A84A3FE5@my-deja.com}
\end{address}
Un poco movidas han estado las noticias el día de ayer.

La esposa del presidente de la Asamblea Departamental del Valle del Cauca haciendo eco a su marido o al comandante
 Augusto del frente 30 de las FARC, para que suspendieran los operativos de rescate, mientras unos periodistas de RCN
 perecían entre el fuego cruzado.

Mientras tanto, la Corte Constitucional (¿o fue la Corte Suprema?) declaró inconstitucional al estatuto
 antiterrorista...

Ante estos dos episodios, los noticieros de radio y televisión de ayer por la tarde y anoche estuvieron casi
 monopolizados por los acontecimientos de Venezuela y CNN internacional no mostraba nada distinto al viaje del
 Secretario de Estado Powell a Israel y Palestina.

Los presidenciales están presentando sus propuestas y anoche vimos a Uribe contestando preguntas en directo sobre sus
 planes.  Entre más escucho los planes de Uribe Velez, menos me convence pero aún así me parece mucho más aterrizado
 que Serpa Uribe.  Ante esto veo cada vez mejores las propuestas de Sanín Posada quien sigue bajando en las encuestas.

Quien sí ha venido subiendo en las encuestas ha sido Lucho Garzón, aunque más a costa de Noemí que de los dos punteros,
 lo cual está moviendo a muchos a hablar de un acuerdo entre estas dos candidaturas. La gran ventaja que tiene Garzón
 es que es respetado por los empresarios, quienes aún disintiendo de sus puntos de vista cuando fue presidente de la
 CUT lo reconocen como un interlocutor serio con el cual se puede llegar a acuerdos beneficiosos a ambas partes.  Su
 gran problema: aún habla de paz negociada cuando los colombianos queremos oir hablar de ganar la guerra y aún así
 inscribió a Vera Grabe como su vicepresidente mientras muchos la recuerdan disparando desde el Palacio de Justicia
 (cosa que, por cierto, no sucedió).

En Palestina las cosas no pintan mejor.  Entre un Sharon enceguecido por su punto de vista de solución netamente
 militar y un Arafat que por ineptitud o complicidad está permitiendo que la Autonomía Palestina se convierta en la
 fuente de ataques terroristas contra los ciudadanos (que no las instituciones estatales) del Estado de Israel, son los
 pueblos israelí y palestino los que están sufriendo.   No es un ejemplo muy alagador, la verdad, si vemos lo que está
 sucediendo en nuestra Colombia, pero las situaciones no son análogas...

Pues bien.  Aquí seguimos tratando de sobrevivir este país, aún cuando compañeros de trabajo han estado durmiendo a
 pocas cuadras de bombas como la de Villavicencio y recorriendo el país tratando de llevar la infraestructura que otros
 quieren destruir. ... porque al fin y al cabo los que trabajamos para multinacionales extranjeras somos unos
 vendepatrias (aunque nuestros salarios sean inferiores en promedio que los de nuestros pares de empresas públicas ...)

— Carlos Th
[E-mail: for contacting me by email replace my-deja for yahoo.]

\chapter{Preguntas para definir derechas e izquierdas.}
\begin{metadata}
	Published by \anchor[chlewey]{chlewey} on \anchor[http://ewey.co/B229]{Mon, 06 May 2002 21:47:53 +0000}\\
	\categories{derecha, usenet, izquierda, opinion}\\
	Shorthand: \anchor[http://blog.chlewey.net/2002/05/preguntas/]{preguntas}
\end{metadata}

\begin{address}
\anchor[http://groups.google.com/group/soc.culture.colombia/msg/9d5f612268322384]{news:3CD6FA09.AD3DB067@my-deja.com}
\end{address}
Los que me han visto por aquí durante algún tiempo, sabrán que uno de mis temas recurrentes es la división de las
 ideologías entre ``derecha'' e ``izquierda'', el gran problema de esta división es que no hay una escala comunmente
 aceptada que permita, de acuerdo a la gran cantidad de factores que un partido o una persona pueda pregonar, que
 permitan calificarla en la escala de dereha/izquierda.

En la prensa he visto que comúnmente hablan de los candidatos presidenciales como:
Alvaro Uribe  - derecha
Nohemí Sanín
  - centro
Horacio Serpa - centro-izquierda
Lucho Garzón  - izquierda

En cierta forma así se autodenominan Sanín, Serpa y Garzón (lo que veo como un prejuicio a la noción de ``derecha'').
 Pero si vemos puntos individuales, en lo que estos personajes piensan sobre:
¿Qué tanto debe intervenir el estado en
 decidir por sus ciudadanos?
(y cual de estas respuestas siginifican ``izquierda'' o ``derecha''.
¿Cual debe ser el
 alcance/tamaño del estado y la burocracia?
¿Debe ser el estado proteccionista o laxo económicamente?
¿Qué es el
 neoliberalismo y si este es malo o bueno?
¿Qué es la globalización y si esta es mala o buena?
¿Qué es peor: las
 guerrillas, los paras o los políticos corruptos?
¿Debe combatirse a la subversión militarmente o se debe negociar con
 ella?
¿Debe legalizarse la droga?
¿El congreso debe ser más pequeño?
¿Los hospitales deben ser autosuficientes
 económicamente?
¿Las parejas homosexuales tienen los mismos derechos que los matrimonios heterosexuales?
¿Es la mujer
 dueña de su cuerpo? ¿Se despenalizaría el aborto?
¿La moral católica es la guía suprema? ¿debería incluirse el
 decálogo en el código de ética del congreso?
¿Es el estado la fuente de todos lo males?
¿Debemos pensar en la patria
 primero o en la humanidad primero?
¿Somos los seres humanos los amos de la creación?
¿Tienen derechos los
 animales?
¿Tienen derechos los estados?
¿Es mejor una democracia participativa o una democracia representativa?
¿Es la
 democracia el sistema de gobierno ideal?
¿Debería la población civil colaborar con las autoridades frente al conflicto
 armado?
¿Está demente la guerrilla?

Finalmente:
¿Cual debería ser la postura de ``extrema derecha'', ``derecha moderada'' ``centro-derecha'', ``centro'',
 ``centro-izquierda'', ``izquierda moderada'', ``extrema izquierda'' sobre cada una de esas preguntas?

y

De acuerdo a esta escala ¿dónde está cada candidato?

Bueno, y ya que hemos definido a los políticos:
¿dónde se encuentran las FARC?
¿dónde se encuentra el ELN?
¿dónde se
 encuentran las AUC?
¿dónde se encuentra Pastrana?
¿dónde se encuentra Tapias?

Bueno, pero antes de empezar a tacharlos, debemos definir bien la escala.  Si no, estamos haciendo trampa para acomodar
 las respuestas.

¿Quién se le mide?

— Carlos Th
[E-mail: for contacting me by email replace my-deja for yahoo.]

\chapter{Mi voto.}
\begin{metadata}
	Published by \anchor[chlewey]{chlewey} on \anchor[http://ewey.co/B230]{Wed, 22 May 2002 03:10:47 +0000}\\
	\categories{usenet, elecciones, personal}\\
	Shorthand: \anchor[http://blog.chlewey.net/2002/05/mi-voto/]{mi-voto}
\end{metadata}

\begin{address}
\anchor[http://groups.google.com/group/soc.culture.colombia/msg/e2c23d06c7768aac]{news:acf2bg\$okr4v\$1@ID-83976.news.dfncis.de}
\end{address}
Mi voto no será, tal vez, por quien pudiera ser el mejor presidente para Colombia en los próximos cuatro años.  Ninguno
 de los actuales candidatos me parece que sea el presidente perfecto en este aspecto y si por ello fuera, Noemí Sanín
 parece tener la mejor propuesta (y no sólo la menos mala).

Mi voto va a ser, más bien, por la democracia.  Voy a aportar con mi voto por un concepto algo más abstracto que un
 hombre o una mujer que ocuparía el solio de Bolívar.  Pienso votar por que la palabra democracia tenga aún algún
 sentido en este país.

Las encuestas me dicen que la presidencia la ganará Alvaro Uribe Velez, bien de una en estas elecciones, o ante Horacio
 Serpa en una segunda vuelta.  Podría pensarse que si apoyo con mi voto a Uribe, contribuyo a que no haya necesidad de
 una segunda vuelta con el consecuente gasto en recursos del estado y en un par de semanas más de desgaste de las
 campañas en los medios, con el peligro de la guerrilla queriendo sabotear nuevamente unas elecciones, más el agua
 sucia que se seguirán hechando las campañas de Uribe y Serpa.

Pero la democracia no son las encuestas.  Tampoco lo es, exclusivamente, el sufragio.  La democracia es la
 participación y, a menos que las encuestas sean capaces de lograr la participación del pueblo todo, las encuestas no
 son la democracia.  Si así fuera, mejor ni hagamos primera vuelta: al fin y al cabo si Uribe no gana en la primera
 ganará en la segunda, las encuestas indican que ganará de todas formas.  Votar así sólo porque las encuestas dicen
 algo, el así llamado voto útil, es un gran peligro para la democracia.  Es votar por Uribe para que no haya segunda
 vuelta o votar por Serpa para atajar a Uribe.  Es polarizar al país, no ante un par de ideas contrarias, sino ante el
 temor de la guerrilla y de cómo debemos reaccionar ante ella.

Lo peor de todo es que es un voto equivocado.

Tanto Serpa como Uribe proponen básicamente lo mismo: pelear contra la guerrilla si la guerrilla sigue en la misma
 tónica actual y dialogar con la guerrilla si esta da muestras de voluntad de paz.  Ese cuento de que Uribe es la
 guerra y Serpa es el diálogo no se refleja ni en los discursos ni en los hechos.  Tanto Uribe como Serpa han dialogado
 con la guerrilla, han estado involucrados con las convivir, han criticado la voluntad de paz de las FARC, prometen
 enfrentar a las FARC, dialogar con el ELN y, si las FARC muestran voluntad de paz: dialogar con las FARC, etc.  Hubo
 una diferencia en la posición respecto al proceso de paz de Pastrana, pero esa diferencia no indica que uno signifique
 la guerra y el otro el entreguismo.

Muchos de los que dicen que votarán por Uribe, lo hacen porque creen que Uribe representa la mano fuerte contra los
 bandidos de la guerrilla. Eso no es lo que promete Uribe.  Uribe, según sus propias palabras, ofrece fortalecer la
 autoridad del estado con el objeto de disuadir la guerra, no de aumentarla.  Pero no deja claro cómo va a realizar
 esto de una forma que logre disuadir la guerra, por lo que en últimas puede terminar creando una guerra más cruenta de
 la actual sin que esta sea claramente favorable al estado ni a los ciudadanos que se acojan a la protección del estado.

Por otro lado Serpa busca votos precisamente asustando con la perspectiva de guerra de Uribe.  Insinuando que Uribe va
 a traer la guerra, cuando la guerra ya está aquí.  Pero, salvo en detalles, no propone nada distinto a lo que propone
 Uribe siendo sus ideas menos claras e igualmente sin indicarnos el cómo.  Nos dice que no necesitamos más que las
 fuerzas armadas que tenemos hoy en día, pero que han probado que aún no son suficientes para acabar con el terrorismo
 a pesar de los grandes avances que han tenido en el último año.  No nos habla de cómo fortalecer estas fuerzas
 armadas, pero igualmente nos dice que va a enfrentar a las guerrillas sin cerrar la puerta del diálogo.

Ante estas dos perspectivas sólo queda clara una cosa: ambos insinúan que van a seguir jugando a la guerra dejando
 abierta una posibilidad de nuevos diálogos cuando la guerrilla así lo decida, sin dejar en claro con cuales elementos
 contará el estado por ellos presedido para negociar con la guerrilla.  Ante estas propuestas de guerra y paz, Noemí no
 se aleja demasiado: oponerse a la guerrilla y dejar abierta la puerta del diálogo; aunque insinúa que tomará la
 iniciativa en lugar de esperar a que sea la guerrilla la que decida... mientras no deje claro como tomará esa
 iniciativa no está diciendo nada substancialmente distinto.

Pero la guerra y la paz no es el único problema que nos aqueja.  Hay una economía con problemas y no sólo achacables a
 la guerrilla o cuya solución toque dejarla para cuando la guerrilla se desmovilice.  Hay un problema de desempleo que,
 independientemente de si la guerra lo causó no puede esperar a que la guerra se acabe.  Y así muchos etcéteras.

Los problemas de la educación y la salud tienen que ver más con las políticas fiscales de los gobiernos que con la
 guerra misma, así que, como ciudadanos responsables debemos ver esos problemas a la par de la guerra.  La corrupción:
 cuando un empleado público roba recursos del estado (directa o indirectamente) no lo hace porque en Colombia hay
 guerrilla.  Y no nos engañemos que el paramilitarismo es un problema más viejo que la guerrilla misma.

Respecto a la economía, Serpa a sugerido que Uribe representa al neoliberalismo y, con éste, que Uribe acabará con el
 sistema de salud y con el Sena.  Lo absurdo es que Uribe fue el ponente de la ley 100.  Hoy en día pocos se atreven a
 denominarse neoliberales en Colombia, siendo esta una palabra que la iglesia satanizó.  Bueno, yo me declaro aquí como
 un neoliberal que votará por un candidato de izquierda.  En gran medida porque creo que el neoliberalismo, a
 diferencia del capitalismo salvaje que nos han impuesto, es la mejor forma de lograr que la iniciativa individual
 genere riqueza.  En Colombia el neoliberalismo nunca ha existido, lo único que hemos tenido ha sido la feria de
 privatizaciones que no es lo mismo.  Pero Uribe no representa este neoliberalismo.  Al menos no lo ha dicho.  No ha
 dicho cómo piensa ser la relación entre el estado y el sector privado.  Pero eso sí, se la pasa insinuando de la
 necesidad de la inversión social que vendrá cuando se acabe la corrupción.

Bueno, por ahí ya dejé escapar por quién va a ser mi voto.  Mi voto no va a ser por la guerra, pero sí por la mejor
 lectura sobre la guerra y sobre qué debe hacer un estado cuando se enfrente a buscar la paz a través del diálogo con
 una discidencia alzada en armas.  En este aspecto mi voto va por un negociador que se forjó negociando pliegos de
 peticiones y aprendiendo a ceder y a lograr que sus interlocutores pudieran confiar en su palabra.  Mi voto va por una
 persona cuyas propuestas económicas han sido aplaudidas por los industriales, a pesar de que por mucho tiempo fue su
 contraparte.  Yo no sé mucho de economía pero cuando una persona formada en la izquierda sindical es capaz de hacerse
 aplaudir en temas económicos por industriales, sin que sus compañeros de izquierda lo tachen de traidor, debe estar
 diciendo algo bueno.

No sé qué tal será mi candidato como presidente.  No creo que sea el menos malo.  Quien creo que es el menos malo es
 Noemí.  Pero, si las encuestas reflejan algo de lo que será la realidad de las urnas, no me tendré que preocuparme de
 los problemas que pueda tener mi candidato una vez electo presidente.  Y si llegare a ocurrir que mi candidato queda
 elegido presidente... bueno, tragando saliva esperaré a que las cosas se resuelvan para bien.  Al menos, ante la
 perspectiva de los demás candidatos creo que el sacudón que significaría puede ser hasta para bien.

Mi voto no es para lograr que mi candidato llegue a presidente, sino para contribuir a la democracia.  Mi voto es para
 no votar por alguien que no me convence del todo dizque para que no haya una segunda vuelta. Mi voto es para dar un
 apoyo a perspectivas diferentes pero inteligentes.  Mi voto es para una persona con el cual no comulgo políticamente
 pero con el que encuentro coincidencias fundamentales.  Mi voto es por la mejor perspectiva que hay de paz ante tantas
 perspectivas de guerra.  Mi voto es también una manifestación de que si la guerrilla se desmoviliza y pide perdón
 podemos otorgar ese perdón.  Mi voto es, en fin, por Luis Eduardo Garzón.

Una vez expresado mi voto invito.

No invito a que voten por el mismo candidato o por las mismas razones por las que yo voto.  Cada cual tendrá sus
 razones.  Invito a que nos atrevamos a analizar otros motivos y a que, con nuestro voto, podamos expresar algo.

Que el que vote por Uribe, no vote porque traerá la paz a través de la guerra, o porque creará un congreso unicameral
 (al fin y al cabo todos los congresistas son políticos corruptos, ¿no?), o porque debemos atajar al elefante de Serpa.
  Ni mucho menos porque es el mesías.

Que el que vote por Serpa no vote porque es el único que podrá manejar los hilos del poder de la política, ni porque es
 la única alternativa a la guerra de Uribe.

Atrevámonos a desencasillarnos.  Noemí Sanín tiene tal vez el mejor equipo, el único vicepresidente preparado para
 asumir una eventual presidencia y un mejor programa económico que Serpa y Uribe.  Si con nuestro voto queremos decir
 que un presidente, que un candidato debe llegar con propuestas serias y preparadas un voto por Noemí es una
 manifestación al respecto.

Harold Bedoya se opuso al despeje desde mucho antes que Uribe.  Desde mucho antes que Pastrana, incluso.  Si nuestra
 razón de votar es por decirle no a un eventual nuevo despeje ese mensaje quedaría más claro con Bedoya que con Uribe.

Por otro lado si queremos darle un mensaje más directo a las FARC.  O si creemos que la corrupción hay que combatirla
 de maneras creativas.  Si creemos que un presidente debe estar dispuesto a correr riesgos por defender sus ideas, o si
 símplemente queremos votar por el único candidato al que los demás candidatos no han criticado: ahí tenemos a Ingrid
 Betancurt.

Si no es porque no recuerdo el nombre, al candidato que se inscribió no para que lo elijan a él sino para promover el
 voto en blanco deberíamos seleccionarlo con la esperanza que tenga más votos que don blanco.

En fin, la democracia está aquí es para que nos expresemos y no para legitimar un status quo.  No nos preocupemos que,
 en últimas, gane quien gane, el país no se va a acabar ni se va a salvar.  Somos los ciudadanos y no un presidente,
 quienes en últimas sacaremos adelante a nuestro país o lo seguiremos dejando a la deriva.

Supongo que votaré por Uribe en la segunda vuelta.  Para eso es la segunda vuelta: para que votemos por el menos malo
 de las dos alternativas más populares.  Por la primera vuelta pienso votar por quien más me gusta, por quien mejor
 expresa lo que pienso de mi país o de su futuro, por decir que creo que la paz es posible.

Y sin duda, si mi candidato pasa a la segunda vuelta junto con Uribe: por fin veremos un debate civilizado entre dos
 perspectivas realmente distintas, sin el ruido de falsas insinuaciones.

— Carlos Th

\chapter{Detalles de la votación}
\begin{metadata}
	Published by \anchor[chlewey]{chlewey} on \anchor[http://ewey.co/B234]{Mon, 27 May 2002 16:11:53 +0000}\\
	\categories{usenet, elecciones, information}\\
	Shorthand: \anchor[http://blog.chlewey.net/2002/05/detalles-de-la-votacion/]{detalles-de-la-votacion}
\end{metadata}

\begin{address}
\anchor[http://groups.google.com/group/soc.culture.colombia/msg/02008baf3c3f66d8]{news:3CF25AC9.4874E026@my-deja.com}
\end{address}
He aquí un detalle de las votaciones de ayer, departamento por departamento.
\begin{verbatim}
escrutado potencial  particip   ganó      votos
Amazonas:  100  \%      26.959     11.397  Serpa     5.460
Antioquia:
 99,0\%   3'071.272  1'438.300  Uribe   925.369
Arauca:    100  \%     115.989     29.760  Uribe    14.544
Atlantico: 100
  \%   1'300.929    459.723  Serpa   224.569
Bolivar:    99,1\%   1'019.182    345.798  Serpa   175.827
Boyacá:     99,3\%
     747.344    407.807  Uribe   187.048
Caldas:     99,9\%     672.005    353.378  Uribe    67.829
Caquetá:   100  \%
  207.440     51.877  Uribe    24.175
Casanare:   99,3\%     144.685     71.764  Uribe    32.441
Cauca:      95,9\%
 636.104    224.724  Serpa   105.863
Cesar:     100  \%     481.537    192.235  Serpa   100.886
Chocó:      90,5\%
 183.810     60.022  Serpa    40.359
Córdova:    99,5\%     796.309    372.851  Serpa   196.881
C/marca:    99,3\%
 1'223.196    657.224  Uribe   346.220
Bogotá:    100,0\%   3'837.203  2'146.940  Uribe 1'207.877
Guainía:    95,3\%
 13.442      3.866  Serpa     1.735
Guaviare:  100  \%      38.971      8.175  Uribe     3.968
Huila:     100  \%
 539.817    240.074  Uribe   125.403
La Guajira: 98,4\%     326.069     98.421  Serpa    63.625
Magdalena:  99,6\%
 614.259    248.609  Uribe   112.631
Meta:       99,1\%     405.620    184.547  Uribe   115.258
Nariño:     96,3\%
 764.099    286.035  Uribe   129.112
N. Sant.:   96,7\%     775.923    352.595  Uribe   205.067
Putumayo:  100  \%
 136.837     40.124  Uribe    16.430
Quindío:   100  \%     355.021    195.003  Uribe   126.459
Risaralda: 100  \%
 553.622    227.149  Uribe   190.851
San Andrés:100  \%      38.198     11.206  Serpa     5.380
Santander: 100,0\%
 1'206.944    693.556  Serpa   353.298
Sucre:      99,8\%     456.028    198.044  Serpa   118.650
Tolima:     96,0\%
 817.892    360.895  Uribe   178.852
Valle:      99,3\%   2'496.988  1'083.088  Uribe   603.584
Vaupés:     84,4\%
 13.488      3.320  Serpa     1.903
Vichada:    96,3\%      25.498      6.096  Serpa    58.082
\end{verbatim}

Exterior:   99,2\%     165.470    106.705  Uribe    89.699

\par% p
TOTAL:      99,0\%  24'208.150 11'244.288  Uribe 5'829.958
(100,0\% indica aproximación >99,95\%.  100  \% indica totalidad
 de mesas excrutadas.)

De esos 11'244.288 votos, 10'991.531 fueron votos válidos (por candidato o en blanco) y 252.757 fueron no válidos
 (nulos y no marcados).  De los votos válidos 195.465 fueron en blanco.

Uribe obtiene así el 53.04\% del total de votos válidos, algo más del 50\% del total de votos y menos del 25\% del
 potencial electoral.  Es decir que más del 75\% creyó en otra alternativa, no le importó o no pudo votar.

El mapa político así:

Serpista:
La Costa Atlantica: Córdova, Sucre, Bolivar, Atlántico, Cesar y Guajira.  Excepción: Magdalena.
Los extremos
 del país: San Andrés, Chocó, Amazonas, Vaupés, Guainía y Vichada.
Santander (bueno, el departamento natal de
 Serpa).
Cauca.

Total: 14 departamentos.

Uribista:
La zona andina: excepción de Cauca y Santander.
La región Oriental excepto el extremo oriental: Arauca,
 Casanare, Meta, Caquetá, Putumayo y Guaviare.
Magdalena (salvo Santa Marta).
El exterior.

Total: 18 departamentos, Bogotá y el Exterior.

Los más abstencionistas:
Las zonas de mayor conflicto y más perisféricas... esto sugeriría presión de los grupos
 armados ilegales contra la elecciones, producto de los desplazamientos previos o debido al traslado de mesas a las
 cabeceras municipales.  Aún así, ningún departamento tuvo una participación superior al 60\%, lo que indica todavía un
 gran grado de apatía o escepticismo.  (O que los muertos de la registraduría no están votando).

En todos los departamentos el primer y segundo lugar lo disputaban Serpa y Uribe.  El tercer y cuarto lugar entre
 Garzón y Sanín.  (No estoy seguro si en todos pero en todos los que he visto el quinto y sexto lo disputaron entre
 Betancourt y Bedoya.)  En varios países europeos Garzón superó a Serpa y en Francia el tercer lugar lo ocupó Betancurt.

Faltaría ver municipio a municipio, para ver si las tendencias se confirman.

Por el momento Álvaro Uribe Vélez supera en votación a la de Horacio Serpa en la segunda vuelta de 1998, lo que
 convierte a Uribe en el liberal más votado en Colombia.  Aún así está por debajo de la votación de Pastrana en esa
 ocación.

Serpa con sus 3'486.384 votos está en el orden de votos que obtuvo en la primera vuelta de 1998.

Estos datos están basados en el boletín 34, emitido el domingo 26 de mayo a las 22:16.

— Carlos Th
[E-mail: for contacting me by email replace my-deja for yahoo.]

\chapter{Fwd: Lo que siento !!}
\begin{metadata}
	Published by \anchor[chlewey]{chlewey} on \anchor[http://ewey.co/B237]{Sun, 16 Feb 2003 03:49:17 +0000}\\
	\categories{usenet, guest}\\
	Shorthand: \anchor[http://blog.chlewey.net/2003/02/lo-que-siento/]{lo-que-siento}
\end{metadata}

\begin{address}
\anchor[http://groups.google.com/group/soc.culture.colombia/msg/833389d2882d78b6]{news:b2n1g5\$1e2kt4\$1@ID-83976.news.dfncis.de}
\end{address}
En parte esto refleja lo que yo mismo siento y cuando alguien puede decirlo con mejores palabras que las mías, bien
 puedo compartirlas.

\par% p
— Carlos Th

\par% div% {'style': 'background: #eee; color: #009;'}
——- Original Message ——-
From: ``MATE''
Sent: Friday, February 14, 2003 11:24 PM
Subject: Lo que siento !!~

Me resulta difícil ir a dormir sin antes expresar  mis sentimientos ...... ahora estoy muy triste, estoy dolida.....
 por la situación que vivimos en Colombia.. en menos de nueve días hemos sido victimas de dos atentados terroristas y
 lo que mas me duele es la indiferencia, la apatía y sobretodo la gran cantidad de niños que ven fulminados sus sueños
 ya sea por la muerte de sus padres o la de ellos mismos........  mucha gente expresa despectivamente: ``ya era hora que
 atentaran contra los de estrato 6'' .......''pero que a mi que me da si no vivo ni tengo familiares en Neiva'' .... ``tan
 pronto como pueda me largo de este puto país''.. todo esto me duele...
Me duele que la gente ya no crea en lo
 nuestro.....que no pueda ni quiera construir sueños...

Me detengo un momento entre mi ir y venir, para pensar en todo esto y me pregunto una y otra vez ¿a dónde vamos a
 llegar ? ¿qué futuro puede brindarle este país a mis hijos? ¿con que seguridad puedo salir de mi casa y saber que voy
 a regresar?

Para ser muy sincera muchas veces siento que Yo no puedo hacer nada, que todo se me sale de las manos y que este país
 que durante 24 años me ha dado identidad.....no tiene solución !!!...... Hoy lloro por mi Colombia y con sentimiento
 de patria mi alma trata de desfallecer lentamente...

No se si ahora soy mas sensible y las cosas me afecten mas.. no se si este en mis días pero en verdad que al ver las
 noticias se me llenan los ojos de lagrimas y un nudo en la garganta me impide respirar....

Para concluir solo quiero decirles que no dejemos de creer y luchar por lo nuestro por esa sangre colombiana que un día
 llenó de calor nuestro cuerpo... no desfallezcamos y ayudémonos para levantarnos cuando sintamos que este pais no
 tiene solución....Yo necesito la ayuda de ustedes y ustedes pueden contar con mi mano siempre que quieran levantar con
 orgullo nuestro tricolor Colombiano.

Bueno eso es lo que siento ahora a la camita

Los quiere

MATE LOCA !!!

\chapter{Reeligiendo}
\begin{metadata}
	Published by \anchor[chlewey]{chlewey} on \anchor[http://ewey.co/B242]{Mon, 19 Apr 2004 21:12:45 +0000}\\
	\categories{usenet, opinion}\\
	Shorthand: \anchor[http://blog.chlewey.net/2004/04/reeligiendo/]{reeligiendo}
\end{metadata}

\begin{address}
\anchor[http://groups.google.com/group/soc.culture.colombia/msg/4eec45489456d848]{news:c61fcf\$6ssi2\$1@ID-83976.news.uni-berlin.de}
\end{address}
Uribe quiere que lo reelijan.  Finalmente lo dijo claramente después de que muchos de sus congresistas amigos lo
 hubieran propuesto.

Los puntos a favor de que el congreso pase la enmienda consititucional que permita la reelección incluyen el mismo
 deseo ciudadano; la libertad de elegir (que podamos elegir a quien sí nos gustó para que repita el mandato), etc.

Los puntos en contra... bueno, está la historia latinoamericana reciente y los últimos cien años de historia colombiana.

Comenzaré con los ejemplos:

Hugo Chávez Frías cambió la constitución para que lo pudieran reelegir... y lo reeligieron...

Alberto Fujimori cambió la constitución para que lo pudieran reelegir... y lo reeligieron...

Carlos Menem cambió la constitución para que lo pudieran reelegir... y lo reeligieron...

Fernando Cardoso cambió la constitución para que lo pudieran reelegir... y lo reeligieron...

Bueno...

Rafael Reyes cambió la constitución para que lo pudieran reelegir al siguiente periodo... y tuvo que salir con el rabo
 entre las piernas antes de que pudieran reelegirlo.

Gustavo Rojas Pinilla cambió la constitución para prolongar su mandato...

Al único presidente colombiano reelegido en el siglo XX, Alfonso López Pumarejo, sin que tuviera que cambiar la
 constitución porque en esa época se permitía la reelección mientras esta no fuera inmediata... no pudo terminar su
 segundo período.

En conclución, a los colombianos y a los latinoamericanos no nos ha ido bien intentando reelegir presidentes.  (Cardoso
 medio se salva.)

¿Por qué el caso de Uribe sería diferente?

Bueno, porque Uribe sí le está ganando a la guerrilla (lo mismo decían de Fujimori o de Rojas).

Bueno, porque Uribe trabaja en lugar de hacer lo de los políticos de siempre (hmmm. ¿no decían eso mismo de Reyes y de
 Rojas?, ¿no eran Fujimori y Chávez también antipolíticos?)

Pero, por otro lado... ¿Por qué quien mustre ser un buen presidente no puede ser premiado con un nuevo mandato?

Bueno, tengo algunos puntos de vista, pero me gustaría primero escuchar lo que otros piensan...

\chapter{Odio a tanto odio}
\begin{metadata}
	Published by \anchor[chlewey]{chlewey} on \anchor[http://ewey.co/B7]{Tue, 09 Oct 2007 13:10:00 +0000}\\
	\categories{usenet, facebook, odio, opinion}\\
	Shorthand: \anchor[http://blog.chlewey.net/2007/10/odio-a-tanto-odio/]{odio-a-tanto-odio}
\end{metadata}

Odiamos a los motociclistas y a los taxistas.  Odiamos a Victor Hugo Aristizabal y a Gustavo Petro, y al gobierno de
 Uribe.  Odiamos a los toreros y a quienes no logran comprender que los animalitos también sufren.  Nos da rabia, nos
 emputa... en fin.

\par% p
En Facebook hay \anchor[http://www.facebook.com/s.php?q=\%22odio\%20a\%22\&k=200000010]{más de 500 grupos} con la frase ``odio a''.  Una buena proporción de los mismos son grupos colombianos. ¿Por qué tanto odio? ¿por qué tanto
 empute?

\par% p
Hace varios años descubrí en internet los grupos de discusión y desde hace cerca de ocho años empecé a participar
 activamente en \anchor[news:soc.culture.colombia]{soc.culture.colombia} así como en otros foros en \anchor[http://es.wikipedia.org/wiki/Usenet]{Usenet}. No diré que eran un paraíso, pues no lo eran.  Las diferencias y las discrepancias eran evidentes y vicerales, y
 expresadas por personas de todos los pelambres y niveles culturales.  Pero cada grupo tenía una serie de \relax{% {'style': 'font-style: italic;'}
habituales} que subía el nivel del discurso, que desbarataban con buenos argumentos o con sarcasmo los argumentos del contrario.
 Al final, aquellos cuyo único argumento era el odio terminaban por aburrirse e irse.

Pero de un tiempo para acá me encuentro con foros como los de El Tiempo y con los grupos creados por colombianos en el
 Facebook, por poner sólo dos ejemplos.  ¡Qué diferencia!

Supongo que en gran medida Usenet partió del mundo académico, y el formato facilita la discusión detallada de cada uno
 de los puntos de un mensaje, y eso logra que, ante un montón de basura, se encuentren verdaderas perlas de discusiones
 interesantes apoyando una u otra postura.  Los blogs tienen otro formato y, aunque creo que en su gran mayoría son
 profesionales quienes los usan, el rigor académico ha sido dejado de lado porque, entre otras cosas, no es fácil
 aplicarlo.

\par% p
...Y Google ya convirtió a Usenet en \anchor[http://groups.google.com/]{otro blog}.

Todavía me sorprende ver cómo los grupos y blogs colombianos están cargados de odio y de deseos de muerte y, sobre
 todo, cuando quienes participan parecen ser profesionales.  Pero, tal vez estoy equivocando la demografía y no sean
 profesionales sino adolecentes de colegio quienes compongan el grueso de tanto odio y tanto empute.  Un par de
 incidentes me hacen pensar que esta teoría puede no estar equivocada.  Pero eso es aún más preocupante.

¿Será posible que nuestros profesionales y estudiantes de colegio aprendamos a argumentar sin tanto empute y tanto
 odio?  ¿A expresar nuestras discrepancias con citas, evidencias y sarcasmo en lugar de con deseos de que maten a
 nuestros contrincantes?  ¿A imponer nuestros puntos de vista a punta de lenguaje cuidado y no de groserías?  Hay cosas
 en las que me muestro optimista... pero, desafortunadamente, esta no es una de ellas.

\chapter{Promocionando a Wikimedia y OLPC en Colombia}
\begin{metadata}
	Published by \anchor[chlewey]{chlewey} on \anchor[http://ewey.co/B8]{Wed, 17 Oct 2007 19:40:00 +0000}\\
	\categories{olpc, personal, wikimedia, wikipedia}\\
	Shorthand: \anchor[http://blog.chlewey.net/2007/10/wikimedia-y-olpc/]{wikimedia-y-olpc}
\end{metadata}

\par% p
Detrás de mi familia, mi trabajo y la construcción de mi red social en el Facebook (¡maldito vicio!), algunas de mis
 actuales planes incluyen la fundación del capítulo colombiano de la \anchor[http://wikimediafoundation.org/]{Wikimedia Foundation} (\anchor[http://meta.wikimedia.org/wiki/Wikimedia\_Colombia]{Wikimedia Colombia}), la promoción informal de \anchor[http://www.laptop.org/]{OLPC} y el XO \anchor[http://www.colombiaolpc.org/]{en Colombia}, y una cruzada \anchor[http://www.facebook.com/group.php?gid=4663039183]{en contra del voto útil}.

\par% p
Los tres son proyectos políticos aunque de naturalezas disímiles.  Dos de ellos están relacionados con la educación.
 El otro... nada (usando esa muletilla que tienen muchos de los entrevistados en \anchor[http://www.radionica.gov.co/index.php?option=com\_content\&task=view\&id=50\&Itemid=70\&lang=es]{Días de radio}).

¿Qué tan conveniente son Wikimedia y OLPC en Colombia?

\relax{% {'style': 'font-weight: bold;'}
Wikimedia} es la fundación que tiene a cargo a \anchor[http://es.wikipedia.org/]{Wikipedia} y otros proyectos similares que buscan la difusión de contenidos educativos libres: \anchor[http://es.wikinews.org/]{Wikinoticias}, \anchor[http://es.wikiquote.org/]{Wikicitas}, \anchor[http://es.wiktionary.org/]{Wikcionario}, \anchor[http://es.wikibooks.org/]{Wikilibros}, \anchor[http://es.wikisource.org/]{Wikisource}, \anchor[http://commons.wikimedia.org/]{Commons} y la \anchor[http://es.wikiversity.org/]{Wikiversidad}, entre otros, entendiéndose por libres a contenidos educativos que pueden ser usados y redistribuídos con pocas
 restricciones, salvo aquellas que garanticen que los contenidos sigan siendo libres.

\relax{% {'style': 'font-weight: bold;'}
OLPC} (One Laptop Per Child) es una fundación que busca desarrollar computadores portátiles a prueba de niños que permitan
 que los niños pobres tengan acceso a la tecnología y aprendan a través de ella.  Su producto es el XO, conocido antes
 como el láptop de 100 dólares.  A diferencia de programas como \anchor[http://www.computadoresparaeducar.gov.co/]{Computadores para educar} (CPE), OLPC no recicla los computadores viejos que las empresas sacan como basura, sino que desarrolla productos
 totalmente nuevos.

Más que compatibles, estos proyectos son complementarios en muchos aspectos.  Los contenidos que todos nosotros podemos
 crear a través de Wikipedia y los demás proyectos de Wikimedia serán sin duda la base para que nuestros niños,
 utilizando el XO o los computadores reciclados, tengan acceso a una buena fuente de información que no costará más que
 el acceso a la misma, y OLPC ofrece un buen medio para acceder a esta información.

\par% p
Pero el desafío sigue siendo grande.  No hay, de momento, la voluntad política (ni el conocimiento suficiente de parte
 de nuestros políticos) para que el gobierno colombiano (o patrocinadores privados) invierta 20 millones de dólares en
 traer los primeros cien mil XO.  Todavía no encuentro los datos de cuánto han costado los \anchor[http://www.computadoresparaeducar.gov.co/cpe\_en\_cifras.html]{casi cien mil} computadores asignados por CPE en siete años de operación.  Claro que hay que reconocer que mientras OLPC busca un
 computador por cada niño (cada portátil beneficiaría a un solo niño), los PC reacondicionados por CPE servirían cada
 uno a más de diez niños sin los riesgos de convertir a nuestros niños pobres en blancos de la delincuencia.

\par% p
Y aún faltan los contenidos y las aplicaciones.  Sobre las aplicacionesya hay gente trabajando en \anchor[http://www.el-directorio.org/Squeak/OLPC]{Squeak} y otras herramientas.  Como editor de Wikipedia, Wikilibros y la Wikiversidad sé que aún faltan contenidos libres
 suficientes para Colombia y, principalmente, para los niños de Colombia, y es aquí donde un capítulo de Wikimedia en
 Colombia podría ser impulsor de la iniciativa.

\par% p
Ahora: ¿Cuanto estás dispuesto a invertir en los niños de Colombia? ¿\anchor[http://www.computadoresparaeducar.gov.co/donantes.html]{Un computador usado}? ¿\anchor[http://www.xogiving.org/]{399 dólares}? ¿\anchor[http://es.wikipedia.org/wiki/Ayuda:C\%C3\%B3mo\_puedo\_colaborar]{tu conocimiento}? ¿\anchor[http://www.colombiaolpc.org/?q=node/12]{tus habilidades}? ¿tu tiempo? ¿tus contactos?

¿Qué esperas a cambio? ¿ciudadanos y consumidores críticos? ¿el bienestar de tu pueblo? ¿deshacerte de lo que te sobra?
 ¿reconocimiento? ¿una carrera política?

¿O simplemente no te interesa?

\chapter{¿El voto en blanco o una tercera opción?}
\begin{metadata}
	Published by \anchor[chlewey]{chlewey} on \anchor[http://ewey.co/B10]{Thu, 25 Oct 2007 15:59:00 +0000}\\
	\categories{opinion}\\
	Shorthand: \anchor[http://blog.chlewey.net/2007/10/voto-en-blanco/]{voto-en-blanco}
\end{metadata}

Si los dos punteros no convencen completamente a ti y a gran parte del electorado ¿debemos buscar una tercera opción o
 votar en blanco?

Personalmente Enrique Peñalosa me parece un candidato serio.  Es serio con lo que plantea y con la visión de ciudad que
 quiere.  No me molestaría que repita alcaldía.  Sin embargo, Enrique Peñalosa se dejó meter en una campaña de guerra
 sucia ante su principal contendor, el Anapista Samuel Moreno Rojas, y de esta guerra y su posición fuerte que en este
 país confunden con arrogancia, Peñalosa ha perdido muchos votos que difícilmente se recuperarán.

\par% p
No me gusta de su propuesta el enfoque casi exclusivo a TransMilenio sin plantear de fondo la reorganización del
 transporte público colectivo y la exploración hacia un verdadero sistema de transporte masivo sobre rieles, ni la
 continuidad de su política restriccionista respecto al carro particular.  Pero no me preocupa del todo.  En cuanto a
 los colegios por concesión, no entiendo la alharaca.  Aquí parece que la palabra \relax{% {'style': 'font-style: italic;'}
privatizar} es pecado porque ``las cosas dejan de ser nuestras''.  Personalmente entre Cafam y Fecode prefiero a los primeros como
 dueños de la educación, aunque, tal vez mi concepto esté viciado por haber estudiado con los Hermanos de las Escuelas
 Cristianas, quienes con colegios para los más pudientes como La Salle subsidian educación de calidad para otros
 estratos en colegios como el Instituto Técnico Central o el San Bernardo.

Pero si tu modelo de ciudad e inclusión social se orienta más al asistencialismo que a la construcción de espacios,
 entiendo que no te guste Enrique Peñalosa.

El actual favorito es Samuel Moreno Rojas quien promete la continuación de la política social de Lucho Garzón, la
 construcción del Metro y el fin de las restricciones al carro particular.  No me disgusta ninguna de estas tres
 propuestas pero los mecanismos de financiación no han sido claros.  No hablaré de sus 50 votos, ya que no creo que
 podamos definir toda su carrera con base en una pregunta algo capciosa, respondida al filo de la media noche el día de
 cierre de campaña.  Pero lo que si me preocupa es que en su hoja de vida como congresista no figure un proyecto serio
 y que la actuación de su hermano en la alcaldía de Bucaramanga, de la cual Samuel fue asesor de campaña haya sido
 considerada desastrosa.

No veo una propuesta de ciudad en la campaña de Moreno y sí muchos intereses en quienes lo apoyan.  Al detener el
 desarrollo de TransMilenio mientras se define un metro que no se va a dar de la forma planteada quienes se benefician
 son los transportadores tradicionales y no los transportados que somos nosotros.

\par% p
Si compartes mis dos preocupaciones, tanto por Moreno como por Peñalosa, tienes varias opciones para votar:

\begin{enumerate}

\item Escoger el menos malo de ellos dos y darle tu voto, con la esperanza de que el otro no llegue a la alcaldía.
\item Buscar la tercera opción con mayor probabilidad: William Vinasco Ch.
\item Votar en blanco (o abstenerte de votar).
\item Revisar la hoja de vida de los otros tres candidatos y darle tu voto de confianza, así estés seguro de que no quedará.
\item Revisar la hoja de vida de los otros tres candidatos y tras convencerte de que ninguno de los seis sirve, entonces
 votar en blanco (o abstenerte... o retomar las opciones 1 o 2).

\end{enumerate}

Por convicción no creo en las tres primeras alternativas (doy una concesión a la alternativa 2 si esta vino de una
 reflexión sobre el candidato Vinasco y no de votar por él sólo por que no es ni Moreno ni Peñalosa).

Votar por el menos malo, casi siempre significa que estamos votando por alguien a quien consideramos malo y, si gana y
 aquellos aspectos que nos hacían dudar se manifiestan, surgen los arrepentimientos.  Pero la principal razón por la
 que no aconsejo esto es porque se evidencia una falta de cultura democrática en la cual esperamos a que sean las
 encuestas y los medios de comunicación quienes nos digan por quién votar.  No somos así dueños de nuestro voto sino
 que se los estamos vendiendo a los formadores de opinión y, muchas veces, sin siquiera recibir una contraprestación
 económica a cambio.

Si tu única razón para votar por Vinasco es que no es ni Moreno ni Peñalosa, estamos ante el mismo dilema.  No estamos
 pensando a quién damos nuestro voto sino que, a partir de a quién no se lo queremos dar, dejamos que las encuestas
 decidan por nosotros.  Ahora, si realmente te gusta William Vinasco Ch. o sus propuestas, pues vota por él.
 Personalmente no me gusta su propuesta de legalizar las ventas callejeras o ampliar la restricción vehicular
 (temporalmente, pero así han sido todas las anteriores restricciones y aumentos: temporales).  Pero eres libre de
 pensar distinto que yo.

Si no te gustó Peñalosa, Moreno o Vinasco, es muy probable que te quedes ahí ¿no?  Votas por el menos malo, o por el
 que tiene menos posibilidades con tal de no contribuir con tu voto a elegir a alguien que no te gusta, o símplemente
 pasas a la siguiente etapa: el voto en blanco o no votar.

Hay algunas diferencias entre el voto en blanco y no votar (p. ej. abstenerte, votar sin marcar, o anular
 concientemente tu voto), pero legalmente sólo si el voto en blanco supera al mejor de los candidatos hay consecuencias
 jurídicas.  Sólo repito, espero que tu voto en blanco o tu no voto no sea una consecuencia de que agotaste tu análisis
 democrático en solo dos o tres candidatos, aquellos que los encuestadores te dictaron.  Hay más candidatos y si por
 pereza mental no quisiste enterarte de los que proponían debes ser consiente de que tu abstención o tu voto en blanco
 corresponde a tu falta de análisis y no a una decisión responsable.

\par% p
Pasemos a los otros tres: Antonio Galán, Juan Carlos Flórez y Jorge Leyva.  De Leyva conozco muy poco, pero tanto Galán
 como Flórez no son aparecidos en la escena política bogotana.  Ambos han sido concejales y se han destacado como
 tales.  Ambos han estudiado la ciudad y se han preparado para ser alcaldes.  Sus propuestas, junto con las de Leyva,
 están disponibles:

\begin{itemize}

\item \anchor[http://www.tejiendocomunidad.com/]{Tejiendo Comunidad} (Antonio Galán)
\item \anchor[http://www.juancarlosflorez.com/]{Juan Carlos Flórez}
\item \anchor[http://www.jorgeleyva.com/]{Jorge Leyva}

\end{itemize}

Ahora, ya no tienes excusa.  Ya comparaste todas las posibles terceras opciones, y si estás seguro de que ninguno de
 esos seis candidatos reúne lo que esperas para alcalde, entonces ahí si tienes mi bendición (que nunca necesitaste)
 para votar por el menos malo de los dos punteros, el menos opcionado de los tres punteros, el voto en blanco, el voto
 al azar o, simplemente no votar.

Vota a conciencia.  La mejor forma de no votar tu voto es pensando que tu voto es tuyo y lo diste a quien consideraste
 la mejor opción.

\chapter{¿Nos tomaron Chávez y las guerrillas?}
\begin{metadata}
	Published by \anchor[chlewey]{chlewey} on \anchor[http://ewey.co/B11]{Wed, 31 Oct 2007 16:14:00 +0000}\\
	\categories{farc, guerrilla, izquierda, opinion, polarizacion, samuel-moreno, uribismo}\\
	Shorthand: \anchor[http://blog.chlewey.net/2007/10/nos-tomaron/]{nos-tomaron}
\end{metadata}

\par% p
Manifiesto primero que no voté por \anchor[http://www.samuelalcalde.com/]{Samuel Moreno Rojas} ni por el \anchor[http://www.polodemocratico.net/]{Polo Democrático Alternativo}, y ese candidato y ese partidos no eran siquiera los segundos o terceros en mi lista, por muchas causas que incluyen
 que no los creo las personas más idóneas para continuar manejando la ciudad en la que vivo.

\par% p
En parte desconfío de Moreno Rojas porque muchas de sus propuestas no son aterrizadas.  Yo sí creo factible un sistema
 de \anchor[http://es.wikipedia.org/wiki/Metro\_\%28ferrocarril\%29]{tren metropolitano} en Bogotá, pero no con la ligereza de cifras y argumentos que Moreno exponía.  También concuerdo en que Bogotá
 necesita integrar todos sus sistemas de transporte público y en que la ampliación del \anchor[http://www.elnuevodorado.com/]{Aeropuerto Eldorado}, se está quedando corta.  Pero no veo al Polo tomando una posición firme frente a los transportadores o aprobando los
 impuestos que se necesitan para la adecuación real del aeropuerto, el arreglo y ampliación de las calles, o la
 construcción del metro.

\par% p
Pero una cosa es no creer que Moreno y su partido sean los adecuados y otra muy distinta es sugerir, como he escuchado,
 incluso de parte de nuestro \relax{% {'style': 'font-style: italic;'}
amado} presidente o sus cercanos colaboradores, que por medio del Polo la guerrilla se va a tomar a Bogotá y al país.

\par% p
Las pruebas son sin duda espurias.  Se basan en leer a \anchor[http://www.anncol.nu]{ANNCOL} y extrapolar de ahí hipótesis poco sustentadas.  A ANNCOL hay que leerla con cuidado porque es un órgano de propaganda
 y como tal miente y desfigura la realidad de la misma forma que los medios de propaganda del estado.  Creer
 selectivamente lo que dice ANNCOL o selectivamente buscar ahí pruebas de comportamientos o actitudes no es serio.  Por
 análisis selectivo me refiero a que si ANNCOL dice que el gobierno es paramilitar asumimos que es propaganda falsa
 pero si dice que el Polo es afín a la lucha guerrillera, entonces ahí sí están diciendo la verdad, y así apoyar la
 tesis de que los del Polo son guerrilleros de amarillo.

Que las FARC digan preferir a Samuel Moreno Rojas como alcalde de Bogotá no dice nada bien o mal de Samuel Moreno
 Rojas.  Pero si queremos partir de ahí para asegurar que Moreno tiene vínculos con las FARC, estamos entonces
 utilizando la misma lógica de quienes insisten que Álvaro Uribe Vélez tiene vínculos con los paramilitares porque
 Mancuso o Castaño se hayan expresado a favor de su elección o reelección.  Es más, la tesis del Uribe paramilitar
 tiene mucho más sustento que la tesis del Moreno guerrillero, porque no se basa sólo en las declaraciones de los
 milicianos ilegales, sino en afinidades geográficas y de clase.  Si usted y yo estamos de acuerdo en que Álvaro Uribe
 no es un paramilitar, debemos reconocer entonces que las acusaciones de Samuel Moreno como guerrillero son simplemente
 espurias.

Y esto si es que realmente creemos al leer a ANNCOL que las FARC prefieren a Samuel Moreno Rojas y su Polo Democrático
 Alternativo.

\par% p
La otra gran prueba del gobierno, que todos los furibistas siguen al pie de la letra sin cuestionar, es una vieja tesis
 de \anchor[http://www.polodemocratico.net/\_Carlos-Gaviria-Diaz\_]{Carlos Gaviria Díaz}, cuando fue Magistrado de la Corte Constitucional, en la que permitía la conexidad entre el homicidio y la rebelión
 para tratar al homicidio político como delito político.  Bueno, aquí habría que tener en cuenta un poco de filosofía
 para atacar esta idea por la idea en sí, y a continuación expongo mi punto de vista:

\begin{enumerate}

\item Los estados son necesarios.
\item Todo estado debe evitar la disensión que pueda llevar a su fin como estado, o no sería estado.
\item Todo modelo de estado tiene fallas; varias de esas fallas son subsanables por los métodos que el mismo estado ofrece.
\item Habrá fallas que no sean subsanables por este método, hay ahora un conflicto entre la falla del modelo y la necesidad
 del modelo de autosustentarse.
\item Esto da lugar al derecho legítimo (que no legal) a la rebelión.
\begin{itemize}

\item Nota aclaratoria: esto no es excusa para proponer una rebelión cada vez que las fallas del modelo puedan ser subsanadas
 por el modelo mismo.
\item Nota aclaratoria dos: tan legítimo es el derecho a la rebelión como el derecho del estado a preservarse.
\item Nota aclaratoria tres: hay varias rebeliones que sustentan al estado actual en Colombia, incluyendo las de 1957, 1953,
 1886 y 1819, entre muchas otras.  Así que nuestro estado es tan legítimo como lo puede ser el derecho a la rebelión.

\end{itemize}

\end{enumerate}

La rebelión es así un acto político que se legaliza una vez triunfe o llegue a un acuerdo con el estado previo.
 Igualmente, la legitimidad de cada rebelión debe ser relativa a la incapacidad del estado de reformarse a si mismo.  Y
 antes de que me ataquen de guerrillero debo aclarar que esta es una posición filosófica y no estoy en ningún momento
 legitimando aquí a las FARC o al ELN.

Asumiendo así que la rebelión puede ser legítima y que es un acto político, varios de los medios utilizados por la
 rebelión son sin duda conexos a la rebelión, y esto puede incluir el asesinato.

\par% p
Hay otro planteamiento filosófico que expone el gobierno y sus seguidores y que deslegitima de raíz el derecho a la
 rebelión.

\begin{enumerate}

\item Colombia es un estado de derecho democrático.
\item El modelo (estado de derecho democrático) está libre de fallas conceptuales, porque es el único tipo de estado en el
 cual las fallas de implementación son subsanables por el mismo modelo conceptual.
\item Por lo tanto no existen conflictos: cualquier cosa que amenace al estado de derecho democrático es ilegítima.
\item No existe por ende un derecho a la rebelión.  Sólo amenazas al estado de derecho democrático.

\end{enumerate}

Mi punto aquí no es, sin embargo, indicar que mi filosofía es más adecuada que la del gobierno, sino puntualizar las
 fallas en la lógica misma del gobierno y de sus seguidores.

El gobierno, a través de la ley de Justicia y Paz, pretendió vendernos la idea de que la lucha de los paramilitares era
 una lucha política y por lo tanto sus acciones eran acciones políticas cobijadas bajo el delito de sedición.  Cuando
 la Corte Constitucional falló en contra de considerar al paramilitarismo como sedición, el gobierno ha movido sus
 maquinarias para lograr que se tipifique la sedición de manera que favorezca a los paramilitares y pueda perdonar así
 a los delitos conexos, incluyendo al homicidio.

Es decir: el gobierno reconoce la conexidad del homicidio con la sedición, que al ser un delito político es indultable,
 y por otro lado promueve que se trate al paramilitarismo como sedición (y por ende como un delito político indultable
 que puede cobijar a otros delitos denominados conexos); por otro lado el gobierno sataniza a quien expone que un
 delito, conexo en principio con la rebelión, siendo la rebelión el delito político por excelencia, es efectivamente un
 delito conexo al delito político.

Esta lógica asimétrica del gobierno, que sus seguidores siguen sin cuestionar, me parece más peligrosa que la lógica
 asimétrica de la guerrilla, por que al fin y al cabo este gobierno encabeza al estado que yo considero legítimo y sus
 seguidores son las personas con las que yo convivo a diario.

Si esta es toda la evidencia que el gobierno puede ofrecer, aún asumiendo que sea cierta, entonces los vínculos entre
 el Polo Democrático Alternativo y las Fuerzas Armadas Revolucionarias de Colombia son simplemente especulación.

Viene ahora la cuestión de la credibilidad y de los intereses de ANNCOL y las FARC.  La Agencia de Noticias Nueva
 Colombia, ANNCOL, es una organización domiciliada en Suecia y Dinamarca que reproduce los comunicados de las FARC.  Es
 parcialmente independiente de las FARC en el sentido de que su línea editorial puede publicar también (y lo hace)
 noticias y artículos de otras fuentes, pero claramente publica sólo lo que es afín a su línea editorial: justificar la
 existencia y la lucha de las FARC.

Así, si ANNCOL encuentra un comunicado, una tesis, o un artículo en la que un filósofo o un magistrado ya retirado
 expone la conexidad entre el homicidio y la rebelión, no dudará en publicarla, porque eso justifica a las FARC.  Esto
 no demuestra, en punto alguno, que Carlos Gaviria Díaz haya entrado en contacto con las FARC y les haya pedido que
 publicaran su artículo en ANNCOL.

Por otro lado ANNCOL, también publica artículos de otros pensadores de izquierda.  Desde luego que hay afinidad entre
 alguien que justifica a la guerrilla que lucha contra un estado opresor (sus palabras) con la línea política de una
 oposición democrática a tal estado opresor.  Mal que bien, el Polo es percibido dentro y fuera del país, y tanto por
 aúlicos como por detractores del gobierno, como la oposición al gobierno de Uribe.   Que un colaborador de ANNCOL
 decida expresar que la alcaldía de Samuel Moreno conviene a Bogotá y a Colombia, no dice absolutamente nada de las
 afinidades de Moreno Rojas con las FARC.  Incluso si ese colaborador de ANNCOL fuesen las FARC mismas.

Pero hay que recordar que ANNCOL es un medio de propaganda, y hay que mirar con sal sus comunicados y artículos.

Las FARC son las primeras interesadas en que el Polo Democrático Alternativo fracase, y podemos recordar un poco de
 historia.  Bajo la tregua de Belisario Betancur, se abrió la posibilidad de que las FARC formaran un partido político
 a modo de opción (o brazo si así prefieres llamarlo) democrática.  Así se formó la Unión Patriótica bajo el comando
 inicial de Jacobo Arenas, líder ideológico de las FARC.  La Unión Patriótica no sólo aglutinó a militantes de las FARC
 sino a otras personas que se identificaban con la línea política de las FARC, más no con la rebelión armada, y entre
 estas personas surge Jaime Pardo Leal.   Jaime Pardo, se opuso a la dirigencia de Jacobo Arenas porque buscaba una
 independencia entre la Unión Patriótica y las FARC, y la línea de Pardo triunfó.  Pero Pardo Leal fue asesinado, como
 más tarde lo serían José Antequera y Bernardo Jaramillo, y muchos de los líderes regionales de la UP.

La masacre de la UP, propiciada por Gonzalo Rodriguez Gacha y los paramilitares bajo su mando, y apoyada por fuerzas
 políticas del estado, marcó un cambio en los objetivos de las FARC, que pasaron así de una guerrilla romántica a la
 fuerza combativa que tuvo en jaque al país a finales de los años 1990.  Esa degradación de la lucha guerrillera es tan
 causa de los paramilitares como de las mismas FARC.   El exterminio de la Unión Patriótica le probó a las FARC que no
 hay alternativa real de diálogo con el estado colombiano salvo el desmonte del mismo estado.  Es por ello que fallaron
 las conversaciones de Andrés Pastrana y es por ello que las FARC ni se molestan en conversar con Álvaro Uribe Vélez.

Pero las FARC necesitan todavía autolegitimarse.  Y el Polo Democrático Alternativo les resta legitimidad.  Si el Polo
 es capaz de traer (por si mismo o en alianza con las derechas del país), la justicia social y la consecución de las
 otras banderas de lucha de las FARC, sin necesidad de la rebelión armada, las FARC pierden entonces su justificación
 de ser.  Si no tienen por qué luchar, y por otro lado tienen un legítimo temor a desmovilizarse, entonces las FARC
 pierden sentido, no tienen salida ni futuro.

\par% p
Muchas veces se ha llamado a los izquierdistas democráticos \relax{% {'style': 'font-style: italic;'}
idiotas útiles} de la izquierda radical, porque prefieren recalcar los problemas del estado a condenar los desmanes de la guerrilla, y
 en últimas a justificar a la guerrilla y sus desmanes como consecuencias de las inequidades del estado.  Pues bien,
 ahora los \relax{% {'style': 'font-style: italic;'}
idiotas útiles} de las FARC son los propios uribistas, quienes al rechazar a la izquierda democrática, de una forma tal que bien nos
 acerca a lo que pasó en los años 1980 con la UP, justificarán ante la comunidad internacional la existencia de las
 propias FARC.

Necesitamos cerrar la brecha entre el uribismo y la izquierda democrática, porque esta es la única forma de salvar a
 Colombia y de deslegitimar a los radicales.  Esa brecha se cierra reconociendo que Samuel Moreno Rojas no es un
 guerrillero, y que Gustavo Petro dejó de serlo hace ya varios años.  Esa brecha se cierra buscando que por medio del
 control político Samuel Moreno Rojas no arruine a Bogotá y no buscando, junto con las FARC, que el Polo fracase.

\chapter{Modelos de estado}
\begin{metadata}
	Published by \anchor[chlewey]{chlewey} on \anchor[http://ewey.co/B12]{Wed, 28 Nov 2007 20:39:00 +0000}\\
	\categories{derecha, information, izquierda, libertad}\\
	Shorthand: \anchor[http://blog.chlewey.net/2007/11/modelos-de-estado/]{modelos-de-estado}
\end{metadata}

Si hubiera una sola visión válida de la vida todo sería sencillo y el juego político simplemente se ceñiría a ella,
 pero la vida es muy compleja y los intereses de cada uno de nosotros se complementan, se alinean, chocan o nos separan
 de los intereses de nuestros semejantes.

Estoy con un compañero de viaje por la campiña y vemos, tal vez yo primero, una manzana en lo alto de un árbol.  No
 tengo hambre pero me provocó la manzana, trepo y la bajo, y la guardo para más tarde.  Mi compañero, por el contrario
 no hizo nada para conseguir la manzana, lo más probable fue porque fui yo quien tomó la iniciativa y el sólo observó,
 o tal vez porque estaba débil.  Observó con hambre pues hacía tiempos no comía.  ¿Quién tiene más derecho a esa
 manzana? ¿él que la necesita pero no la pudo conseguir o yo que hice el esfuerzo?  Tal vez yo debería ser generoso y
 dársela, pero ese debe ser un acto mío libre y no una imposición.

Aquí tenemos un dilema ético y un dilema político. Cada uno de nosotros tiene intereses y cada uno de nosotros cree
 tener el derecho de hacer valer sus intereses.  Algunos hemos trabajado por la manzana, otros la necesitan para paliar
 su hambre, y hay ocasiones donde el trabajo de unos impide a los otros obtener de primera mano sus beneficios
 esperados.  Un sistema político puede tender a favorecer al que adquiere derechos por sí mismo, otro puede preferir
 favorecer al más necesitado. El primero favorece la desigualdad; el segundo se impone sobre nuestras libertades.  En
 principio ninguno de los dos sistemas políticos es más o menos moral, simplemente se basan en principios distintos.
 Cada uno tiene sus defensores y sus detractores, y cada uno favorece a intereses diferentes de la población.

Hay otros tipos de conflictos: la tradición contra la innovación; la seguridad contra la libertad; la visión común
 contra el derecho a disentir.

Mucho se habla sobre los impactos del cambio en una organización.  Cuando las cosas no cambian se forman expectativas
 basadas en lo que vivimos.  Tenemos, por ejemplo, nuestros empleos y con ellos sostenemos a nuestros hijos y a
 nuestros padres y esperamos que cuando seamos viejos y no podamos seguir trabajando sean nuestros hijos quienes nos
 sostengan.  Pero nuestros hijos pueden tener otro plan de vida, tal vez impuesto por el estado que para garantizar la
 inequidad de esta situación interviene para que cada uno de nosotros trabaje su propia pensión.  Nuestros hijos no
 esperarán que sean nuestros nietos quienes los sostengan de viejos y no esperan tampoco tener que sostener a sus
 propios viejos.  Cuando se hizo el cambio ya estábamos muy viejos para tener nuestro propio ahorro y el estado no nos
 garantiza una pensión y nuestros hijos están pagando por sus propias pensiones.  El cambio, tal vez bien intencionado,
 afecta mis expectativas, y rompen con lo que yo esperaba.  Eso no es justo.  Pero continuar la tradición, en la cual
 los hijos cuidan de sus padres, no es justo tampoco con quienes se hacen viejos sin tener hijos que cuiden de ellos.
 ¿Tradición o innovación?  Algunos nos beneficiaremos de una y otros se beneficiarán de la otra.

Cuando todos profesamos una misma religión, podemos basar nuestro estado en ese sentimiento religioso y calcar los
 principios éticos de la religión en el funcionamiento del estado.  Pero cuando no todos profesamos la misma fe, o no
 todos entendemos esa fe de la misma forma, surgen nuevos problemas.  Religiones como el cristianismo da respuestas
 simples a inquietudes simples y da respuestas compleja a inquietudes más complejas.  Los dirigentes del estado y de la
 iglesia promueven la sumisión a la fé, pero los intelectuales y los teólogos nos recuerdan que en la prédica de Cristo
 nos invitaba a sobresalir y a escoger.  La misma religión, pero dos visiones enfrentadas.  Y en el sentimiento
 religioso, lo que para uno es un valor, para otros es una tara.

El creyente cree que el demonio enceguece al librepensador y no le permite ver la verdad de Dios.  El librepensador
 cree que el creyente está enceguecido por su propia creencia impuesta por la tradición y los líderes religiosos.  Es
 un conflicto que se hace evidente cuando definimos un modelo de estado.  El modelo secular, con separación de iglesia
 y estado, permite definir principios comunes para los no creyentes y para los creyentes de diferentes religiones, pero
 eso da pie también a un conflicto personal cuando cada uno de nosotros debe escoger entre la ley suprema de nuestro
 dios, o la ley de los hombres.

Y viene el asunto de la seguridad. Yo quiero disponer de mi manzana y para mí es una amenaza el estado que me impone
 compartirla con el vecino, pero también es una amenaza mi otro vecino que simplemente me la quiere robar.  Para mi
 vecino con hambre, el estado que le impide quitarme la manzana es una amenaza también.  El estado no sólo debe
 brindarme unas reglas claras que me permitan decidir si comparto o no mi manzana con mi vecino hambriento, también le
 exijo que me proteja de quien me la quiere arrebatar por fuera de los conductos legales. Al estado le exijo seguridad:
 seguridad contra los rateros, seguridad contra el crimen organizado, seguridad contra las amenazas naturales,
 seguridad contra una invasión extranjera, etc.

Pero mis otros vecinos también exigen seguridad para sus propios intereses, y si el estado debe suplir todas esas
 necesidades de seguridad para todos nosotros, en una sociedad enorme donde no todos nos conocemos personalmente, se
 corre el riesgo de que el estado nos trate a todos como sospechosos.  Ese estado al que yo le exijo seguridad, atenta
 contra mi libertad.  Y esto no sucede sólo a nivel de la relación entre el estado y sus súbditos, sino a nivel
 personal también.  Si yo quiero proteger mis intereses termino formando una jaula alrededor mío, pero esa misma jaula
 me impide salir con libertad.

Cada cara del prisma, cada dilema ético y social, es un dilema político.  La forma de resolver uno u otro problema van
 dando forma a cada estado y cada modelo de estado favorece a unos y perjudica a otros.  Yo puedo creer que lo ideal
 sería un estado más justo con todos, pero la justicia es un término relativo.  ¿Qué es lo justo?  ¿Favorecer los
 derechos adquiridos y las expectativas dictadas por la tradición? ¿o combatir la inequidad favoreciendo siempre al más
 necesitado?  Si la justicia siempre se pone del lado del más débil, no hay incentivos para el progreso personal: es
 más cómodo y más fácil ser ese débil o aparentar serlo.  Es mejor no destacarse.  Si por el contrario, la justicia
 favorece siempre a quien consigue las cosas para sí mismo, aumenta la desigualdad económica y crece la insatisfacción
 de los débiles quienes, ante una justicia que no los favorece, se sentirán titulados a tomar las cosas por sus propios
 medios.

No existen modelos únicos de izquierda y derecha.  estamos ante un prisma multifacético donde cada pregunta individual
 se encuentra en una dimensión distinta.  Algunos pensadores han definido un esquema de dos dimensiones basados en el
 grado de libertad económica y el grado de libertad social.  Un eje tiene que ver con qué tanto se mete el estado en
 nuestros propios bolsillos para imponernos gravámenes u otorgarnos subsidios, para permitirnos actuar o para regular
 nuestra actuación.  El otro eje muestra el tamaño del estado en otros asuntos tales como las libertades individuales.
 ¿Debe el estado imponernos una moral o dejarnos en libertad de que cada uno de nosotros encuentre esa moral?

Dos ejes son todavía insuficientes.  Dos estados pueden ser altamente intervencionistas en el aspecto económico y sin
 embargo muy disímiles: por ejemplo uno puede promover impuestos progresivos (a mayor riqueza mayor porcentaje que se
 impone en los gravámenes) y una serie de subsidios para compensar a los más pobres, otro estado prefiere esquemas
 impositivos más simples, pero se mete en el funcionamiento interno de las empresas definiendo al detalle una serie de
 regulaciones medioambientales y de responsabilidad social que debe cumplir para poder operar.  Ambos modelos buscan
 que los grandes generadores de capital (los ricos, las empresas) favorezcan a los menos favorecidos por medio de
 imposiciones, pero de formas muy diferentes que dan lugar a conflictos y a oportunidades diferentes.  No creo que
 estén al mismo sentido de un eje izquierda-derecha el modelo de estado que exige a una empresa que no contamine, a un
 estado que grava enormemente a los dueños de la empresa y reparte la contribución a los trabajadores y desempleados,
 ni mucho menos el estado que expropia la fábrica y la reparte entre los empleados.  Estas diferentes formas de
 intervenir en las libertades económicas no son necesariamente todas iguales.

No creo que haya un modelo único y perfecto de estado del cual todos los demás sean aberraciones. Cada uno de nosotros
 encontrará nuevas oportunidades y nuevos conflictos de intereses en cada modelo de estado.  Y cada uno de nosotros
 considerará injusto un modelo de estado que privilegie los intereses de los demás sobre los propios.

\chapter{Y después de Uribe ¿quién?}
\begin{metadata}
	Published by \anchor[chlewey]{chlewey} on \anchor[http://ewey.co/B14]{Thu, 29 Nov 2007 18:47:00 +0000}\\
	\categories{opinion, uribismo}\\
	Shorthand: \anchor[http://blog.chlewey.net/2007/11/despues-de-uribe/]{despues-de-uribe}
\end{metadata}

\par% p
Alguien hacía \anchor[http://www.facebook.com/topic.php?uid=11465590289\&topic=3203]{esta pregunta} a un contertulio de Facebook, seguida de una retaíla en la cual hablaba de que sólo Uribe había trabajado por el país
 y que nadie, y mucho menos los críticos al uribismo, estamos a la altura de dar una propuesta.

Bueno, personalmente creo que si Uribe no ha dado las condiciones para que el país siga sin él (y hablo del país, no
 sólo del uribismo), ese es un fracaso de la gestión de Álvaro Uribe Vélez.

Siempre he insistido que nuestro país requiere de institucionalidad, más que de caudillos. Lo ideal es que el estado,
 representado por instituciones, sea fuerte e invisible. Que el estado sea invisible pero que esté ahí. Que no nos
 moleste con trabas, pero que cuando lo necesitemos para solucionar un conflicto, esté ahí y podamos contar con él. Y
 este modelo de estado no se logra con un caudillo mesiánico. Es más, el caudillismo es la antítesis del estado fuerte
 e invisible: es personificar al estado en un sólo hombre que tendrá, por ello, que echarse todo el peso del país
 encima.

Pero creo, que a pesar de que muchas fuerzas del país estén tratando de condicionarnos a que un solo hombre es y
 seguirá siendo la solución (una solución que no soluciona nada estructural porque dejará de ser indispensable), hay
 todavía muchas personas que tienen el país en la cabeza y pueden reemplazar sin problemas a nuestro actual presidente.

Dentro del mismo uribismo, y entendiendo por uribismo la continuidad de la agenda pro-seguridad democrática, pro-TLC,
 pro-no negociación con el terrorismo de las FARC, pro-negociación con el terrorismo de las AUC, vemos a personas como
 Germán Vargas Lleras o el mismo Andrés Felipe Arias. Actualmente afines a Uribe pero evolucionando están Noemí Sanín y
 Juan Manuel Santos.

En la oposición o desde la independencia, pero con esperanza de continuar los logros de la seguridad democrática, pero
 con un sentido más inclusionista, podremos ver a Antanas Mockus, a Rafael Pardo, a Rodrigo Rivera, a Sergio Fajardo o
 a Enrique Peñalosa, entre otros. Ninguno de ellos desmontará lo bueno construido por Uribe, ni tendrá miedo de
 desmontar lo nocivo del actual gobierno.

Ideológicamente opuestos hay también figuras presidenciables tales como Carlos Gaviria y Lucho Garzón. Ninguno de ellos
 acabará el país y se lo entregará a Pedro Marín y compañía o a Hugo Chávez, porque aunque haya algunos acuerdos
 ideológicos tanto Gaviria como Garzón son constitucionalistas y seguidores de las reglas del juego, tal como lo han
 demostrado en sus carreras políticas. Tal vez extrañemos palabras duras de condena de Gaviria contra las FARC, pero la
 propuesta política de las FARC es completamente contraria a lo que ha sido la vida pública de Carlos Gaviria. Lo mismo
 sucede con respecto a Chávez.

Tras analizar las cabezas, podemos ver a quienes los acompañan. El Álvaro Uribe de 2006 fue acompañado a la presidencia
 por partidos como el Partido Social de Unidad Nacional (Social y Nacional... una combinación con no muy gratos
 recuerdos), Colombia Democrática, Cambio Radical (que alguien analizaba que ni es cambio ni es radical) entre otros.
 Particularmente en el partido de la U decantó la clase política tradicional que salió del partido Liberal. En 2006 la
 U tenía más lastre de la política tradicional que la L. Y en todo este uribismo, no sólo está representada la vieja
 política sino, particualmente, la así llamada Parapolítica. Si hay algo que el Uribe de 2002 nos incumplió fue
 precisamente librar a Colombia de la corrupción política y ahora gran parte de esa corrupción política está
 acompañando a Uribe.

Vargas Lleras no arrastra la totalidad del Uribismo, y parte de la carga de la que se desprende está gran parte de esa
 política tradicional, pero el tiene su propia tradición política heredada y de llegar a la presidencia todo ese lastre
 hoy uribista se volverá vargasllerista. (La verdad se volverá cualquier cosa: hasta amarillo en el caso de que Carlos
 Gaviria llegue al poder: la política tradicional no se nutre de la oposición sino de la lambonería.)

Andrés Felipe Arias también tiene oposición interna en el uribismo y por otro lado no tiene maquinaria propia. La
 verdad, lo único que tiene en común el uribismo es a Álvaro Uribe: retirado el caudillo hay muchos intereses
 personales. Pero lo mismo que con Vargas Lleras: si de aquí a dos años y medio Arias puntea las encuestas, veremos a
 todo el Uribismo apoyándolo. El mismo análisis vale para Sanín y para Santos o para cualquier otro político hoy dentro
 del uribismo.

La gran ventaja que tendría un uribista diferente a Uribe es que tiene una oportunidad de sacudirse de cierto lastre
 uribista y crear un equipo nuevo y quien mejor oportunidad tiene, en mi concepto, es Noemí Sanín.

Mockus y Fajardo no tienen muchas malas compañías. Igual vale para Peñalosa. Ellos tienen la ventaja de no ser ni
 uribistas y anti-uribistas (Mockus se definía en 2006 como post-uribista) y como tales pueden ser una oportunidad de
 reconciliación. Conozco mejor la carrera de Mockus y por ello me centro en él. Su gran ventaja: no posee lastre
 político. Su gran desventaja: no posee lastre político. No llegará mal acompañado, pero igualmente, tiene un trabajo
 muy arduo para ser visto por todo el país como el sucesor de Álvaro Uribe Vélez (sucesor en el país, no en el
 uribismo, repito).

Por los lados de Pardo y de Rivera, a su alrededor está el Partido Liberal. Personalmente y lo que me gusta, no es el
 partido liberal de Serpa y Córdoba, sino el partido liberal de Gaviria y el peñalosismo. Es el partido liberal que
 trata de convertirse en una opción de centro. El que no despotrica contra Uribe, el que es afín a continuar lo bueno
 de la seguridad democrática, a considerar las oportunidades económicas del país antes que la ortodoxia neoliberal o
 marxista, en últimas al Partido Liberal que es liberal, y no que pretende ser social-demócrata o uribista. Pero en
 últimas es la misma razón social que nos ha gobernado con Samper (hoy Uribista y Samuelista), con Gaviria, con Barco,
 con Turbay (quien murió Uribista y tratando de refundar a la Patria), con López, etc.

Y sin dudas, si el Partido Liberal se fortalece da aquí a dos años y una de sus cabezas (Pardo, Rivera, Peñalosa,
 Gaviria) se perfila como uno de los más opcionados contendientes en la carrera presidencial, podríamos ver a la U y a
 Cambio Radical regresando a la bandera roja del liberalismo.

En cuanto a Carlos Gaviria y Lucho Garzón, he dicho que son constitucionalistas y como tal, aunque no lo digan, no son
 afines a las FARC, pero dentro del Polo Democrático Alternativo sí hay sectores afines a esa guerrilla, y hay focos de
 corrupción política y de clientelismo. Tenemos dos años para ver cómo se comportan Samuel Moreno y Antonio Navarro en
 sus nuevos cargos para ver qué tanto pesa el partido y qué tanto sus antiguas y nuevas alianzas. De Antonio Navarro
 tenemos el buen antecedente de su gestión en la alcaldía de Pasto (pero esto fue antes de que el Polo se formara como
 tal); de Moreno Rojas, el menos afortunado antecedente de la gestión de su hermano en la alcaldía de Bucaramanga, y el
 hecho que que Garzón tuvo que gobernar en gran medida sin su propio Polo.

En conclusión, sí hay quien pueda reemplazar a Uribe. Y hay varias opciones para escoger, tanto en el uribismo como
 fuera de él. Tal vez sea claro que mi favoritismo está en el centro (Mockus, Pardo), pero a la derecha y a la
 izquierda también hay buenas opciones, y hay muchas más opciones de las que he mencionado: en Colombia hay mucha gente
 inteligente a las que le cabe el país en la cabeza y no pretenden convertirse ellos mismos en el país... y también hay
 políticos.

\chapter{Worst Case Scenario}
\begin{metadata}
	Published by \anchor[chlewey]{chlewey} on \anchor[http://ewey.co/B15]{Mon, 14 Jan 2008 14:18:00 +0000}\\
	\categories{alvaro-uribe, farc, fiction, futuro}\\
	Shorthand: \anchor[http://blog.chlewey.net/2008/01/worst-case-scenario/]{worst-case-scenario}
\end{metadata}

Este es un caso netamente hipotético y esperemos que el futuro de Colombia esté lejos del mismo.  Es, sin embargo, un
 escenario posible y plausible, y como tal es nuestro deber buscar los mecanismos para que esto no suceda.

En los EE.UU. están en plena campaña electoral y todos los candidatos, incluyendo los republicanos, quieren apartarse
 del fiasco de George W. Bush en Irak.  En el caso de los demócratas, los dos favoritos: Hillary Clinton y Barak Obama,
 han sido críticos del TLC con Colombia y quieren apartarse en lo posible de involucrarse directamente en una nueva
 lucha antiinsurgente, o de relaciones con gobiernos que no hacen lo suficiente para proteger a sus propios sindicatos.

En noviembre gana el tiquete Clinton-Obama (u Obama-Clinton).

Mientras tanto, la ofensiva de Chávez empieza a materializarse.  Si bien la Unión Europea no saca a las FARC de la
 lista de grupos terroristas, algunos gobiernos tales como el danés declaran que no tienen jurisdicción absoluta sobre
 cualquier subdito propio que financie a las FARC.  Ecuador, Bolivia, Nicaragua y otros países declaran que las FARC no
 son terroristas y les reconocen estatus de fuerzas beligerantes pudiendo tener representaciones en sus propios países.
  Desde luego: Venezuela encabeza la lista.

En respuesta, el gobierno colombiano rompe relaciones diplomáticas, bien oficialmente o bien de facto (llamado
 permanente a consultas a los embajadores), perdiendo así interlocusión con tales gobiernos.  Venezuela no sólo permite
 la presencia de representantes de las FARC sino que permite la instalación de campamentos permanentes. Ecuador llega
 también a un acuerdo con las FARC que les permite usar el territorio ecuatoriano con la condición de que no atenten
 contra ciudadanos o intereses de ese país.  El apoyo de Venezuela no se limita a la representación y al territorio
 sino que empieza a ofrecer armamento moderno a las FARC.

El gobierno continúa con los bombardeos y el acoso permanente a las FARC dentro de Colombia y ante la gravedad de la
 situación diplomática vuelven a aumentar los falsos positivos, producto de la presión del gobierno a las fuerzas
 militares.  Hay un ligero aumento en el paramilitarismo producto del desencanto ante la Ley de Justicia y Paz y varios
 sectores sindicales son víctimas de estos grupos emergentes o de los falsos positivos.  Aunque el ejecutivo tiene una
 real convicción de perseguir a los paramilitares emergentes y proteger a los sindicatos sus esfuerzos no son
 suficientes y la percepción es que el gobierno patrocina tales actos.  Nacionalmente la opinión pública justifica
 estos crímenes porque si los matan es porque están relacionados con las FARC.  Internacionalmente estos crímenes son
 inadmisibles.

Tras ese 2008, llega enero de 2009 con la posesión del tiquete demócrata.  Las denuncias de crímenes contra
 sindicalistas han venido aumentando y en conclusión no sólo no se aprueba el TLC sino que tampoco se renueva el APDEA
 ni el Plan Colombia.  Esto, aunado con el vaivén económico que se empieza a manifestar en 2008 y el cierre de las
 fronteras venezolanas para productos colombianos generan un impacto negativo en las industrias colombianas, las cuales
 pierden completamente la confianza en el gobierno de Álvaro Uribe Vélez.  Varias empresas quiebran, otras se reducen,
 y en general aumenta el desempleo.  No solo los empresarios sino todo el pueblo se desilusiona de Uribe.

En medio de la crisis, la ventaja estratégica y logística que logran las FARC manteniendo campamentos permanentes en
 Venezuela y Ecuador aunado con un mejor armamento, les permite tomar la iniciativa en algunos puntos, llegando incluso
 a obtener corredores seguros y control territorial real en algunas partes del territorio.  Misiles tierra-aire les
 permiten conjurar la supremacía aérea de las fuerzas constitucionales.

Los secuestrados ``canjeables'' corren una de tres suertes posibles: son entregados a representantes internacionales como
 muestras de buena voluntad, mueren en intentos o supuestos intentos de rescate (o por enfermedades), o continúan a
 todo lo largo de 2008 y 2009 como prisioneros de las FARC.  Al aumentar la capacidad de control territorial, las FARC
 recuperan la habilidad de lograr secuestros de políticos o militares.  Durante el 2009 las FARC mantienen una cifra
 más o menos estable de 20 a 25 canjeables, aún descontando bajas y entregas gota a gota.  Íngrid Betancourt muere en
 cautiverio por problemas de salud y sus hijas y su madre declaran en todo el mundo que la culpa es de Uribe por no
 acceder a un acuerdo humanitario.

Algunos golpes espectaculares de las FARC contra las fuerzas constitucionales, dejan la sensación de que el gobierno no
 es capaz de detener a las FARC.  Por otro lado, y gracias a las nuevas ventajas estratégicas, las FARC concentran sus
 objetivos contra las fuerzas militares constitucionales y representantes del estado.  La población civil puede temer
 el secuestro extorsivo pero no atentados directos contra la vida (p. ej. bombas).  Esta mayor selectividad, que busca
 recuperar imagen internacional, también le permite a las FARC mejorar ligeramente la imagen interna. el pueblo raso
 deja de ver a las FARC como una gran amenaza y ante la crisis económica que evidencia los desaciertos de la
 administración Uribe en ese campo, parte del discurso de las FARC empieza a permear a la población civil.

Llega 2010.  El país está en una crisis económica, las FARC han retomado la iniciativa.  El gobierno ha caído
 dramáticamente en las encuestas y su credibilidad internacional está en el piso, el desempleo supera el 20\% y no se ve
 salida.  Ante tanta incertidumbre algún candidato con propuestas populistas se impone en las elecciones de mayo y se
 reafirma en las de junio.

La clase empresarial que retiró el apoyo a Uribe, ve al nuevo presidente con aun mayor incertidumbre.  Las medidas
 tomadas por el nuevo gobierno para evitar los despidos genera la quiebra de varias empresas y en su retórica
 populista, el nuevo gobierno culpa a los empresarios.  En su intención de congraciarse con Venezuela, el nuevo
 gobierno inicia conversaciones con las FARC, que incluyen un alto el fuego unilateral de parte del gobierno y un área
 despejada.

Ante estas circunstancias, las FARC consolidan su posición bélica y durante las negociaciones comprometen al gobierno
 en un cambio estructural en el modelo del estado.  Las fuerzas paramilitares empiezan a consolidarse.  Las fuerzas
 armadas, sintiéndose traicionadas por el gobierno, hacen muy poco para controlar al nuevo paramilitarismo y lo
 alientan.  Las ordenes del gobierno para someter a las autodefensas (parte por convicción, parte por exigencia de las
 FARC), son infructuosas.  Antes de completar un año de gobierno, hay una guerra civil de gran escala con las FARC
 (apoyadas ya abiertamente por Venezuela) por un lado, y una combinación de antiguas fuerzas armadas constitucionales y
 autodefensas por el otro.  El gobierno, no puede comprometerse a apoyar a las FARC por medio de las fuerzas militares
 leales a la constitución y en sus titubeos es derrocado por la milicia.  La tradición colombiana descarta un gobierno
 militar, y en su afán de parecer defensores de la constitución de 1991, los golpistas proclaman a algún presidente
 civil interino mientras se convoca a elecciones.

Internacionalmente ninguna de las partes del conflicto: las FARC por un lado, las Fuerzas Militares golpistas por el
 otro, representan un gobierno legítimo.  Ambas fuerzas controlan territorio.  Ambas fuerzas utilizan el comercio
 ilegal de drogas como parte de su financiación.  Ni la OEA ni los EE.UU. apoyan abiertamente a las fuerzas golpistas,
 pero tampoco apoyan a las FARC.  Las fuerzas golpistas tienen un mayor poder militar, pero tienen menos apoyo
 internacional.  En este limbo termina de transcurrir 2010 y 2011, y en diciembre de 2011, Cháves toma una maniobra
 audaz: invade a Colombia con el pretexto de restaurar el orden constitucional.

Presionado por el estado de las cosas y a la luz de las mayorías republicanas en el Congreso estadounidense conseguidas
 en noviembre de 2010, los EE.UU. toman la decisión de apoyar al gobierno interino tras la invasión venezolana.  Este
 apoyo no se materializa en acciones militares directas, sino en prestar asesorías, apoyo logístico y armamentos a las
 fuerzas defensoras.

Venezuela suspende el envío de petróleo a los EE.UU. lo cual alienta a los halcones de Washington a declarar
 completamente inadmisible la presencia de Chávez en Venezuela y deciden aplicar la táctica Jadafi: bombardean centros
 importantes de producción petrolera de Venezuela, suspendiendo completamente la capacidad de exportar por los próximos
 dos años.

Dos años son un tiempo suficiente para que la chequera de Chávez pase a saldo en rojo, y como tal su diplomacia de
 petrodólares pierde efectividad.  En Colombia esto se refleja en que las fuerzas bolivarianas de las FARC pierden el
 apoyo de Venezuela y por lo tanto su capacidad de derrotar a las antiguas fuerzas militares constitucionales, sin que
 estas tengan todavía la capacidad de derrotar a las FARC y sus aliados.  Esta situación de un país dividido y en
 guerra civil se prolonga hasta 2014 y se ha expandido a Venezuela y Ecuador.

Cuando Venezuela recupera la capacidad de exportar crudo, el gobierno de Chávez está muy debilitado y el pueblo exige
 proteger la integridad de Venezuela de la guerra en Colombia (de cualquiera de las partes), lo cual incluye romper
 completamente las ayudas a las FARC y permitirles usar territorio venezolano.

En el territorio controlado, el gobierno interino pretende organizar elecciones en 2014.  Estas elecciones se llevan a
 cabo con una alta abstención en las zonas calientes y una participación moderada en las zonas aseguradas por el
 gobierno interino y las antiguas fuerzas constitucionales.  El depuesto presidente de 2010 reconoce los resultados de
 estas elecciones, lo cual da una vía libre para que la comunidad internacional reconozca al nuevo gobierno como
 legítimo.  Sin embargo las FARC controlan grandes cantidades de territorio, tienen un armamento equiparable al de las
 fuerzas constitucionales (incluyendo una pequeña fuerza aérea), y están muy lejos de una derrota, aun cuando pierdan
 todo el apoyo internacional que han ganado desde 2008.  Tampoco tienen un camino seguro a una victoria.

El gobierno de 2014 tiene un mandato claro para lograr la legitimidad: deshacerse del paramilitarismo y buscar un
 armisticio con las FARC.  Se establece para la negociación que las FARC pueden ejercer control territorial en los
 territorios que controlan sin disputa, e igualmente las fuerzas constitucionales y la población civil no deben ser
 agredidas en las zonas de pleno control.  El gobierno logra negociar también un cese de hostilidades de parte de las
 autodefensas ilegales.  En las zonas en disputa se declara un alto el fuego como preámbulo a las negociaciones.

Los alto el fuego son frágiles y se rompen y restablecen todo el tiempo, pero las partes respetan las zonas declaradas
 controladas por la contraparte.  Las FARC se comprometen a la renuncia total de practicas consideradas terroristas y
 se someten a los protocolos de Viena y de Ginebra sobre derecho de guerra y derecho humanitario.  Igualmente se
 comprometen a desmontar la producción y el tráfico de estupefacientes.  Otro tanto le corresponde al gobierno
 constitucional respecto a los grupos paramilitares.

Los acuerdos parecen avanzar.  Ambas partes adquieren reconocimiento internacional.  Las FARC tienen un control
 efectivo y legalizado de parte del territorio y lo mismo sucede con el gobierno institucional.  Todo 2015 transcurre
 en medio de esta armisticio y de conversaciones.  Pero a mediados de 2016, cuando las partes están optimistas de
 lograr un acuerdo, un nuevo rompimiento del alto el fuego en una de las zonas disputadas, escala pronto en una ruptura
 de las negociaciones, y la guerra se restablece.

Finalmente, sobre 2020 se logra un acuerdo definitivo entre las FARC y las fuerzas constitucionales.  Se aprueba una
 nueva constitución la cual es refrendada en las urnas.  Sin embargo, tras una década de guerra civil, hay demasiados
 agentes armados.  Varios grupos se oponen a dejar el lucrativo negocio del narcotráfico, otros se conforman en
 pandillas inspirados en los maras centroamericanos, controlando distintos tipos de negocios ilegales.  El poder
 militar de estos grupos, que no pretenden control político, es tal que las nuevas fuerzas constitucionales
 (conformadas por una combinación de las antiguas fuerzas constitucionales y combatientes de las FARC) son incapaces de
 controlar estos grupos.  En la práctica se prefiere ignorarlos y pretender que son una amenaza menor al nuevo orden
 constitucional.

\chapter{Injusticia social: Bullshit!}
\begin{metadata}
	Published by \anchor[chlewey]{chlewey} on \anchor[http://ewey.co/B16]{Thu, 17 Jan 2008 13:57:00 +0000}\\
	\categories{farc, injusticia-social, opinion}\\
	Shorthand: \anchor[http://blog.chlewey.net/2008/01/injusticia-social-bullshit/]{injusticia-social-bullshit}
\end{metadata}

\par% p
La principal bandera de lucha de las FARC y otros grupos alzados en armas es la injusticia social que hay en Colombia o
 en sus respectivos países.  Pues bien, parafraseando a Penn \& Teller, eso es \anchor[http://www.sho.com/site/ptbs/home.do]{\relax{% {'style': 'font-weight: bold;'}
bullshit}}.  Esa tesis no es más que basura.  No es una basura traída de los cabellos y tiene una pequeña base en la realidad,
 pero es basura.

Las FARC como tal y el ELN surgieron en 1964, junto con muchos otros grupos guerrilleros en toda América Latina, junto
 con los movimientos anticolonialistas en África y en medio de la guerra fría en Asia.  Todo ello no es inconexo.

Colombia tiene un largo historial de violencia.  Tras la Pax Hispánica que hubo entre la conquista y el levantamiento
 de los Comuneros, empezaría un rosario de violencias: la lucha independentista entre 1810 y 1822, guerras entre
 antiguas colonias como la guerra de 1829 contra Perú, guerras civiles entre regiones o concepciones distintas del
 estado entre 1840 y 1902, un período de relativa calma en la cual la violencia bipartidista empezó poco a poco a
 ebullir hasta 1949 cuando, tras el asesinato de Jorge Eliécer Gaitán, el partido liberal le declaró la guerra al
 oficialista partido conservador en lo que se conoció como La Violencia, así en mayúsculas.  Amnistiados los liberales
 en 1953 bajo el régimen de Rojas Pinilla, y unidos a los conservadores en 1957 para deshacerse de Rojas Pinilla, la
 violencia entre liberales y conservadores quedaría finalmente conjurada con el Frente Nacional.

La injusticia social fue siempre mayor en Colombia en los años previos al frente nacional de lo que fueron después.  En
 1957 se logró finalmente el sufragio universal para todos los hombres y mujeres de Colombia.  En épocas anteriores no
 sólo se le negaba el voto a las mujeres sino también a los pobres.  Por casi 150 años de vida republicana, durante la
 época del Frente Nacional ya muchas de las injusticias sociales habían sido superadas.  Y entre 1964 hasta 1991 y
 desde entonces hasta hoy, se ha seguido superando los problemas de inequidad e injusticia social.  No completamente,
 pero sí ha sucedido.  Las FARC han existido, por ende, durante el cuarto de historia colombiana en que menos
 injusticia social ha existido.

Las violencias anteriores no revindicaban a la corrupción, la exclusión social o la injusticia social.  Invocaban
 modelos de estado o colores políticos.  La misma independencia fue una lucha por el poder y por ideales políticos
 abstractos.  El poder de las clases pudientes (los criollos, la clase mantúa) que tenían todo menos el poder político
 por el único pecado de haber nacido a este lado del charco, y el ideal político del ``maestro artesano'', del francmasón
 que inspiró también la independencia de los EE.UU. y la revolución francesa.  Una igualdad política que parte no del
 concepto abstracto de que todos somos iguales, sino de que por nuestros méritos podemos igualarnos.  Cuando Bolívar
 ofrecía igualdad al ex esclavo o al indio no lo hacía por un concepto de justicia social, sino porque podía valerse de
 sus méritos, para lograr sus propósitos.

Luego, una serie de ideales políticos abstractos y luchas por el poder motivaron las violencias sin que nadie invocara
 algo así como la justicia social o la lucha contra la corrupción como banderas de su lucha.  No porque no hubiera
 justicia social, no porque no hubiera corrupción excluyente, sino porque esos conceptos no importaban.

Empezaron a importar con Marx y, particularmente, tras el triunfo de la revolución rusa.  Pero quienes tomaron esas
 luchas en América Latina, personas como Jorge Eliécer Gaitán o como Juan Domingo Perón, no se inspiraban sólo en
 Lenin, sino también en Mussolini.  Su ideal no era tomarse el poder por las armas, sino en las calles.  Perón lo
 logró, a Gaitán lo mataron.

Luego sería todo el partido liberal, no sólo la línea gaitanista, la que renunció a las elecciones de 1950 y decidió,
 más bien, irse a luchar en el monte.  No por la justicia social, sino porque consideraban que los conservadores no
 ofrecían garantías políticas.  En fin, una lucha de colores políticos.  Lucha que se superaría en 1957 con la creación
 del Frente Nacional.

El socialismo ha existido en Colombia desde los años 1920 y tras la creación del frente nacional, varios sectores
 liberales de la línea más socialista conservaron sus armas y fundaron comunidades bajo sus propios ideales.  Con
 distinto grado de politización, algunas simplemente se dedicaron al bandolerismo, pero otras empezaron a definir sus
 reglas aparte del juego del Frente Nacional.  Era la época de la guerra fría: Cuba había pasado de ser el Casino de
 los EE.UU. a una nación hostil donde los soviéticos planeaban colocar misiles nucleares.  Fue la época del
 mackartismo.  El Che Guevara, uno de los líderes de la revolución cubana se había ido a Bolivia a exportar la
 revolución.  Lo último que querían los EE.UU. era que se repitiera la historia de Cuba y en respuesta presentaron el
 Plan LASO.

Era la época de la guerra fía: Cuba había pasado de ser el Casino de los EE.UU a una nación libre y amparada por la
 Unión Soviética, y lo habían logrado los mismo cubanos a través de la guerra de guerrillas.  Colombia no era el casino
 de los EE.UU. pero igual tenía problemas de injusticia social.  No tenía la dictadora de Fulgencio Batista, pero igual
 tenía un adefesio llamado Frente Nacional.  Era necesario en Colombia crear entonces un nuevo ejército, un Ejército de
 Liberación Nacional, inspirado en Cuba.

El Plan LASO, tenía por objeto suprimir los grupos comunistas en América Latina y eso, en Colombia, significaba acabar
 con las ``Repúblicas Independientes'' como aquella fundada en el Tolima bajo el nombre de Marquetalia, comandada por un
 tal Pedro Antonio Marín quien se hacía llamar como algún líder campesino anterior: Manuel Marulanda Vélez, e
 ideológicamente dirigido por un comunista de línea dura llamado Jacobo Arenas.  El ejército constitucional, arremetió
 contra la comunidad de Marquetalia, la cual resistió lo suficiente para que muchos de sus líderes escaparan.  Arenas,
 quien no se encontraba allí, no tuvo problema en mostrar como el estado reprimía las opciones de socialismo y que el
 único camino era alzarse en armas en contra del régimen y así logró reclutar a varios estudiantes universitarios para
 irse al monte y formar junto con los sobrevivientes de Marquetalia, las Fuerzas Armadas Revolucionarias de Colombia.

Más que la injusticia social que existía antes y más pronunciada, y que existía y sigue existiendo en muchos países sin
 guerrillas, las FARC y el ELN fueron un producto de una época, fueron producto de la guerra fría, de las políticas y
 temores de Washington, del triunfo de la revolución cubana, del concepto existente en los años 1960 que la lucha
 armada era una opción real para luchar contra algo abstracto llamado injusticia social.

La misma guerra fría no sólo produjo estos grupos sino que los alimentó por años.  En países como la Argentina, también
 hubo guerrillas, pero estas luchaban contra algo más concreto: las dictaduras militares.  Entre la mano fuerte de las
 dictaduras y ante el fin de las mismas, la democracia fue el conjuro para que las guerrillas terminaran
 desmovilizándose.  En El Salvador, fue tal el poder de las guerrillas que a mediados de los años 1980 controlaban la
 mitad del país, su éxito les permitió negociar de igual a igual con el gobierno llegándose a acuerdos de paz cuando la
 Unión Soviética se desmoronaba.  En el Perú, el gobierno populista de Alan García, les quitó muchas de sus banderas y
 ante la falta de apoyo popular, en lugar de grandes ejércitos los guerrilleros terminaron convertidos en grupúsculos
 que usaban el terrorismo para compensar su tamaño.  El trabajo policial y de inteligencia, más que los grandes
 presupuestos de guerra, lograron la captura de Abimael Guzmán y el desmantelamiento de Sendero Luminoso y el MRTA.

¿Qué pasó en Colombia durante todo ese tiempo? En Colombia nunca hubo una dictadura de mano dura que por un lado
 contuviera a las guerrillas y por otro, que su terminación representara el triunfo de los movimientos armados que los
 condujera al desarme.  Por otro lado la geografía quebrada, la disponibilidad de recursos como el petróleo, y pronto
 las drogas, les dio una forma de sobrevivir y de sostenerse que les permitió sobrevivir la caída misma de la Unión
 Soviética.

Adicionalmente, los intentos de paz de los años 1980, fracasaron, quedando como mayor recuerdo el genocidio de la Unión
 Patriótica.  Genocidio en gran parte patrocinado por una revancha personal de Gonzalo Rodríguez Gacha, un esmeraldero
 y narcotraficante más conocido como ``El Mexicano'', pero igualmente ignorado o patrocinado por personas de la clase
 política tradicional y de las Fuerzas Armadas.  La utilización del secuestro extorsivo y el boleteo, generaron una
 reacción en diferentes estamentos de la sociedad y es así como los ejércitos privados de Rodríguez Gacha y de Pablo
 Escobar se unieron con ganaderos, políticos y sectores de las fuerzas armadas para formar distintos tipos de
 autodefensas.

Al terminar la guerra fría, el EPL y el M-19 ya se habían desmovilizado, pero muchos de sus líderes, incluyendo al
 candidato presidencial Carlos Pizarro León-Gómez, fueron asesinados.  Las autodefensas aumentaban su poder y fueron
 responsables de varios de los asesinatos de líderes desmovilizados, por no hablar de la larga tradición de asesinatos
 de miembros de la UP.  Colombia estaba pasando de convertirse en un procesador de cocaína a un productor de coca, con
 colonos y campesinos en las zonas de influencia de las FARC, dedicados al cultivo o recolección de coca y el
 transporte y procesamiento de la coca y la cocaína.  La industria del secuestro se estaba perfeccionando.

Sin la caída de una dictadura que justificara su desmovilización, con la necesidad cada vez menor de una financiación
 directa de la Unión Soviética, con un territorio que le permitía sobrevivir, con un enemigo que jugaba a sus mismas
 tácticas de guerra sucia, y que de paso le servía para justificar sus reclutamientos, las FARC y el ELN no dejaron de
 existir tras el fin de la guerra fría.

La injusticia social no tiene nada que ver con su existencia.  Eso es basura: bullshit.  Tampoco existen porque sean
 simples mafias: organizaciones criminales, sin ideología, ni su único propósito es la producción y exportación de
 drogas ilegales.  Existen por inercia.  Existen porque creen en algo más allá de su propia existencia.  Porque se
 quedaron en 1964 cuando en medio de la guerra fría creían que la única forma de combatir la injusticia social era por
 las armas, y porque Colombia, en su posición geográfica privilegiada, en el extremo norte andino, sus selvas, sus
 mares, su conexión entre Américal de Norte y América del Sur, su riqueza, ha sido un lugar donde pueden sobrevivir con
 esa mentalidad.

Sabiendo ahora por qué sobreviven, podemos saber también cómo lograr su fin.  Y estar seguros de que tras su
 desaparición o desmovilización, no surgirán nuevos grupos luchando por la injusticia social.

Un problema más complejo es lograr el fin de las FARC y del ELN y las AUC sin que por ello surja un nuevo problema
 similar a los maras centroamericanas.

\chapter{Si hoy pudieras decirle algo a las FARC}
\begin{metadata}
	Published by \anchor[chlewey]{chlewey} on \anchor[http://ewey.co/B17]{Mon, 04 Feb 2008 11:12:00 +0000}\\
	\categories{activismo, actualidad, farc, opinion}\\
	Shorthand: \anchor[http://blog.chlewey.net/2008/02/si-hoy-pudieras/]{si-hoy-pudieras}
\end{metadata}

Bueno, primero sabemos que las FARC no escuchan, o más exactamente escuchan lo suficiente para desvirtuar o ignorar al
 adversario o para aprovechar como propaganda las medias tintas de lo que les conviene.

Si hoy puedieras decirle algo a las FARC ¿qué le dirías?

¿Le dirías que su lucha, enmarcarda dentro de la corrupción del estado y las otras formas de violencia presentes en
 Colombia, no la compartes pero la justificas, o le dirías que ni la corrupción ni las otras formas de violencia son
 excusa para su barbarie?

¿Les dirías que aunque repudias el secuestro, pueden continuar con esa práctica mientras el gobierno no ceda en sus
 inamovibles y de luz a un canje de prisioneros; o le dirías que ellos son los únicos responsables del secuestro y que
 su obligación es liberar sin condiciones a todos los rehenes civiles y permitir la verificación de la Cruz Roja a los
 prisioneros de guerra combatientes que conserven?

¿Es tu mensaje a las FARC que, como eres oposición al gobierno, serás complaciente con sus tácticas y mantendrás
 siempre, incluso este 4 de febrero, una férrea crítica a la administración Uribe y a todo el pueblo uribista?  ¿O es
 tu mensaje que aunque te opongas al gobierno y a sus políticas, dejas claro y sin restricciones que te opones
 fírmemente a la anacrónica combinación de formas de lucha, al secuestro y al terrorismo de las FARC?

\chapter{No hay conflicto}
\begin{metadata}
	Published by \anchor[chlewey]{chlewey} on \anchor[http://ewey.co/B18]{Wed, 20 Feb 2008 21:59:00 +0000}\\
	\categories{derecha, guerrilla, izquierda, opinion, uribismo}\\
	Shorthand: \anchor[http://blog.chlewey.net/2008/02/no-hay-conflicto/]{no-hay-conflicto}
\end{metadata}

Soy un agnóstico político.  No creo que exista un sistema político único y perfecto que sirva a todas las situaciones y
 creo que todos los sistemas políticos existentes han sido respuesta a una realidad.

Creo que lo más parecido a un modelo político perfecto es la democracia constitucional liberal, entendiendo por
 democracia un modelo de estado en el que se consulta al pueblo sobre sus destinos, por constitucional que es regida
 por una serie de principios y por liberal que es incluyente y basada en el respeto a las libertades individuales; pero
 con todo tengo mis dudas que una democracia constitucional liberal se adapte a todas las situaciones y convenga a
 todos.

Colombia está definida en su constitución como un estado social de derecho.  Un estado de derecho es aquel regido por
 unas leyes y no sobre el capricho de sus gobernantes.  Por social se entiende que se incluye la sociedad toda y se
 busca compensar a los menos favorecidos por las posibles inequidades que puedan surgir.  Yendo más allá, Colombia
 incorpora las bases de una democracia liberal constitucional.

Tomando la letra de nuestra constitución y creyéndonosla, tenemos que somos un estado liberal, social, constitucional,
 democrático y de derecho.  Se respetan las libertades individuales y más de una decena de artículos de la constitución
 están consagrados a explicarnos nuestros derechos.  Se favorece la opinión ciudadana y existen mecanismos para que los
 ciudadanos podamos participar en las decisiones del estado.  Y existiendo así mecanismos para que el pueblo participe
 dentro de la constitución y la ley, entonces cualquier intento de participar por fuera de la ley se convierte en un
 despreciable acto de sabotaje.

Pero por muy incluyente que sea el estado en el papel, éste no podrá incluir todas las opciones.  El conflicto es
 inherente en las sociedades humanas ya que todos nosotros requerimos cosas que pueden llegar a estar en conflicto con
 los requerimientos del otro.  La forma en la que la sociedad resuelve estos conflictos nos da un modelo de estado.  En
 la democracia liberal, se parte de la libertad de los individuos para que estos decidan, bien por voto directo, o bien
 a través de un representante, cómo será ese modelo de estado: cómo se han de resolver los conflictos.

Colombia es una democracia representativa, pero ¿qué tan representativos son nuestros representantes?  ¿El senador que
 elegí sí vota como yo hubiera querido?  ¿O, la verdad, no tengo idea siquiera cómo se desempeña mi senador?  ¿O no
 salió elegido y mi voto sólo ayudó a otro de la lista en quien confío menos?  ¿O no voté?

No conozco las últimas encuestas pero hace unos pocos años el congreso era una de las instituciones en las que los
 colombianos menos creíamos, y eso refleja una cosa: no creemos que el congreso nos represente en un sentido positivo.
 Y cuando nuestros representantes no nos representan, la democracia está viciada.  Pero aunque no lo estuviera, en la
 democracia finalmente deben tomarse decisiones, decisiones que serán tomadas por la mayoría de quienes se les ha
 otorgado la responsabilidad.  Pero siempre habrá una minoría.

Una democracia liberal garantiza derechos a la minoría, pero toma la decisión de la mayoría.  No hay nada inherente que
 implique que la mayoría tenga siempre la razón; a duras penas hay una correlación entre lo que la mayoría decide y lo
 que es conveniente para esta mayoría.  Pero, aunque se respeten los derechos democráticos de la minoría, tales como el
 derecho a expresarse, es claro que cuando una decisión modela a un estado y a la forma como se administran los
 conflictos en esta, una decisión implicará la sistemática resolución de los conflictos en contra de unos y en
 beneficio de otros.

Aunque garantizar el derecho a la voz a las minorías podrá eventualmente conducirlas a convencer a la mayoría, es claro
 que el sistema no es perfecto.  El conflicto es inherente, y la forma como se manifiesta el conflicto puede llegar en
 ocasiones a generar violencia.

Este gobierno nos ha tratado de imponer un lenguaje.  Nos dice que en Colombia no hay un conflicto armado interno
 porque nuestra constitución nos garantiza que somos un Estado Social de Derecho, y por lo tanto cualquier conflicto
 entre conciudadanos tiene una solución por medio de la legalidad y, por lo tanto, cualquier manifestación violenta
 contra el estado no forma parte de un conflicto sino de una amenaza contra el estado y contra la patria.  Entonces las
 FARC y el ELN no son actores del conflicto sino amenazas terroristas.

Pero el hecho es que hay colombianos que mueren y que son mutilados en medio de esta amenaza terrorista.  Colombianos
 que están o no a favor del sistema.   Colombianos que están por convicción y colombianos que están por error.
 Colombianos que tomaron las armas como respuesta a un conflicto social, y colombianos que no creen en las armas pero
 que han muerto por ellas igualmente.  Desde luego que hay conflictos sociales.  Hay conflicto cuando una parte de la
 población requiere de la economía que genera la exportación de carbón y otra parte prefiere devengar del turismo que
 pagará más y mejor cuando disfrute de playas limpias.  Y colombianos que preferirán que nuestros manglares no sean
 tocados ni por los puertos carboneros ni por los complejos turísticos con la esperanza de que exista todavía algo de
 naturaleza para nuestros hijos y nietos.

Hay conflicto cuando requerimos de los agrocombustibles como una alternativa a seguir quemando nuestras fuentes de
 petróleo, pero quienes hoy poseen las tierras para sembrar palma africana son comunidades que no tienen nada más que
 esas tierras ancestrales, y conflicto cuando una vez adquiridas esas tierras, estos empresarios del agro tumban el
 monte de nuestro biodiverso país para reemplazarlo por monocultivos.

Y cuando nuestros representantes votan una ley forestal, o una ley de tenencia de tierras, o de uso de suelos para el
 turismo o la actividad portuaria carbonera, etc., entonces nuestros representantes están decidiendo sobre una forma de
 resolver estos conflictos, favoreciendo a unos y perjudicando a otros.

Y así como hay un conflicto social hay también un conflicto armado.  Podrá ser ilegítimo, poco representativo y ajeno
 al real conflicto social, pero existe.  Ignorarlo no lo hará desaparecer, ni reconocerlo implicará reconocer justa
 causa al otro bando.

La otra parte del lenguaje que nuestro gobierno nos ha impuesto para que repitamos como cotorras es que no hay crímenes
 de estado, simplemente porque no hay una ley que los sancione, o un decreto presidencial que los ordene.

Bueno, si una ley promoviera un acto ilegal, ese acto dejaría de ser ilegal.  El concepto de crimen de estado no hace
 referencia a los actos ilegales consagrados como legales en leyes y decretos.  No es que el estado, que en el caso de
 una democracia es responsabilidad de todo el pueblo, ejecute crímenes amparados en las leyes del mismo estado.

El verdadero concepto de crimen de estado es cuando un crimen, un acto ilegal, es cometido sistemáticamente por agentes
 del estado en representación de ese estado.

Veamos, si un soldado asesina por celos a la novia usando su arma de dotación, eso no es un crimen de estado.  El
 estado es igualmente responsable, pues dio un arma a una persona violenta, pero el grueso de la responsabilidad recae
 en quien tomó la decisión de usar esa arma contra un civil que no representaba amenaza a la sociedad.  A pesar de su
 uniforme y de su rango, el soldado no actuó en persecución de uno de sus objetivos como agente del estado.  Esto es
 claramente un caso de responsabilidad individual que no compromete al estado.

Otro soldado, tal vez en el rango de sargento o teniente, o coronel o general, tiene como misión controlar a las
 milicias ilegales, principalmente a las FARC y al ELN.  En cumplimiento de esta misión y por fuera de lo que los
 estatutos le dictan, este soldado permite que civiles armados acosen a los colaboradores de la guerrilla, obtengan
 información y dejen mensajes en forma de cuerpos mutilados o familias masacradas.  Aquí ya hay una diferencia con la
 novia del soldado celoso y es que los crímenes de este otro soldado, igualmente ilegales, si fueron ejecutados como
 una extensión de su labor como agente del estado.

Cuando esto sucede, y se desprende también un patrón sistemático, no sólo en que los actos se repitan y se acompañen de
 otros similares como las ejecuciones extrajudiciales, la desaparición forzada y la tortura, sino que hay también un
 patrón sistemático de encubrimiento dentro de las fuerzas militares y hasta llegar al comandante en jefe, entonces el
 hecho de que sean ilegales y el hecho de que eventualmente jueces independientes al gobierno dicten sentencias
 condenatorias contra esos agentes del estado, no por ello dejan de ser crímenes de estado.

Pero es más fácil dormir tranquilos sabiendo que no hay conflicto armado ni crímenes de estado de los que debamos
 preocuparnos.

\chapter{¿Están derrotadas las FARC?}
\begin{metadata}
	Published by \anchor[chlewey]{chlewey} on \anchor[http://ewey.co/B19]{Sun, 02 Mar 2008 02:58:00 +0000}\\
	\categories{farc, opinion}\\
	Shorthand: \anchor[http://blog.chlewey.net/2008/03/derrotadas/]{derrotadas}
\end{metadata}

\par% p
Desde mi escritorio en medio de la ciudad no puedo hacer nada más que especular, así que mi lectura de la realidad
 puede estar completamente equivocada.  ¿Están derrotadas las FARC?  Esto lo escribo al final del día en que se conoció
 la suerte de Luis Edgar Devia, alias \relax{% {'style': 'font-style: italic;'}
Raúl Reyes}, y aunque no me alegro por la muerte de 18 colombianos (17 guerrilleros y un soldado), no puedo dejar de pensar que
 algunas cosas pintan más promisorias, aunque todavía el camino es tortuoso.

Las FARC no están derrotadas, ni con este ni con muchos otros golpes grandes que han sufrido en los últimos días.
 Todavía cuentan con un importante apoyo internacional, todavía tienen combatientes y redes de apoyo, y si logran
 resistir lo suficiente hay una posibilidad de que la marea política cambie a una situación que les sea más favorable.
 Pero desde mi escritorio, esa esperanza es muy poco probable, y gran parte del futuro del país depende de como manejen
 las FARC esas probabilidades: si insisten en una guerra o si se preparan para la paz.

Todavía quedan dos años y medio de gobierno de Álvaro Uribe Vélez y no veo en el país signos de que esto cambie.
 Incluso la eventual muerte del presidente, quien lo reemplace, sea Francisco Santos o alguien designado por el
 congreso, continuará la política de Uribe.  Y no veo un candidato hoy, con posibilidades para el 2010, que cambie la
 política a tal punto que la espiral descendente de las FARC se revierta.  Ni siquiera Carlos Gaviria, que a pesar de
 su discurso ambiguo, lo veo mucho más cerca de Lula Da Silva que de Hugo Chávez.

Así que las FARC, si tienen algo de inteligencia, lo único que les resta es administrar la derrota.

¿Están las FARC derrotadas?  Yo creo que para efectos prácticos lo están, y lo saben.  La pregunta ahora es si se
 replegarán esperando pasar de agache hasta el final del gobierno Uribe, con la esperanza de negociar con otro en
 términos más favorables, o seguir quemando sus cartuchos bélicos y políticos con la esperanza de parecer lo
 suficientemente poderosos para ser creíbles en una negociación, o si su orgullo les impedirá mostrarse débil ante un
 presidente de derecha como Álvaro Uribe.  Y el problema de negociar con Uribe, no es sólo reconocerle la victoria a
 quien para ellos ha sido el peor presidente de Colombia, sino que, estoy seguro, Uribe no les despierta confianza.

En ese sentido yo soy más optimista.  Creo que Uribe, ante una propuesta seria y creíble de que las FARC negociarán una
 desmovilización, les ofrecerá todas las garantías.  El problema es qué será una propuesta seria y creíble, porque no
 sólo las FARC desconfían de Uribe, sino que Uribe también desconfía de ``la FAR''.

Sigo pensando que el principal fracaso del Caguán no es que las FARC no tuvieran voluntad de paz, sino que mientras el
 gobierno pretendía negociar una desmovilización, las FARC pretendían negociar toda la política del estado; y mucho me
 temo que hoy por hoy, las FARC sigan pretendiendo eso en una eventual negociación, mientras el gobierno es más reacio
 a negociar algo más que la desmovilización.  Y me temo, por otro lado, que las FARC no querrán pasar de agache los
 próximos dos años y medio.

Aunque las FARC están prácticamente derrotadas, todavía tienen combatientes, algo de apoyo internacional, redes de
 apoyo y, lo más importante, medios de financiación.  Y aunque sepan que ningún cambio plausible de la situación
 política los lleve a una eventual victoria, si creo que querrán mostrar todavía algo de fortaleza cuando llegue el
 momento de negociar con este u otro gobierno.  Pero cada intento que hagan de querer mostrar fortaleza, es una llaga
 más a este maltrecho país.  Y cada vez que las FARC pierdan el pulso con el gobierno, es un paso más lejos a la
 reconciliación nacional, porque sólo probará a los violentos de este país, que sólo la violencia resuelve los
 problemas.

\chapter{Ser uribista o ser uribista acrítico}
\begin{metadata}
	Published by \anchor[chlewey]{chlewey} on \anchor[http://ewey.co/B20]{Mon, 10 Mar 2008 15:52:00 +0000}\\
	\categories{alvaro-uribe, opinion, uribismo}\\
	Shorthand: \anchor[http://blog.chlewey.net/2008/03/uribista-acritico/]{uribista-acritico}
\end{metadata}

\par% p
Yo no soy uribista, ni antiuribista, la verdad soy muy poco anti-cualquiercosa, salvo que sea anti-anticualquiercosa.
 En ocasiones \anchor[http://chlewey.blogspot.com/2008/02/no-hay-conflicto.html]{me he definido} como un ``agnóstico político'' porque no creo que haya una verdad absoluta en la política.  Creo que algunos
 planteamientos de \anchor[http://es.wikipedia.org/wiki/\%C3\%81lvaro\_Uribe\_V\%C3\%A9lez]{Uribe} son convenientes y otros inconvenientes, pero igual creería lo mismo de cualquier otro presidente que haya estado o
 pudiera haber estado en la \anchor[http://www.presidencia.gov.co]{Casa de Nariño}.

\par% p
La primera vez que voté en unas elecciones presidenciales fue en 1994.  Voté por \anchor[http://es.wikipedia.org/wiki/Humberto\_de\_La\_Calle]{Humberto de La Calle} como candidato por el \anchor[http://www.partidoliberal.org.co/]{Partido Liberal}, y luego por el candidato presidencial \anchor[http://es.wikipedia.org/wiki/Ernesto\_Samper]{Ernesto Samper}, tanto en primera como en segunda vuelta.  Cuando estalló el escándalo del \anchor[http://es.wikipedia.org/wiki/Proceso\_8.000]{Proceso 8000}, en el fondo defendía a mi presidente, supongo que en gran medida por no sentir que equivoqué mi voto. En esa época,
 sin embargo, había cosas del gobierno de Samper que me gustaron y otras que me disgustaron.  Yo fui samperista pero
 aún así crítico frente a las actuaciones de Ernesto Samper que me parecían inconvenientes, comenzando por la forma
 como buscaba mantenerse en el poder cediendo ante cualquier tipo de presión.  Mi samperismo terminó, sin embargo, el 6
 de julio de 1996.  Esa noche fue el debate final y votación en la Cámara de Representantes sobre si se procesaba o no
 a Ernesto Samper por su vinculación con el Proceso 8000.  La forma como se llevó a cabo el debate y las decisiones
 tomadas me mostraron a una clase política comprada y a favor del oficialismo.  Para destacar: la conducción del debate
 por parte del presidente de la cámara \anchor[http://es.wikipedia.org/wiki/Rodrigo\_Rivera]{Rodrigo Rivera}, a la final uno de los pocos liberales que votó en contra del presidente.

\par% p
¿Es posible ser seguidor de un líder y sin embargo ser crítico con el mismo y con quienes le rodean?  Desde luego que
 sí.  De haber podido votar en 1990 es posible que hubiera votado por \anchor[http://es.wikipedia.org/wiki/C\%C3\%83\%C2\%A9sar\_Gaviria]{César Gaviria} y, en términos generales, durante su mandato fui gavirista, sin que eso me impidiera ver las contradicciones con las
 que se manejó el país.  Recuerdo dos episodios en particular: cuando el Partido Conservador criticaba algo del
 presidente liberal, Gaviria les recordó que el conservatismo estaba usufructuando del poder y como tal no tenían
 derecho a tales críticas.  Lo primero que pensé es: ¿y es que acaso el presidente cree que el poder es para
 usufructuar?

\par% p
El siguiente episodio fue a raíz del abatimiento de \anchor[http://es.wikipedia.org/wiki/Pablo\_Escobar]{Pablo Escobar Gaviria}.  Mal podría yo considerarme seguidor de Escobar, pero las declaraciones presidenciales en las cuales se veía la
 muerte del capo como un gran logro me dejó un sabor agridulce. Tal vez el país sería mejor sin Escobar, pero no por
 ello su muerte debería ser razón de júbilo por parte de nuestro Jefe de Estado.  No puedo dejar de pensar en la
 repetición de esa escena este \anchor[http://web.presidencia.gov.co/sp/2008/marzo/01/01012008.html]{primero de marzo}.

\par% p
Fui gavirista, pero aún así creo que tuve el criterio de ver cosas en César Gaviria que no me gustaron, tal como poner
 el servicio diplomático de Colombia para un proceso personal.  Fui samperista, al menos entre el 7 de agosto de 1994 y
 el 6 de julio de 1996, pero aun así pude darme el lujo de ver las inconsistencias de la administración Samper antes de
 mi desencanto.  Nunca fui Pastranista, ni en 1988 (cuando no podía votar pero todos mis compañeros del colegio lo
 favorecían), ni en 1998 a pesar de haber votado en segunda vuelta por \anchor[http://es.wikipedia.org/wiki/Andr\%C3\%83\%C2\%A9s\_Pastrana]{Andy Rabbit}.  A nivel local he sido \anchor[http://es.wikipedia.org/wiki/Antanas\_Mockus]{Mockusista} y \anchor[http://es.wikipedia.org/wiki/Enrique\_Pe\%C3\%B1alosa]{Peñalosista} y me he he atrevido a no tragar entero sus administraciones.  Definitivamente creo, por experiencia personal, que ser
 seguidor de un líder o de una tendencia política no nos hace necesariamente acríticos.

No digo que, necesariamente, mi criterio haya sido siempre acertado, no sólo cuando elegí a quien no debí elegir, sino
 también al apoyar o criticar a mis líderes favoritos o no.  Pero creo firmemente que aunque nuestro criterio pueda
 fallar no debemos renunciar a nuestra capacidad de ser críticos.

Aunque no soy uribista, creo que el uribismo es una opción política (bajo el concepto de opciones políticas
 personalistas) tan válida como cualquier otra.  Pero me asusto al comprobar como cada vez más y más el uribismo pierde
 autocrítica.  Uribe mismo es una persona muy compleja y llena de contradicciones, pero el uribismo sale a aplaudir
 como borregos cualquier posición del presidente así esta contradiga la posición anterior.

\par% p
Sólo puedo recordar una caricatura de \anchor[http://es.wikipedia.org/wiki/Quino]{Quino} donde decía, palabras más o palabras menos \relax{% {'style': 'color: #006600;'}
«Cuando un gobernante se ama a sí mismo más que a su pueblo es una desgracia, pero cuando un pueblo ama a su gobernante
 más que a sí mismo es peligroso»}.  A mi no me da miedo Álvaro Uribe Vélez ni me da miedo el uribismo.  Lo que si me causa pavor es tanto uribismo
 acrítico.

\chapter{Estrenando blog}
\begin{metadata}
	Published by \anchor[chlewey]{chlewey} on \anchor[http://ewey.co/B247]{Fri, 18 Apr 2008 02:18:36 +0000}\\
	\categories{usenet, facebook, personal}\\
	Shorthand: \anchor[http://blog.chlewey.net/2008/04/estrenando-blog/]{estrenando-blog}
\end{metadata}

Una de las ideas de adquirir un dominio en Internet es para usarlo.

\par% p
La principal razón por la que adquirí \anchor[http://chlewey.net]{Chlewey.\relax{% {'style': 'font-variant:small-caps;'}
net}} fue para experimentar programando para \anchor[http://developers.facebook.com]{Facebook}, así como recuperar el sitio web personal que alguna vez tuve.  Lástima que no pude recuperar mi antiguo dominio de
 Chlewey.\relax{% {'style': 'font-variant:small-caps;'}
org}, pues algún traficante de dominios (un \anchor[http://es.wikipedia.org/wiki/Ciberokupa]{ciberokupa} o \emph{cybersquatter}) decidió quedarse con él para distribuir publicidad dudosa.

\par% p
Pero todavía no he creado un buen sitio personal, mientras, por otro lado, he venido manejando un blog en Blogger: \anchor[http://chlewey.blogspot.com]{chlewey.blogspot.com}, tras haber escrito algo en \anchor[http://chlewey.spaces.live.com]{Live Spaces} (ex MSN) y, desde luego, algún interesante paso por Usenet.  Últimamente estoy participando también de varios foros en
 Facebook... cierto placer encuentro en discutir de varios temas, muchas veces centrados en la política.

Pues bien.  Ya que tengo el dominio decidí probar dejar mi blog aquí, recuperando lo escrito en Blogger, en Live
 Spaces, lo interesante de Usenet y, lo que pueda incluir de otros sitios.  Algún día me pondré a escarbar en Facebook
 a ver qué hay rescatable de todo lo que he dicho.

Así que aquí me tienen, para que me dejen sus inquietudes, critiquen mi diseño o mis ideas (o mi ortografía, como
 quieran) y pasen un rato que espero sea agradable, o ladrilludo, no sé.

\chapter{Homo conspirator}
\begin{metadata}
	Published by \anchor[chlewey]{chlewey} on \anchor[http://ewey.co/B248]{Thu, 10 Jul 2008 03:46:55 +0000}\\
	\categories{opinion, teorias-conspirativas}\\
	Shorthand: \anchor[http://blog.chlewey.net/2008/07/homo-conspirator/]{homo-conspirator}
\end{metadata}

El hombre es un animal que conspira.
 Conspirar significa respirar juntos.
 Son pocas las especies animales que sabemos que conspiran, que son capaces de unirse para lograr un objetivo en contra de otros animales de su misma u otra especie, más allá de los grupos de caza.
 Ponerse de acuerdo, planear, preparar, y luego atacar.
 Así conspiramos los seres humanos, los chimpancés y los delfines nariz de botella. La guerra no es más que una
 manifestación de nuestras conspiraciones: la confabulación máxima.

Sin embargo, ¿son nuestras realidades el resultado de una gran conspiración de unos pocos seres humanos privilegiados que mueven el mundo tras bambalinas haciéndonos creer que el libre albedrío existe y que nosotros elegimos a nuestros gobernantes?
 Personalmente estoy convencido que sí existen grupos de poder muy poderosos por encima de las estructuras de poder de
 nuestros estados, pero creo que tales grupos de poder no tienen un control omnipotente de todo lo que sucede en el
 mundo.

La mayoría de las teorías conspiracionistas creen descubrir las motivaciones de tales grupos.
 Personalmente creo que la gran mayoría de las teorías conspiracionistas están erradas y que esa gran conspiración de poderosos es mucho más prosaica de lo que se están imaginando.
 La mayor parte de nuestras grandes desgracias pueden ser más fácilmente explicadas por que todos nosotros y nuestros
 gobernantes nos equivocamos, cometemos errores, o simplemente cometemos estupideces, sin que sea necesario que fuerzas
 obscuras hayan maquinado tales desgracias.

\chapter{Esta ficción de la realidad}
\begin{metadata}
	Published by \anchor[chlewey]{chlewey} on \anchor[http://ewey.co/B249]{Sun, 25 May 2008 23:26:09 +0000}\\
	\categories{actualidad, personal, proyeccion-y-carrera, wikipedia}\\
	Shorthand: \anchor[http://blog.chlewey.net/2008/05/esta-ficcion/]{esta-ficcion}
\end{metadata}

Que por un lado tembló, que por otro confirmaron la muerte de Tirofijo, que la farcpolítica, que el abrazo entre Uribe
 y Chávez con la UNASUR, que está haciendo un hermoso sol afuera mientras que tengo que dedicarme a los atrasados
 proyectos finales de la maestría... en fín, pasa mucho y ya no quedan muchas ganas de hablar de eso.  Pero sí hay
 ganas de hablar, de todas formas.

\par% p
Alguna vez, en mis épocas de wikipedista (y últimamente en la \anchor[http://tinyurl.com/5uf7no]{acción malpensante}) me preguntaba porqué mis contribuciones en temas de mi ámbito profesional habían sido mínimas.  Poco hablo de mi
 trabajo o de mis temas de interés en mi blog, o en Wikipedia, o en mis demás ámbitos de expresión.

\par% p
¿Qué me lleva a disociar estos temas?  Tal vez en gran medida es que soy un soñador de mundos, pero esto podría ser una
 irrelevante conclusión.  A mi mente le encanta inventar, crear cosas que no son pero que podrían ser.  La glosopoesia
 o \anchor[http://es.wikipedia.org/wiki/Lengua\_construida]{conlanguísmo} me han llevado a aprender de \anchor[http://es.wikipedia.org/wiki/Ling\%C3\%BC\%C3\%ADstica]{lingüística} y \anchor[http://es.wikipedia.org/wiki/Filolog\%C3\%ADa]{filología}.  \anchor[http://en.wikipedia.org/wiki/Worldbuilding]{Crear mundos} e \anchor[http://es.wikipedia.org/wiki/Historia\_contrafactual]{historias alternas} me han llevado a interesarme y aprender de \anchor[http://es.wikipedia.org/wiki/Sociolog\%C3\%ADa]{sociología}, \anchor[http://es.wikipedia.org/wiki/Antropolog\%C3\%ADa]{antropología}, \anchor[http://es.wikipedia.org/wiki/Historia]{historia}, \anchor[http://es.wikipedia.org/wiki/Cultura]{cultura}, \anchor[http://es.wikipedia.org/wiki/Vexilolog\%C3\%ADa]{vexilología}, \anchor[http://es.wikipedia.org/wiki/Pol\%C3\%ADtica]{política}.  Tal vez sueño con poder crear todo un mundo como la Tierra Media de Tolkien, y publicar historias que residan allá.
 Pero en todas estas ¿dónde está el mundo real en el que vivo?

Recientemente he participado en varias discusiones políticas.  Este blog está lleno de especulación política.  Pero
 manejo esas discusiones con la misma distancia con la que me separo de mis mundos imaginarios.  Los mimo, les dedico
 tiempo, camino por la calle pensando en estos problemas reales o imaginarios, tomo de la realidad para la ficción y
 sigo elucubrando.

Sé la diferencia entre la ficción y la realidad.  Entre mi imaginación y lo que realmente pasa, y entre la especulación
 sobre lo que pasa y lo que podemos asumir con un grado aceptable de certeza.  Profesionalmente sé donde estoy, sé lo
 que debo lograr, y la realidad sigue alimentando mi ficción.  Desde que tengo trece años he elaborado diseños de un
 imaginario computador ideal, de un sistema operativo, más tarde de un circuito electrónico, de un elemento de red, de
 una red compartimentada y segura, de un firewall, de un virus, de un procedimiento.

Pero a pesar de toda la riqueza que puedo tomar de mi imaginación, a pesar de todo lo que mi capacidad creativa me
 lleva a conocer mejor de lo que es la realidad y aprender de ella, he aprendido a callar.  El mismo impulso que me
 lleva a aprender, a buscar, a crear, es el mismo impulso que me lleva a aislarme, a querer protegerme de quienes creen
 que mis elucubraciones no son más que un escape de la realidad y que por lo tanto las censuran.  Y al cerrar esa
 puerta resulta que mis elucubraciones, que eran una aproximación a la realidad, me conducen a refugiarme de una
 realidad hostil.

\par% p
En ocasiones puedo tomar distancia y traducir un artículo sobre \anchor[http://es.wikipedia.org/wiki/Los\_burgueses\_de\_Calais]{los burgueses de Calais} o la \anchor[http://es.wikipedia.org/wiki/Expedici\%C3\%B3n\_\%C3\%A1rtica\_de\_Andr\%C3\%A9e]{expedición ártica de Andrée}.  Hay una distancia segura entre lo que soy, mi espíritu creativo y la realidad.  Pero con mi vida profesional tal
 distancia se reduce.  Y muy probablemente por ello no la expreso.

\chapter{Derechizados}
\begin{metadata}
	Published by \anchor[chlewey]{chlewey} on \anchor[http://ewey.co/B250]{Wed, 06 Aug 2008 16:41:22 +0000}\\
	\categories{derecha, opinion}\\
	Shorthand: \anchor[http://blog.chlewey.net/2008/08/derechizados/]{derechizados}
\end{metadata}

Evidentemente hay un importante grado de derechización en el país, sobre todo de la percepción de que está bien autodenominarse ``de derecha''.
 No hace muchos años, el término ``de derecha'' tenía una connotación algo negativa y muchos preferían considerarse
 ``conservadores'' o ``tradicionalistas''.

¿Cuándo cambió esto?
 ¿Han sido estos seis años de uribismo suficientes, no tanto para cambiar la ideología del país sino para interpretarla
 de otra forma?

\par% p
Si te consideras de derecha (o conservador, o tradicionalista, o símplemente uribista) sería interesante saber cuál es
 tu postura respecto a algunas preguntas básicas:

\begin{enumerate}

\item ¿Crees en la Santa Madre Iglesia Católica, y crees que los valores de la iglesia han de ser la guía de los valores
 nacionales?
\item ¿El libre mercado y la sana competencia son la mejor forma de producir buena calidad, bajos precios y, en últimas, el
 bienestar y la riqueza de toda la sociedad?
\item ¿Estás orgulloso de ser un colombiano de bien, y muestras tu orgullo portando el tricolor, cantando el himno, luciendo
 un sombrero vueltiao y agradeciendo a nuestros soldados?
\item ¿Consideras aberrantes el matrimonio homosexual, el aborto, la unión libre y otras sinvergüencerías?
\item ¿Debe el M-19 pedir perdón y reparar a las víctimas o de lo contrario comer callados frente al proceso de paz con los
 paramilitares?
\item ¿La mejor forma de sensibilidad social es trabajar honestamente y construir país, y no andar desprestigiando la
 dignidad de nuestro estado?

\end{enumerate}

Una de las preguntas que me he hecho para autodenominarme agnóstico político es poder entender tanto a la izquierda como a la derecha política sin juicios de valor: un juicio de valor, sin duda, implica una adhesión política por una doctrina u otra.
 Por mucho tiempo mi visión de la derecha fue la de un pensamiento retrógrado, sin sensibilidad social, de disciplina
 férrea que se impone sobre el pensamiento individual.

\par% p
Hoy siento que este juicio es injusto.
 No me he derechizado, ni me he acentrado... símplemente reconozco que en la autoidentificación política de cada uno de
 nosotros, no partimos del dogma o del interés personal \relax{% {'style': 'font-size: 90%; color: #999999;'}
(si soy rico soy de derecha porque la derecha protege mis intereses, si soy pobre soy de izquierda, porque la izquierda
 redistribuye)}, sino de la forma como cada uno de nosotros, por educación o experiencia, considera lo mejor para nuestros semejantes.

¿Será posible que nuestros políticos, polistas o uribistas, entiendan que la otra visión es otra visión y no una
 aberración personal de nuestros enemigos políticos?

\chapter{Pro-blogging soon}
\begin{metadata}
	Published by \anchor[chlewey]{chlewey} on \anchor[http://ewey.co/B251]{Fri, 05 Sep 2008 21:53:47 +0000}\\
	\categories{blogs, personal, proyeccion-y-carrera}\\
	Shorthand: \anchor[http://blog.chlewey.net/2008/09/pro-blogging/]{pro-blogging}
\end{metadata}

\par% p
I kind of \anchor[http://blog.chlewey.net/2008/05/esta-ficcion]{was complaining} that I seldom participate in fora, write on this blog, or make contributions to Wikipedia in my professional field.
 This should be about to change, as one of the goals in my work will include blogging a little on professional basis.

So you probably will find me soon on this other aspect of my life.

I will not ruin the surprise yet for those of you who do not already know what my pro field is, not that it would be
 any hard to find out as there are too many information about me scatered around, and most of you who are reading this
 blog probably already know who I am and what I do.

I am still terrified.~ But I hope I will not disappoint you.

\chapter{chlewey.org}
\begin{metadata}
	Published by \anchor[chlewey]{chlewey} on \anchor[http://ewey.co/B257]{Fri, 19 Sep 2008 14:57:34 +0000}\\
	\categories{personal, web}\\
	Shorthand: \anchor[http://blog.chlewey.net/2008/09/chlewey-org/]{chlewey-org}
\end{metadata}

\par% p
Tras haber recuperado mi viejo dominio de \anchor[http://chlewey.org]{chlewey.org}, hasta no hace mucho secuestrado por \anchor[http://es.wikipedia.org/wiki/Ciberokupa]{ciberokupas} quienes me pedían USD 250,00 de rescate, no puedo dejar de estar feliz por tenerlo de vuelta sin haber pagado más que
 el costo de registro ante NIC.

\par% p
Me encuentro, sin embargo, ante un nuevo reto: reorganizar y completar todo el contenido que he pretendido tener en mi sitio web, ahora multiplicado en varios sitios.
 Una de las ideas es mover los artículos, páginas y contenidos de carácter más personal a \relax{% {'style': 'font-variant:small-caps'}
chlewey.org} y dejar en \anchor[http://chlewey.org]{chlewey.net} el contenido más \emph{profesional}.~ Esto implicaría en gran medida que o bien este blog se movería nuevamente o bien se dividiría en dos.

\par% p
Bueno.~ Mi blog ya pasó por \anchor[http://chlewey.spaces.live.com/]{MSN Spaces} (hoy Live), \anchor[http://chlewey.blogspot.com/]{Blogspot} y ahora aquí.~ Están \anchor[http://www.facebook.com/notes.php?id=659019663]{mis notas en Facebook}, alguna entrada de \anchor[http://www.hi5.com/friend/profile/displayJournal.do?ownerId=40025753]{blog en Hi5}, mis antiguas discusiones en Usenet, etc.
 Alguna vez creé una cuenta en Eltiempo.com que nunca he utilizado y ya debería ser tiempo de dejar un lugar definitivo para mi espacio de reflexión.
 Pero no sólo eso, por requerimientos laborales posiblemente esté pronto publicando profesionalmente en un blog.
 ¿Qué hacer con tantos sitios?

Aun no tengo respuestas.
 Si alguien quiere sugerirme algo escucho opiniones.
 Por lo pronto estoy seguro que me interesa conservar este blog con sus enlaces internos a artículos y que aún me queda
 un tiempo largo para poner en orden todas mis ideas y lograr ese sitio web que tanto añoro.

\chapter{gunning, germing, and steeling}
\begin{metadata}
	Published by \anchor[chlewey]{chlewey} on \anchor[http://ewey.co/B440]{Mon, 04 Aug 2008 03:22:09 +0000}\\
	\categories{facebooknote, historia, information}\\
	Shorthand: \anchor[http://blog.chlewey.net/2008/08/gunning-germing-and-steeling/]{gunning-germing-and-steeling}
\end{metadata}

Casi una semana de lectura (entre otras ocupaciones), para terminar de leer mi último libro: devorarme las quinientas y
 pico de páginas de «Armas, gérmenes y acero» de Jared Diamond.

Pienso que debería leer más.  Mi repertorio de libros leídos, sin incluir los que he tenido que estudiar (o aún ellos),
 es relativamente bajo.  En gran parte lo que leo y ese conocimiento general con la que descresto a amigos e incautos
 (los que me aguantan) es por todo lo que fragmentariamente he leído por Internet o visto por televisión.

Reviso entonces los tres últimos libros que he comprado y leído.  Además de «Armas, gérmenes y acero» están «Ursúa» de
 William Ospina y «Cuentos chinos» de Andrés Oppenheimer.  ¿Hay un patrón no intencional que vincule a todos ellos?

«Ursúa» es un tratado sobre la conquista de Sudamérica septentrional bajo la excusa de una biografía novelada de Pedro
 de Ursúa.  Cómo la cultura y la ambición de los españoles choca ante la cultura indígena de las Américas para
 desgracia de esta última.  Choque que se intenta revisar desde otra óptica, más científica, por Diamond en su obra
 ganadora del premio Pulitzer.

¿Era inevitable esta suerte?

Lo poco que sé de mi historia personal es que nací en una sociedad que habla español, una lengua indoeuropea (con
 raíces, tal vez, en la moderna Ucrania) desarrollada en Iberia central tras siglos de dominación romana (latín),
 visogoda (germanos que ya no hablaban germano) y árabe.  Que mi abuelo hablaba inglés, siendo él descendiente de
 negros africanos que adquirieron el inglés trabajando como esclavos para los ingleses en las Antillas.  Bautizado yo
 dentro del catolicismo: religión que se remonta al creciente fértil cuando este ya estaba poblado de afroasiáticos
 originarios, tal vez, de Etiopía.  Mi abuela, de fenotipo bastante indígena, chibcha presumo, se apellida García, en
 toda su vida sólo ha hablado español y practicado la fe católica.

Por los lados de mi madre y sólo juzgando fenotípicamente, habrá algún grado de mestizaje pero predomina la ascendencia
 europea (visigoda? romana? mora? fenicia? íbera?) que son más evidentes en los ojos verdes que tenía mi abuela.
 Siglos y milenios de mestizaje que convergen en mí, quien ama el pan y la pasta (trigo del creciente fertil), el arroz
 (otro pasto, esta vez de china meridional), las arepas y tortillas (maíz mesoamericano), la papa (tubérculo
 centroandino), la carne (de los descendientes de los uros domesticados por aparte en India y en el creciente fértil),
 etc.

Y todos mis compatriotas, comparten historias similares.  Personas que hemos convergido en la esquina noroccidental de
 América del Sur, con distintas historias genealógicas, y compartiendo un futuro para nosotros y nuestros hijos.

Es ahí donde converge el tercer libro.  Oppenheimer compara varias culturas actuales para descubrir porqué están
 generando o no crecimiento.  Hay muchos elementos entre lo que este argentino nacionalizado estadounidense nos cuenta
 que complementa la línea argumental de Diamond.  Diamond nos habla de la influencia ecológica y geográfica como motor
 de la historia, por encima de consideraciones raciales.  En «Cuentos chinos» Oppenheimer nos propone dejar de lado el
 color político y analizar, más bien, las decisiones de los pueblos y sus gobernantes.  Muchos elementos del
 determinismo geográfico (Diamond odiaría esta alocución), podrían considerarse borrados hoy cuando las
 telecomunicaciones y los medios de transporte nos liberan de las barreras geográficas que hicieron inevitable que
 Pedro de Ursúa y sus contemporáneos españoles se hiciesen a estas tierras, pero aún hoy América Latina tiene una
 desventaja frente a otros continentes con miras al futuro.

\par% p
Nuestra desventaja está en nuestra mentalidad.  Una mentalidad que nos lleva a que nos estemos peleando entre si
 debemos extender TransMilenio, crear un tren de cercanías o construir un metro, discusión en donde no he escuchado aún
 si finalmente ampliaremos la trocha de nuestros ferrocarriles.  Donde la discusión una reforma política se hunde por
 una silla vacía y si esta conviene o no al cálculo político del gobierno.  Donde a los que no somos uribistas nos
 tachan de chavistas... o donde a los que no son chavistas los tratan de oligarcas vendidos a los intereses del
 Imperio. \begin{m}% {'guerra': '', 'galaxias': '', 'de': '', 'las': ''}

\end{m}
.

Donde está el futuro nuestro como colombianos o como latinoamericanos.  Ya muchos de mis amigos de Facebook votaron con
 los pies.  Los que aquí permanecemos... ¿nos tocará como pueblo la suerte de Pedro de Ursúa quien se hizo grande
 frente a quienes le precedieron en este rincón del mundo, para morir finalmente en medio de la selva y de las intrigas
 políticas?  ¿Hicieron mal nuestros antepasados, indios o españoles, ante el futuro nuestro y de nuestro pueblo, en
 asentarse en este país geográfica y económicamente quebrado?

\chapter{Polarización}
\begin{metadata}
	Published by \anchor[chlewey]{chlewey} on \anchor[http://ewey.co/B441]{Fri, 18 Jan 2008 22:34:41 +0000}\\
	\categories{derecha, izquierda, opinion}\\
	Shorthand: \anchor[http://blog.chlewey.net/2008/01/polarizacion/]{polarizacion}
\end{metadata}

\par% p
Apenas empecé a ver que algunos de mis amigos se unían al grupo \anchor[http://www.facebook.com/group.php?gid=6684734468]{Un millón de voces contra las FARC} solo pensé que era un grupo más de los anti-FARC que por Facebook pupulan, y a los cuales me he negado
 sistemáticamente a pertenecer.

Al leer la introducción, no dejé de pensar que había algunas premisas falsas.  No creo, por ejemplo, que las FARC hayan
 querido engañarnos con el episodio de Emanuel y entre más leo, releo, reveo y analizo los últimos acontecimientos,
 menos creo que las FARC hayan urdido un engaño.

No por ello, desde luego, se justifica lo que las FARC han hecho o su misma existencia 43 años después de los sucesos
 de Marquetalia.

Pero en la incertidumbre de esos primeros días de enero, salió una propuesta dentro de ese grupo: manifestarnos en
 contra de las FARC, y me pareció que valía la pena, me pareció que pese a algunas premisas que no compartía, los
 colombianos podríamos ponernos a favor de una causa: decirles a las FARC que ellas no son nuestro ejército del pueblo.
  Que las FARC y la comunidad internacional pudieran entender que nuestra voz como colombianos es una en rechazo a las
 FARC.

Iluso ¿no?

Nuestra voz como colombianos es una en rechazo a las FARC, pero el diablo está en los detalles.

Primero, yo quiero ver a las FARC desmovilizadas y si eso implicará algún grado de impunidad e incluso el
 reconocimiento político de sus líderes, no me importa.  Sí, preferiría que pagaran cárcel por lo que han hecho, pero
 antes que nada pido que silencien sus fusiles.  Pero muchos otros abogan por que símplemente maten a todos los
 guerrilleros y, ya que estamos en esas, que también maten a Piedad Córdoba y a los del Polo.

Segundo, yo pretendo analizar la realidad en su complexidad.  Así como creo que lo de Emanuel no fue un engaño, tampoco
 creo que Oscar Morales, el creador del grupo, sea un agente de la CIA o del DAS cuyo único objetivo haya sido poner al
 país a hablar de la marcha del 4 de febrero en lugar de los últimos escándalos del presidente Uribe.  No conozco a
 Oscar Morales.  No sé quien sea ni para quien trabaja.  Pero conozco a muchos de mis amigos quienes comparten su
 pensamiento y tal vez el mío, en parte, y en mis seis meses en Facebook creo saber cómo funciona esto.  Esto es una
 red de personas, personas que tienen iniciativa individual y que no requieren que un gobierno o una subversión les
 dicten qué iniciativas tener.

La realidad, generalmente, está compuesta de respuestas sencillas y en reconocer en las personas que nos rodean y en
 aquellas que están lejanas las mismas complejidades que nosotros mismos tenemos.  Alguna vez leía que no hay que
 buscar conspiraciones en lo que se puede explicar con la estupidez humana, y creo que quien dijo eso tiene mucha
 razón.  Casi siempre es posible explicarnos un comportamiento extraño de otra persona cuando nos preguntamos si
 nosotros mismos, con la información disponible, no hubiéramos actuado igual.  Así, lo que parece descabellado o que
 obedece a un plan maquiavélico, resulta fácil de explicar.

Sin embargo nuestra tendencia es a simplificar la visión del mundo y volver complejas las explicaciones.  La teoría de
 fractales y el principio de la navaja de Occam suelen decirnos que es al revés, la realidad es muy compleja pero sus
 explicaciones son, generalmente, simples.

Al simplificar nuestra visión del mundo, entonces vemos a la izquierda política (lo que quiera que eso signifique) como
 algo intrínsecamente bueno (justicia social) o intrínsecamente malo (terrorismo), y desde esa visión extrema sólo nos
 cabe pensar que el otro, o no el otro sino la contraparte, el adversario, hace parte de una compleja trama en contra
 mía, en contra nuestra.

Y vemos entonces cómo los extremos se tocan.  Sus actitudes son similares, pues ambos parten de simplificar la visión
 del mundo a través de complejas explicaciones conspiracionistas.

Cuando los furibistas más furibundos hablan de la guerrilla, se inventan epitetos, los cargan de adjetivos,
 distorsionan el nombre; pero no paran ahí, siguen hablando de la p...erra de la Piedad Córdoba o del macaco de Hugo
 Chávez, pero no rebajan un solo adjetivo desobligante, y claramente ese adjetivo no está ahí puesto como una forma
 burda de jocosidad, sino seriamente puesto para insultar.

\par% p
Pero luego paso a \anchor[http://www.anncol.nu]{ANNCOL}, el principal medio de propaganda de las FARC, y veo exactamente lo mismo pero para el otro lado.  Parece imposible
 para ellos hablar del ``presidente Álvaro Uribe Vélez''  sino que tienen que hablar del ``narcopresidente Uribe Balas'' o
 cosas así.  Palabras deliberadamente puestas para insultar.  Para lograr una ovación de la tribuna que piensa de forma
 similar.

Pero ese insulto termina opacando el mensaje.  Un artículo en el cual se pueden exponer los hechos y dejar que estos
 hablen por sí solos se convierte ahora en un panfleto propagandístico, que emociona a la tribuna pero genera rechazo
 en quien no comparte del todo la visión.

Y finalmente el espacio del medio se reduce.  Quienes queremos ver el mundo en su complejidad, y buscar la belleza de
 la sencillez de las explicaciones, terminamos en la necesidad de estar dando explicaciones y defendiéndonos de los
 ataques de lado y lado.  Si quiero reconocer que en todo lo nefasto que han sido las FARC para Colombia, creo que en
 el episodio de Emanuel no hubo un engaño deliberado, entonces el furibismo me atacará por no darme cuenta del obvio
 engaño, porque las FARC siempre engañan y son mentirosas y ..., y tendré que defenderme de quienes  piensan como
 ANNCOL por haber utilizado la palabra ``nefasto'' para referirme a  las FARC.

Bueno, si fuiste capaz de leer esto te invito a que intentes ver que el otro no está conspirando en tu contra, sino que
 sólo comparte una visión diferente.

¡Saludos!

\chapter{No more}
\begin{metadata}
	Published by \anchor[chlewey]{chlewey} on \anchor[http://ewey.co/B262]{Thu, 09 Oct 2008 23:38:37 +0000}\\
	\categories{personal, proyeccion-y-carrera}\\
	Shorthand: \anchor[http://blog.chlewey.net/2008/10/no-more/]{no-more}
\end{metadata}

\par% p
Unfortunately \anchor[http://blog.chlewey.net/2008/09/pro-blogging]{the hope I once had to actually make a difference} was truncated for my resistance to just do the trivial things I am expected to do.

\anchor[http://commons.wikimedia.org/wiki/Image:Colca-condor-c03.jpg]{\includegraphics[width=300\px,height=197\px]{blog/condor-01-300x197.jpg}}

\chapter{Nuestras caras en el Libro de Caras}
\begin{metadata}
	Published by \anchor[chlewey]{chlewey} on \anchor[http://ewey.co/B266]{Mon, 27 Oct 2008 21:01:51 +0000}\\
	\categories{facebook, opinion}\\
	Shorthand: \anchor[http://blog.chlewey.net/2008/10/nuestras-caras/]{nuestras-caras}
\end{metadata}

\par% p
Estaba pensando si escribir sobre un tema serio de filosofía política o de actualidad nacional, sobre un tema más personal, o sobre un tema frívolo.
 Hoy ganó el tema frívolo: mis amigas que en el \anchor[http://www.facebook.com/]{Facebook} no tienen en su perfil una foto propia sino la de sus hijos.~ (Bueno, amigas y amigos.)

\begin{wrapfigure}{l}{240\px}\centering% {'src': 'http://news.chlewey.net/wp-content/uploads/2008/10/kids1.png', 'title': 'kids1', 'height': '240', 'width': '240', 'alt': '', 'class': ['alignleft', 'size-medium', 'wp-image-267']}
\includegraphics[width=240\px,height=240\px]{blog/kids1.png}
\end{wrapfigure}

Quienes me conocen a fondo saben que puedo llegar a ser bastante quisquilloso a veces, con cosas triviales sobre como las cosas deberían ser.
 Personalmente creo que la foto del perfil debe ser una foto identificable de la persona, no una foto de la mascota, ni del marido (o la marida), ni caricaturas, ni un afiche promocional de la empresa, ni paisajes, ni grupos, etc. sino una foto de la persona.
 La principal razón de la foto es que podamos reconocer a nuestros amigos cuando los buscamos o cuando publican historias en el Facebook.
 Facebook significa, precisamente, libro de caras.

\par% p
Pero, no porque yo crea que así deben ser las fotos de perfil de Facebook, significa que ustedes o mis amigos deban hacerme caso.
 Finalmente yo también he puesto como imagen de mi perfil ocasionalmente imágenes que no son fotos identificables mías.
 En el MSN uso una bandera de Colombia de una foto que le tomé al ARC Gloria.~ En Hi5 tengo \anchor[http://www.hi5.com/friend/40025753--Carlos--Profile-html]{una foto de grupo}.~ El 6 de marzo de 2008 reemplacé mi foto de perfil de Facebook por un \anchor[http://www.facebook.com/home.php?ref=home\&\#/photo.php?pid=721414\&id=659019663]{afiche de Luis Carlos Galán}.

\begin{wrapfigure}{r}{240\px}\centering% {'src': 'http://news.chlewey.net/wp-content/uploads/2008/10/kids2.png', 'title': 'kids2', 'height': '240', 'width': '240', 'alt': 'Mosaico de perfiles 2', 'class': ['alignright', 'size-medium', 'wp-image-269']}
\includegraphics[width=240\px,height=240\px]{blog/kids2.png}
\end{wrapfigure}

Pero al menos sí hay un punto donde deberíamos cuidarnos y es poner a nuestros hijos como nuestra propia imagen.
 Sí, estamos muy orgullosos de ellos.
 Son la razón de nuestras vidas.
 Hemos renunciado a ser nosotros mismos por convertirnos en una extensión de ellos.~ En fin.~ Somos ellos.

Pero ¿si tanto los queremos, por qué los exponemos tanto?

A decir verdad no deberíamos publicar en forma, valga la redundancia, pública y promiscua a nuestros hijos.
 Si tenemos un álbum de Facebook dedicado a ellos, deberíamos mirar en las opciones de privacidad para saber quienes
 pueden ver o no a nuestros preciados tesoros, a nuestras vidas y, definitivamente, no deberíamos ponerlos en el lugar
 más público de todos: poniéndolos como nuestra propia identidad.

\par% p
Ya, si queremos mostrar que nuestro cónyuge es más importante que nosotros mismos, que la razón de nuestra vida es nuestra mascota, que no nos creemos lo suficientemente fotogénicos como para poner nuestra foto y ponemos entonces un paisaje o una caricatura, valga y venga.
 Pero antes que publicarlos deberíamos proteger a nuestras razones de vida.

\par% div% {'style': 'font-size:85%'}
P.D. Si alguien reconoce aquí a su propio hijo, por favor avíseme para que sea menos reconocible (pero recomiendo también hacer lo propio en su perfil, primero).
 Si la foto es de un sobrino, estemos al menos seguros de que los papás estén de acuerdo.
 Y si son fotos de nosotros cuando chicos... bueno, \anchor[http://www.myspace.com/chlewey]{eso no es problema}.

\chapter{Un socialista islamista al control del mundo}
\begin{metadata}
	Published by \anchor[chlewey]{chlewey} on \anchor[http://ewey.co/B276]{Thu, 06 Nov 2008 18:21:01 +0000}\\
	\categories{barack-obama, derecha, estados-unidos, izquierda, opinion}\\
	Shorthand: \anchor[http://blog.chlewey.net/2008/11/socialista-islamista/]{socialista-islamista}
\end{metadata}

\anchor[http://en.wikipedia.org/wiki/John\_McCain]{John McCain} fue un muy buen candidato.
 Personalmente de mis afectos por representar el lado más liberal del partido republicano en contraste con el
 conservadurismo religioso de \anchor[http://en.wikipedia.org/wiki/George\_W.\_Bush]{George W. Bush}.~ Sin duda si McCain hubiese sido el presidente de los EE.UU. este hubiera sido un cambio positivo.

\par% p
McCain representa el ala más progresista del \anchor[http://en.wikipedia.org/wiki/Grand\_Old\_Party]{Grand Old Party}, pero aun así McCain llegó como candidato de los republicanos con todo el elefante a cuestas, incluyendo a su fórmula
 vicepresidencial la ultraconservadora \anchor[http://en.wikipedia.org/wiki/Sarah\_Palin]{Sarah Palin}, y con la carga de ser el candidato republicano tras el gobierno republicano de Bush Jr.

Personalmente creo que McCain equivocó su estrategia al intentar apelar al conservadurismo religioso del GOP, ya que
 estos, ante la perspectiva de un negro socialista e islamista sin duda hubieran vencido sus temores de votar por un
 liberal republicano como McCain, así este hubiese concentrado sus esfuerzos por apelar a los independientes e
 indecisos.

\par% p
No dejo de creer que el mundo hubiese sido un poco mejor si hace ocho años McCain hubiese sido el \anchor[http://en.wikipedia.org/wiki/Republican\_Party\_(United\_States)\_presidential\_primaries,\_2000]{candidato republicano}, bien sea que los republicanos hubiesen subido al poder o que Al Gore hubiera continuado el mandato demócrata.

Pero McCain no fue el presidente de los EE.UU. en 2001, ni lo será en 2009.
 El turno de relevar al conservadurismo bushiano corresponde ahora a Barack Hussein Obama, Jr.

Sin conocer bien quien era Obama, no niego que fue mi favorito frente a Hillary Rodham Clinton primero y frente a John McCain después, y en la medida en la que me fui enterando mejor quien era, más atractivo se me hacía.
 Como todo el mundo, incluyendo la mitad de los EE.UU. me he sentido atraído por el carisma de este mulato quien fuese
 presidente del principal periódico especializado en derecho (el Harvard Law Review), un gran orador, y una de las
 personas más inteligentes que ha disputado en la carrera por la Casa Blanca.

Hijo de un político kenyano, criado en su primera infancia por su padrastro indonesio, y luego por sus abuelos blancos
 en Hawaii, líder de los derechos civiles en Chicago, senador de Illinois y después por Illinois, la historia de Obama
 no deja de ser interesante por si misma, pero lo que sí es claro es que Obama es un maestro de los medios y del
 marketing.

\par% p
En su discurso de aceptación del triunfo, cuando la mayor parte de los candidatos en los EE.UU., Colombia, e imagino la
 mayor parte del mundo, se sube a la tarima con la familia, el jefe de campaña, el vice y todo el circo para dar su
 discurso en medio de una fiesta, Obama se paró solo frente al podio y pronunció un discurso emocional y efectista, que
 en ocasiones me recordó a ese otro presidente oriundo de Illinois quien hace más de siglo y medio liberó a los
 esclavos y pronunció uno de los dos discursos más famosos más famosos en los EE.UU.: el \anchor[http://en.wikisource.org/wiki/Gettysburg\_Address]{discurso de Gettysburg}.~ El otro discurso famoso en los EE.UU. fue el ``\anchor[http://www.americanrhetoric.com/speeches/mlkihaveadream.htm]{I have a dream}'' de Martin Luther King, y sin duda Barack Hussein Obama tomó elementos de Lincoln y de King para armar su propio
 discurso y arrancarle lágrimas a la audiencia.

Pero así como me enamoré de Obama sin saber quién era, que proponía, o qué decía; ahora que será el nuevo presidente de
 la nación más poderosa del planeta es bueno que nos enteremos quién es y qué propone... y finalmente qué es lo que va
 a hacer.

Barack Obama es un agnóstico bautizado dentro del catolicismo, como yo.
 Una mezcla de razas con una formación juvenil internacional, como yo.
 Pero más allá de estas concordancias, y de decirnos que sí se puede, y que él es el cambio, es más lo que sabemos de
 sus políticas por lo que los detractores dicen que lo que los medios han reportado directamente sobre él.

Bueno, lo primero que puedo decir es que Obama no es ni remotamente socialista.
 Sí, le coquetea a algunos aspectos de la socialdemocracia europeo-canadiense, en cuestiones como la seguridad médica universal.
 Pero ¿alguien en su sano juicio podrá acusar a Adenahuer o a Pinochet de socialistas por procurar una seguridad médica universal?
 En algunos aspectos los EE.UU. ha sido atípico frente a los otros países del G7: los siete países económicamente más
 poderosos del planeta, y pretender defender el sistema de los EE.UU. y considerar a todo lo demás como socialismo es
 una aberración de la forma como los EE.UU. han sido forjados, principalmente desde la era Reagan.

Y, luego, que acusen de islamista a un señor que ha propuesto bombardear, de ser necesario, a Pakistán para perseguir a los fundamentalistas musulmanes, creo que tiene un concepto un poco distorcionado de lo que es el islamismo.
 (Bueno, sobresimplifico, pues los musulmanes se han venido matando entre ellos de la misma forma como los cristianos
 lo hemos hecho entre nosotros.)

Si Fidel Castro ha escrito que admira a Obama (y no, no pidió el voto por él), o si Chávez dijo querer que Obama sucediera a Bush ¿qué tiene esto que ver con Obama?
 Estas frases no dicen nada sobre el objeto sino sobre el sujeto que las pronuncia.
 O bien ellos han sido embrujados como el resto del mundo por el carisma de Barack Hussein Obama, o bien, tras ocho
 años de personalizar su antiimperialismo yanki en la figura de George Walker Bush, simplemente ven en lo que Obama
 representa una opción más palatable, pero, particularmente, una derrota de Bush y su partido y sólo por eso,
 independientemente de lo que Barack Obama realice en política internacional, es que la izquierda ha preferido a Obama
 sobre McCain.

Lo cual en nada prueba que Obama sea de izquierda.

\chapter{Social networking}
\begin{metadata}
	Published by \anchor[chlewey]{chlewey} on \anchor[http://ewey.co/B284]{Sat, 13 Dec 2008 03:31:36 +0000}\\
	\categories{facebook, information, social-media, twitter}\\
	Shorthand: \anchor[http://blog.chlewey.net/2008/12/social-networking/]{social-networking}
\end{metadata}

What does web-based social networking means for you?

\par% p
Every one of us live in a social network composed by family, friends, neighbors, classmates, coworkers, people you meet
 every day in the street, store attendants, business partners, people in the media that we either admire or loath, and,
 if we are celebrities ourselves, the people that admire, support, or follow us.  We might have joined a club, or a
 school play, meeting people who share an interest and who otherwise we might not have met.  People that come in and
 out in our lives but that leave a strong impression, people we met by chance or by mistake but with who we connect.
 People we probably never met physically, but to who we pen-pal, meet at \relax{% {'style': 'font-variant: small-caps'}
bbs}, Usenet, or web forums, or just chat. So that is a social network, the sum of all people we connect, and the way we
 connect with them.  Not every people in our social network means the same for us, and not to everyone we approach
 equally.  I might trust my banker my money, but not my drinking friends.  But I would trust my drinking friends
 details of my family I will not tell my banker.  On a more practical approach, our social network is the set of people
 we rely on for different aspects of our lives: being fed, pursuing goals together, having fun, having our backs
 scratched, etc.  On the other hand, social networking services are not social networks.  They are just enablers of
 social networking.  Sometimes I see my over 700 \emph{friends} in \anchor[http://www.facebook.com/]{Facebook}.  Quite a big number, but most of them do mean something to me.  There are a few people I had never met in person, but
 who I price for having met them in virtual communities, including people I met in Facebook itself.  There are people
 from high school I barely met back then but who represent an important part of my existence, and who I would anyhow
 trust with a few secrets.  But there is also some people who is very important in my social network who just refuse to
 join Facebook, from privacy and philosophical concerns to simple tech-fobia (or just tech-apathy).  But Facebook is
 not my only social networking service.  I \anchor[http://twitter.com/chlewey]{twitter} (not much, I am just beginning, and actually I had no followers to tweeter to).  I have some activity in \anchor[http://www.linkedin.com/in/chlewey]{Linked In}, less so in Hi5, Tagged, or MySpace.  I barely use Windows Live Messenger (MSN Messenger) to chat with family for
 family business (and to get the occasional virus link from an infected friend).  An, of course, I blog.  So, I ask
 myself what web-based social networking means for me.

\begin{itemize}

\item It meant my previous job, but probably also contributed to losing it.
\item  It meant a few congratulations in my recent birthday, some friends that confirmed to come to my birthday meeting, and
 none of them actually coming (well, to be fair, my \anchor[http://luzelenathompson.blogspot.com/]{sister} did came).
\item  It means a probable opportunity to meet with my high school friends, or at least the one forth that did confirm but
 had not agreed on the actual time to meet. \begin{small}
(see postdatum bellow)
\end{small}

\item It means a blog that is barely read and very seldom commented about.
\item But, for me, it actually means hope.
 Hope that one day I can actually make a difference through the people I had met, either personally or through the web.
 Hope that society can be changed for better through people coming together for a greater aim.
 Hope that social networking services prove to be a real enabler of social transactions, an enabler of society, and not
 just a bare communication channel for our eclectic but limited social networks of our own.

\end{itemize}

Of course, hope is not enough, and hope without action is just an empty painkiller to which you can get addicted.  So
 probably I should just take a radical approach and cut myself from social networking services: freezing this blog,
 shutting down my Facebook profile, removing silent contacts from my Windows Live Messenger, and twitter no more.  I
 would rather monetize my hope, but I had a little problem with money: I do not want money (I just need it, this
 society still is a monetary society and some \anchor[http://www.zeitgeistmovie.com/]{non-monetary utopias} are just utopias, and I do want things money can get, but money itself I just don't care much about).  So, basically
 this is an early farewell for my followers, 'cuz at the time I actually decide to cut myself from social networking
 services, no warning will precede.  For the time being, I am just interesting in knowing: what does web-based social
 networking means to you?

\horrule{}Postdatum: ~90 people graduated with me from Highschool.  26 confirmed through FB.  4 actually appeared. \begin{small}
— 15 December, 2008
\end{small}

\chapter{Feliz…}
\begin{metadata}
	Published by \anchor[chlewey]{chlewey} on \anchor[http://ewey.co/B292]{Sun, 28 Dec 2008 12:58:47 +0000}\\
	\categories{fechas, personal}\\
	Shorthand: \anchor[http://blog.chlewey.net/2008/12/feliz/]{feliz}
\end{metadata}

\relax{% {'style': 'font-size:1.35em;font-family:fantasy'}
Feliz \anchor[http://es.wikipedia.org/wiki/Solsticio\#El\_solsticio\_de\_diciembre]{solsticio austral}, feliz \anchor[http://es.wikipedia.org/wiki/Perihelio]{perihelio}, feliz \anchor[http://es.wikipedia.org/wiki/Navidad]{navidad}, feliz \anchor[http://es.wikipedia.org/wiki/J\%C3\%A1nuca]{jánuca}, felices \anchor[http://es.wikipedia.org/wiki/D\%C3\%ADa\_de\_los\_Santos\_Inocentes]{inocentes}, felices \anchor[http://es.wikipedia.org/wiki/Saturnales]{saturnales}, feliz inti yaco, felices ferias, felices reyes, feliz \anchor[http://es.wikipedia.org/wiki/Fiesta\_de\_Yule]{yule} o \anchor[http://es.wikipedia.org/wiki/Litha]{litha} (escoje tu hemisferio), feliz \anchor[http://es.wikipedia.org/wiki/Kwanzaa]{cuansa}, feliz \anchor[http://zh.wikipedia.org/wiki/\%E5\%86\%AC\%E8\%87\%B3]{dongzhi}, feliz \anchor[http://es.wikipedia.org/wiki/A\%C3\%B1o\_Nuevo]{año nuevo}, ... en fín: Felices Fiestas.
}
\par% div% {'style': 'text-align: center;', 'class': ['aligncenter']}
\anchor[http://www.flickr.com/photos/chlewey/3131821128/]{\includegraphics[width=300\px,height=225\px]{blog/dsc00225-300x225.jpg}}

\chapter{Y he aquí el 2009}
\begin{metadata}
	Published by \anchor[chlewey]{chlewey} on \anchor[http://ewey.co/B302]{Wed, 07 Jan 2009 04:35:02 +0000}\\
	\categories{activismo, fechas, personal, proyeccion-y-carrera, social-media}\\
	Shorthand: \anchor[http://blog.chlewey.net/2009/01/y-he-aqui-el-2009/]{y-he-aqui-el-2009}
\end{metadata}

Los inicios de año son convenciones más o menos arbitrarias que poco significan más allá del significado que querramos
 darle, pero aún así son fechas convenientes para recapitular nuestros pasados y mirar nuestros porvenires.

Este año me enfrento ante un dilema: aventurar sobre caminos desconocidos y posiblemente tortuosos de ser independiente
 y a hacer lo que quiero, a mi ritmo y explotando plenamente mis capacidades, o simplemente buscar a toda costa un
 camino seguro, caminos estos últimos que mi experiencia han mostrado no ser tan seguros.

\par% p
El truco aquí es convertir a Internet, a Web 2.0, a las \anchor[http://blog.chlewey.net/social-networking]{redes sociales}, a este blog, etc. en algo más que el gran escape y convertirlos en potenciadores de nuevos retos, para mí y para la
 sociedad en la que vivo.

Tres conceptos claves: educación, política y sociedad.

Ya voy atrasado en tomar ese rumbo.  Pero aquí vamos.

\chapter{Happy first year!}
\begin{metadata}
	Published by \anchor[chlewey]{chlewey} on \anchor[http://ewey.co/B308]{Thu, 08 Jan 2009 14:01:40 +0000}\\
	\categories{familia, personal}\\
	Shorthand: \anchor[http://blog.chlewey.net/2009/01/happy-first-year/]{happy-first-year}
\end{metadata}

\par% p
One of the best things in my live had been to be a parent.  It is also demanding, as I am responsible for the care and
 nursery of  two brand new lives.  Today my youngest kid is celebrating her first birthday, and this has been an
 amazing year.  Yet, I want to share all my joy with you guys.

\par% p% {'style': 'text-align: center; font-size:1.5em;'}
\includegraphics[width=480\px,height=640\px]{blog/dsc00496.jpg}¡Feliz primer año, Ana María!

\chapter{Buscando mis raíces}
\begin{metadata}
	Published by \anchor[chlewey]{chlewey} on \anchor[http://ewey.co/B312]{Sun, 11 Jan 2009 14:38:03 +0000}\\
	\categories{activismo, futuro, personal, proyeccion-y-carrera, social-media, web}\\
	Shorthand: \anchor[http://blog.chlewey.net/2009/01/buscando-mis-raices/]{buscando-mis-raices}
\end{metadata}

Uno de los consejos que he recibido en estos últimos días es que debo regresar a mis raíces.
 Lo ha dicho una de las personas que más me ha marcado en mi vida, que más ha influido en mí y a quien yo más admiro. Y
 no es sólo porque sea mi padre.

La pregunta, sin embargo, es cuáles son mis raíces.
 En alguna época de mi vida me destaqué por las matemáticas.
 Nunca fui un estudiante ejemplar, pero mi facilidad y luego pasión por las matemáticas me hicieron destacar, viajar y triunfar.
 Lo intenté más tarde, pero coyunturas de mi vida laboral se convirtieron en un obstáculo para algo que entonces me
 seguía gustando pero que no me apasionaba del mismo modo.

Aun quiero seguir con las matemáticas.
 La pasión no será la misma que en mi adolescencia pero sigue ahí, pero hay otras pasiones que también han surgido, que
 han aflorado.

\par% p
Tengo una necesidad imperiosa de dar de lo que sé —desafortunadamente no es una necesidad que se manifieste en una actitud activa.
 Participaba en \anchor[news:alt.usage.spanish]{alt.usage.spanish}, me hice bibliotecario de la Wikipedia, y he coqueteado con la docensia por este motivo.
 Tal vez sea también una cuestión de ego (¿qué blogger no lo tiene?), pero en últimas más que una cuestión de créditos,
 es una cuestión de querer sinceramente que este sea un mundo mejor gracias a lo que pueda aportar.

Pero lo que puedo aportar en un newsgroup, o una enciclopedia virtual, o un blog eclectico, no es realmente mucho,
 pero, particularmente, no es algo que retribuya a la fuente de una forma que la fuente pueda vivir tranquilamente y
 sacar adelante una familia.

En los últimos tres meses (desde antes, incluso) he tenido que replantearme todo el concepto de quién soy y qué quiero hacer para lograr los objetivos materiales y espirituales de mi vida y los de mi familia.
 Que haya tenido que replanteármelos no significa que lo haya hecho de forma consistente y tan solo flashes de lo que
 quiero es lo que he logrado.

Una de ellas es que tengo que replantear completamente mi carrera profesional.
 Salí de la universidad a convertirme un técnico de soporte; y creo que desde un principio fue claro que el soporte técnico no era mi vocación pero pasé siete años de mi vida profesional en esa labor.
 Hoy en día esa es la experiencia profesional que tengo para mostrar, y replantear mi carrera implica comenzar desde el principio.
 Pero ya tengo 36 años, una familia y muchas deudas.
 Laboralmente soy viejo y costoso para un empleador, salvo que tuviera referencias profesionales y experiencia que no
 tengo.

Así que tengo que regresar a mis raíces.
 Tengo que buscar mis raíces.~ ¿Son mis raíces las matemáticas?~ ¿o es cuestión de buscar mis nuevas pasiones?

Las matemáticas siguen siendo una pasión, pero una pasión que aún no sé como monetizar, comenzando porque los únicos
 estudios formales de mátemáticas como carrera fue una maestría que comencé y nunca terminé.

La otra pasión que mencioné es el dar intelectualmente.
 Es poder dar de lo poco que sé y de lo que puedo analizar, para que otros puedan aprender y mejorar sus propias vidas.
 Desafortunadamente otra de mis frustraciones profesionales ha sido la docencia, tanto por no poder ejercerla como por
 no haber sido un excelente comunicador cuanto pude intentarlo.

La tercera pasión es la política.
 Soy uno de los muchos niños que soñó con ser presidente, pero ciencias sociales no era mi fuerte, así que ganó la ingeniería cuando escogí mi carrera.
 Hoy sigue importándome qué pasa y qué posición, crítica o no, puedo tomar ante los distintos temas de actualidad
 nacional e internacional.

Por eso participé opinando en las marchas del 4 de febrero y seguí opinando en el grupo de Facebook en el que se gestaron hasta que resulté incómodo a la versión de la realidad que ahí querían mostrar.
 Por eso la mitad de las entradas en este blog trata de temas políticos mientras no hay entradas de temas ingenieriles.
 ¿Puedo buscar mis raíces obviando que soy un animal político?

Pero la política, tal cual la estoy ejerciendo, tampoco me aporta económicamente.

Ya es hora de que tome mis pasiones y empiece a exprimirles el jugo.
 A buscar mis raíces y lograr que ellas sean el sustento de mi vida y de la de mi familia.

Así que si crees que yo alguna vez te he dado algo, comienza por enviarme un dólar.
~ (O al menos un dólar virtual en forma de un abrazo virtual.)

He dicho.

\chapter{Lomo de cerdo al curry-BBQ con papas al autovapor}
\begin{metadata}
	Published by \anchor[chlewey]{chlewey} on \anchor[http://ewey.co/B315]{Mon, 19 Jan 2009 17:09:33 +0000}\\
	\categories{cocina, information}\\
	Shorthand: \anchor[http://blog.chlewey.net/2009/01/al-curry-bbq/]{al-curry-bbq}
\end{metadata}

Bueno, aquí viene la primera entrada culinaria de este blog.
 No tendría sentido hacer un blog culinario porque aunque me gusta experimentar en la cocina, no es que le dedique el
 suficiente tiempo ni siempre valga la pena comentar los resultados.

\par% p
Pero quería comentar los resultados del experimento de ayer.

\subsection{Lomo de cerdo al curry-barbecue}
Ingredientes:
\begin{itemize}

\item Una libra de lomo de cerdo, tajado.
\item Un litro de Bretaña
\item Especias y sal al gusto.
 En mi caso fue Especias para paella (clavo, pimienta, nuez moscada, romero y tomillo) y Curry (undisclosed
 ingredients) de La Barraca.
\item Salsa barbecue lista.~ En mi caso Honey Mustard Barbecue Sause de Hunt's

\end{itemize}

Preparación:

Poner en una olla la bretaña y poner a calentar a fuego alto.
 Agregar el cerdo.
 Cuando comience a hervir, agregar las especies y la salsa de barbacoa y bajar a fuego medio.
 Dejar hervir hasta que la mezcla de bretaña, salsa y especias tenga consistencia de salsa.

\par% p
Antes de servir pescar la carne y pasarla por la plancha, y el resto de la salsa servirla en una salsera.

\subsection{Papas al autovapor}
Ingredientes:
\begin{itemize}

\item Una libra de papa sabanera.
\item Mayonesa
\item Eneldo.

\end{itemize}

Preparación:

Se lavan las papas (de ser necesario) y se pelan.
 Se introducen en una bolsa plástica y se sella bien asegurándose de sacar el máximo de aire.
 Se mete la bolsa con las papas por 15 minutos en el horno microondas.

(La receta original era para papa criolla, y con la criolla bastan 5 minutos para una libra larga de papa.)

\par% p
Una vez termina y se comprueba que las papas están en su punto, se cubren de mayonesa y eneldo al gusto.

\subsection{Sírvase}
Puede servirse el cerdo pasado a la plancha, las papas, la salsa curry-bbq en una salsera, y acompañarse de una ensalada.
 En mi caso fue cohombro con lechuga romana aderezados con aceite de oliva y vinagre balsámico.

Disfrútese.

\chapter{Un año}
\begin{metadata}
	Published by \anchor[chlewey]{chlewey} on \anchor[http://ewey.co/B324]{Wed, 04 Feb 2009 14:51:05 +0000}\\
	\categories{facebook, farc, opinion, polarizacion}\\
	Shorthand: \anchor[http://blog.chlewey.net/2009/02/un-anno/]{un-anno}
\end{metadata}

\par% p
Hace un año participé en la gran marcha en contra de las FARC y tangencialmente en su organización, y ante la crítica
 de algunos sectores sobre la visión limitada de esa iniciativa, mi posición es que \anchor[http://blog.chlewey.net/2008/02/si-hoy-pudieras]{había algo que decir}, así no fuera todo lo que hubiera que decir.

\par% p
Hoy recibimos este aniversario tras una serie de eventos.~ Las muertes de \emph{Raúl Reyes} (abatido en confusos hechos), \emph{Iván Ríos} (asesinado) y \emph{Manuel Marulanda} (por causas naturales) en marzo.
 El rescate de 4 rehenes civiles y 11 uniformados en la Operación Jaque, incluyendo la así llamada joya de la corona: Íngrid Betancourt.
 Deserciones como la de \emph{Karina} primero, y luego las de cuidadores de reenes que le garantizaron la libertad a Óscar Tulio Lizcano, entre otros.
 Hay muchos signos que apuntan a un debilitamiento de la guerrilla.

\par% p
Hoy asistimos a la liberación “unilateral” de seis secuestrados, que dejarán a la guerrilla sin rehenes políticos y tan
 solo con oficiales, suboficiales y \emph{retenidos }(secuestrados extorsivos).
 Cuando escribo esto llevamos ya dos de tres actos: la liberación de los uniformados el domingo y la de Alan Jara el martes.
 Este jueves será la liberación de Sigifredo López si no hay contratiempos.
 Dos actos llenos de espectáculos.
 Los periodistas Jorge Enrique Botero y Hollman Morris se encargaron de amenizar el domingo.
 No sabemos si intencionalmente o utilizados por la guerrilla; y en el caso de Morris, aumentado por la actitud del
 gobierno quien pretendió enmendar su falta de objetividad por medio de la censura y no del rebatimiento de argumentos.

\par% p
Ayer, martes, Alan Jara en sus declaraciones libres no fue en momento alguno elogioso con el gobierno y acusó a Uribe
 de prolongar innecesariamente la situación de los secuestrados \emph{canjeables}.
 No estoy de acuerdo con Jara y su tesis de homicidio por omisión, pues considero en este caso al gobierno responsable mas no culpable.
 Y no más responable que todos nosotros como sociedad.

\par% p
Sí algo he aprendido en este año es a que mi respeto por Álvaro Uribe Vélez (la persona que ocupa el cargo de Presidente de la República, no el cargo en sí) ha venido disminuyendo.
 Salvo su política de seguridad y el micromanagement de los consejos comunitarios, ese hombre no tiene propuestas políticas claras, sino que improvisa a cada rato.
 No tiene una política clara siquiera en si sigue desbaratando la constitución de 1991 para reelegirse.
 Considera que su principal virtud es no oir consejos.~ Pero si hay algo que me asusta más que un presidente así, es el \anchor[http://blog.chlewey.net/2008/03/uribista-acritico]{seguimiento ciego al caudillo}.  Es tal el punto de ese seguimiento ciego que personas que han sido ideológicamente afines con Uribe pero se han
 distanciado de él ante la propuesta segunda reelección, tales como Germán Vargas Lleras y \anchor[http://www.ginaparody.com/]{Gina Parody}, son tratados como traidores.

Y en cuanto a las FARC.
 Hoy las FARC son mucho menos de lo que eran en el año 2000, en número de combatientes, control territorial, poder armamentista, capacidad de hacer daño, etc.
 Pero hoy las FARC son todavía más de lo que eran en 1980, con un posible agravante: quienes aun están (los que no se
 han fugado) están políticamente más comprometidos que nunca.

Las FARC nunca han estado a punto de tomarse el poder en Colombia.
 Ni en su punto más fuerte en 1997-99.
 Pero tampoco están a punto de ser derrotadas.
 Creo que todavía es posible la derrota militar de las FARC, a un muy alto costo.
 A muchos desde la comodidad de nuestras ciudades o de las carreteras despejadas no nos importa asumir ese costo.
 Particularmente porque, sin una derrota de las FARC, ellos no querrán nada distinto a acomodar la constitución a su
 antojo como parte de una negociación.

\par% p
Desafortunadamente tras un año de haberles dicho \emph{¡No a las FARC!}, aun no veo el fin del problema y sí una polarización creciente entre los que no quieren a las FARC y por ello apoyan
 a Uribe y los que no quieren a las FARC pero tampoco al costo de más Uribe.

\chapter{Análisis asimétrico}
\begin{metadata}
	Published by \anchor[chlewey]{chlewey} on \anchor[http://ewey.co/B330]{Mon, 09 Feb 2009 23:53:40 +0000}\\
	\categories{actualidad, farc, opinion, uribismo}\\
	Shorthand: \anchor[http://blog.chlewey.net/2009/02/analisis-asimetrico/]{analisis-asimetrico}
\end{metadata}

\par% p
«La guerra es la continuación de la política por otros medios», decía \anchor[http://es.wikipedia.org/wiki/Carl\_von\_Clausewitz]{Carl von Clausewitz}.  \anchor[http://es.wikipedia.org/wiki/Carl\_Schmitt]{Carl Schmitt} respondía que «la política es la continuación de la guerra por otros medios».  Nuestro gran filósofo político \anchor[http://es.wikipedia.org/wiki/José\_Obdulio\_Gaviria]{José Obdulio Gaviria} simplemente nos propone que aquí no hay guerra.

\par% p
Cuando se habla de guerras se propone el concepto de \anchor[http://en.wikipedia.org/wiki/Asymmetric\_warfare]{guerra asimétrica}, esto es un conflicto bélico entre una parte que posee un ejército regular y otra que dice que no puede financiar tal, y por ende ajusta sus prácticas a obtener mayores beneficios con pocos gastos: emboscadas, sabotaje, apoyo (y mimetismo) en la población civil, etc.
 En otras palabras guerra de guerrillas y terrorismo.
 Desde luego que hablar de guerra asimétrica implica reconocer cierta legitimidad al combatiente menor.
 Quien resta tal legitimidad preferirá hablar de \anchor[http://es.wikipedia.org/wiki/Terrorismo]{amenaza terrorista}.

Así que no sólo existen guerras asimétricas, sino también un uso asimétrico del lenguaje en el análisis de un conflicto. La parte mayor se verá tentada a no hablar de guerra sino de terrorismo y llamar así terroristas al contendiente menor.
 Esto pasa en el conflicto palestino-israelí y pasa en el conflicto interno colombiano.
 Así el contendiente menor deja de ser un contendiente y se convierte en un simple criminal, con toda la carga propagandística que esto conlleva.
 El contendiente menor insistirá en que es una guerra legítima, buscando las contradicciones en el discurso del
 contendiente mayor.

\par% p
Un ejemplo claro de esta asimetría en el lenguaje parte del problema del intercambio humanitario, que parecía enterrado
 tras la Operación Jaque, pero cobra actualidad tras la reciente liberación \emph{unilateral} de Jara, López y los cuatro uniformados.

\par% p
Primero, recordemos que el término \emph{intercambio humanitario} no lo propusieron ni las FARC ni el gobierno, sino el ex presidente Alfonso López Michelsen.
 El término propuesto por las FARC ha sido \emph{canje de prisioneros}, y el del gobierno \emph{soltar terroristas por rehenes}: asimetrías del lenguaje.

\par% p
Las FARC se autoconsideran un \emph{estado en gestación} (palabras de \emph{Simón Trinidad} durante los diálogos del Caguán) y como tales se autoconsideran con derecho a tener combatientes, prisioneros, e impuestos.
 Un ejemplo de los impuestos es su ley 002, por medio de la cual pretenden justificar la extorsión (\emph{colaboración a la causa} la llaman ellos) a personas naturales y jurídicas pudientes bajo la amenaza de que si estas no cumplen sus \emph{obligaciones}, serán retenidas hasta que sus allegados paguen.

En principio ante un estado legítimo, si no pagamos nuestros impuestos podríamos ser encarcelados por incumplir nuestras obligaciones, así no estemos de acuerdo con la destinación que nuestros gobernantes dan a nuestros tributos.
 Declararme libertario no es objeción de conciencia legal para eludir el pago de impuestos.
 Esto hace que, desde el punto de vista de ellos, de las FARC, su ley 002 sea legítima y consideren legítimo el
 secuestro extorsivo.

Así mismo, como parte combatiente (beligerante) que ellos se consideran, se arrogan el derecho de tener prisioneros de guerra, entre los combatientes (soldados y policías) e ideólogos (políticos) del otro bando.
 Y se consideran con el derecho de pedir un canje con prisioneros que el otro bando retiene.

Claramente (y lo hemos visto), cuando un soldado o un policía es liberado por las FARC, regresa como miembro pleno de las fuerzas armadas constitucionales.
 Cuando un político es liberado, regresa a hacer política.
 De conservar la simetría de diálogo que pretenden las FARC, los guerrilleros liberados seguirán siendo parte de las FARC, con plenas funciones de combatiéndes o ideólogos.
 De eso se trata un canje.

Nuestro gobierno hace una lectura distinta de la situación.

Primero, para el gobierno las FARC no son una fuerza beligerante legítima, sino un grupo de delincuentes y terroristas, quienes perdieron cualquier pretención de representar una causa cuando empezaron a recurrir a hechos terroristas.
 De ahí que proclamas como la tal ley 002 no es más que una burda justificación del delito de secuestro extorsivo, uno
 de los delitos más duramente perseguidos y castigados en Colombia.

\par% p
Igualmente, y como las FARC son un grupo terrorista, todos sus miembros son por ende terroristas y hampones (no importa
 si han participado o no de presuntos actos terroristas, su sola vinculación a las FARC los hace terroristas y
 hampones), y no se puede justificar que estos terroristas y hampones condenados (o en espera de condena) salgan libres
 a cambio de ciudadanos de bien y legítimos representantes del estado.

\begin{tabular}{lllllllll}% FIX

Ente &
Versión FARC &
Versión Gobierno \\

Gobierno constitucional &
gobierno ilegítimo &
legítima representación del pueblo \\

FARC &
estado en gestación y ejército del pueblo &
grupo terrorista \\

Conflicto &
guerra civil (asimétrica) &
amenaza terrorista \\

Civiles privados de la libertad por las FARC &
retención por el no pago de contribuciones a la causa &
secuestro extorsivo \\

Políticos\emph{canjeables}privados de la libertad &
prisioneros políticos &
secuestro de civiles \\

Uniformados privados de la libertad &
prisioneros de guerra &
secuestro de policías y soldados \\

Guerrilleros privados de la libertad por el estado &
prisioneros de guerra y prisioneros políticos &
terroristas y hampones \\

Intercambio de personas privadas de la libertad &
canje de prisioneros &
soltar terroristas y hampones a cambio de rehenes \\

\end{tabular}

Las asimetrías del análisis van más allá del problema de los secuestrados y criminales convictos, y tocan también los alcances de una posible solución negociada.
 Mientras las FARC insistirán en negociar de tú a tú con el gobierno para llegar a un nuevo acuerdo constitucional;
 para el gobierno la única opción es el sometimiento de las FARC a las leyes actuales.

\par% p
Pero para las personas que hoy en día están sufriendo prisioneros en el monte, esta asimetría del análisis se vuelve en una sinsalida.
 El gobierno ha insistido en que sólo soltará guerrilleros que renuncien a ser guerrilleros, lo cual es inaceptable
 para las FARC, y las FARC insisten en que sólo accederan al canje bajo la premisa de que todos los combatientes de las
 FARC queden libres, lo cual no sólo es inaceptable para el gobierno, sino imposible cuando guerrilleros clave como \emph{Simón Trinidad} y \emph{Sonia} están prisioneros bajo la jurisdicción del departamento de justicia de los Estados Unidos.

\par% p
Hay mucho más de fondo en este análisis asimétrico, porque el lenguaje del gobierno es lo suficientemente vago como
 para insinuar que todo el que disienta de sus propias tesis es un criptoguerrillero o, como lo diría este pasado fin
 de semana: miembro del frente intelectual de las FARC.

\par% div% {'style': 'font-size: 91%;'}
Sólo por claridad del discurso, he aquí mi propio lenguaje: el gobierno constitucional encabezado por Álvaro Uribe Vélez es legítimo (si bien viciado).
 Las FARC son un grupo guerrillero que ha recurrido al terrorismo pero que tiene pretenciones legítimas que no justifican que sigan alzadas en armas.
 Los secuestrados por las FARC son secuestrados y la ley 002 es una ley ilegal que no emana de ningún estado.
 Los políticos, militares y policías secuestrados son rehenes y son secuestrados.
 Y los secuestrados (extorsivos, políticos o uniformados) seguirán siendo secuestrados mientras las FARC no garanticen principios mínimos de los que gozan los prisioneros del estado, como son visitas de familiares, asesoría legal, verificación por la defensoría del pueblo y las ONG, debido proceso, etc.
 Los guerrilleros en prisión son prisioneros políticos si su único cargo es rebelión, o reos convictos si lo son de otros delitos no conexos a la rebelión.
 Los actos de guerra dirigidos contra población civil son actos terroristas y no son conexos a la rebelión.
 El intercambio entre los rehenes que tienen las FARC y los guerrilleros convictos por el estado no deslegitima los principios de la seguridad que compromenten al estado y puede ser una opción práctica para garantizar que algunos colombianos recuperen su libertad.
 Desde luego, sería mucho mejor si las FARC renunciaran al secuestro como método de lucha, soltando a todos los
 secuestrados actuales sin contraprestación.

\chapter{Proyección comunitaria}
\begin{metadata}
	Published by \anchor[chlewey]{chlewey} on \anchor[http://ewey.co/B333]{Fri, 06 Feb 2009 17:27:28 +0000}\\
	\categories{olpc, personal, wikimedia, wikipedia}\\
	Shorthand: \anchor[http://blog.chlewey.net/2009/02/proyeccion-comunitaria/]{proyeccion-comunitaria}
\end{metadata}

\par% p
Hace más de un año \anchor[http://blog.chlewey.net/2007/10/wikimedia-y-olpc]{tenía algunas ideas en mente}, que por diversas razones no he logrado concretar.
 Hoy, por motivos diferentes, he tratado de definirme como una persona que busca facilitar los procesos de formación de comunidad por medio de las tecnologías.
 Hay cosas muy vagas en eso, y como pinta la situación, no mucha plata para mí y mi familia, pero sin duda muchas cosas
 se juntan.

\par% p
Anoche hablaba con una persona que ha trabajado con implementaciones tecnológicas en la comunidad \anchor[http://es.wikipedia.org/wiki/Nasa]{nasa (paez)}.~ Estoy todavía buscando información sobre la escritura del \anchor[http://www.google.com.co/search?q=\%22nasa+yuwe\%22]{nasa yuwe} o lengua paez, pero aparentemente sus muchos sonidos se reflejan en una ortografía no apropiada para un teclado \relax{% {'style': 'font-variant: small-caps;'}
qwerty} y los estudios de usabilidad muestran que los íconos del \anchor[http://laptop.org/en/laptop/]{XO} no son los más intuitivos.
 Pero antes de ir más allá debemos preguntarnos primero qué es lo que queremos lograr implantando tecnologías de la
 información en una sociedad principalmente rural.

\par% p
Tratando de no asumir una actitud paternalista donde yo, como parte de la comunidad \emph{mainstream} intento imponer mis valores sobre una cultura amenazada, o intento preservarla como a una reserva ecológica, debo
 pensar (y preguntarles) en qué se beneficiarían asimilando las tecnologías dentro de su propia cultura.

Yo puedo creer que permitiendo que ellos se apropien de la tecnología, y pudiendo publicar contenidos en su propia lengua, pueden al mismo tiempo conservar lo escencial de su cultura e insertarse en el mundo moderno.
 Puedo creer que un niño paez hoy tiene que escoger entre preservar la lengua y cultura de sus abuelos, o pasar a la modernidad donde pueda explorar nuevos horizontes, y esto mismo es una amenaza a la cultura en la medida en la que nuevos niños crezcan queriendo irse.
 Pero esta escogencia no necesariamente tiene que ser así.
 Este mismo niño deberá poder explorar esos nuevos horizontes del mundo moderno desde su propia cultura ancestral.

Pero eso es lo que yo creo, porque nunca tuve ese dilema.
 Antes de tomar una decisión es importante saber qué creen ellos.

\chapter{Un día para reflexionar sobre el papel de media humanidad}
\begin{metadata}
	Published by \anchor[chlewey]{chlewey} on \anchor[http://ewey.co/B419]{Sun, 08 Mar 2009 14:07:24 +0000}\\
	\categories{familia, fechas, opinion}\\
	Shorthand: \anchor[http://blog.chlewey.net/2009/03/un-dia-para-reflexionar/]{un-dia-para-reflexionar}
\end{metadata}

Más que caer en lugares comunes, de que no debería ser un día sino todo un año, etc. lo importante es que un día como
 hoy podamos reflexionar sobre la importancia de la mujer en nuestra sociedad, y de cómo esta misma sociedad aún les
 niega presencia y las invisibiliza.

Personalmente me gustaría ver un mundo donde el sexo biológico de las personas (no me gusta el término ``género'' porque
 este hace referencia original a un concepto gramatical, no social) no sea relevante como etiquetador.  Nadie sería más
 o menos por ser hombre o mujer.

He tratado de que mi vida sea así y muchas de las personas importantes en mi vida son y han sido mujeres: mi mamá, mi
 hermana, mis tías y primas, profesoras, amigas, mi esposa y mi hija.  En mi círculo más cercano ha habido más mujeres
 que hombres, en gran medida por la vocación de servicio de las mismas.

Esta igualdad soñada está todavía lejos de ser realidad, si bien reconozco que he vivido en un mundo y un tiempo donde
 ya no parece una utopía.

Así que no me resta más que desearle un muy feliz día a todas mis amigas y lectoras desprevenidas de estas notas.

\chapter{Politiqueando a lo 2.0}
\begin{metadata}
	Published by \anchor[chlewey]{chlewey} on \anchor[http://ewey.co/B425]{Tue, 24 Mar 2009 21:59:16 +0000}\\
	\categories{elecciones, facebook, opinion, social-media, twitter}\\
	Shorthand: \anchor[http://blog.chlewey.net/2009/03/politiqueando/]{politiqueando}
\end{metadata}

\par% p
¿Cuál es el papel real de los medios sociales (social media, web 2.0, o como queramos llamarle) en el acontecer
 político nacional?  ¿Cual es su potencial?  ¿Hasta donde un candidato o un promotor puede \anchor[http://pulsosocial.com/2009/03/02/politicos-quieren-ganar-terreno-en-la-web/]{desarrollar una campaña }por estos medios?

\par% p
Tras mi \anchor[http://patadas-de-ahogado.blogspot.com/2009/01/de-la-ingenuidad-la-hipocresa-y-la.html]{participación tangencial} en la organización de las marchas del 4 de febrero de 2008 en contra de las autodenominadas Fuerzas Armadas
 Revolucionarias de Colombia-Ejército del Pueblo y la forma como estas se llevaron a cabo; tras haber visto la campaña
 presidencial en los EE.UU. y el papel que jugaron las comunidades de internautas; es claro que algo importante hay
 dentro de la forma como se está desarrollando Internet y la política.

Yo soy un gran promotor de los medios sociales, tanto por lo que han significado en mi vida, extrapolando a la vida de
 otros, como por los nuevos paradigmas que estos proponen.  Por otro lado creo que el ser humano es esencialmente un
 animal político, aún cuando decimos muchas veces no entender o incluso odiar a la política.  El sólo hecho de que nos
 planteemos una mejor forma de hacer las cosas y discutamos o intriguemos por lograrlo, tanto en nuestras casas o
 lugares de trabajo, son formas de hacer y asumir la política.  Muchas veces lo que odiamos es realmente la
 politiquería.

Pero, con todo esto, soy aún escéptico de que la política 2.0 sea ya y ahora.

\anchor[http://www.facebook.com/]{Facebook} se ha convertido en Colombia en un referente obligado sobre en qué andan los profesionales y los jóvenes.  Se ve ahí
 desde la banalidad de qué comemos en \anchor[http://www.facebook.com/group.php?gid=2401609457]{Crepes and Wafflers} o qué tanto nos gusta \anchor[http://www.facebook.com/pages/dormir/29316764549]{dormir} o \anchor[http://www.facebook.com/pages/BANARSE-EN-PAREJA/35140311911]{bañarnos en pareja}, cómo nos reímos de quienes creemos que no están a nuestro nivel, o cómo \anchor[http://blog.chlewey.net/2007/10/odio-a-tanto-odio]{odiamos} un sinnúmero de cosas.  Y desde luego, es un lugar donde se apoyan o se odian a políticos y propuestas políticas.  Y
 así mismo es un lugar donde los políticos quieren estar para tener contacto con sus adeptos.

\par% p
Una serie de \anchor[http://www.eltiempo.com/enter/internet/home/twitter-sube-al-podio-de-las-redes-sociales-en-todo-el-mundo\_4819380-1]{artículos} acaba de poner a \anchor[http://twitter.com/]{Twitter} dentro del panorama, y así ya he contado al menos \anchor[http://twitter.com/German\_Vargas]{tres} \anchor[http://twitter.com/rafaelpardo]{políticos} \anchor[http://twitter.com/AntanasMockus]{colombianos}, presidenciables, que han incursionado en este nuevo medio.  Las páginas web han sido usadas por los políticos ya por
 bastante tiempo y muchos han optado también por la moda de los blogs.

Es un medio y un medio cada vez más importante en la medida en la que cada vez más estos políticos utilicen estos
 medios para lograr una comunicación de dos vías, en la que escuchen a sus seguidores al tiempo que expresen sus
 pensamientos y sus propuestas.  Y conservando, también, el criterio de filtrar las respuestas, pues no siempre lo que
 dicen quienes participan refleja lo que el público objetivo silencioso realmente quiere.

Las redes sociales basadas en web, en correo electrónico, etc. pueden ser un medio de bajo costo para divulgar nuevas
 ideas e inquietudes, para impactar.  Casi todos le creemos más a un amigo que a un extraño, y más a un extraño que es
 como uno (un desconocido) que a una celebridad.  Pero la convocatoria final está todavía en manos de celebridades y,
 en el caso particular de Colombia, en celebridades como Julio Sánchez Cristo, Jota Mario Valencia y Juan Gossain.

\par% p
El 4 de febrero de 2008 nació en Internet. \anchor[http://comunidades.semana.com/wf\_infonoticia.aspx?IdNoticia=934]{ Nació en Facebook}.  Nació bajo la red de amigos y contactos de una persona relativamente desconocida en Barranquilla, pero que, dada las
 circunstancias del momento, tuvo un rápido eco trascendiendo voz-a-voz dentro de otras redes de amigos,
 particularmente cuando \anchor[http://www.facebook.com/event.php?eid=7284103926]{se concretó} una forma de hacer sentir ese descontento.  Se convirtió originalmente en una causa con un propósito y una acción
 compartida de voz a voz, y en la medida en la que tomó fuerza logró la atención de los medios.

La causa pudo haber sido deformada de muchas formas.  El mensaje que trató de controlarse, se tergiversó en algunos
 casos, se consolidó de formas excluyentes en otros, o simplemente se reinterpretó por quienes apoyaron y quienes
 contradijeron el fenómeno.  Pero partió de ahí, del voz a voz dentro de redes de amigos y así se consolidó como un
 proyecto del cual participaron finalmente millones de personas en todo el mundo.

Las causas por los derechos de los animales, por el software libre, por combatir la desnutrición, etc. se difunden por
 estos medios sin pagar un peso adicional al que ya se paga por nuestra propia procrastinación.  Tal vez ninguno tenga
 la resonancia del 4 de febrero de 2008, pero el potencial está ahí.

Y ese es el potencial que los políticos en campaña deberían aprender a aprovechar.  Cómo lograr sembrar la semilla para
 que el voz a voz resuene.

\par% p
Personalmente me gustaría ver más debate, y menos confrontación.  Más exposición de ideas y menos \anchor[http://www.juglardelzipa.com/wordpress/2009/03/13/candidatos-e/]{bitácoras de viaje}.  Cosas que valgan la pena retrinar, compartir por Facebook o \anchor[http://friendfeed.com/]{Friendfeed}, y no tanto referentes para criticar.  Lenguajes que lleguen al corazón de las personas y que estas repliquen porque
 les gustó y no porque son funcionarios de la campaña.

Aun esto no será suficiente para ganar elecciones, pero pueden hacer una diferencia, algo que sirva a los propios
 candidatos para ver cómo sus respuestas realmente influyen.  Suponiendo que eso realmente les interesa y no
 símplemente los botines burocráticos del poder.

\chapter{Aguas tibias y el derecho a opinar sobre lo inopinable}
\begin{metadata}
	Published by \anchor[chlewey]{chlewey} on \anchor[http://ewey.co/B436]{Thu, 26 Mar 2009 16:22:24 +0000}\\
	\categories{debate, facebooknote, opinion, twitter}\\
	Shorthand: \anchor[http://blog.chlewey.net/2009/03/aguas-tibias/]{aguas-tibias}
\end{metadata}

Muchas veces es importante citar a otros, simplemente por respeto a sus ideas o sus planteamientos, pero siempre se
 corre el riesgo de ofender si alguien se siente mal citado.

\begin{wrapfigure}{r}{202\px}\centering% {'width': '202', 'align': 'alignright', 'id': 'attachment_463', 'caption': 'No nos tomemos tan en serio'}
\includegraphics[width=202\px,height=300\px]{blog/obamicon-le-202x300.png}
\caption{No nos tomemos tan en serio}
\end{wrapfigure}

\par% p
A raíz del \anchor[http://barcamp.org/BarCampBogota2]{BarCamp Bogotá 2}, y de la charla que quise dar sobre política en medios sociales quedé debiendo un resumen del mismo que publiqué en \anchor[http://blog.chlewey.net/2009/03/politiqueando]{mi blog}.  Esto coincidió con un artículo de mi ciberamigo y escritor Julián Ortega Martínez (@\anchor[http://twitter.com/julianortegam]{julianortegam}) en \anchor[http://globalvoicesonline.org/2009/03/24/colombia-are-politicians-making-the-most-out-of-twitter/]{Global Voices} donde recopila comentarios en Blogs y en Twitter sobre la participación de dos políticos colombianos presidenciables
 en \anchor[http://twitter.com/]{Twitter}, y finalmente coincidió con las inquietudes de otro entusiasta de los nuevos medios y bloguero, Diego Leal (@\anchor[http://twitter.com/qadmon]{qadmon}) sobre el \anchor[http://www.diegoleal.org/social/blog/blogs/index.php/2009/03/25/politica-en-linea-y-objetividad?blog=5]{papel de los medios sociales en la política}.

\par% p
No son las únicas coincidencias, ni tan coincidentes.  En Colombia ya estamos en campaña por congreso y presidencia de
 2010.  En 2008 sucedió el hecho de una \anchor[http://www.facebook.com/event.php?eid=7284103926]{marcha multitudinaria} cuya semilla comenzó en la plataforma de red social \anchor[http://www.facebook.com/]{Facebook}, formando un hito.  Estos nuevos medios tuvieron un papel importante, aunque discutiblemente decisivo, en la campaña
 presidencial de los EE.UU. en 2008.  Estos tres artículos: de Diego, de Julián y el mío, no son ni los primeros ni
 serán los últimos sobre el tema, ni mucho menos serán el referente, de lo que se viene.

Pero tema político aparte, viene la razón de este post: ¿qué tan dispuestos estamos para escuchar al otro sin
 prevenciones?

En el artículo de Diego, con el que estoy mayormente de acuerdo, veo dos problemas grandes: 1) tomar (o no dejar claro)
 que el artículo de Julián no refleja la opinión del autor, sino que recopila las opiniones de otros tuiteros.  2) un
 excesivo sarcasmo sobre la forma de ``contaminar ``de la twittósfera colombiana.

El primer problema causó una justificada (aunque en mi opinión exagerada) reacción de Julián.  Justificada porque su
 labor, como editor de Global Voices Online, no es opinar sino recopilar opiniones, y por lo tanto no debería
 insinuarse que es su opinión.  Creo que es importante que Julián defendiera, como lo hizo, esta posición.

Esto no significa, sin embargo, que lo que escribió Julián no fuesen opiniones.  No fueron opiniones personales de él,
 pero sí es una compilación de opiniones de otros.  Opiniones que en conjunto parecen formar una gavilla que dicta la
 forma en que se debe participar en Twitter.  Y aunque Julián no plantea elementos para iniciar un debate, las voces
 que compila (y la mía misma en mi blog) sí son un llamado al debate político.

\par% p
Pero es el segundo problema, el que ha causado mayor alineación.  Y no es la primera vez.  Cuando el gurú del blogueo
 colombiano Rafael Bayona (@\anchor[http://twitter.com/Patton]{Patton}) escribió en su \anchor[http://patton.blogdeldia.com/item/836]{blog personal sobre Twitter}, lo que para mí fueron consejos interesantes a adoptar o no, para muchos fueron una afrenta personal porque alguien
 les decía cómo escribir.  Su mayor contribución, acuñar el término ``contaminar'' para describirnos.

\par% p
Cada uno de nosotros decide cómo \anchor[http://blog.chlewey.net/2008/10/nuestras-caras]{exponerse a la red}, qué tanto de su vida personal dejarán saber.  Muchos cuidan su identidad digital con esmero de protejer su presencia
 física y la de sus familias, mientras otros son libros abiertos que, bien por ellos, pero también exponen a los demás.
  Yo, por ejemplo, trato de mantener los asuntos de mis hijos dentro del círculo más privado de mi muy pública
 presencia en red, pero mi hermana es feliz publicando fotos de sus sobrinos. (Y no me molesta)

Muchas personas en Twitter son felices desahogándose de lo que no pueden compartir con el círculo más cercano de
 conocidos por sus posibles consecuencias: hablar de sus problemas en el trabajo, en su familia, con su pareja, etc. y
 poco a poco esto establece relaciones de confianza donde compartimos nuestras penurias o las alegrías por nuestras
 mascotas.  Tal vez Twitter, en este aspecto, se convierte en una reunión de alcohólicos anónimos donde podemos
 desahogarnos casi que en directo.

\par% p
Pero al mismo tiempo Twitter es una herramienta de promoción y una herramienta de diálogo.  Twitter es un medio que
 puede usarse de formas muy diversas.  Yo sigo, por ejemplo a @\anchor[http://twitter.com/Wordpress]{Wordpress} y no me importa que no me siga, porque por ahora no me interesa insistir en dialogar con ellos, pero sí me interesa
 enterarme de lo que tienen que decir.  Otras personas insisten en seguir sólo a quienes les sigan, es parte de la
 libertad que cada quién tiene de usar Twitter o cualquier otra herramienta.

Así que yo me arrogo el derecho de opinar sobre política y los políticos y de burlarme de los políticos que no cumplen
 mis expectativas.  Y me arrogo el derecho de criticar o burlarme de, o símplmente opinar sobre, mis compañeros
 twitteros.  (Tan arrogante yo, ¿no?)

Por el momento sólo estoy cansado de que si alguien opina de lo curioso de otras opiniones, quieran descalificarlo por
 él mismo opinar.

\chapter{Chololatómanos}
\begin{metadata}
	Published by \anchor[chlewey]{chlewey} on \anchor[http://ewey.co/B437]{Fri, 28 Nov 2008 15:48:59 +0000}\\
	\categories{facebooknote, drogas, opinion}\\
	Shorthand: \anchor[http://blog.chlewey.net/2008/11/chololatomanos/]{chololatomanos}
\end{metadata}

¿Te gustan los chocolates? ¿Eres adicto a las gomitas? ¿te quedas sin almorzar si no hay carne en el almuerzo?

Hay adicciones y hay adicciones.  Un heroinómano sufrirá efectos psíquicos y físicos si no consume.  Son tan fuertes
 estos efectos que en últimas su adicción estará cada vez menos relacionada con el placer que produce el consumo y más
 con la incapacidad de vivir con la abstinencia.  Pero, incluso cuando logre superar su síndrome de abstinencia, habrá
 casi perdido la posibilidad de sentir placer con algo diferente de la heroína y todo el resto de su vida vivirá bajo
 el yugo de que en cualquier momento puede recaer en el vicio.  Bueno, hay casos de recuperaciones casi totales, cuando
 logra sublimar el placer físico por otros medios que dan sentido a la vida, pero ese no es el caso de la mayoría.

Si por el contrario eres adicta al chocolate, sentirás un placer inmenso en su consumo, gracias a las endorfinas y
 otras drogas que contiene el cacao.  Pero ese placer inmenso no reemplaza, salvo en casos patéticos, los otros
 placeres de la vida.  No hay síndromes de abstinencia relevantes.  No hay pérdida de sensaciones.  El chocolate es
 simplemente algo placentero pero rara vez será algo que produzca una adicción clínica.

¿Dónde está el límite entre una adicción real y algo que simplemente nos gusta mucho?  No está en si me creo o no capaz
 de dejarlo.  En el alcoholismo social, por ejemplo, el bebedor cree que puede dejar su vicio, porque efectivamente
 puede vivir cinco o 12 días sin alcohol y sin sentir la privación, pero retorna en lo que cree que es voluntario.  Un
 maniaco de los chocolates puede creer que su vida no tiene sentido si se priva de su placer.  Pero el alcohólico
 social se engaña a sí mismo creyendo que no es un adicto.  El chocolatómano sólo le gusta mucho y exagera ante una
 posible ausencia de su pequeño vicio.

¿Dónde está Internet?  ¿Es un vicio estilo chocolate o es un vicio estilo alcohol o heroína?  ¿El adicto a Internet es
 clínicamente adicto o es simplemente un fanático de estar conectado?  La verdad no tengo idea porque no he encontrado
 estudios concluyentes sobre la adicción a Internet.

El vicio a Internet puede tener los efectos sociales y económicos atribuidos a otros vicios fuertes, como la pérdida de
 la productividad o el aislamiento social del enviciado, quien pasa de ser un miembro activo y aportante de la sociedad
 a un lastre para sus conciudadanos, comenzando por su propia familia.  Desde este punto de vista, el vicio a Internet
 es más nocivo que el vicio al cigarrillo, o al café, o al chocolate.  Pero los vicios al café o al chocolate no son
 clínicamente adicciones, mientras que el aparentemente inofensivo vicio al tabaco sí lo es.  Separar un vicio simple
 de una adicción no tiene tampoco mucho que ver con los efectos negativos a la sociedad causados por el enviciado.

Pero aún así caemos en una posible falacia.  ¿Es el ostracismo social una manifestación de lo nocivo que puede ser el
 vicio a Internet?  Muchos acusan a Internet y otras tecnologías de reemplazar el contacto físico entre personas
 reales.  Los ciberómanos se sienten más a gusto interactuando con nombres de usuario sin caras, que con las personas
 de carne y hueso con las que conviven.  Eso lo veo como un gran riesgo a las nuevas generaciones nativas del
 ciberespacio, quienes desde que tienen uso de razón han convivido con Internet.  Pero como una persona que conoció a
 Internet cuando ya era adulto (aunque toda su adolecencia la compartió con computadores), en muchos sentidos siento
 que Internet, antes que negar mi vida social la ha liberado.  Dentro de mi personalidad introvertida la interacción
 social con las personas de carne y hueso que me rodean siempre ha sido limitada.  Desde la primaria, antes de tener mi
 primer computador.  Durante el bachillerato, en que no dedicaba más tiempo a los computadores que el que mis
 compañeros dedicaban al 2600.

Los computadores, primero, e Internet después, se convirtieron en una forma de escapar de las limitaciones de mi mundo
 real.  Una forma de crear, de sentir diferente, de encontrar personas con las cuales compartir inquietudes que las
 personas a mi alrededor inmediato no compartían.

Pero lo mismo dice un heroinómano.  Lo mismo dice un consumidor de Ecstasis.  El Yague y la mariguana tienen el mismo
 efecto de liberar nuestras mentes de las limitaciones del mundo físico.  No puedo, entonces, pretender que esta
 liberación sea una justificación real de mi vicio.

Pero hay algo que sé, y es que con Internet siendo un simple vicio o un vicio adictivo, aún necesito del mundo de carne
 y hueso, el mundo con el que interactúo con mis hijos, y en el que me gustaría interactuar con mis amigos.

Y en últimas sólo eso es lo que quería decir.

\chapter{Þanx}
\begin{metadata}
	Published by \anchor[chlewey]{chlewey} on \anchor[http://ewey.co/B438]{Sun, 09 Nov 2008 20:33:21 +0000}\\
	\categories{facebooknote, facebook-note, personal}\\
	Shorthand: \anchor[http://blog.chlewey.net/2008/11/thanx/]{thanx}
\end{metadata}

\anchor[http://www.facebook.com/photo.php?pid=1633253\&l=7379f13410\&id=659019663]{\begin{wrapfigure}{r}{240\px}\centering% {'src': 'http://blog.chlewey.net/wp-content/uploads/2008/11/catleyas-300x220.jpg', 'title': 'catleyas', 'height': '176', 'width': '240', 'alt': 'catleyas', 'class': ['alignright', 'size-medium', 'wp-image-467']}
\includegraphics[width=240\px,height=176\px]{blog/catleyas-300x220.jpg}
\end{wrapfigure}
}For those who are worried, let me tell you I am something finer now.  Thank you very much for caring, and for all
 support voices that I got.

\par% p
Please \anchor[http://chlewey.net/]{keep tuned}.  Soon I hope I can deliver great news to y'all.

\chapter{In plain panic}
\begin{metadata}
	Published by \anchor[chlewey]{chlewey} on \anchor[http://ewey.co/B439]{Mon, 03 Nov 2008 05:18:10 +0000}\\
	\categories{facebooknote, facebook-note, personal}\\
	Shorthand: \anchor[http://blog.chlewey.net/2008/11/in-plain-panic/]{in-plain-panic}
\end{metadata}

Since my adolescence I have been kind of depressive.  Probably not clinically depressed but once a shrink confirmed my
 suspicion.  It might had happened when I first felt in love and got my heart broken, something that kept happening
 again an again, until I finally found someone who I could love and loved me back, without submerging me in a
 depression crisis.  «Malparidez existensial» was a name people in Colombia had used to refer to that or similar
 feelings: «existensial sonofabitchness».  Giving a combination of my depressive nature and my introvert personality,
 it usually manifested as social isolation.

An introvert personality is a personality type.  Shyness, on the other hand, is a defect.  Unfortunately introvert
 people trend to be shy, and I am, myself, not an exception.  For a while I have been building around me a fortress to
 hide my real feelings, and then I happen not to know my real feelings.  I might even think that I suffer from
 Asperger's syndrome, given that I am not only unable to read my feelings, but I am usually unable to read other
 people's.  Well, no shrink has yet give me a hint that I suffer from Asperger's, this could just be my way to
 rationalize my inexcusable shyness.

The bottom line: even if I recognize reality from fiction, I do not feel comfortable with reality, particularly with
 the reality that is closer to me.  I can isolate in a crowd of friends and comrades.  Well, actually, just in a crowd
 of comrades as it happens that I do not have a real friend (some of the people tagged in this note are the closest
 things to real friends I have ever had).  I can push people away just when I most need them.  The real problems that I
 am supposed to face and solve, are just things I keep procrastinating, because it is easier to me to deal with a
 virtual, distant, or just fictitious reality, than dealing with my closer reality.

\par% p
I am in plain panic.  For some time I had not feel that kind of depression I used to feel before I met Beatriz.  This
 time it is not a broken heart (no yet, at least).  This time it is just the realization that I am just a failure.

\begin{blockquote}
«The saddest thing in life is wasted talent.»
— Lorenzo Anello (A Bronx Tale)
\end{blockquote}

However, I hope this panic status will soon be over.  There is always a time for hope whenever you can still steal
 life a smile.

\chapter{Rage... o 25 cosas sobre mí}
\begin{metadata}
	Published by \anchor[chlewey]{chlewey} on \anchor[http://ewey.co/B470]{Wed, 20 May 2009 18:30:33 +0000}\\
	\categories{familia, odio, personal, proyeccion-y-carrera}\\
	Shorthand: \anchor[http://blog.chlewey.net/2009/05/rage-o-25-cosas-sobre-mi/]{rage-o-25-cosas-sobre-mi}
\end{metadata}

\par% p
Aunque no faltará quien se sienta desilusionado porque volví a escribir, creo que este ensayo es un proceso importante de autosanación mental.
 Podría en estos momentos contenerme y concentrarme en mis deberes, pero eso sólo aplazaría más la solución de lo que
 me impide funcionar en esta sociedad del \anchor[http://www.google.com.co/search?q=''trabajar+trabajar+y+trabajar'']{trabajaar, trabajaar y trabajaar}.

\begin{enumerate}

\item Soy una esponja de conocimientos relativamente inútiles.~ La \anchor[http://es.wikipedia.org/wiki/Historia]{historia}, la \anchor[http://es.wikipedia.org/wiki/Pol\%C3\%ADtica]{política}, la \anchor[http://es.wikipedia.org/wiki/Vexilolog\%C3\%ADa]{vexilología}, la \anchor[http://es.wikipedia.org/wiki/Astrof\%C3\%ADsica]{astrofísica}, la \anchor[http://es.wikipedia.org/wiki/Epistemolog\%C3\%ADa]{epistemología}, el \anchor[http://es.wikipedia.org/wiki/Cine]{cine}, etc. muchas cosas que realmente sirven poco para lo que he escogido como carrera.
 Igual, lejos estoy de ser una eminencia en cualquiera de estos o muchos otros campos.
\item Me encuentro en un debate entre mi deber ser y mi querer ser; principalmente porque siento que el deber ser que me han querido inculcar no es más que la presión social de anular la identidad individual.
 ¿Por qué debo aplazar el querer ser para cuando ya sea viejo y no pueda disfrutarlo?
 Por otro lado también es complicado el querer ser cuando no sé aun que es lo que en últimas quiero.
\item Soy una persona \anchor[http://twitter.com/chlewey/status/1163734461]{sin grandes pasiones}.
 Muchas veces envidio a quienes tienen un ídolo o un amor platónico; un hobbie; un llamado religioso; algo externo que les de sentido a la vida.
 De las muchas cosas que me gustan, pocas me apasionan y ninguna es mi gran pasión.
\item Así como son pocas las cosas que me apasionan, también son \anchor[http://blog.chlewey.net/2007/10/odio-a-tanto-odio]{pocas} las \anchor[http://www.google.com.co/search?q=odio+site\%3Achlewey.net+inurl\%3Ascripts]{cosas} que aborrezco; e igualmente tampoco tengo un odio favorito.
\item En muchos aspectos soy demasiado racional.~ De ahí que sea agnóstico teológica, filosófica y \anchor[http://blog.chlewey.net/2008/02/no-hay-conflicto]{política}mente.~ Supongo que de ahí deriva también mi falta de pasiones.
\item Ser racional con respecto a las pasiones humanas no significa en nada que yo actúe racionalmente.
 Casi todas mis decisiones las tomo con información insuficiente y no siempre mi actuar sigue mis decisiones.
 Y dos fallas casi nunca son un acierto.
\item He leído mucho menos de lo que me gustaría haber leído.
\item Me gusta leer, ver televisión, ver cine, navegar por internet.
 Y me gusta recrear en mi mente y ocasionalmente en papel o en la pantalla de mi computador, lo que leo, veo y aprendo,
 y lo que invento de todo ello.
\item Me gusta comer bien y me gusta comer mucho.~ Son dos placeres diferentes y no tengo preferencia del uno sobre el otro.
\item Otro placer que disfruto es descargar la adrenalina frente al volante de un automóvil.
 Son unos pocos minutos en los que sentir mi vida en el límite me permite mantener mi mente enfocada.
 Es, desde luego, un placer peligroso porque dependo no sólo de mis limitadas capacidades sino de otros conductores,
 problemas mecánicos, el estado de las vías y los animales que se atraviesen.
\item La última vez que estuve en un parque de diversiones no sentí mayores descargas de adrenalina, ni mayor vértigo.
 Creo que llegué a racionalizar tanto el estado de los juegos mecánicos que no me causaron mayor sorpresa.
\item Puedo llegar a ser un poco terco en defender algunas de las cosas que me parecen justas o correctas.
 Incluso creo sentir un cierto placer si percibo que a quien yo considero injusto se desespera ante mi actitud.
 El problema es que muchas veces no mido las consecuencias que esta actitud mía puede traer en un país tan violento
 como en el que vivo.
\item Sólo recuerdo unas tres veces en el colegio y ninguna en mi vida adulta en la que haya peleado con \relax{% {'style': 'color:#600;'}
rabia}.~ Prácticamente nunca exteriorizo mi \relax{% {'style': 'color:#600;'}
rabia} en forma de violencia física ni hacia las personas ni hacia las cosas.
\item Tampoco me gusta lastimar verbalmente.
 Tal vez porque tampoco sea bueno en ello.
 Esto no significa que no sea capaz de herir a las personas, sólo que casi siempre es inconscientemente y a través de
 mecanismos como la indiferencia.
\item Igualmente tampoco he ejercido actos de violencia intencionada contra mi mismo.
 La velocidad y otras conductas potencialmente autodestructivas no tienen como objeto causarme daño sino lograr el
 placer último de salir airoso de ellas.
\item Me he hecho mucho daño.
 No sólo soy indiferente con las personas que me rodean sino también conmigo mismo, y esa indiferencia hace que varias
 enfermedades me estén invadiendo y que en ocasiones no observe una conducta segura al conducir o tratar con las demás
 personas.
\item Así como muchas veces soy inquieto intelectualmente y mi atención salta aleatoriamente de una actividad a otra, también puedo, en ocasiones, enfocarme por horas o incluso días en pequeños o casi insignificantes detalles.
 El problema es que casi nunca esa atención se centra en ese deber ser del que hablaba antes.
\item Me gusta caminar.
 Y me gusta también el ejercicio físico las pocas veces que lo hago, aunque mi vida sea más bien sedentaria.
 Pero el placer de caminar va más allá de la actividad física en sí.
\item No fumo (salvo pasivamente y bajo protesta), bebo muy poco porque simplemente no me gusta en trago.
 Nunca he metido drogas ilegales.
 Me gusta el café y ocasionalmente me hace falta.
 En cuanto a las drogas ilegales la verdad es que nunca me las ofrecieron aunque es muy probable que las hubiera
 rechazado llegada la ocasión, así como la presión social nunca logró hacerme tomar cuando no quise.
\item Soy pésimo tomando drogas recetadas. Ese estereotipo que que los latinos creemos que el médico no sirve si no receta algo no me aplica.
 No sé como hice para tener cierto juicio con el tiamazol durante más de un año pero el juicio se acabó una vez comencé
 a recuperar mi sobrepeso.
\item Tengo problemas para comunicarme bien con otras personas.
 Soy casi incapaz de expresar sentimientos; mido tanto las palabras que quiero decir que al final no digo nada o termino diciendo lo primero que se me viene a la mente... casi siempre lo más imprudente para decir.
 Tal vez por ello me siento mucho más cómodo escribiendo y particularmente escribiendo de cosas \anchor[http://blog.chlewey.net/2008/05/esta-ficcion]{que poco tienen que ver} con lo que siento o pienso.
\item Algunos de mis diagnósticos (propios o de otros) me han situado como \anchor[http://es.wikipedia.org/wiki/Depresi\%C3\%B3n]{depresivo}, \anchor[http://es.wikipedia.org/wiki/S\%C3\%ADndrome\_de\_Asperger]{aspérgico} o \anchor[http://es.wikipedia.org/wiki/Trastorno\_por\_d\%C3\%A9ficit\_de\_atenci\%C3\%B3n\_con\_hiperactividad]{hiperactivo}.
 El problema de confirmarse un diagnóstico es que me sienta cómodo con un «yo soy así» en lugar de buscar mejorar, particularmente porque me gusta mucho vivir en la zona de comfort.
 Un diagnóstico más interesante es que sufro de \relax{% {'style': 'color:#600;'}
rabia} contenida lo cual me lleva a sabotear mi vida.
\item Cuando se puso de moda hablar de la inteligencia emocional empecé a temer que tengo la inteligencia equivocada.
 Cuando escucho de muchos otros niños genios con problemas de adaptación al mundo real me convenzo cada vez más de que
 tengo esa inteligencia equivocada, aunque creo que, la verdad, hasta ahora he salido relativamente bien librado.
\item Siempre he visto a la plata como un medio, nunca como un fin, y dado que siempre he vivido en una zona de comfort, se me hace sumamente difícil pensar en cómo monetizar lo que sé o lo que soy.
 Me siento más a gusto dedicando mi tiempo a cosas que sé que no van a dar plata.
\item Siento que la política me coquetea cada vez más y no sólo para analizarla y comentar sobre ella sino para hacerla.
 Lo que no sé es quien, tras leer estos 24 puntos anteriores, confíe lo suficiente como para votar por mí.

\end{enumerate}

\chapter{Sobre Chlewey}
\begin{metadata}
	Published by \anchor[chlewey]{chlewey} on \anchor[http://ewey.co/B481]{Mon, 08 Jun 2009 21:37:58 +0000}\\
	\categories{chlewey, internet, personal}\\
	Shorthand: \anchor[http://blog.chlewey.net/2009/06/sobre-chlewey/]{sobre-chlewey}
\end{metadata}

Alguna vez me preguntaba qué tan diferente es mi identidad en internet: Chlewey, con mi persona de carne y hueso, y mi
 conclusión es que no hay tal diferencia, tal dicotomía.  Más que un alterego, Chlewey no es más que un apodo.
 Chlewey, Chlewey Thompin, Carlos Thompson, Carlos Eugenio Thompson Pinzón o cualquier combinación o transliteración
 corresponden a una misma persona, a una misma identidad.

\includegraphics[width=300\px,height=79\px]{blog/Chlewey-Thompin_logo.png}
Chlewey es más corto y más único que cualquier combinación de mi nombre legal.
 Es por lo tanto el nombre de usuario que suelo escoger para identificarme en el mundo de internet, pero en ningún
 momento pretende ser una identidad exclusiva de internet, como sí lo manejan otras personas.

\par% p
Muchos se preguntan el origen de la palabra \emph{Chlewey}.
 Este surgió en 1992, antes de que yo siquiera supiera que existe algo llamado internet.
 Quería buscar un nombre alternativo que combinara ser lo suficientemente único y representativo, y como tal me puse a
 jugar con las letras de mi nombre o, más bien, de la traducción de mi nombre al inglés.

\par% p
Mi nombre de pila traducido al inglés sería \emph{Charles Eugene} \ipa{% {'class': ['ipa']}
/tʃærlz ˌjuˈdʒin/}.~ Estos nombre, a su vez, tienen apodos comunes tales como \emph{Charly} \ipa{% {'class': ['ipa']}
/ˈtʃær·li/} o \emph{Chuck} \ipa{% {'class': ['ipa']}
/tʃʌk/} para \emph{Charles}, y \emph{Ewey} \ipa{% {'class': ['ipa']}
/ˈju·i/} para \emph{Eugene}.~ Igualmente, la \emph{ch} del inglés tiene tres pronunciaciones posibles: \ipa{% {'class': ['ipa']}
/tʃ/} como en Charles, \ipa{% {'class': ['ipa']}
/k/} como en \emph{chlorine} (cloro), o \ipa{% {'class': ['ipa']}
/x/} como en \emph{loch} (lago) o \emph{chi} (ji), esto es como la pronunciación de la «j» española aunque no como la pronunciamos en Colombia.
 Como una forma de representar a Carlos/Charles tomé la combinación \emph{chl} con la pronunciación \ipa{% {'class': ['ipa']}
/xl/}.

\par% p
Así quedó \emph{Chlewey}, pronunciado \ipa{% {'class': ['ipa']}
/ˈxlju·i/}, pronunciación que podría transcribirse en español como «jliúi» o, si es muy difícil «cliúi».

\par% p
En cuanto al \emph{Thompin}.~ La historia data de mis años del colegio (escuela secundaria), cuando mis compañeros empezaron a llamarme \emph{Tompi} o \emph{Tompin} como una abreviatura de mi apellido.~ Normalizando la «n» de \emph{Pinzón}, y la ortografía con «Th», quedó como el apellido para Chlewey: \emph{Chlewey Thompin}.

\par% p
La principal desventaja que veo, sin embargo, con \emph{Chlewey} es que a pesar de su gran unicidad e identificabilidad es difícil de pronunciar correctamente o de recordar su correcta ortografía.
 No sé si este artículo sirva a los interesados, si es que los hay, para que la etimología los ayude a pronunciar y
 escribir correctamente la palabrita.

Saludos a todos.

\chapter{Indocumentado}
\begin{metadata}
	Published by \anchor[chlewey]{chlewey} on \anchor[http://ewey.co/B491]{Mon, 22 Feb 2010 17:28:30 +0000}\\
	\categories{cedula, cedula-de-ciudadania, documento-de-identidad, ineficencia, ineficiencia-estatal, personal}\\
	Shorthand: \anchor[http://blog.chlewey.net/2010/02/indocumentado/]{indocumentado}
\end{metadata}

El cuento comenzó en 2005.  Mi cédula original, aquella que tengo desde 1991 (hagan cuentas de qué tan viejo soy) ya
 presentaba un deterioro apreciable ya que no era un documento que se adaptara a los tarjeteros de las billeteras y
 tocaba sacarla cada vez que uno entraba a cualquier edificio para identificarse. —Esos coletazos de la cultura mafiosa
 y de cómo la ciudadanía se adapta a ella.

Mi cédula ya hasta tenía bolsillo propio.

En 2005 ya se tenía previsto que el nuevo modelo de cédulas sería el «definitivo» y que eventualmente el proceso de
 renovación sería gratuito, pero entonces tocaba pagar \$33.000 para renovar la cédula.  Así que tras una larga espera
 pagué en junio y en octubre fui a la Registraduría local de Usaquén para hacer el trámite.  Sin mayores novedades, la
 nueva cédula estaría lista en máximo 18 meses.

Todo iba bien.  Con mi cédula original seguía identificándome ante cada instancia en la que requerían tal documento.
 La última vez que ello sucedió fue el 28 de diciembre de 2006.

En un evento que sigo creyendo era totalmente innecesario, mi supervisor de etonces me hizo ir a las bodegas de Huawei
 (mi entonces empleador) en Zona Franca Bogotá a las siete de la noche, donde tuve que utilizar la cédula a la entrada,
 buscar un cable duplicado en 180 cajas para meterlo en una caja donde faltaba y luego sí ir a la casa.

No tuve más noticias de mi cédula original.  Ni en portería de Zona Franca, ni en las bodegas de Huawei, ni en mi
 carro, ni en ninguna otra parte apareció.  Bonito día de inocentes.

A partir de ahí tuve que bandearme con una contraseña no refrendada y otros documentos como un pase (licencia de
 conducción) peor de deteriorado que mi cédula.  En algunos sitios ponían más problemas que en otros, pero mal que bien
 pude sobrevivir: entrar a donde tenía que entrar y pagar lo que tenía que pagar.  Cierto alivio hubo cuando renové mi
 pase, pero en sitios como Alkosto siempre ponían problema.

Mientras tanto, nadie daba razón de mi nueva cédula.  En abril de 2007 recuerdo haber preguntado en un puesto de
 cedulación de esos nuevos que habían puesto para que renováramos gratuitamente nuestras cédulas viejas y me dijeron
 que parecía haber un problema pero que no sabían que era.  El caso es que no estaba lista.

Finalmente en septiembre de 2007 saqué un tiempo para ir a que me autenticaran la contraseña.  Este es un trámite
 necesario para que la contraseña de que la cédula está en elaboración pueda usarse como documento de identificación
 (si bien no como documento electoral).

En la oficina de autenticaciones de la Registraduría de Bogotá me dijeron que había un problema y que tenía que ir a la
 Registraduría Nacional.

Así que fui.

Tras una larga pero relativamente decente fila la respuesta fue que no coincidía el cotejo dactilar que me tomaron en
 2005 con el original tomado en 1990.  Cuando pregunté que cómo así revisaron nuevamente y descubrieron que el
 funcionario de Usaquén en 2005 se equivocó y tomó las huellas de mi mano derecha en el espacio de las de la izquierda
 y viceversa.  El funcionario que me estaba atendiendo me propuso radicar un derecho de petición para que se arreglara
 el problema.

Presenté el derecho de petición y en el tiempo previsto me dieron la respuesta (el estado parece funcionar): una orden
 a la registraduría local para que sin dilación me rehicieran el proceso.

Fui, me tomaron las huellas (bien) y los demás pasos.  Me expidieron una nueva contraseña que autentiqué prontamente y
 fin problemas y de nuevo fui un ciudadano documentado.

Bueno, lastimosamente no pude votar en las elecciones locales de septiembre de 2007 pero al menos ya podía
 identificarme.

Los primeros meses no le presté mayor atención pues por muy prioritario no esperaba que lo que se demora 18 meses me lo
 resolvieran en dos.  Pero mis papás que renovaron en Usaquén en diciembre de 2007 tuvieron sus cédulas en diciembre de
 2008.

Luego, a mediados de 2009 me quisieron poner problemas nuevamente en Alkosto aduciendo que la autenticación de la
 contraseña debía refrendarse anualmente.  En cambio no tuve problema alguno renovando mi pasaporte ni obteniendo mi
 visa estadounidense.

\par% p
En algún momento cambiaron la página de la Registraduría y el mensaje cambió de que mi cédula no estaba lista a

\begin{blockquote}
«Su solicitud del Documento Número \textbf{XXXXXXXX} está próximo a iniciar proceso de producción, por favor consulte esta base de datos en los próximos días para
 verificar el estado de su documento.»
\end{blockquote}

Se acercaba el 1 de enero de 2010, fecha en la cual todos los colombianos deberíamos tener la cédula nueva y
 definitiva.  El 23 de diciembre de 2009 fui a la oficina de autenticaciones de la Registraduría de Bogotá en la 12 con
 octava, en pleno centro.

Me rechazaron la petición porque mi cédula estaba rechazada. ¡¿Otra vez?!  ¡¿Y por qué sólo hasta ahora me avisan?!  Me
 dijeron que fuera a la Registraduría local de La Candelaria que quedaba cerca para averiguar el problema y subsanarlo.
  De allí me remitieron a la de Usaquén, en la 123 con octava en el Norte de Bogotá.

Tras una espera de dos horas para hablar con el Registrador local y revisar  nuevamente el sistema vimos que el rechazo
 era por el trámite de 2005 y que el trámite de 2007 no tenía ningún problema.

Adicionalmente el registrador ne decía que las contraseñas se autentican una sola vez, sin refrendaciones anuales ni
 nada.  Con una nueva contraseña y una nota del registrador local al jefe de autenticaciones regresé al centro donde
 sin problemas me autenticaron esa nueva contraseña.

¡Que bien!  Nuevamente estaba identificado, aunque para los de CMR Falabella la contraseña no servía para nada.

\par% p
Mientras tanto, y hasta la última vez que revisé hace pocos minutos, el mensaje sigue siendo el mismo:

\begin{blockquote}
«Su solicitud del Documento Número \textbf{XXXXXXXX} está próximo a iniciar proceso de producción, por favor consulte esta base de datos en los próximos días para
 verificar el estado de su documento.»
\end{blockquote}

Nadie ha sido capaz de darme una explicación convincente por la cual un proceso que físicamente demora pocos días (y
 que me confirman, para algunas personas duró apenas pocos días), por el cual pagué, ante el cual se dio una orden de
 prioridad, aun no concluye.

Mientras tanto todo el mundo dice que así son las cosas.  Que eso a veces se tarda.  Todos parecen conformarse con eso.

Pues no.  O bien hay un caso de delito contra el sufragio o bien es un caso de desdeño administrativo que raya en lo
 criminal.

Y debemos decirlo.

\chapter{Reflexiones niponas}
\begin{metadata}
	Published by \anchor[chlewey]{chlewey} on \anchor[http://ewey.co/B507]{Tue, 23 Feb 2010 21:48:36 +0000}\\
	\categories{nihon, opinion}\\
	Shorthand: \anchor[http://blog.chlewey.net/2010/02/reflexiones-niponas/]{reflexiones-niponas}
\end{metadata}

He estado en Japón dos veces.  La primera fue en 1990 y yo era un adolescente que competía en las Olimpiadas
 Internacionales de Matemáticas que ese año se celebraban en Pekín (RPC) y mi plan era simplemente aprovechar con mis
 compañeros el turismo de algunos pocos días en Tokio.  La segunda vez fue en 2009 cuando en medio de una crisis
 personal mis papás me invitaron a pasar con ellos su último mes de una estadía de 10 años en Yokohama.

\anchor[http://blog.chlewey.net/wp-content/uploads/2010/03/l\_640\_480\_D6A74188-2349-419A-96F1-74D593221490.jpeg]{\begin{wrapfigure}{l}{192\px}\centering% {'src': 'http://blog.chlewey.net/wp-content/uploads/2010/03/l_640_480_D6A74188-2349-419A-96F1-74D593221490.jpeg', 'alt': '', 'height': '144', 'class': ['size-full', 'alignleft'], 'width': '192'}
\includegraphics[width=192\px,height=144\px]{blog/l_640_480_D6A74188-2349-419A-96F1-74D593221490.jpeg}
\end{wrapfigure}
}En esos 19 años muchas cosas cambiaron.  Quién soy yo y que espero de la vida fue un cambio.  Lo que esperaba del
 viaje.  Cómo conocer lo que iba a conocer.  La misma cultura japonesa cambió en muchos aspectos.

\anchor[http://blog.chlewey.net/wp-content/uploads/2010/03/l\_640\_480\_F7887341-B62F-442A-B138-ED247D0FCFDE.jpeg]{\begin{wrapfigure}{r}{192\px}\centering% {'src': 'http://blog.chlewey.net/wp-content/uploads/2010/03/l_640_480_F7887341-B62F-442A-B138-ED247D0FCFDE.jpeg', 'alt': '', 'height': '144', 'class': ['size-full', 'alignright'], 'width': '192'}
\includegraphics[width=192\px,height=144\px]{blog/l_640_480_F7887341-B62F-442A-B138-ED247D0FCFDE.jpeg}
\end{wrapfigure}
}Son muchas las experiencias.  Son muchos los contrastes entre allá y acá y entre estos y lo que recuerdo de la otra
 cultura en la que he estado inmerso: la sueca.  Las personas que conocí.  La iniciada que tuve con Nueva York.  En
 fin.  Mucho para una sola entrada.

\chapter{Votando verde}
\begin{metadata}
	Published by \anchor[chlewey]{chlewey} on \anchor[http://ewey.co/B519]{Fri, 12 Mar 2010 04:44:53 +0000}\\
	\categories{cedula, elecciones, opinion, uribismo}\\
	Shorthand: \anchor[http://blog.chlewey.net/2010/03/votando-verde/]{votando-verde}
\end{metadata}

\par% p
Desafortunadamente no podré votar este domingo 14 de marzo, por razones \anchor[/indocumentado]{ya expuestas}.

Tampoco pude votar el 23 de septiembre de 2007, por ese mismo motivo. Entonces mi intención fue votar por Juan Carlos
 Flórez quien me parecía una mucho mejor alternativa a Samuel Moreno, Enrique Peñalosa y William Vinazco. (Sobre todo
 mucho mejor que Moreno y Vinazco.) Para el Concejo Distrital pensaba votar por Lariza Pizano y no recuerdo si
 finalmente definí o no mi voto para la Junta Administrativa Local (JAL) de Suba.

Este domingo vamos a elegir a nuestros representantes ante el Senado, la Cámara de Representantes y el Parlamento
 Andino. No he tenido tiempo de escudriñar todas las campañas así que opto por otra estrategia de elección: buscar
 hasta encontrar un buen candidato y decidir por él o ella mi voto; y si encuentro otro mejor, redefinir así el voto.
 El voto ha de ser por una persona con una buena trayectoria previa, que te inspire confianza y cuyo programa esté
 acorde a lo que esperas. Es perfectamente válido votar por tus intereses (por ejemplo el candidato que represente a tu
 gremio) o por intereses más generales regionales o nacionales. Al fin y al cabo es tu voto.

Si pudiera votar, mi voto por el Senado está entre John Sudarsky (3 del Partido Verde) y Juan Carlos Flórez (7 de
 Compromiso Ciudadano por Colombia). Ellos son dos ciudadanos que representan la cultura ciudadana y el cambio positivo
 de Bogotá. Flórez apoya a Sergio Fajardo quien también representa esa misma cultura ciudadana y cambio positivo en
 Medellín.

\par% p
Para la Cámara de Representantes por Bogotá mi voto va por la lista cerrada del Partido Verde. Y me interesa votar
 verde porque significa la construcción de un partido, no por guiño presidencial como el Partido \st{Nacional Socialista}Social de \textbf{U}nión Nacional, o de una maquinaria política alrededor de una propuesta personal como Cambio Radical, sino de una
 propuesta ideológica de concertación. Es un bonito experimento que me gustaría bastante que funcione, y funcionará si
 votamos por ella.

Por el Parlamento Andino tengo menos preferencias, pero me recomendaron la cabeza de lista (501) del Partido Verde, lo
 cual agregaría coherencia a mi voto.

Hay otras dos cosas que se definen este domingo: las consultas internas del Partido Conservador y el Partido Verde, y
 la Papeleta Caribe.

Primero diré por qué no votaré por la Papeleta Caribe. Primero, porque no pertenezco a la Región Caribe de Colombia, ni
 tengo mayores vínculos con la Costa, esto haría que si la Papeleta Caribe fuese de sí o no, no votaría ni por el sí,
 ni por el no, ni en blanco. Simplemente no la pediría y no depositaría nada en la urna, como una forma de no
 inmiscuirme en un proceso que es de ellos (y que conste que no digo 'ellos' de forma excluyente). Segundo, porque
 igual no estoy votando en la Costa; no sé si en mi mesa de votación existan papeletas para los que quieran votar por
 esa iniciativa, pero en ese caso será para que la pidan los interesados. Tercero, porque hasta ahora nadie ha logrado
 explicarme qué ventajas tiene para la Costa más allá de aumentar un nivel de burocracia. Cuarto, por lo primero que
 dije: no podré votar.

Ahora diré por qué no votaré en la consulta interna del Partido Conservador; además de no poder votar. Mejor dicho diré
 por qué no votaría. No votaría porque no me convence ninguno de los candidatos.

\par% p
Andrés Felipe Arias me parece un candidato muy poco serio, ya que él mismo se consideró todo este tiempo como un
 comodín de Uribe, guardándole el puesto de candidato presidencial del partido conservador al liberal discidente Álvaro
 Uribe. Cuando la probabilidad de la reelección empezó a minar, sugirió una alianza con el Partido de la U que parecía
 motivada a que el Partido Conservador se marginara por tercera vez de llevar un candidato propio a las urnas. En otras
 palabras, si yo fuera militante conservador no necesariamente uribista no votaría por Arias. Pero, además, Arias
 representa una visión política que definitivamente no comparto: la visión de que el estado debe concentrarse en los
 ricos con la esperanza de que estos en su sabiduría y misericordia ayuden a la gran masa de brutos pobres. \anchor[/derechizados]{Tal vez funcione}\anchor{, pero no la comparto.}

Por Noemí Sanín voté en la primera vuelta de las elecciones presidenciales de 1998, más por el equipo que estaba
 formando con Mockus, Lleras de La Fuente y Valdivieso, que por Sanín misma. En 2002 le reconocía su fortaleza en su
 discurso económico, aunque mi voto entonces fue por Lucho Garzón. Hoy veo a una Noemí Sanín muy desdibujada, sin
 propuestas novedosas; y enfrascada en una pelea casi personal con Arias. Hoy Noemí no contaría con mi voto ante muchas
 mejores propuestas.

Álvaro Leyva me parece un tipo más bien serio, pero no le veo ese talante presidencial y no estoy seguro bien a qué
 jugaría tras estos ocho años de seguridad democrática. Igual que con Noemí, mi voto iría por alguna de las mejores
 propuestas. Con José Galat sí sé bien por qué no votaría por él: es un hombre muy coherente con sus ideas políticas y
 sociales, lo cual admiro, pero no son mis ideas políticas y sociales.

Finalmente queda Marta Lucía Ramírez, y confieso que si tuviera intención de votar dentro del conservadurismo, Ramírez
 sería mi elección.

Pero no votaría en la consulta interna del Partido Conservador porque mi idea sería votar en la consulta interna del
 Partido Verde donde ya tengo mi candidato: el profesor Antanas Mockus Zivikas.

\par% p
Así que ya que por motivos de fuerza mayor no podré votar este domingo, le pido a todos los que leen esto que si aún no
 tienen candidato me ayuden a poder votar por Mockus el próximo 30 de mayo. Esto mediante dos procedimientos:

\begin{enumerate}

\item Si no tienes candidato presidencial definido, o tu candidato no es conservador y te gustaría ver a Mockus debatiendo
 con tu candidato, pide el tarjetón de la consutal verde y marca a Antanas Mockus.
\item Si no tienes candidato al senado, sigue mi consejo y vota por John Sudarsky (número 3), o quien más te convenza o
 símplemente marca el logo del Partido Verde sin seleccionar un número. Si no tienes candidato a la Cámara de
 Representantes vota por el girasol amarillo en fondo verde. Si no tienes candidato a Parlamento Andino: vota por el
 Partido Verde.

\end{enumerate}

Te pido votar por el Partido Verde en las elecciones parlamentarias porque, primero: es una propuesta seria, así que
 no sería botar tu voto, y segundo porque así buscaremos que el Partido Verde consiga el umbral necesario para tener
 representación parlamentaria y no pierda su personería jurídica. Si el partido verde perdiese su personería jurídica,
 ni Antanas Mockus, ni Enrique Peñalosa, ni Luis Eduardo Garzón podrán competir en las elecciones presidenciales.

Desde luego, que si ya tienes tu candidato en cada corporación, no te pediré que cambies el voto. Tu voto es tu voto,
 es tu poder. Úsalo como mejor te parezca. Sólo te pido que no cometas el error de votar en consulta interna del
 Partido Conservador con el objetivo de votar en contra de alguien (p. ej. en contra de Arias o en contra de Noemí)
 siendo que te inclinas más por un candidato verde (Mockus, Peñalosa o Garzón). Vota por el que más te guste.

Y si ya decidiste por Petro, Pardo, Santos o Vargas y no te identificas por ningún verde o ningún azúl. Pues vota
 apenas para Senado, Cámara y Parlamento. (O no.) Me haces un mayor favor votando positivamente por quien sí te
 convenza que votando por quien yo digo sólo porque yo lo digo.

\textbf{Coletilla:} Si eres antiuribista el mejor voto contra Arias no es Noemí Sanín (salvo que realmente seas noemicista, pero ya no
 será un voto en contra sino uno a favor), sino \textbf{no} votar en la consulta interna del Partido Conservador (salvo que sí quieras votar por Ramírez, Leyva o Galat, en cuyo
 caso deberías votar por ellos y no por Sanín). En la situación actual, si Arias queda de lider de un Partido
 Conservador con no demasiados votos, el voto uribista se repartirá entre Arias y Santos. No veo una alianza para la
 primera vuelta entre el partido Conservador y la U.

Y si has decidido no votar poque todos deben ser lo mismo, piénsalo dos veces. Revisa candidato por candidato para
 estar seguro de que absolutamente ninguno vale la pena. Si te parecen demasiados para ir uno por uno, escucha
 consejos: ya dí mis dos favoritos para el Senado: Flórez y Sudarsky, pero hay más: está Juan Manuel Galán (1 del
 Liberal), Rodrigo Lara (7 de Cambio Radical), está MIRA (incluso si no eres cristiano no denominacional), está el
 Partido Verde. (MIRA y el Partido Verde son los únicos que están libres de parapolítica, farcpolítica, etc.) Tal vez
 alguno de ellos te inspire la suficiente confianza.

Si todavía ninguno te convence lo suficiente, pero te caigo bien: bueno, vota por los que yo quería votar y no podré.
 Un favor personal que no tendrá ninguna contraprestación politiquera.

Y, si finalmente eres anarquista, anarco-socialista, anarco-capitalista, activamente apático, o simple y llanamente
 apático: sal a hacer una caminata ecológica, siéntate en casa a ver televisión internacional, o haz lo que realmente
 quieras... o ¿Quién soy yo para darte consejos?

\chapter{La retórica de la libertad}
\begin{metadata}
	Published by \anchor[chlewey]{chlewey} on \anchor[http://ewey.co/B522]{Wed, 17 Mar 2010 14:55:17 +0000}\\
	\categories{censura, libertad, opinion}\\
	Shorthand: \anchor[http://blog.chlewey.net/2010/03/la-retorica-de-la-libertad/]{la-retorica-de-la-libertad}
\end{metadata}

\par% p
Recordemos que Internet fue \anchor[http://es.wikipedia.org/wiki/ARPANET]{originalmente diseñado} por el \anchor[http://defense.gov/]{Departamento de Defensa} (DoD) de los Estados Unidos con el objetivo de que fuese un sistema de comunicaciones de datos que siguiera
 funcionando aún cuando gran parte de la infraestructura fuere destruída ante una catástrofe o un ataque nuclear.  La
 solución fue crear una red abierta, descentralizada y ubicua.

El DoD pronto cedió el control de Internet a la academia y a principios de los años 1990 entró con fuerza la empresa
 privada a seguir creciendo el Internet.  El DoD se beneficiaría de Internet aunque no lo controlara.  Es más, se
 beneficia de Internet porque no lo controla.  Son las universidades, los grandes empresarios, los gomosos, quienes
 construyen Internet para el servicio del DoD o de cualquier otra persona que lo quiera usar.  Con Internet se
 inventaron el medio, el aire, que todos \&mdash;incluídos ellos\&mdash; podemos usar para transmitir nuestra voz y
 escuchar a los demás.

El carácter libre de Internet es lo que hace que funcione.  Es lo que lleva a que los empresarios quieran invertir en
 él, a que los usuarios creen contenidos, pagando o no servicios de hospedaje, y busquen información, pagando o no su
 conexión.  Los que conocemos el medio nos encontramos con un terreno anárquico, donde puede encontrarse de todo: desde
 un manual para fabricar bombas hasta compilaciones de poemas cursis; desde pornografía de todos los gustos y todos los
 precios hasta noticias y opinión sobre los sucesos mundiales; desde tiernas y chistosas fotos de gatitos hasta
 imágenes explícitas de la última masacre.

¿Debe Internet ser regulada o controlada? ¿o su espíritu libre debe predominar sobre todas las cosas?

Internet es un medio como el aire.  Como en el aire, nadie controla lo que podemos decir o escuchar, pero eso no
 necesariamente nos permite decir o escuchar lo que se nos venga en gana.  Algún aforista gringo, tal vez Twain, decía
 que el derecho a la libre expresión no te permite gritar ¡Fuego! en un teatro lleno de gente.  No por ser Internet un
 medio abierto y desregulado éste nos da derecho a decir o hacer lo que las leyes nacionales nos impiden decir o hacer
 en otros medios.

En los Estados Unidos la libertad de expresión es uno de sus derechos más sagrados.  Tan sagrados que este derecho
 ampara incluso a la apología de delitos de odio.  Una persona puede ir a los Estados Unidos y montar allá un sitio web
 por medio del cual, basado en teorías pseudocientíficas, justificar que los colombianos son una raza superior a los
 venezolanos y que todos estos últimos deben ser esterilizados para el bien de la humanidad.  O que los hijos del
 Presidente de la República deben morir para poder garantizar el pan diario a todos los demás colombianos.  O que los
 periodistas como Daniel Coronell deben ser acallados a toda costa para la supervivencia de la Patria.  Todo eso lo
 ampara el derecho a la libre expresión.  Lo que no ampara el derecho a la libre expresión es la coordinación de planes
 concretos para lograr la eliminación de los venezolanos, los hijos del presidente o los periodistas críticos al
 gobierno.

La libertad de expresión en los Estados Unidos tampoco ampara a la pornografía infantil.  Una persona puede ser acusada
 en los Estados Unidos por sólo tener pornografía infantil en su computador, aun cuando no tuviera intenciones de
 compartirla; y sin duda será condenada si usa ese medio tan libre y desregulado de Internet para distribuirla.

El medio es libre, pero las personas que usan el medio son responsables moral, civil y penalmente de lo que expongan en
 el medio.  Esta responsabilidad no es mayor ni menor a la responsabilidad de usar otros medios tales como el aire.  La
 libertad del medio no es una excusa para hacer trampas a la ley.

En Alemania, el derecho a la libre expresión es también un derecho fundamental; pero tiene un límite claro: la apología
 a la violencia y a grupos de odio es un delito que no puede ampararse bajo la libre expresión.  Si la ley prohíbe
 vender objetos con símbolos nazis, entonces es natural que la ley prohiba albergar contenidos con fines de
 distribución de apología nazi, y un sindicado no puede ampararse en la libre expresión o en la libertad intrínsica de
 Internet como defenza a su delito.

\par% p
Recientemente el gobierno venezolano \anchor[http://alt1040.com/2010/03/hugo-chavez-en-planes-para-censurar-internet-en-venezuela]{dijo} que no permitiría que Internet fuese utilizada en Venezuela para difundir noticias falsas.  Y no se refería a parodias
 como \anchor[http://www.theonion.com/]{The Onion}, sino que se basaba en un caso en el cual un \anchor[http://www.noticierodigital.com/]{medio virtual} que pretendía ser serio publicó rumores de que un ministro había muerto como si fuesen noticias confirmadas.

Independientemente de si nos gusta o no el gobierno venezolano (a mi no me gusta) o si nos guste o no la censura
 (tampoco me gusta), no podemos caer en el facilismo de decir que Chávez no entiende que Internet debe ser libre.

Internet no es libre.  Internet ya está regulado de muchas formas.  Tenemos esa falsa idea de que es libre porque en
 países como Colombia no existen filtros nacionales de contenido y eso nos expone a cualquier tipo de basura o de
 perlas que existe en Internet.  Pero toda esa basura y esas pocas perlas ya han pasado muchos filtros.  Hay
 regulaciones sobre la dirección IP y el nombre de dominio que usan esos sitios web.  Hay regulaciones sobre los
 protocolos que se manejan.  Hay censura en los lugares de origen.  Hay responsabilidades civiles y penales que aplican
 a quienes publican; y en ocasiones a quienes reciben ciertos contenidos.  Hay líneas editoriales.

Puede que no nos gusten las leyes de censura de un país en particular.  Yo puedo creer sinceramente que si en Alemania
 quitaran las restricciones a la apología nazi y antisemita, se pudiera dar un verdadero debate que permita que la
 mayor parte de la población se dé cuenta por cuenta propia que el nazismo y el antisemitismo son perversos y no
 símplemente prohibidos.  Yo puedo creer sincesaramente que los únicos responsables de la pornografía infantil son
 quienes la producen y distribuyen y no los enfermos mentales que la consumen.  Yo puedo creer sinceramente que si un
 medio distribuye desinformación perderá más por falta de credibilidad que por una censura impuesta estatalmente.  Pero
 así no me gusten las leyes de censura de otros países o del mío propio, no puedo desconocer que las leyes de censura
 son leyes y que todos los países, aun los más abiertos, tienen leyes de censura.

E Internet es sólo un medio de comunicación.  Es un medio que por su propia naturaleza es muy difícil de controlar o
 regular, pero no por ello Internet es un territorio sin ley.  Lo que digamos por Internet está sujeto a lo que las
 leyes de nuestros países digan que podemos decir por cualquier medio.  Lo que digamos por Internet es responsabilidad
 nuestra, y está sujeto a las responsabilidades legales o paralegales que nos imponga nuestra sociedad.

Chávez podrá decir muchos disparates.  Pero pedir responsabilidad en Internet no es uno de ellos.

\chapter{Mira, mamá, ya soy un gurú de Twitter y doy consejos}
\begin{metadata}
	Published by \anchor[chlewey]{chlewey} on \anchor[http://ewey.co/B524]{Sat, 17 Apr 2010 01:32:38 +0000}\\
	\categories{opinion, social-media, twitter}\\
	Shorthand: \anchor[http://blog.chlewey.net/2010/04/soy-guru-de-twitter/]{soy-guru-de-twitter}
\end{metadata}

\par% p
Las distintas plataformas de red social en Internet, por concepción o características de diseño tienen diferentes vocaciones.
 \anchor[http://www.facebook.com/]{Facebook} es el lugar para mantener en contacto tus amigos y familia.~ \anchor[http://www.linkedin.com/]{LinkedIn} para establecer tus relaciones profesionales.~ \anchor[http://badoo.com/]{Badoo} (entre muchas otras) para encontrar nuevos amigos. ~\anchor[http://es.wikipedia.org/]{Wikipedia} para construir colectivamente conocimiento.

\anchor[http://twitter.com/]{Twitter}, básicamente, es un lugar para obtener y difundir información variada en pequeñas cápsulas.

\par% p
Estas vocaciones de las distintas plataformas no impiden que puedan usarse de formas novedosas y creativas.
 Muchas personas han hecho nuevos amigos a través de Facebook, o tienen su cuenta allá para hacer negocios y difundir información en una red de amigos o de fans.
 Twitter \anchor[http://www.formspring.me/chlewey/q/195770985]{no es la excepción}.
 Entre las personas que sigo por medio de Twitter veo que hay una gran actividad social desarrollándose, más allá de
 difundir u obtener información.

Así, que si aun te preguntas qué puedes hacer en Twitter, o incluso si llevas un tiempo y quieres replantear tu uso, puedes seguir estos consejos, tan libre o al pie de la letra como quieras.
 Todo lo que yo diga aquí está abierto a controversia, pues al fin y al cabo yo no he sido más que un observador que no
 se las sabe todas, y un partícipe con su propio sesgo.

\par% p
Así que comenzaré con la regla \#1.

\par% p% {'style': 'padding-left: 30px;'}
\relax{% {'style': 'color: #993300;'}
\textbf{Regla \#1:}}\textbf{ Twitter no tiene reglas.} Todo consejo que doy acá no es más que un consejo el cual eres libre de seguir o ignorar, y siempre habrá quien haga
 todo lo contrario al consejo y lograr creativamente resultados.

Bueno, con la primera regla clara lo primero que tienes que preguntarte es qué tipo de usuario quieres ser.~ \textbf{¿Quieres usar tu cuenta en Twitter para un propósito profesional o más bien personal?} Los propósitos profesionales incluyen las cuentas institucionales, las cuentas personales de quienes tienen su propio negocio y quieren desarrollarlo por Twitter a nombre propio, o las cuentas personales de quienes quieren cuidar su imagen profesional, aún cuando no usen Twitter directamente para su negocio.
 Los propósitos personales son aquellas donde te interesa más la sociabilidad de la plataforma o, simplemente, mamar
 gallo.

Claramente si tu propósito es profesional debes tener más cuidado con usar correctamente Twitter y romper las reglas sólo cuando estés seguro de que experimentar vale la pena para tus propósitos.
 Si tu propósito es más personal, lo importante es que te sientas cómodo, aunque escuchar consejos y decidir si los
 sigues o no tampoco hace daño.

\par% p
La segunda pregunta, que es crucial cuando tu propósito es profesional, es \textbf{¿quieres usar Twitter para difundir o para obtener información o una combinación de ambas?}, y acto seguido pregúntate \textbf{¿Qué nivel de interacción quiero?} Estas preguntas no sobran para cuentas de propósito personal.

Por ejemplo, si eres el encargado de Twitter de un medio de comunicación o de un blog profesional ya establecido, tal
 vez sólo te interese difundir información (p. ej. los titulares de las nuevas notas con un link a la nota respectiva)
 y no necesites mayor interacción.

Si eres un candidato en campaña política, tu interés principal podría ser difundir el programa y las actividades de la
 campaña, pero puede convenir crear un espacio para escuchar inquietudes y responder dudas específicas sobre la campaña
 y el programa.

Por otro lado, si eres un diseñador o programador de web, tu interés profesional requiere que te enteres de las últimas novedades y crear una red de profesionales afines con quienes despejar dudas.
 Aquí eres, principalmente un consumidor de información que busca amplia interactividad.

O simplemente buscas la sociabilidad.
 La información que buscas es cómo se encuentran otras personas para compartir tus inquietudes personales.
 La interacción aquí es el eje de tu presencia en Twitter.

\par% p
Muy probablemente no seas un único tipo de usuario.
 Puedes partir de ser simple promotor de tu propia información y volverte más interactivo si descubres que hay un potencial o un gusto por escuchar a los demás y darles respuestas.
 O puedes combinar una labor profesional propia con elementos de sociabilidad.

\par% p% {'style': 'padding-left: 30px;'}
\relax{% {'style': 'color: #993300;'}
\textbf{Regla \#2:}}\textbf{ Entiende por qué sigues a quienes sigues}.
 Cuando sigues a alguien en Twitter, esto se refleja técnicamente en dos aspectos.
 Verás lo que esa cuenta dice dentro de tu flujo de datos (\emph{timeline} o línea de tiempo), y le permitirás a esa persona que te envíe mensajes directos.
 Pero seguir también tiene otros aspectos más sociales: das importancia a la persona que sigues y lo estás invitando a que te siga.
 Estas dos cosas no son necesariamente ciertas.

Si tu intención es obtener información, claramente si sigues a muy pocos usuarios recibirás pocas fuentes de información, pero si sigues a demasiados, no será práctico.
 También es clave saber a quienes sigues.
Sí sólo te interesa seguir a los principales diarios o a ciertos blogs, tal vez Twitter no sea tu mejor herramienta
 sino un agregador de \anchor[http://es.wikipedia.org/wiki/Fuente\_web]{feeds}: obtienes mejor información por cada entrada, que un titular machacado para caber en 140 caracteres.
 Un posible consejo si apenas estás comenzando: busca las cuentas personales en Twitter de tus blogueros favoritos y síguelas.
 Por lo que ellos dicen y con quien ellos interactúan regularmente puedes descubrir qué otros usuarios reportan
 información relevante y empezar a seguirlos.

Si tu intención es divulgar información y ya eres un medio establecido, o un artista famoso, no es necesario hacer mucho.
 Primero empieza por decir cosas interesantes.
 Luego puedes usar los servicios de búsqueda de Twitter para ver quiénes podrían estar interesados en lo que dices, y empiezas a seguir a un puñado de ellos pero que parezcan tener influencia.
 Aquí usas la táctica de seguir a modo de tarjeta de invitación, pero con la esperanza de que estos usuarios empiecen a hablar de ti o de tu medio, empiecen a retwittear lo que dices y a promocionarte.
 A partir de ahora depende de tu estrategia de interactividad, para determinar a quién más seguir.

Si quieres divulgar información pero no eres tan reconocido, básicamente inicia de una forma similar a la anterior,
 pero incluso si no te interesa la interactividad debes continuar siguiendo nuevas cuentas para garantizar tu
 visibilidad.

Cada cuenta que sigues te representa una serie de tweets en tu flujo de entrada, y necesitarás tiempo para leerlos todos.
 Si sigues a una veintena de usuarios personales, tal vez con media hora al día o cinco minutos cada hora puedas enterarte de todo lo que dicen.
Si sigues a diez mil usuarios, es imposible leer todo lo que dicen en tu flujo de entrada.
 A partir de qué número se vuelve inmanejable el flujo de entrada depende de qué tan activos sean tus seguidos y cuanto tiempo puedas disponer.
 También depende de qué tan rápido puedas reconocer mensajes que no te interesan sin terminarlos de leer.

Hay varios tipos de filtros que te permiten hacer manejable lo inmanejable.
 Integrado dentro de la misma plataforma de Twitter están las listas.
 Puedes crear una lista de los usuarios que siempre te interesa leer, otras de los usuarios que lees cuando buscas
 información de un tema específico. Cuando tienes un tiempo extra puedes leer todo lo último que han dicho en tu flujo
 principal de entrada, pero la mayor parte de las veces sólo lees las listas.

Herramientas como TweetDeck permiten crear columnas donde filtras palabras claves para ver o no ver lo que te interesa.

\par% p
En una estrategia de difusión, al final puedes optar por distintas estrategias de seguimiento:

\begin{enumerate}

\item Sigues a nadie o a muy pocos.
 Esto sólo funciona si ya eres una fuente establecida de información relevante para tus seguidores.
~ No das vanas ilusiones de que te interesan leer todo lo que dicen tus seguidores.~ También puedes hacer creer que te
 crees una diva.
\item Sigues a varios, pero definitivamente muchos menos de quienes te siguen.
 Esto suele indicar que estás interesado en lo que dicen ciertas personas, por ejemplo, que estás buscando información al tiempo que difundes la tuya.
Con seguridad también creeran que te crees una diva.
\item Es similar el número de personas que te siguen a las que sigues, y los números de seguidos parecen manejables.
 Bueno, esto significa que aún no eres un faro de información, pero con seguridad los que te siguen creerán que eres
 una persona interesante.
\item Números similares, pero inmanejables: los usuarios novatos podrán creer que los tienes en cuenta, pues los sigues, pero no nos engañemos: es imposible que sepas lo que ellos dicen cuando no se dirigen a ti.
 La ventaja es que estás permitiendo que tus seguidores te envíen mensajes directos, lo que no sucede en los dos
 primeros casos.

\end{enumerate}

Si te siguen varios miles de usuarios, tu principal labor es de difundir información, pero te interesa la
 interactividad, a la hora de la verdad no importa a cuantos sigan sino si tienes en cuenta a quienes se dirigen a ti.

\par% p
Desafortunadamente si eres una celebridad de la farándula, los deportes o la política, muchas de las veces que te mencionan no se dirigen realmente a ti, así que es importante filtrar todos esos mensajes que no esperan respuesta de aquellos que sí.
 Si son muchos, y así tú quieras twittear personalmente, posiblemente necesites contratar a alguien que te indique qué
 debes o no debes contestar.

\par% p% {'style': 'padding-left: 30px;'}
\relax{% {'style': 'color: #993300;'}
\textbf{Regla \#3:}}\textbf{ Prueba que escuchas.} Aún cuando tu estrategia no requiera interactividad ocasionalmente te harán una observación respetuosa encaminada a que mejores algún aspecto de tu comunicación.
 Aunque pongas tu diario o tu blog de 100 entradas diarias a cargo de un bot que se limite a replicar tu feed, de vez
 en cuando revisa las menciones (sobre todo aquellas que inician con tu arroba) para saber si hay algo que te ayude a
 mejorar y aunque no tengas tiempo para mejorarlo, agradece.

Desde luego que si tu estrategia se basa en tener mayor interacción, muestra que estás pendiente de lo que te dicen y
 responde toda solicitud o mensaje que te hayan hecho con respeto.

Adicionalmente: siempre que preguntes algo, revisa quién te contesta, y agradece la respuesta.
 No es sólo un acto de cortesía, es que ese usuario que te contesta es alguien que puede aburrirse de escuchar lo que
 dices si no siente que lo tienes en cuenta.

\par% p
Desde luego hay ocasiones en las que no es práctico responder individualmente.
 Un mensaje al aire donde expreses tu gratitud a todos los cientos de usuarios que te respondieron podría ser
 suficiente.

\par% p% {'style': 'padding-left: 30px;'}
\textbf{\relax{% {'style': 'color: #993300;'}
Regla \#4:} No te olvides del contenido.} Quienes te siguen lo hacen porque les representas algo.
 Si no son bots de spam, o cuentas de difusión nuevas, generalmente eso que representas es un contenido de calidad.
 Mantén esa calidad.

Por calidad no me refiero necesariamente a que todos tus tweets sean trascendentales o importantes, sino que sea el tipo de material que esperan los que te siguen.
 Si tu uso es más social que profesional, está bien que cuentes cómo te fue en el día, qué estás comiendo o que te duele la cabeza.
 Entre más sincero parezca lo que dices, mayor calidad tendrá eso que dices.
 Pero si tu uso es más profesional, tu medida de calidad será acorde.

Si tu interés es difundir información al mayor número de usuarios, es más efectivo cuando esos usuarios son receptivos a tu información.
 Muéstrales qué tipo de información tienes para ofrecer y si están interesados te seguirán para obtener más información.
 No esperes que con sólo tu nombre se vuelvan adictos a lo que tienes que decir, a menos que sepas, desde luego, que no
 tienes mucho para decir y que tu nombre es tu mayor activo.

Si tu interés es obtener información de una forma activa, es necesario que quienes puedan responder escuchen tus
 preguntas, y para ello es bueno dar algo a cambio: comparte tus propias experiencias en el área y crea así una
 audiencia.

\par% p
Desde luego, si sólo eres un stalker de celebridades, pues no es necesario que digas nada, o no tienes que cuidar qué
 dices, pues lo importante aquí no es quién te sigue sino a quién sigues.

\par% p% {'style': 'padding-left: 30px;'}
\relax{% {'style': 'color: #993300;'}
\textbf{Regla \#5:}}\textbf{ Determina si realmente necesitas un candado.} Si tu cuenta tiene un propósito profesional de difusión o de búsqueda activa de información, definitivamente es mala idea proteger tus tweets.
 Si te interesa interactuar con personas desconocidas, es mejor que dejes tu cuenta abierta.
 Pero si tu propósito es más social personal y quieres tener la libertad de quejarte de tu jefe ante tu séquito de
 seguidores el candado es necesario.

Si aún no estás seguro de para donde quieres ir, pero tu propósito es profesional, es mejor que tengas tu cuenta
 abierta (sin proteger tweets) y te asegures de no decir nada de lo que deberías arrepentirte.

Si tu propósito es más personal y social, pero aún no sabes bien qué tipo de cosas querrás decir, puedes jugar con
 proteger o desproteger tu cuenta hasta encontrar de qué forma te sientes más seguro.

Generalmente una cuenta abierta atrae más seguidores.
 Tanto seguidores legítimos (personas que se interesan en lo que dices) como bots de spam.
 Pero también te puede leer tu jefe actual o futuro, tu ex novio, o personas que no quieres que se enteren de lo que
 dices.

Primero no te preocupes si te sigue una cuenta de spam.
 La cuenta de spam sólo será molesta si tú la sigues.~ No es necesario siquiera que te preocupes de bloquearla.

\par% p
Básicamente la única razón para proteger tu cuenta es porque piensas decir algunas cosas personales y quieres estar seguro de quién te lee.
 Pero esto sólo funciona si estás seguro de quién te lee.
 Cuando decidas proteger tu cuenta es conveniente que hagas una purga de seguidores.
 Y cuando te interese tener más de 500 seguidores olvídate que tienes control en cada uno de ellos: no te ilusiones con
 el candado, no sirve para nada, mejor deja abierta la cuenta.

\par% p% {'style': 'padding-left: 30px;'}
\relax{% {'style': 'color: #993300;'}
\textbf{Regla \#6:}}\textbf{ Reduce el spam a su mínima expresión necesaria.} Hay dos tipos de spam en Twitter: que te la pases promocionando algo, o lograr que los demás te promocionen a ti.
 Si sólo promocionas algo corres el riesgo de que te bloqueen o ignoren.

Claramente la cuenta \anchor[http://twitter.com/cocacola]{@cocacola} promocionará a la marca Coca Cola.~ Esto no es spam.~ Quienes siguen a \anchor[http://twitter.com/cocacola]{@cocacola} es porque están interesados en Coca Cola.~ Pero si la cuenta \relax{% {'style': 'color: #0000ff;'}
@marilynjoe} se crea para seguir a muchos usuarios con la esperanza de que sigan un vínculo a \emph{Enlarge Your Penis}, entonces eso sí es un spam descarado, y muy probablemente será bloqueada y cerrada por denuncias de los usuarios.

Si tu interés es promocionar un producto, hazlo lo más abiertamente posible y no por medio de tácticas subrepticias.
 No lo hagas desde una cuenta que pretende ser de una persona haciendo contactos personales con otros twitteros, sino
 desde una cuenta creada para el producto en mención.

Pero si lo vas a hacer subrepticiamente hazlo bien.
 Sólo que eso requerirá mucho trabajo.
 Dí cosas interesantes además de la promoción de tu producto, de tal forma que la promoción de tu producto parezca ser una más de las cosas que tienes que decir.
 El problema es que a la larga te termines enviciando al aspecto social de Twitter y te olvides de tu producto.

Un spam más efectivo es logrando que los demás twitteros hablen de ti.
 La forma más fácil es crear concursos en los cuales los usuarios tengan que repetir tu marca para ganar un premio.
 El problema es que esto también genera cierta alienación hacia tu marca.
 Más trabajo es lograr que mencionen tu marca junto a contenidos de calidad, por ejemplo con datos trivia sobre la
 industria bajo la cual se enmarca tu producto.

\par% p
Desde luego que la calidad percibida del producto permita que unas estrategias sean mejores que otras: Sipote Burrito
 puede salirse con la suya haciendo concursos de spam mientras que si quieres promocionar un substituto genérico al
 Viagra, te reportarán como spam de una, por ingenioso que quieras ser.

\par% p% {'style': 'padding-left: 30px;'}
\relax{% {'style': 'color: #993300;'}
\textbf{Regla \#7:}}\textbf{ Diviértete.} Si tu cuenta es personal, aun cuando haya un propósito profesional en ella, tú estás en Twitter porque te sirve y te gusta.
 Si tienes que usar Twitter profesionalmente, también puedes usarlo para desahogarte por medio de una cuenta personal.
 O no.

Si quieres crear una red de amigos nuevos diferentes en perfil a los que puedes encontrar en Facebook u otras redes similares, en Twitter puedes intentarlo.
 Hay mucha gente de distintos perfiles que estará interesada en seguirte.

Si tienes espítitu de troll, trolea.
 Si quieres trolear pero quieres proteger tu imagen profesional, crea un alterego y trolea desde allí.

Si sientes que tienes chispa y que le puedes sacar apuntes a cada situación: con seguridad que podrás encontrar una
 buena audiencia en Twitter y podrás alimentar tu ego con los retweets que recibes.

Si eres feliz contando tus cuitas Twitter es perfecto: te escucharán un conjunto de personas que será lo
 suficientemente cercano para escuchar pero no tanto como para que te de pena.

Sigue a quienes digan algo que tú quieras escuchar y a tantos como tu creas que puedes manejar.

Responde lo que te dé la gana responder, sea que te pregunten o no.

Muestra tu lado más obscuro o tu lado más sublime: el que te divierta más mostrar.

Pon candado.

Quítalo.

Participa en las rifas que quieras, así tus seguidores amenacen con dejarte de seguir.

Cuida a tus seguidores.

Sorpréndelos.

Búrlate de ellos.

\anchor[http://mauriciogonzalez.tv/no-todos-tienen-algo-para-vender-en-las-redes-sociales/comment-page-1/\#comment-844]{Diviértete.}

\chapter{\#antesdeuribe Santos ya era Santos}
\begin{metadata}
	Published by \anchor[chlewey]{chlewey} on \anchor[http://ewey.co/B532]{Thu, 06 May 2010 20:07:54 +0000}\\
	\categories{elecciones, juan-manuel-santos, opinion, uribismo}\\
	Shorthand: \anchor[http://blog.chlewey.net/2010/05/antesdeuribe/]{antesdeuribe}
\end{metadata}

\par% p
Siento que mis amigos más uribistas, así como a mucha de las personas que leo en Twitter y otros foros, consideran
 votar por \anchor[http://es.wikipedia.org/wiki/Juan\_Manuel\_Santos]{Juan Manuel Santos} como la única forma de preservar el legado de \anchor[http://es.wikipedia.org/wiki/\%C3\%81lvaro\_Uribe\_V\%C3\%A9lez]{Álvaro Uribe Vélez} y de continuar su obra.
 Finalmente ese es el posicionamiento que Santos ha querido dar a su campaña (antes y después del cambio de imagen).

\par% p
Este post no va para mis amigos de la \anchor[http://twitter.com/\#search?q=\%23olaverde]{Ola Verde}.~ Ni para mis amigos del \anchor[http://www.polodemocratico.net/]{Polo}; o quienes permanecen con el trapo \anchor[http://www.partidoliberalcolombiano.info/inicio/]{rojo} o \anchor[http://www.partidoconservador.org/]{azul}, o que piensan en \anchor[http://es.wikipedia.org/wiki/Rafael\_Pardo]{Pardo} o Sanín a pesar del trapo rojo o azul.~ Mucho menos para los seguidores de \anchor[http://es.wikipedia.org/wiki/Germ\%C3\%A1n\_Vargas\_Lleras]{Germán Vargas Lleras}.
 Este artículo ni siquiera va para los seguidores de Juan Manuel Santos que creen en Santos por ser Santos y no por ser
 el candidato del uribismo.

\par% p
Juan Manuel Santos apareció en mi radar cuando era Primer Designado (título que después sería reemplazado por el de
 Vicepresidente) durante el gobierno de \anchor[http://es.wikipedia.org/wiki/C\%C3\%A9sar\_Gaviria]{César Gaviria Trujillo}, presidente liberal.~ Lo recuerdo en una caricatura de \anchor[http://www.semana.com/noticias-gente/osuna-50-anos-rasgos-rasgunos/121464.aspx]{Osuna} cuyos detalles son irrelevantes en este momento. Si hoy creemos que los vicepresidentes son figuras que no sirven para mucho y que no hacen mucho, salvo que por enredos en lo que digan comprometan al gobierno; los primeros designados eran figura más obscuras aún.
 Pero ahí se veía ya a un personaje político y con ambiciones.
 Juan Manuel Santos proviene de una familia de periodistas, los entonces dueños de \anchor[http://www.eltiempo.com/]{El Tiempo}, con \anchor[http://es.wikipedia.org/wiki/Eduardo\_Santos]{un Presidente} en su pasado.

\par% p
Durante el siguiente período, Santos pasó a conformar el triunvirato que regiría al Partido Liberal, entonces en el
 poder con \anchor[http://es.wikipedia.org/wiki/Ernesto\_Samper]{Ernesto Samper}; sin embargo Santos fue crítico de Samper y aparece en \anchor[http://www.google.com.co/search?q=santos+samper+mancuso]{algunas investigaciones y declaraciones} como parte del complot que quería tumbar al presidente.
 Se retiró de la dirección colegiada del partido para aspirar a la presidencia pero ante la apabullante popularidad de \anchor[http://es.wikipedia.org/wiki/Horacio\_Serpa\_Uribe]{Horacio Serpa}, fiel escudero de Samper,~ retiró su aspiración y formó la organización \anchor[http://www.buengobierno.com/]{Buen Gobierno}.

\par% p
Desde Buen Gobierno promulgaba la tesis de la \emph{tercera vía}.~ Algo así como un punto intermedio entre el \anchor[http://es.wikipedia.org/wiki/Neoliberalismo]{neoliberalismo} de Gaviria y la \anchor[http://es.wikipedia.org/wiki/Socialdemocracia]{socialdemocracia} del serpo-samperismo; y que encontraba inspiración en el gobierno laborista de \anchor[http://es.wikipedia.org/wiki/Tony\_Blair]{Tony Blair}.~ El término ``\anchor[http://es.wikipedia.org/wiki/Tercera\_V\%C3\%ADa]{tercera vía}'' se había usado antes para designar a los gobiernos socialdemócratas que, como el de Suecia, durante la guerra fría
 buscaban una política social desde el capitalismo; pero ese significado se había corrido tras la caída de la cortina
 de hierro.

Muchos veían a Buen Gobierno como una plataforma política para proyectar una candidatura presidencial.
 Su aspiración ya era clara.~ Pero en esos años se presentaba a sí misma como un observatorio político.

\par% p
En 2000, Santos se retira de Buen Gobierno para aceptar el Ministerio de Hacienda que le ofreció el presidente \anchor[http://es.wikipedia.org/wiki/Andr\%C3\%A9s\_Pastrana\_Arango]{Andrés Pastrana}.
 No podríamos saber lo que pensaba de la contienda política que en 2002 enfrentaría a Horacio Serpa con Álvaro Uribe
 Vélez, pues desde el ministerio no podía participar, pero al término del mismo regresa a El Tiempo como columnista, a
 retomar las tesis de la tercera vía y atacar al presidente venezolano \anchor[http://es.wikipedia.org/wiki/Hugo\_Ch\%C3\%A1vez]{Hugo Chávez Frías} por su modelo político.

Cuando empezó a sonar la reelección de Uribe, impulsada por declaraciones de la entonces embajadora en España Noemí Sanín, Juan Manuel Santos se confesaba como uribista pero contrario a que se hicieran reformas constitucionales a nombre propio.
 En otras palabras se confesaba contrario a la reelección de Álvaro Uribe.
 Pero una vez aceptada la reelección por el Congreso y la Corte Constitucional, Santos atendió el llamado de Uribe de formar un partido político que recogiera a los liberales que apoyaban a Uribe (y que no quisieran adherirse al entonces existente Cambio Radical).
 Santos pasó así de antirreeleccionista a jefe del partido de la reelección: el Partido Social de Unión Nacional o
 Partido de la U.

El premio por su labor: el Ministerio de Defensa.

Un paréntesis.

\par% p
En Twitter y en Facebook he preguntado a mis amigos y seguidores de qué nos han servido los cuatro años adicionales de
 Álvaro Uribe Vélez, y como única respuesta he recibido algunos \emph{like}.~ Ninguno de mis uribistas amigos se atrevió a contestar.

\par% p
Hay un punto para el cual sirvió el segundo cuatrenio: romper la esperanza de las \relax{% {'style': 'font-variant: small-caps;'}
Farc} de que el chapuzón del gobierno de Uribe se pudiera aguantar con permanecer cuatro años en repliegue estratégico.
 Pero creo que esto lo hubiera logrado también un sucesor de Uribe y sin tanto desgaste.
 En 2006 el sucesor de Uribe tenía nombre propio: Germán Vargas Lleras.

\par% p
Uribe llegó al poder en 2002 con varias banderas.~ La más conocida era su lucha contra las \relax{% {'style': 'font-variant: small-caps;'}
Farc}, pero también quería impulsar un saneamiento de la política.
 Amenazaba con reformar el Congreso y tranzó con los congresistas que el Congreso seguiría funcionando sin necesidad de
 que se intercambiaran puestos o se otorgaran otras prevendas.

\par% p
Pero llegó el embeleco que la reelección y con ello a las prácticas de intercambiar favores en el Congreso con favores del ejecutivo con los congresistas.
 La bandera del saneamiento político se escondió discretamente mientras que los defensores de turno del gobierno
 explicaban que era necesario transar con el Congreso por aquello de la \emph{Realpolitik}.~ Igual no importaba, los triunfos de la política de la Seguridad Democrática (su lucha contra las \relax{% {'style': 'font-variant: small-caps;'}
Farc}) hizo que los electores pasados olvidaran su doble promesa y se enfocaran sólo en la primera.
 A Uribe lo eligieron en 2002 por su postura fuerte contra las FARC y porque pretendía sanear la política.
 En 2006 aquello de sanear la política era apenas una promesa incómoda que era mejor esconder, por que al fin y al cabo
 la culebra seguía viva.

\par% p
En el primer cuatrenio de Uribe, la capacidad operativa de las \relax{% {'style': 'font-variant: small-caps;'}
Farc} se vio seriamente golpeada.
 Todas las cabeceras municipales habían sido recuperadas en nombre del estado constitucional.
 Sólo faltaban los golpes contundentes, pero el verdadero trabajo ya estaba hecho.

\par% p
Entre los posibles candidatos de 2006, algunos prometían continuar la política de seguridad.
 Sin duda Lleras lo hubiera hecho.
 Pardo también era afín (a pesar de las falsas acusaciones, ventiladas por Santos, de vínculos con las \relax{% {'style': 'font-variant: small-caps;'}
Farc}).~ Serpa ya había prometido mano firme contra las \relax{% {'style': 'font-variant: small-caps;'}
Farc} en 2002.
 Sólo en el naciente Polo Democrático Alternativo había una postura de replantear la seguridad democrática.
 Carlos Gaviria, candidato del Polo, resultó favorecido con el segundo puesto lo que hizo a muchos asustar con era indispensable que Uribe se hubiera lanzado, pero gran parte del triunfo de Gaviria fue haberse enfrentado a Uribe.
 Estoy casi seguro que si Uribe no hubiera sido candidato en 2006 hubiera ganado un candidato de línea dura frente a
 las \relax{% {'style': 'font-variant: small-caps;'}
Farc}.

A Uribe sólo le faltaban los golpes contundentes.
 Y estos llegaron durante su segundo cuatrenio con Juan Manuel Santos a la cabeza del Ministerio de Defensa.
 Con esto se cierra el paréntesis.

Juan Manuel Santos lo que hizo fue cosechar el trabajo del primer cuatrenio de Uribe, y no sin poner en jaque algunos
 aspectos del gobierno uribista.

Primero, sus posiciones como columnista, le trajeron un inmediato recelo del gobierno venezolano quien casi rompe relaciones con Uribe sólo por haber nombrado a Santos ministro de defensa.
 (Igual esto no es culpa de Santos ni de Uribe, Chávez siempre tuvo una relación de amor y odio con Uribe: odio en los
 micrófonos y amor tras cada encuentro personal entre los dos mandatarios.)

\par% p
Luego vendría la Operación Fénix en la que el Ejército y la Policía de Colombia abaten a \emph{Raúl Reyes} y a casi todos sus acompañantes.
 Santos dio la orden de proceder y luego da las primeras declaraciones con el parte de victoria.
 Parte de victoria que contenía una mentira clave: dijo que la acción fue iniciada por la guerrilla desde territorio colombiano y que luego los guerrilleros se replegaron al campamento en Ecuador donde culminó la acción.
 Esa versión mentirosa fue la que Santos comunicó a Uribe cuando este habló por primera vez con el presidente Correa de Ecuador.
 Esa mentira le costó credibilidad diplomática a Colombia, por más que el uribismo crea que la diplomacia colombiana triunfó en la Cumbre de Rio y en la OEA.
 Esa mentira provocó el deterioro de las relaciones diplomáticas entre Colombia y Ecuador.

\par% p
La muerte de \emph{Iván Ríos} y la de \emph{Manuel Marulanda}, no son estrictamente triunfos de Juan Manuel Santos.~ \emph{Ríos} murió asesinado por uno de los suyos, producto del hostigamiento de las fuerzas armadas constitucionales.
 Y Santos, se apresuró a declarar que su asesino, un guerrillero con prontuario propio de actos de terrorismo y barbarie más allá de los justificables bajo el concepto de rebelión, era digno de una recompenza por entregar a un miembro del Secretariado de las FARC.
 Igual, supongo, los uribistas que me lean aquí dirán que esto fue un acto de guerra, justificable bajo la gravedad de
 la amenaza que las FARC significan al país.

\par% p
Por su parte \emph{Marulanda }(alias \emph{Tirofijo}) murió de causas naturales.~ Podría argumentarse que el hostigamiento de las fuerzas armadas evitó que \emph{Marulanda }tuviese una atención médica adecuada.
 Pero el papel de Santos es, por mucho, tangencial.
 Realmente el único papel de Santos en el hecho fue haber anunciado en una entrevista que \emph{Tirofijo} esataba muerto, antes de cualquier otra información revelada al público.

El otro gran triunfo de Santos fue la Operación Jaque, pero el verdadero papel de Juan Manuel Santos fue aprobar el
 plan ya elaborado por el grupo de Inteligencia del Ejército y comunicarlo al presidente y al embajador de Estados
 Unidos para recibir su visto bueno.

La Operación Jaque fue posible porque la red de comunicaciones de la guerrilla estaba vulnerable, producto del hostigamiento de las fuerzas militares.
 Y fue posible gracias a los presedentes de las liberaciones mediáticas de algunos políticos secuestrados.

\par% p
Ahora, el hostigamiento de la fuerzas militares a la guerrilla es algo que venía desde antes.
 Muchos recuerdan al gobierno de Andrés Pastrana Arango (y del cual Santos hizo parte) como el gobierno que le entregó
 el país a las \relax{% {'style': 'font-variant: small-caps;'}
Farc}.~ Esto no es del todo cierto.~ Pastrana desmilitarizó un extenso territorio con el objetivo de dialogar con las \relax{% {'style': 'font-variant: small-caps;'}
Farc} lo que les facilitó cierta logística a las mismas; pero al mismo tiempo Pastrana gestionó el Plan Colombia que fue la base de la profesionalización de las Fuerzas Armadas constitucionales.
 Tras un pico de secuestros en 2000, el número de secuestrados por las \relax{% {'style': 'font-variant: small-caps;'}
Farc} disminuyó en la segunda mitad del mandato de Pastrana, en parte por un Gaula mejor entrenado y con mejores equipos.
 Pastrana también creó los batallones de Alta Montaña que permitieron recuperar el Sumapaz y bajo su mandato se
 disminuyó en una parte importante la presencia de las guerrillas en Cundinamarca.

\par% p
Clausurada la Zona de Distensión (ese territorio desmilitarizado por el estado constitucional), las \relax{% {'style': 'font-variant: small-caps;'}
Farc} iniciaron una ofensiva terrorista de mucho impacto mediático, pero las cantidades de explosivos decomisados por las Fuerza Militares en esos meses anteriores a la elección de Uribe fueron mucho mayores que los explosivos utilizados en atentados terroristas.
 Con Pastrana ya había un ejército funcional capaz de enfrentar a las \relax{% {'style': 'font-variant: small-caps;'}
Farc} y ganarles, cosa que no existía al principio de su mandato.

A ese ejército profesionalizado y con equipos renovados gracias al Plan Colombia, Uribe le agregó la Doctrina de la
 Seguridad Democrática.

\par% p
La Doctrina reza que, siendo Colombia un estado de derecho bajo una constitución liberal y democrática, la guerra
 revolucionaria no tiene justificación alguna y por lo tanto no existe tal guerra revolucionaria, no existe conflicto
 armado (pues todo conflicto político podría dirimirse en los escenarios democráticos) y por lo tanto lo que hacen las \relax{% {'style': 'font-variant: small-caps;'}
Farc} no es más que una amenaza terrorista.

\par% p
No existiendo conflicto interno, entonces tampoco existen actores por fuera del conflicto.
 Nadie puede declararse neutral frente a lo que hay, y por lo tanto todos y cada uno de los ciudadanos de colombia o hace parte de la legalidad o apoya la amenaza terrorista.
 Ningún ciudadano puede substraerse a ser parte de la lucha contra las \relax{% {'style': 'font-variant: small-caps;'}
Farc} y quien lo haga es entonces amigo de los terroristas.

\par% p
En gran medida creo que el componente de fortalecimiento, profesionalización y equipamiento de las Fuerzas Armadas no tienen nada que ver con la Doctrina y sería igual de eficaz con Doctrina o sin Doctrina.
 La Doctrina por mucho ha permitido la justificación del pago de recompenzas a civiles; siendo aun debatible si las recompensas eran necesarias para garantizar la colaboración ciudadana.
 Parece ser que el mayor efecto de las recompenzas no fue lograr la colaboración ciudadana sino fomentar la deserción
 al interior de las \relax{% {'style': 'font-variant: small-caps;'}
Farc}; y para esto no se necesitaba la Doctrina.

\par% p
Con sólo haber mantenido una política de mano fuerte, con unas Fuerzas Armadas bien administradas (Marta Lucía
 Ramírez), bien equipadas (Plan Colombia) y profesionalizadas (otro legado de Pastrana), tendríamos ese hostigamiento a
 las \relax{% {'style': 'font-variant: small-caps;'}
Farc} que dieron fruto entre marzo y julio de 2008.
 Con Juan Manuel Santos de Ministro o con cualquier otro; con Uribe de presidente o con algún sucesor.

\par% p
Ahora, no sé hasta que punto sea culpa de Santos o no, pero esa gran ventaja estratégica que dio fruto en la Operación Jaque, se perdió cuando se revelaron todos los detalles de la misma.
 Tal vez lograr otra Operación Jaque hubiera sido imposible -al perro no lo capan dos veces- pero hay un detalle que
 Inteligencia Militar debió esconder: la vulnerabilidad de las comunicaciones de las \relax{% {'style': 'font-variant: small-caps;'}
Farc}.

Con todo ello, hay un señalamiento que tal vez sí sea injusto con Juan Manuel Santos: los falsos positivos.

En todo sistema de medición, un positivo es el hecho que una vez medido determine una acción.
 Por ejemplo para un sistema de detección de virus cibernéticos, un positivo se da cuando detecta un virus.
 En un sistema de reconocimiento de huellas dactilares, el positivo se da cuando determina que la huella sí corresponde a una persona autorizada.
 Un falso positivo corresponde cuando se detecta como virus algo que no es, o se autoriza a una persona que no se debía.
 Un negativo se da cuando la medida indica que no se debe tomar la acción (dejar correr el programa, no autorizar el paso de la persona).
 También hay falsos negativos (un virus indetectable, una persona autorizada a la que el sistema no le reconoce la
 huella).

Las Fuerzas Militares tienen en Colombia un objetivo principal: garantizar la vida, honra y bienes de los colombianos.
 Desde hace varios años la principal amenaza a la vida, honra y bienes, ha provenido de los grupos armados revolucionarios.
 Y por lo tanto la principal labor de las Fuerzas Militares ha sido enfrentar a las guerrillas.
 Se ha querido poner medidas de desempeño a la labor de las Fuerzas Militares, y eso requiere establecer parámetros para medir.
 En un mundo ideal, la medida sería cuántas vidas de colombianos salvaron y cuántos bienes conservaron.
 Pero esto es muy difícil de medir.
 Un colombiano muerto por cáncer no debe ser un indicador negativo del desempeño de las fuerzas militares.
 Así que las mediciones se han enfocado no a los objetivos finales sino a los objetivos intermedios de su función
 operativa: enfrentar a las guerrillas; y su método de medición: contar bajas.

Un guerrillero muerto en combate es una baja (un positivo).
 Un guerrillero capturado es una baja (otro positivo), un guerrillero desmovilizado es otra baja (un positivo pero que no puede atribuirse a un batallón, no entra en la cuenta).
 Un civil muerto y hecho pasar por guerrillero contará como un positivo (un falso positivo).
 Un civil arrestado y procesado por guerrillero también sería un falso positivo.
 Pero ese civil arrestado podrá hablar y liberarse de su falso cargo.

Por ello, es este esquema de medición, heredado de la guerra fría (conteo de cuerpos) y regulado por el antecesor a
 Santos el ex ministro Ospina, el que ha producido los escándalo de los falsos positivos: asesinato extrajudicial de
 civiles hechos pasar como bajas guerrilleras para mejorar los índices de desempeño de unidades militares.

Hay indicios de que Santos intentó arreglar esta situación, aunque no sé si de la forma correcta (pues esto no se arregla con cursos de derechos humanos y con declarar que se prefieren guerrilleros caputrados a guerrilleros muertos).
 Por lo menos Santos tomó una acción cuando el escándalo salió a la luz pública.

Una acción que ha sido dúramente criticada por los defensores de oficio de la Doctrina, tales como Fernando Londoño
 Hoyos, hoy ferviente defensor de la candidatura de Juan Manuel Santos.

Juan Manuel Santos tiene un pasado antes de Uribe.
 Tuvo sus propias convicciones políticas como líder del Partido Liberal y como líder de Buen Gobierno.
 Se adaptó a Gaviria y a Pastrana y a Uribe y a estos dos últimos tras ser crítico de sus gobiernos o políticas.
 De antirreleccionista pasó a liderar el partido de la reelección y tras haber metido al gobierno en un par de
 problemas diplomáticos, uno de ellos por mentir, ahora posa como el único continuador de la política de Seguridad
 Democrática.

Santos le debe poco a Uribe y a su Seguridad Democrática, salvo porque ello fue un buen pantallazo.
 Si eres uribista sólo o principalmente por la Seguridad Democrática y de la Seguridad Democrática lo único que te interesa es que las Fuerzas Armadas constitucionales no detengan su ímpetu contra las guerrillas (y no su componente ideológico), entonces cualquiera de los candidatos actuales ha prometido continuar con esta política.
 Está Mockus y su principio de autoridad no negociable.
 Está Vargas Lleras quien apoyó la política uribista desde el primer día y si hoy no figura como uribista es por disentir con respecto a la segunda reelección.
 Está Sanín quien propone revivir el fuero militar que Londoño Hoyos tanto añora.
 Está Pardo quien nunca ha sido amigo de las FARC como lo demuestra su acción en Casaverde.
 Hasta Petro ha prometido continuar con la política de seguridad de Uribe.

Si fuiste uribista por sus promesas de sanear el estado: tal vez el Santos de Buen Gobierno sería una opción, pero el
 Santos de hoy se olvidó de Buen Gobierno por posar como el sucesor de Uribe, el mismo Uribe que traicionó a los
 colombianos en el saneamiento estatal para pedir que los parapolíticos votaran los proyectos gubernamentales antes de
 ir a la cárcel y negociar de todo en aras de dos reelecciones.

En fin.
 Si yo fuera uribista, Juan Manuel Santos no sería mi candidato.
 Por trayectoria y credibilidad preferiría darle mi voto a Germán Vargas Lleras quien en un momento dado prefirió el
 estado de derecho a seguir jugando a la ambigüedad del continuismo de la doctrina en cabeza propia.

Igual no soy uribista.
 Nunca lo he sido.
 Tampoco fui antiuribista y dado el caso, en 2002 o 2006 hubiera votado por Uribe en segunda vuelta frente a Serpa o Gaviria.
 Si tuviera que elegir entre Uribe y Santos, sin embargo, no sabría cual elegir principalmente porque no sé cuál de
 todos los Juan Manuel Santos es Juan Manuel Santos.

\chapter{Hora de tomar decisiones}
\begin{metadata}
	Published by \anchor[chlewey]{chlewey} on \anchor[http://ewey.co/B549]{Sat, 29 May 2010 21:24:49 +0000}\\
	\categories{elecciones, opinion}\\
	Shorthand: \anchor[http://blog.chlewey.net/2010/05/hora-de-tomar-decisiones/]{hora-de-tomar-decisiones}
\end{metadata}

\anchor[http://registraduria.gov.co/Informacion/images/tarj\_primer\_presi.jpg]{\begin{wrapfigure}{r}{250\px}\centering% {'src': 'http://blog.chlewey.net/wp-content/uploads/2010/05/tarjeton.png', 'title': u'Tarjet\xf3n electoral', 'height': '193', 'width': '250', 'alt': '', 'class': ['alignright', 'size-full', 'wp-image-552']}
\includegraphics[width=250\px,height=193\px]{blog/tarjeton.png}
\end{wrapfigure}
}Este domingo 30 de mayo de 2010 son las elecciones presidenciales en Colombia.  Todos los colombianos tenemos una parte de la decisión sobre quién nos guiará y nos
presidirá durante los próximos cuatro años.
La decisión final tal vez no se tome este domingo si ninguno de los candidatos logra el 50\% de los votos válidos
 tendremos otra oportunidad de decidir el 20 de junio entre los dos más plurales del 30 de mayo.

\par% p
Un voto no hace mucha diferencia.
Un voto más un voto más otro voto y así pueden sumar diferencias significativas.
Por ello lo importante es que tomes la decisión de por quién votar (o de no votar) con la mejor sinceridad.
Tus razones no tienen que ser del todo las correctas, pero sí es importante que te sientas conforme, que te sientas
 bien contigo mismo, por la decisión que tomes.

\subsection{No tienes el poder de decidir.}
Si no eres colombiano, aun eres menor de edad, estás en las fuerzas armadas del estado, perdiste tu cédula o no alcanzaste a inscribir en la ciudad en dónde vives, igual puedes invitar a los demás colombianos a participar (o a que decidan no participar).
Si la ley no te~prohíbe~expresar tu opinión (p. ej. no estás en las fuerzas armadas), divúlgala.
\subsection{Decides no decidir.}
No decidir, pudiendo hacerlo, es una opción válida.
Sólo te pido que estés seguro de hacerlo por un motivo correcto (para ti, no necesariamente correcto para mí).

Por ejemplo, si tu religión te prohíbe votar y para ti es más importante seguir los preceptos de la misma de lo que pueda ser cumplir un deber ciudadano no obligatorio, pues nada que hacer.
Tu conciencia estará más a salvo con la abstención que con un voto.

\par% p
Puedes no creer en el concepto del voto.
Creer que tu voto no hace la diferencia (y la verdad tu voto no la hace, casi nunca, la probabilidad de que un
 candidato gane o pierda por \relax{% {'style': 'text-decoration: underline;'}
un} voto es muy baja).
Considerar que el desgaste de ir a una chichonera a que te requisen por un celular y hacer filas y filas no sea un plan de domingo.
Yo respeto todas estas excusas aunque quisiera convencerte de que tu voto sí hace la diferencia.

\par% p
Una de las excusas que más escucho en los abstencionistas y votoblanquistas es que ningún candidato vale la pena.
Si esta es tu excusa sí te pido que tengas en cuenta lo siguiente.

\par% p% {'style': 'padding-left: 30px;'}
Primero:
¿ya agotaste una a una las opciones? ¿o partes de un prejuicio como ``todos los políticos son iguales'', ``todos se la pasan diciendo mentiras''?
Yo no creo que todos sean iguales.
Una de las cosas de la que gozamos los colombianos este 30 de mayo es de una gran diversidad de personalidades y propuestas.
Si te da pereza revisar las propuestas entonces búscate una excusa más creíble como ``tengo mejores planes con mi cama
 este domingo''.

\par% p% {'style': 'padding-left: 30px;'}
Segundo: Agotado el análisis de las opciones sientes que te gustan algunas cosas de algunos pero todos tienen una falla.
Sólo te digo: no se trata de elegir a un candidato perfecto.
Se trata de que quiéraslo o no, se va a elegir a un presidente y tú tienes la opción de expresar en las urnas cuál se acerca más a tu concepto de sociedad.
No lo pienses como que te toca elegir el menos malo, piensa cual crees que es la opción preferible.

\par% p% {'style': 'padding-left: 30px;'}
Tercero: estás seguro que no hay opción preferible.
Que todas las opciones tienen un vicio insondable (p. ej. no hay un solo anarco-capitalista, o ninguno que prometa retomar diálogos con la subversión).
En ese caso te recomiendo el voto en blanco sobre la abstención.

Habiendo decidido que no vas a elegir tienes básicamente tres opciones: abstenerte, votar en blanco, o anular
 explícitamente tu voto.
\begin{enumerate}

\item \textbf{La abstención.} Abstenerte significa que no participas, que no tomas parte del proceso de selección, que te marginas.
Lo puedes hacer por convicción, por pereza, por desidia, o por fuerza mayor.
La abstención no tiene efectos electorales: finalmente se decide entre los que sí votan, pero tiene efectos para tí: no
 recibirás los beneficios que la ley otorga a los votantes, tales como medio día de trabajo pago si eres empleado,
 descuentos en el pasado judicial, etc.
\item \textbf{El voto en blanco.} El voto en blanco puede tener uno de dos efectos electorales: puede ser una carta blanca (delegas el poder de decisión en quienes no voten en blanco) o puede ser un voto de castigo (una expresión de que efectivamente no te gusta ninguno de los candidatos).
De acuerdo a la legislación colombiana actual, el significado es el segundo.\\
Sólo en caso de que el voto en blanco supere al voto individual de cada uno de los candidatos el significado electoral tendrá un efecto electoral: se invalidan las elecciones y se debe convocar a nuevas elecciones con nuevos candidatos.
De lo contrario su efecto es más bien simbólico y entre más candidatos pierdan frente al voto en blanco el mensaje es
 que esos candidatos no debieron haber competido.
\item \textbf{La anulación explícita.} Hay dos tipos de votos nulos que se cuentan por separado: los tarjetones no marcados y los tarjetones con más de una marca o con marcas poco claras.
En teoría se anularía tu voto cuando no es posible determinar qué fue lo que quisiste elegir.
Desde luego tú puedes intencionalmente anular tu voto, depositándolo sin marcar o marcando en más de una casilla (p.
 ej. puedes jugar a tachar todos los candidatos que no te gustan)\\
Para efectos electorales, un voto nulo es casi igual a una abstención.
No tiene el efecto deslegitimizador de la abstención, pero no cuenta para ningún efecto electoral.
Para tí significaría que puedes gozar de los beneficios de haber votado sin realmente haber votado.

\end{enumerate}

\subsection{El voto útil o el voto estratégico}
El concepto de voto útil reza lo siguiente: te gusta un candidato \emph{X}, pero las encuestas te dicen que \emph{X} no tiene mayores opciones y que los candidatos con alguna opción son \emph{Y} y \emph{Z}. ~Entonces vota ahora por el más acorde entre \emph{Y} y \emph{Z} con la esperanza de que no sea necesaria una segunda vuelta entre \emph{Y} y~\emph{Z} o de que pierda el que menos te gusta de esos dos.

\par% p
El voto estratégico tiene más matices.
El voto útil es un tipo de voto estratégico, pero el estratégico puede buscar efectos más sutiles.
Podrías, por ejemplo, votar por un candidato \emph{W} (diferente al \emph{X} que te gusta o a los candidatos \emph{Y} y \emph{Z} que se ven más opcionados) con el objeto, por ejemplo, de prevenir que \emph{Y} (que te gusta menos que \emph{Z}) llegue siquiera a segunda vuelta.

Por ejemplo.
Si te gusta Pardo, pero las encuestas señalan a Mockus y a Santos como los más opcionados.
El voto útil te indicaría votar por Mockus (si eres liberal antiuribista) o por Santos (si prefieres a un liberal de
 cuna).

\par% p
El voto estratégico podría suponer que si crees en la última encuesta que muestra a Mockus \relax, Santos \relax, Vargas Lleras \relax, Petro \relax, Noemí \relax, Pardo \relax; entonces votaras por Vargas Lleras para atajar a Santos.
(Personalmente no sé de donde sale tal encuesta o qué tan creíble es.)

\par% p
Personalmente desaconsejo el voto útil o el voto estratégico y menos en una situación actual donde no hay campañas fuertes promoviendo uno u el otro, las encuestas no son concluyentes y finalmente hay segunda vuelta.
Si tu voto estratégico no tiene el efecto estratégico deseado lo único que hiciste fue botar tu voto y castigar a tu candidato favorito.
¿En serio quieres castigar a tu candidato favorito por insistir?

\subsection{El antivoto.}
El antivoto generalmente consiste en votar a un candidato poco opcionado no porque te guste, sino porque no representa
 lo que no te gusta de los candidatos favoritos de las encuestas.

Hay dos tipos de antivoto: votar por la tercera opción, o votar por un colero aleatorio.

Ejemplos de tercera opción fueron William Vinazco en las pasadas elecciones a alcalde de Bogotá, o Carlos Gaviria en las elecciones presidenciales de 2006.
El efecto del antivoto en 2006 fue tal que la tercera opción se convirtió en la segunda votación.
El grueso de la votación de Carlos Gaviria no fueron de personas con sentir de izquierda o que gustaran de las
 propuestas de Gaviria, sino de personas que quisieron mediar entre un Presidente que ya se sabía lo iban a reelegir, y
 un candidato que se percibía de la maquinaria y la vieja política.

El problema del antivoto (sobre todo por la tercera opción) es que se crezca y a la hora de la verdad no sea lo que buscas.
Pero puede ser una opción más directa que el voto en blanco para expresar tu inconformismo con los punteros.

\par% p
No me gusta el antivoto, pero si lo quieres usar, trata de pensar qué pasaría si tu antivoto se convierte en una opción
 real de poder.

\subsection{Voto positivo.}
Si tienes un candidato, porque te gusta su programa, porque te identificas con los ideales de su partido, porque tienes confianza en su equipo, porque te inspira confianza él o ella, o porque sencillamente sientes que te identifica, vota por este candidato.
Ese es un voto positivo.

No votar por tu candidato por estrategia sólo funciona si haces parte de una estrategia grande y orquestada donde muchos como tú se hayan puesto de acuerdo.
Como no he sentido nada así en esta campaña política, el voto estratégico equivale a votar el voto.

Igualmente ninguna tendencia muestra que algún candidato esté próximo a ganar en primera vuelta, así que recomiendo
 votar en positivo por el candidato que más te convenza con la esperanza de que éste llegue a segunda vuelta y si no
 alcanza, ahí sí el 20 de junio vota por lo que prefieras entre las opciones de entonces: uno de los dos favorecidos,
 en blanco, o abstenerte.

\chapter{Fantasía y ciencia ficción}
\begin{metadata}
	Published by \anchor[chlewey]{chlewey} on \anchor[http://ewey.co/B555]{Fri, 09 Jul 2010 16:20:37 +0000}\\
	\categories{ciencia-ficcion, formspring, fantasia, formspring-me, opinion}\\
	Shorthand: \anchor[http://blog.chlewey.net/2010/07/fantasia-scifi/]{fantasia-scifi}
\end{metadata}

\section{% {'class': ['formspringmeQuestion']}
\anchor[http://www.formspring.me/chlewey/q/781809485]{Aunque ambas son ficción, ¿qué diferencia dirías que hay entre Fantasía y Ciencia-Ficción? ¿O son lo mismo? ¿Qué
 ejemplos de una y otra citarías?}}
\par% p% {'class': ['formspringmeAsker']}
por @\anchor[http://www.formspring.me/apoloduvalis]{apoloduvalis}

\par% p% {'class': ['formspringmeAnswer']}
La ciencia ficción en su sentido más estricto se mantiene dentro del ámbito de lo que es científica y particularmente
 físicamente posible.  Ejemplos como Yo, Robot que explora la conciencia en las máquinas o Gattaca con la manipulación
 genética explora premisas que si bien son cuestionables o tecnológica e ingenierilmente extremas, no controvierten lo
 que sabemos de la física y la biología.

Se considera todavía ciencia ficción cuando nos damos el lujo de permitir una excepción a la ciencia conocida, tal
 como la posibilidad de desplazamientos más rápidos que la velocidad de la luz o de regresar en el tiempo.

Por su parte la fantasía puede obviar por completo el concepto de física.  Por ejemplo una historia fantástica puede
 ocurrir en un universo de tamaño continental donde los fenómenos metereológicos como las mareas, los vientos, el día y
 la noche estén determinados por la magia y no por fenómenos astrofísicos.  La Tierra Media de J. R. R. Tolkien no está
 regida por la misma física que conocemos, de lo contrario esos gigantescos elefantes morirían aplastados bajo su
 propio peso.

La aventura espacial (space opera), es un tema comúnmente catalogado como ciencia ficción, sin que necesariamente lo
 sea.  Puede tener muy poca especulación y fantasía (p. ej. adherirse plenamente a la física conocida) pero estar
 basada en el drama de los personajes, lo cual sería realmente cualquier otro género literario en un escenario de
 ciencia ficción, o puede estar llena de viajes que superen la velocidad de la luz, poderes metafísicos, criaturas
 imposibles, etc. que lo convierten en fantasía espacial.  Basta ver Star Wars para este tipo de aventura espacial que
 tiene que ver más con la fantasía que con la ciencia ficción.

Dicho eso, una de las historias que tengo en primeras etapas de borrador tiene algo de aventura espacial, con muy pocas
 especulaciones con respecto a la física y la biología que conocemos, pero el elemento de ciencia ficción es más una
 premisa.

\par% p
Saludos.

\par% p% {'class': ['formspringmeFooter']}
\anchor[http://formspring.me/chlewey?utm\_medium=social\&utm\_source=wordpress\&utm\_campaign=shareanswer]{Please send me your deepest charge}

\chapter{Carlos Eugenio a bordo}
\begin{metadata}
	Published by \anchor[chlewey]{chlewey} on \anchor[http://ewey.co/B561]{Mon, 19 Jul 2010 16:59:43 +0000}\\
	\categories{opinion}\\
	Shorthand: \anchor[http://blog.chlewey.net/2010/07/carlos-eugenio-a-bordo/]{carlos-eugenio-a-bordo}
\end{metadata}

\anchor[http://tumblr.chlewey.net/post/893660284]{\begin{wrapfigure}{r}{240\px}\centering% {'src': 'http://blog.chlewey.net/wp-content/uploads/2010/07/abordo-sq.jpg', 'style': 'margin-left: .5em;', 'title': 'abordo-sq', 'height': '240', 'width': '240', 'alt': '', 'class': ['alignright', 'size-full', 'wp-image-564']}
\includegraphics[width=240\px,height=240\px]{blog/abordo-sq.jpg}
\end{wrapfigure}
}Cuando nació mi primer hijo, entre los muchos elementos de lencería, dotación, aseo, seguridad doméstica, etc. que
 adquirimos se encontraba un pequeño letrero para colgar en el vidrio trasero del carro advirtiendo que se encontraba
 un “bebé a bordo”. No sé el grueso de los conductores qué tanta atención prestan a esos letreros pero, para lo que
 pudiere pasar, parecía apropiado.
A veces me preguntaba si debía retirar el aviso cuando mi hijo no estaba a bordo del
 vehículo, pero finalmente ahí permaneció durante la mayor parte del tiempo en que mis hijos fueron bebés pequeños.

¿Tiene algún sentido esa advertencia? Quien la mire puede decidir tener más cuidado. No es que otros conductores anden
 pensando en estrellar a cualquiera salvo que sepan que hay niños pequeños u otras posibles víctimas especiales. Pero
 la advertencia sólo funcionaría si es creíble.

Hay quienes denuncian que las ambulancias sirven de servicio de taxi o de mensajería para altas personalidades, por
 ejemplo, usando sus luces de emergencia para avanzar sobre el tráfico y no para lo que fueron hechas: transportar
 pacientes graves que requieren llegar rápido. Cuando creemos que una ambulancia no está en una emergencia real, los
 demás automovilistas nos podemos sentir tentados a no otorgarles ningún privilegio, por ejemplo a no orillarnos para
 permitir que pasen. Con el consiguiente perjuicio a los pacientes realmente graves que transporte una ambulancia seria.

En los años ochenta empezaron a ponerse de moda los letreros de “bebé a bordo” (originalmente en su versión anglo de
 “baby on board”), al poco tiempo el ingenio llevó a que proliferaran letreros de “mascota a bordo”, “recién casados a
 bordo”, “suegra a bordo”, y un largo número de et céteras a bordo. Lo que originalmente era una advertencia se
 convirtió en un chiste.

Afortunadamente ese chiste desapareció y cuando mis bebés fueron pequeños su letrero de “bebé a bordo” era poseedor de
 su completo significado.

Pero hoy vemos una proliferación de “Fulanito a bordo”, con todas las variaciones comunes de nombres. ¿Quién es ese
 Fulanito? ¿El bebé que va a bordo y al que los demás conductores debemos proteger? ¿El hijo del conductor que ya no
 clasifica como bebé? ¿El conductor mismo o su media naranja? Tal vez no sea todavía el chiste de la temporada, pero a
 la ya prácticamente inútil advertencia de “bebé a bordo”, esta nueve proliferación de ingenio la hace más inútil aún.

\chapter{¿Cuál es tu verdad?}
\begin{metadata}
	Published by \anchor[chlewey]{chlewey} on \anchor[http://ewey.co/B569]{Tue, 03 Aug 2010 04:16:08 +0000}\\
	\categories{derecha, injusticia-social, izquierda, opinion, polarizacion}\\
	Shorthand: \anchor[http://blog.chlewey.net/2010/08/cual-es-tu-verdad/]{cual-es-tu-verdad}
\end{metadata}

Se acaba ya el período presidencial de Álvaro Uribe Vélez.
Para mí, no fue ni el mejor ni el peor presidente que haya tenido Colombia ni a lo largo de su historia ni en la que yo he vivido.
Sí fue un presidente singular (lejos decir que fue un presidente como cualquier otro).
Hizo algunas cosas que tenía que hacer (aunque no creo que haya sido la única persona capaz de haberlas hecho) e hizo
 un par de cosas que, en mi opinión, no debió haber hecho.

\par% p
Llega Juan Manuel Santos. ~Una persona que viene preparándose para ser presidente \anchor[http://blog.chlewey.net/2010/05/antesdeuribe/]{desde mucho antes} que Uribe.
Lejos de haber sido mi candidato favorito, hay cosas en lo que ha hecho hasta ahora que me hacen pensar que no será tan malo: que incluso corregirá varios de los desaciertos de Uribe.
Recibo a Santos con el beneficio de la duda.
 Igual, no creo que vaya a ser tampoco el peor presidente de la historia de Colombia.

\anchor[https://twitter.com/Polexia/status/20169802156]{Me entero} que quieren construir \anchor[http://www.elnuevoherald.com/2010/08/02/776090/newt-gingrich-y-alberto-acereda.html]{una mezquita en Nueva York} muy cerca a Ground Zero.
No sería más que un acto controvertido si simplemente fuera un grupo de estadounidenses musulmanes, pero tal parece que quienes están detrás son fundamentalistas musulmanes de los que apoyan supuestas causas ilamistas afines a Al Qaeda y no símplemente estadounidenses que profesan una religión no mayoritaria.
Más que un acto controvertido parece ser un acto deliberadamenta provocador.

\par% p
Islam significa ``\textbf{sumisión}'' y es el plan de los musulmanes más radicales profetizar su verdad para que el mundo se \textbf{someta} a la voluntad de Dios de acuerdo al Corán.
Esa es su verdad y es difícil convencerlos de que conceptos como la tolerancia religiosa son más correctos.
Pero el fundamentalismo no se da sólo en el islam.
El fundamentalismo también está en los grupos cristianos que en EE.UU. quieren imponer la enseñanza del diseño
 inteligente en las escuelas públicas.

¿Si alguien cree tener la verdad, cómo lo convencemos de que hay otras versiones?
Por varios siglos la cristiana Iglesia Católica quemaba a los herejes y convertía (a la brava) a los infieles.
Muchos católicos murieron o fueron desterrados de los países donde triunfaron los protestantes.
Si bien la iglesia católica ha abrazado el ecumenismo y el diálogo con otras religiones aún se autodefine como la poseedora de la verdad, y lo propio hacen los demás cristianismos.
Los fundamentalistas cristianos, católicos o no, al igual que los fundamentalistas islámicos, insisten en imponer su
 verdad o, por lo menos, tratar con condensendencia a los pobres infieles.

Tal vez el hecho de que, hasta hace poco, Colombia fuese un país mayoritariamente católico, no nos ha hecho dividirnos o pelearnos por la religión.
No niego casos de intolerancia religiosa que los ha habido (los raelianos pueden dar fe de persecusiones sistemáticas),
 pero las diferencias entre religiones han estado lejos del conflicto general que nos aquejan.

Pero el conflicto tiene bases afines al sentir religioso.
Se trata de la concepción de la verdad.
En los años 1930, 1940 y 1950, los curas desde sus púlpitos arengaban a favor de matar liberales.
Más tarde, otros curas, abrazando la Teología de la Liberación, justificaban el alzamiento en armas para liberar al pueblo de la opresión del estado burgués.
Pero no quiero hablar de la religión de los curas: quiero hablar de La Verdad.

Los marxistas-leninistas toman como un dogma casi religioso la lucha de clases y el concepto de que el estado burgués es tan intrínsicamente malo que todo lo que hagan para destruirlo e implementar el comunismo es bueno.
La negociación entre el estado y la guerrilla sólo tiene sentido si al final el estado es reemplazado por un estado comunista.
Todo lo que haga el buen comunista para menoscabar al estado burgués vale.
Todo lo que haga cualquier otra persona para buscar un acomodo que permita mejorar las condiciones sociales de la población mientras se mantienen las estructuras del estado es, para aquellos poseedores de la verdad, una falacia y una forma de prolongar al intrínsicamente malo estado burgués.
¿No es esto una verdad religiosa?

También es una verdad religiosa partir del Libro Rojo para asumir que todas las opciones de izquierda son intrínsicamente malas porque atentan contra el Estado.
Creer que el socialdemócrata que aboga por una solución negociada con la guerrilla, es un solapado que quiere imponer
 el comunismo o, en el más amable de los casos, un idiota útil de los terroristas.

\par% p
El conflicto (\anchor[http://blog.chlewey.net/2008/02/no-hay-conflicto/]{o el no conflicto}, pero que lo hay lo hay) colombiano actual, tras estos ocho años de Uribe es un conflicto con visos religiosos.
Se trata de imponer la verdad o, en su defecto, tratar con condesendencia a quienes no la comparten.
Se trata de clamar por la sangre de los infieles.

Hoy los curas no llaman a las armas desde los púlpitos a matar liberales, o a justificar el levantamiento armado de los desposeídos.
Hoy los curas sin sotana predican desde el Palacio de Nariño o desde la oficina del Procurador General, o desde los micrófonos de Todelar, o alguien siempre en Hora 20,
las verdades que el colombiano de bien debe recitar.
Y como buenos fieles religiosos recitamos sin chistar para ver en Uribe al Mesías que salvó a Colombia del desastre.

Cuando confrontamos las evidencias científicas al dogma religioso, el fundamentalista nos dirá que la ciencia no es suficiente y que nuestro celo científico nos impide ver la verdad que va más allá de la ciencia.
Que son sólo datos, apoyados en teorías que se apoyan en teorías pero que no son capaces de explicar lo inexplicable: la verdad revelada.
Llegan, incluso, a afirmar que es el demonio el que nos impide ver la verdad más allá de la ciencia.

Hablando con muchos uribistas, he sentido esa misma línea de pensamiento.
También me ha pasado al hablar con comunistas convencidos.
Es La Verdad, como cada cual se la imagina, la que nos lleva a no querer ver ni evaluar los argumentos del otro.
Cuando se tiene La Verdad, las verdades de los demás son mentiras y el diálogo no tiene sentido si no es para imponer
 La Verdad.

\chapter{I'm not suicidal, but I need the rush}
\begin{metadata}
	Published by \anchor[chlewey]{chlewey} on \anchor[http://ewey.co/B578]{Mon, 09 Aug 2010 15:49:17 +0000}\\
	\categories{muerte, personal, proyeccion-y-carrera, suicidio, vida}\\
	Shorthand: \anchor[http://blog.chlewey.net/2010/08/need-the-rush/]{need-the-rush}
\end{metadata}

Hace tiempos no me daba el gusto de disfrutar de una buena columpiada.
 La mayoría de los columpios con los que me suelo encontrar están diseñados para niños y no para tipos de 200 libras como yo.
 Pero este parecía lo suficientemente sólido y los niños a mi alrededor no estaban interesados.

Parecía.
 Las cadenas del columpio no estaban soldadas y no podía dejar de pensar si alguno de los eslabones empezaría a ceder y a abrirse.
 No pude comprobar que eso sucediera, pero la preocupación fue suficiente para tenerla presente.
 Sobre todo porque no podía mecerme suavemente: tenía la necesidad de hacerlo con fuerza, con todo el impulso que mi
 cuerpo podría transmitirle al columpio.

Necesitaba ese impulso, esa sensación de velocidad. La necesidad de descargar la tensión acumulada en ese instante de vértigo.
 Pero el sentirme inseguro me amarraba.~ Me hizo detenerme en seco en varias ocasiones.

\anchor[http://blog.chlewey.net/2009/05/rage-o-25-cosas-sobre-mi/]{Me gusta el riesgo} y uno de los puntos donde mejor se expresa en en mi velocidad para conducir.
 La velocidad me ha traído algunos accidentes.~ En 1991 corría en mi bicicleta desde Sigtuna \anchor[http://maps.google.com/maps?f=d\&source=s\_d\&saddr=Sigtuna,+Suecia\&daddr=S\%C3\%B3dermalm,+Suecia\&hl=es\&geocode=FTiwjQMdPXIOASl1KAzQULpfRjH9gcrXcdCqZA\%3BFfkSiQMdvcUTASkh37BYwXdfRjFol\_6EAiCBfw\&mra=ls\&dirflg=w\&sll=59.47189,17.89752\&sspn=0.498771,1.783905\&ie=UTF8\&z=10]{hasta} mi casa en Södermalm y en una bajada la velocidad me ganó en una curva.
 Preferí salir del camino antes de arriesgar estrellarme con un poste y solo recuerdo caer de frente dando un bote y con la bicicleta pasando por encima.
 Salí ileso y la bicicleta sólo sufrió en su rueda delantera.

\par% p
A veces me pregunto que pasaría si de nuevo mi vehículo me falla, si no soy capaz de controlar una curva por ir más rápido de lo que puedo controlar.
 Vivo entre esa prevención y la necesidad de correr.
 Y muchas veces no mido.
 A veces me sorprendo cuando tomo decisiones extremas sin mayor tiempo para pensar, meterme entre dos carros que también van rápido.
 Hace unos cinco años, conducía de Bogotá a Melgar y adelante de \emph{La vaca que ríe} había aceite en la vía.
 El carro resbaló, perdí tracción.
 Venía un auto de frente y el mío estaba invadiendo la calzada contraria.
 En instantes tuve los reflejos suficientes para acelerar suavemente, recuperar la tracción y regresar a mi carril con
 un estrecho pero suficiente margen para evitar la colisión.

Años atrás no tuve esos reflejos.
 Regresaba de Chinauta y pasando cerca del Muña perdí tracción, no recuerdo por qué, y mi primera reacción fue frenar y cabretillar con toda para no montarme sobre el separador.
 Sin suerte, una vez el carro recuperó tracción la maniobra extrema le hizo dar una voltereta.
 Me sentí como en una atracción de una ciudad mecánica, amarrado por el arnés mientras veía cómo el mundo giraba a mi
 alrededor hasta quedar patas arriba.

Mi única lesión, al soltar mi cinturón apoyé mi mano sobre el techo del vehículo y una fracción de vidrio se me incrustó en el dedo.
 Mi novia (hoy mi esposa) sufrió un poco más pero salió igualmente sin mayores problemas.
 Una amiga que iba en el asiento trasero y sin cinturón quedó sentada sobre el techo del carro, sana y salva.
 Dos años más tarde murió por intoxicación con monóxido de carbono.

La vida y la muerte.
 La muerte puede venir en cualquier momento, aun por más responsable que uno sea al conducir.
 Puede venir porque esperas que el agua salga más caliente, por un aneurisma silencioso, por un accidente de tránsito o
 porque te quitas la vida.

Conscientemente yo no me quitaría la vida.
 Me gusta sentir la adrenalina, pero lejos de mí está actuar pensando en que me quiero matar.
 Quiero vivir.
 No sé por qué.
 Apego enfermizo al chisme, tal vez, pero quiero vivir lo suficiente como para saber qué pasa.
 Hasta ahora ninguna consecuencia en vida me ha parecido lo suficientemente indeseable como para preferir no
 afrontarla, y eso que soy un cobarde para afrontar consecuencias.

A veces, mezquinamente, fantaseo con mi muerte.
 Una muerte que haga recapacitar a mi familia o al obscuro objeto del mal trato que me han dado o de la forma incorrecta como pretendieron actuar de buena fe.
 Pero no sólo es mezquino.
 Es estúpido.
 Tras el hecho no habrá forma de evaluar las lecciones aprendidas.
 Si hay un motivo para dar la vida, definitivamente no es ese.

Y es que no es lo mismo dar la vida, quitarse la vida o arriesgar la vida.
 El primero es un acto altruista.
 Consiste en permitir tu muerte a cambio de cierta garantía de un bien mayor para otros (tu familia, tu patria, tu dios, lo que quiera que para ti signifique más que tú mismo).
 Podríamos pensar que dar la vida tal vez no sea un acto inteligente y totalmente desinteresado, pero quitarse la vida
 para castigar a otros es definitivamente un acto de soberana estupidez.

Y tampoco soy tan estúpido, lo lamento.

Yo arriesgo mi vida, no porque mi vida no valga, sino porque necesito descargar las tensiones de mi vida en descargas de adrenalina.
 Tal vez también sea estúpido.
 Pero cuando no soy capaz de tomar verdaderos riesgos en mi modelo de vida, sólo me queda el vértigo físico y primario:
 conducir rápido, columpiarme con fuerza, tratar de reventar una pelota de tenis contra un muro: destruir.

Soy muy cobarde para arriesgar muchas cosas y ante eso termino arriesgando otras.

Desde enero llevo twitteando de cuando en vez que quiero irme a Haití a trabajar.
 No lo he hecho.
 No he sido capaz de tomar decisiones realmente radicales como irme de la seguridad de mi hogar y buscar nuevos rumbos, bien se colaborando en la reconstrucción de una zona de desastre, o meseriando en Nueva York.
 Me freno en ese tipo de decisiones de vida.
 A duras penas logré tomarme un café con el obscuro objeto y confesarle, entre dientes, mi alma, y aún no me decido a
 arriesgarlo todo por componer ese aspecto de mi vida.

No me atrevo a arriesgar mi vida en el sentido trascendental de la misma, por eso me la paso arriesgando mi vida en el sentido más físico y prosaico del término.
 Y veo como mi vida se pasa, extinguiéndose, pero aferrándome primitivamente a ella.

Esperando ver cuando las circunstancias terminarán por acabarla.

\chapter{De quereres y gustos}
\begin{metadata}
	Published by \anchor[chlewey]{chlewey} on \anchor[http://ewey.co/B582]{Fri, 20 Aug 2010 09:38:06 +0000}\\
	\categories{amores, deseo, familia, personal}\\
	Shorthand: \anchor[http://blog.chlewey.net/2010/08/de-quereres-y-gustos/]{de-quereres-y-gustos}
\end{metadata}

“Ya sé que no me amas” me ha dicho últimamente mi esposa en distintas oportunidades y con algunas variantes.
 Hace también ya un tiempo que no le digo “te amo”.
 Me pregunto muchas veces que exactamente es lo que siento hacia ella.

Cundo no estamos bien, siento algo de rabia, de resentimiento.
 La culpo por no entender.
 La culpo por no entenderme.
 Porque no se da cuenta que por bien intencionada que crea estar, puede estar haciendo más mal que bien con su proceder
 y sus intentos de corregirme.

Pero aún cuando no estamos bien ella me importa.
 No quiero que se queje de no poder descansar, sino que quiero que descanse.
 Que esté cómoda.
 Que no sufra ni se estrese.
 Que sus cosas en el trabajo le salgan bien.
 Que tome las fotos que quiere tomar.~ Que pase un tiempo con sus hijos.~~ Que pueda divertirse con y sin mí.

\par% p
Y me sigue gustando.~ Me sigue provocando (en el sentido colombiano del término: \anchor[http://www.wordreference.com/definicion/provocar]{provocar} =~\anchor[http://www.wordreference.com/definicion/apetecer]{apetecer}).

No sé si aún la amo.
 No sé si alguna vez la amé.
 Cuando estábamos de novios tardé en decirle que la amaba porque me parecía que era una palabra demasiado grande y yo
 no sabía todavía todo lo que significaba y si lo que sentía hacia ella era eso tan grande.

La pregunta va más allá.~ ¿Amo a mis padres? ¿a mi hermana?

Como padre que soy ahora, estoy seguro de tener algo hacia alguien que puedo llamar amor.
 Amo a mis hijos.
 De eso estoy seguro, aun cuando no sea capaz de expresarlo en acciones.
 Sé que ese es un vínculo que no depende de si se portan bien o mal, que no dependen de mi estado de ánimo, que no
 depende de aquello que nos pueda separar.

\par% p
Lo demás, no lo sé.~ No sé si sea otro tipo de amor, u otro grado de amor.~~ El Gerente nos dice que el \anchor[http://www.blogdelgerente.com/?p=1518]{amor no existe}, que no es más que una decisión.~ Aunque confiesa que sí existe el amor cuando habla de \anchor[http://www.formspring.me/blogdelgerente/q/752362728]{su Tigresa}.
 A veces me he preguntado qué pasaría si mi padre faltara y no sé, no puedo predecir o imaginar siquiera si eso me irá a doler.
 Como si mi corazón se negara a sentir, amor o cualquier otra cosa.

La pregunta no es un deseo.
 La pregunta es tratar de buscar una respuesta.
 En estas épocas donde la homosexualidad es cada vez más aceptada, me he planteado la pregunta de si yo fuere homo o bisexual.
 Y porque me la he planteado y he hecho el ejercicio mental es que llego a la conclusión de que soy completamente
 heterosexual.

\par% p
Hay dos preguntas que me plantea esta situación.~ ¿Qué exactamente es un \anchor[http://www.urbandictionary.com/define.php?term=Mancrush]{\emph{mancrush}}? Y ¿Por qué es tan aceptable ver a mujeres alabando la belleza de otras mujeres mas no así los hombres?
 Alguna vez creí responder la pregunta de que lo que un hombre heterosexual ve atractivo en una mujer parte de su deseo de estar sexualmente con ella, mientras que lo que el mismo hombre heterosexual ve atractivo en otro hombre es el deseo de ser como él.
 Algo análogo sucedería con las mujeres heterosexuales: desean al papacito y envidian a la mamacita.

Pero descubro, con base en mis preguntas, que hay atracciones que no son sexuales.
 Hay mujeres que me gustan más allá de que desee acostarme con ellas y hay varones que me atraen sin que quiera ser
 como ellos y sin que espere intimidad sexual alguna con ellos.

¿Qué me gusta y qué no me gusta de una persona?~ ¿Qué es aceptable que me guste?

Hay un tema de tabú interesante de analizar.
 Mi mente se bloquea automáticamente con la sola pregunta de si me gusta alguien de mi familia consanguínea inmediata.
 La pregunta no es siquiera imaginable.
 Pero no tengo problema alguno a partir del cuarto grado de consanguineidad.
 Dos de mis eternas tragas son primas mías.~ El tabú también va hacia mujeres demasiado viejas y demasiado jóvenes.

Lo de demasiado viejas no sé si coexista ahí una cuestión más de asco que de tabú.

Lo de demasiado jóvenes si hay una varias franjas de tabú que se han venido ajustando con los años.

Hay muchas niñas prepúberes que me parecen muy lindas, lo confieso.
 Pero son eso: físicamente lindas, bonitas.
 Definitivamente no son atractivas.
 Mi fascinación hacia ellas no es ni remotamente sexual.
 Afirmo con certeza que no es genital.
 Plantearme la pregunta genera casi el mismo rechazo que plantearme la pregunta con mi familia consanguínea inmediata.

A medida que llegan a cierta edad o que pasan cierta etapa de su desarrollo ya puedo considerarlas atractivas.
 Pero aquí empieza a surgir un tabú más social que me aleja de considerar a una adolescente como una posible compañera sexual.
 Las puedo considerar atractivas más no deseables.

Ahora, hay adolescentes físicamente desarrolladas que sí podría considerar deseables, pero el sólo hecho de considerar
 su edad les resta puntos en mi escala de deseos.

Así que excluyendo varones y mujeres demasiado jóvenes, demasiado viejas o demasiado consanguíneas ¿Qué queda?
 Mujeres, muchas mujeres.
 Algunas mujeres que me atraen más que otras.
 Mujeres feas y mujeres hermosas.~ Mujeres que me atraen, otras que no inspiran y algunas que dan asco.

\par% p
Pero el grupo de mujeres que me atraen \anchor[http://tumblr.chlewey.net/tagged/tjejer]{es un grupo} muy grande y muy heterogéneo.
 Hay mujeres de veinte y algunas que bordean los 60.
 Hay modelos que sólo he visto en revistas o televisión y hay muchas mujeres cercanas a mi vida cotidiana.
 Las veo en la calle, o visitando a mi hermana, en una conferencia a la que asisto o un taller que dicto.

Una cosa que he notado es que la cercanía de una mujer es un gran atractivo.
 Escucharla, ver cómo se mueve, sentirla cerca, son elementos que me enamoran.

Y estar casado no ha impedido que me siga enamorando.
 Permanentemente me estoy enamorando de una u otra mujer que conozco.
 De unas más que de otras.~ Pero he seguido enamorándome.

Pero hablo de enamorarme en el sentido más relajado del término: sentir atracción hacia una mujer e imaginar su piel contra mi piel, sus labios con los míos y compartir nuestro sexo.
 No implica sentimientos profundos, ni implica tampoco un deseo incontrolable de tener sexo, de consumarlo.
 Puedo, normalmente, vivir con la sola imaginación o el sentir que lo puedo imaginar así ni siquiera lo imagine de
 verdad.

Y en ese sentido, nunca busqué que pasara nada por fuera de mi matrimonio.
 Soñaba con posibles ocasiones, pero no me permití convertirlas en obsesiones.

Porque obsesiones tuve antes de encontrarme con quien se convertiría en mi esposa.
 En cierta forma ella fue una obsesión que reemplazó a otra que estaba entonces vigente.

El matrimonio puede no ser más que una decisión y una costumbre, pero las nuevas obsesiones no tenían sentido así que
 no las hubo.

No las hubo.~ Tiempo pretérito indefinido.

\chapter{Me gustaría…}
\begin{metadata}
	Published by \anchor[chlewey]{chlewey} on \anchor[http://ewey.co/B586]{Thu, 26 Aug 2010 15:31:24 +0000}\\
	\categories{futuro, personal, proyeccion-y-carrera, vocacion}\\
	Shorthand: \anchor[http://blog.chlewey.net/2010/08/me-gustaria/]{me-gustaria}
\end{metadata}

Me gustaría escribir algo más serio que este blog.  Escribir de ficción, de filosofía, de cómo ver el mundo con otros
 ojos, de cómo interpretar la realidad por nosotros mismos.  Me gustaría escribir y publicar y saber que lo que escribo
 importa y ayuda a alguien.

Me gustaría pararme frente a un auditorio y ver los ojos atentos de quienes están sentados, y poder inspirar.  Me
 gustaría ayudar a esas mentes a formarse por sí mismas y que mi papel sea más el de un guía que muestra nuevos caminos
 y alternativas para que ellos puedan decidir seguirlos o no.

Me gustaría entablar conversaciones que desafíen mi intelecto.  Que me hagan pensar.  Que me inviten a resolver
 problemas que no había pensado antes.  A replantear mis conceptos.  A seguir creciendo.

Me gustaría encontrar mi camino, mi vocación.
 Sé que no es en los números o los electrones, pero sí con ellos.
 Es en la gente.
 En las personas.
 En interactuar.
 Y es ahí donde siento el peso de mi gran error de todos estos años de justificar en mi timidez buscar una carrera en
 las máquinas y no con la gente.

Me gustaría explorar mi mente hacia sus límites, y que mi mente, mis ideas, sirvan a los demás.

\chapter{Aunque nos llamen piratas}
\begin{metadata}
	Published by \anchor[chlewey]{chlewey} on \anchor[http://ewey.co/B593]{Fri, 03 Sep 2010 23:10:16 +0000}\\
	\categories{activismo, derechos-de-autor, derechos-intelectuales, information, partido-pirata, patentes}\\
	Shorthand: \anchor[http://blog.chlewey.net/2010/09/aunque-nos-llamen-piratas/]{aunque-nos-llamen-piratas}
\end{metadata}

\par% p
A continuación va el borrador del manifiesto del Partido Pirata de Colombia:

\begin{blockquote}
Reconocemos que Colombia es un estado social de derecho, constituido como una democracia representativa y regido por
 una constitución garantista de origen popular.  Defendemos la Constitución y los derechos y libertades civiles que de
 ella emanan.  Defendemos firmemente el derecho a la libertad de expresión y a la libertad de asociación.
\end{blockquote}

Se reconocen los derechos constitucionales y la constitución en sí misma.
 El Partido Pirata no desconoce ni desafía al estado colombiano ni a su constitución.
\begin{blockquote}
Consideramos que dentro de una democracia la deliberación con argumentos es la vía idónea para expresar nuevas ideas y
 lograr cambios políticos.  Rechazamos los métodos violentos para imponer nuevas ideas políticas.  Rechazamos así las
 vías de hecho tales como el sabotaje, las restricciones a la libertad de locomoción como forma de protesta y al
 alzamiento en armas.
\end{blockquote}

El Partido Pirata defiende las vías de derecho considerando que la Constitución Política de Colombia garantizan el
 libre debate, y nos oponemos por una parte a la lucha armada subversiva, pero también a otras vías de hecho como los
 ataques a sitios web y otras formas de ciberterrorismo, a los bloqueos de vías y a manifestaciones que atenten contra
 la libertad de otros ciudadanos así estos no compartan nuestras ideas.

\par% p
Siempre se discutió si como partido debemos oponernos explícitamente a la lucha armada subversiva, como en el sentido
 de que eso sobre y se cae de su peso, pero este es un eje común en los manifiestos de otros partidos piratas y se
 requeriría de una buena razón de fondo para excluir este punto de nuestra propia manifestación.

\begin{blockquote}
Defendemos y promovemos el derecho a la privacidad. Defendemos el derecho a la igualdad ante la ley y a que nadie sea
 discriminado por su lugar de nacimiento, su raza, su sexo, su religión, su afiliación política, su opinión o por otras
 condiciones sociales o circunstancias personales.
\end{blockquote}

Hasta el momento nada en nuestra declaración de principios que riña con nuestra constitución ¿cierto?~ Pues eso.
\begin{blockquote}
Defendemos el derecho a la vida y a la integridad física y moral: ninguna persona puede ser sometida a tratamientos o
 castigos degradantes o inhumanos o a ningún tipo de tortura.
\end{blockquote}

Otra cosa que sobraría decirla, pues ya está en nuestra constitución.~ Eso somos.
\begin{blockquote}
Creemos que las leyes, incluyendo leyes penales y aquellas enmarcadas en la lucha contra la sedición y el terrorismo,
 deben estar enmarcadas bajo los derechos y libertades civiles reconocidos por la Constitución de Colombia y defendidos
 por el Partido Pirata.
\end{blockquote}

Y sobra decirlo ¿cierto?

Obvio.

Pero llevamos tantos años creyendo que todo vale cuando se trata de combatir a la subversión apátrida que nos olvidamos
 que los guerrilleros también son sujetos a la Constitución de Colombia y a sus garantías.

Pero no debemos ir tan lejos.

Todos los días dejamos que nos vulneren nuestros derechos constitucionales cuando consentimos una requisa o
 identificarnos con nuestra cédula o dejar nuestras huellas dactilares en cualquier formulario trivial.

\par% p
El terrorismo, que rechazamos como Partido Pirata, nos ha hecho aceptar como algo muy normal el antiterrorismo.
 Cedemos a nuestros miedos de buscar que alguien nos proteja así sea a cambio de nuestros derechos básicos tales como el derecho a nuestra intimidad.
 Este miedo es muy difícil de vencer, pero es nuestra obligación vencerlo y superarlo y que nuestras vidas no dependan
 de seguir renunciando a nuestras libertades individuales a cambio de seguir viviendo bajo el miedo.

\begin{blockquote}
Buscamos la transparencia del estado y exigimos que los agentes del gobierno y los representantes ante cuerpos
 colegiados sean responsables política y legalmente por sus acciones y omisiones. Requerimos la transparencia sobre los
 actos públicos que garanticen esta responsabilidad.
\end{blockquote}

Y sí.~ Exigimos y requerimos.~ El control político es un derecho ciudadano y facilitarlo es un deber del estado.
\begin{blockquote}
Manifestamos que nuestra causa política se centra en promover el conocimiento y el disfrute de los descubrimientos y
 creaciones de la humanidad y que consideramos que la humanidad y la democracia se benefician del flujo irrestricto del
 conocimiento que permita seguir creando y construyendo.  Manifestamos que en ningún momento nuestro ideario busca
 limitar los beneficios que reciben los creadores aunque reconocemos que los esquemas de negocios basados en las
 restricciones actuales pueden verse afectados por nuestras creencias.
\end{blockquote}

Aquí sí comienzan los verdaderos puntos diferenciadores del Partido Pirata frente a otros movimientos políticos y organizaciones sociales.
 El conocimiento y la cultura deben ser tan libres como esto nos permita crecer.

Hoy existen muchos negocios legales basados en las restricciones a la libertad de compartir el conocimiento.
 Sin duda un triunfo de las tesis del Partido Pirata entran en conflicto con varios de estos esquemas de negocio.

\par% p
Pero también hay negocios y oportunidades cuando estas restricciones se levanten.
 Y creemos que serán más las oportunidades que surgen de compartir que las oportunidades que surgen de restringir.

\begin{blockquote}
Rechazamos el concepto de ``propiedad intelectual'' por ser un término impreciso que abarca conceptos distintos y se
 presta para abusos en contra de los derechos y libertades civiles.  Las patentes, derechos de autor y registros de
 marca son conceptos dispares que deben tener un tratamiento diferencial en la ley.
\end{blockquote}

El término ``propiedad intelectual'' incluye al menos tres conceptos que son diferentes en sí mismos y por lo tanto da lugar a transferir, a capricho del lobbista de turno, las protecciones restrictivas de un concepto a los otros.
 Una patente es un beneficio que se otorga a una persona natural o jurídica para restringir la comercialización de una tecnología si esta no beneficia al dueño de la patente, y el dueño de la patente puede ser o no el creador de una nueva tecnología.
 El derecho de autor es el derecho que tiene un creador sobre su obra para protegerla o usufructuarse y no puede ser cedido, pues el autor es el autor, no quien compre sus derechos.
 Una marca, si bien puede implicar un acto creativo, se protege no por su valor artístico o ``intelectual'' sino por su
 representación de un producto u organización.
\begin{blockquote}
Manifestamos que el sistema de patentes, si bien fue creado para fomentar la investigación y el desarrollo proveyendo
 derechos y ventajas para los inventores, se ha convertido en una forma de control monopolista y que este control,
 antes que fomentar, restringe la investigación y desarrollo por terceros.  Consideramos que los términos de las
 patentes deben reducirse y nos oponemos a que se patenten ideas o descubrimientos que en sí mismos no sean creaciones
 materiales producto de la invención.
\end{blockquote}

Dos puntos aquí.~ El alcance y los términos de las patentes y lo que puede ser patentado.

Crear, investigar, descubrir y desarrollar cuesta.
 En un esquema capitalista —y somos capitalistas o por lo menos no somos anticapitalistas— muchas veces el creador espera cubrir sus costos de investigación y desarrollo comercializando el producto comercializado.
 Hasta ahí bien.

Pero luego viene otra persona y copia nuestro producto, sin haber invertido en investigación y desarrollo y, por ello, produciendo un producto más barato.
 Eso es injusto.
 Y por ello existen las patentes.
 La patente permite proteger mi investigación y desarrollo obligando a que ese otro productor me reconozca un pago por
 la misma.

Y todo bien.

El problema es que así como protege a un investigador, también puede ser usado para abusos y prácticas restrictivas.
 Una empresa puede bien comprar la patente de una tecnología innovadora que compite con la propia, no para desarrollar
 esa tecnología sino para evitar que otros la usen y mantener así un monopolio con su tecnología obsoleta.

El otro punto es la patente sobre ideas u otros conceptos que no implican productos tangibles.
 Como autor de ficción podría imaginarme una tecnología aun no inventada pero factible en principio y patentarla sin mayores estudios de cómo hacerla realmente posible.
 Eventualmente si alguien —inspirado o no por mi obra de ficción— investiga y consigue hacer práctica esa tecnología
 tendrá que retribuirme económicamente de acuerdo a los términos de mi patente.

\par% p
La patente, cuando se extiende a ideas y no sólo a desarrollos concretos, antes que alentar a la investigación la
 penaliza.

\begin{blockquote}
Consideramos que las biopatentes (patentes sobre descubrimientos biológicos tales como semillas, especies, seres vivos,
 sus partes y genes, incluyendo modificaciones inducidas) son nocivas como concepto, pues privatizan la investigación y
 el acceso al conocimiento; de tal forma las biopatentes deben ser abolidas. Rechazamos las patentes a algoritmos o
 piezas de software; la protección al software puede ser garantizada por los derechos de autor.
\end{blockquote}

Lo de las biopatentes se explica por sí mismo.
 Y, de nuevo, los derechos de autor y las patentes cubren conceptos distintos.
 El software es un proceso creativo que puede ser amparado por derechos de autor y no requiere de ser protegido por
 patentes.
\begin{blockquote}
Creemos que las patentes sobre productos farmacéuticos no deben limitar la distribución de medicamentos o tratamientos
 a grupos sociales ni a poblaciones vulnerables, y mucho menos en casos de emergencias naturales, epidemias o pandemias.
\end{blockquote}

Pues eso.
\begin{blockquote}
Planteamos como uno de los objetivos del Partido Pirata suprimir los monopolios basados en patentes.
\end{blockquote}

Buscamos que las patentes protejan la investigación y el desarrollo, no que restrinjan el acceso de la investigación y
 el desarrollo a la comunidad y a nuevos investigadores y desarrolladores.
\begin{blockquote}
Consideramos que el registro de marcas no corresponden a una actividad creativa y como tal las marcas no deben estar
 protegidas por derechos de autor.  La legislación de protección de marcas debe ser suficiente para proteger las marcas
 frente al fraude, la confusión deliberada, la injuria o el mal uso.
\end{blockquote}

Si bien algunos logos y distintivos pueden provenir de un proceso creativo, las marcas en sí mismas, pero también sus
 manifestaciones gráficas, son ya protegidas de forma adecuada para lo que las marcas han sido hechas.
\begin{blockquote}
Manifestamos que promovemos la cultura y consideramos que un incremento en las formas de compartir la cultura es
 positivo para la sociedad en conjunto. Compartir la cultura no debe ser restringida por medio de gravámenes.  Buscamos
 una reducción de los términos de los derechos de autor.  Consideramos que debe buscarse un balance entre el derecho de
 los creadores de beneficiarse de sus obras y el derecho del público de disfrutar de las mismas.  Objetamos la gestión
 de derechos digitales (DRM).  Propugnamos por la libertad de la distribución no comercial de obras culturales.
\end{blockquote}

Esto es lo que nos concede el título de piratas por la forma como se ha querido demonizar por parte de la industria a
 las actividades que ponen en riesgo sus monopolios.

\par% p
No nos oponemos
 a que los creadores puedan obtener un beneficio por su creación mientras esta no limite la posibilidad de que su obra
 sea difundida.

\begin{blockquote}
Buscamos la universalización del acceso a Internet con el objetivo de eliminar la brecha digital.  Consideramos que
 debe preservarse la neutralidad de la red, esto es que no existan aplicaciones, contenidos o tecnologías que sean
 restringidas por los proveedores de servicio o por el estado, más allá de garantizar el acceso de otros usuarios.  La
 neutralidad tecnológica debe garantizarse en oficinas públicas y organismos del estado mediante el uso obligatorio de
 estándares abiertos.
\end{blockquote}

Y sí, todos los partidos dicen querer la universalización del acceso a Internet y eliminar la brecha digital.

Lo que no todos proponen es que la red sea igualmente accesible para todos independiente de su aplicación y mantener
 los estándares que garanticen que las aplicaciones sean interoperables entre sí y ver esto tanto como una forma de
 empoderar a la ciudadanía, sino como una forma de preservar la transparencia del oficio público.

\chapter{Follow Friday – \#FF}
\begin{metadata}
	Published by \anchor[chlewey]{chlewey} on \anchor[http://ewey.co/B602]{Fri, 10 Sep 2010 22:04:52 +0000}\\
	\categories{follow-friday, opinion, twitter}\\
	Shorthand: \anchor[http://blog.chlewey.net/2010/09/follow-friday-ff/]{follow-friday-ff}
\end{metadata}

\par% p
No recuerdo cuándo es que comenzó en \anchor[http://twitter.com/]{Twitter} la moda de los \emph{Follow Friday}, pero si mal no entendí la idea original consistía en que cada viernes tú, como usuario de Twitter, escogías a un par
 de personas nuevas para seguir y lo anunciabas en tu timeline con la etiqueta \anchor[http://twitter.com/search?q=\%22Follow\%20Friday\%22]{Follow Friday}, \anchor[http://twitter.com/search?q=\%23FollowFriday]{\#FollowFriday} o \anchor[http://twitter.com/search?q=\%23FF]{\#FF}.

Muy pronto (ese mismo día, incluso) evolucionó como una forma de recomendar.  Ya no significaba ``a quién voy a seguir
 este viernes'' sino ``a quien recomiendo seguir este viernes''; o, para muchas personas, simplemente ``a quién voy a
 saludar este viernes''.

\par% p
Partiendo del principio del \emph{Follow Friday} como una recomendación, y sólo como un consejo para que efectivamente lo sea, presento los siguientes consejos.

\begin{enumerate}

\item \textbf{No hagas listados de créditos de cine.}
\begin{blockquote}
\textbf{fulanito} \relax{% {'style': 'color: #0000ff; text-decoration: underline;'}
\#FF} a \relax{% {'style': 'color: #0000ff; text-decoration: underline;'}
@menganito} \relax{% {'style': 'color: #0000ff; text-decoration: underline;'}
@zutanito} \relax{% {'style': 'color: #0000ff; text-decoration: underline;'}
@perengano} \relax{% {'style': 'color: #0000ff; text-decoration: underline;'}
@Alice} \relax{% {'style': 'color: #0000ff; text-decoration: underline;'}
@Bob} \relax{% {'style': 'color: #0000ff; text-decoration: underline;'}
@Carol} \relax{% {'style': 'color: #0000ff; text-decoration: underline;'}
@Dave} \relax{% {'style': 'color: #0000ff; text-decoration: underline;'}
@Eve} \relax{% {'style': 'color: #0000ff; text-decoration: underline;'}
@Fran} \relax{% {'style': 'color: #0000ff; text-decoration: underline;'}
@Gordon} \relax{% {'style': 'color: #0000ff; text-decoration: underline;'}
@Isaac} \relax{% {'style': 'color: #0000ff; text-decoration: underline;'}
@justin}
\end{blockquote}

Al igual que los créditos de cine, nadie lee esos listados, sólo el que trabajó en la película y se detiene a ver su
 propio nombre.  Si el Follow Friday es un saludo para ti, pues no hay problema: la persona saludada verá tu \emph{tweet} en sus \anchor[http://twitter.com/replies]{menciones}.  Pero si lo que quieres es hacer una recomendación, ese tweet será simplemente ignorado por tus seguidores y no
 cumplirá su propósito.
\item \textbf{No inicies con una \emph{arroba}}
\begin{blockquote}
\textbf{fulanito} \relax{% {'style': 'color: #0000ff; text-decoration: underline;'}
@sutano} \relax{% {'style': 'color: #0000ff; text-decoration: underline;'}
\#FF} por que es muy interesante.
\end{blockquote}

Para entender por qué esto no funciona revisemos el siguiente diagrama:\anchor[http://blog.chlewey.net/wp-content/uploads/2010/09/venn.png]{\includegraphics[width=384\px,height=192\px]{blog/venn.png}}
Si \textbf{fulanito} realmente está recomendando seguir a \textbf{sutano}, entonces su audiencia es la parte azul a la izquierda del gráfico, pero este tweet sólo será visto por los usuarios
 en la región morada al centro, es decir que sólo lo verán quienes ya siguen a \textbf{sutano}.  Así que realmente no estás recomendando a nadie.
\item Cuando agradezcas un Follow Friday devolviendo la recomendación, no lo hagas en un @reply.  Pues tendrá el mismo efecto
 que lo explicado en el punto anterior.  Desde luego que si sólo estás agradeciendo el \#FF puedes usar el medio que
 quieras, incluso un mensaje directo o una llamada por teléfono.
\item \textbf{No hagas Follow Friday genéricos.}
\begin{blockquote}
\textbf{fulanito} Follow Friday a todos mis seguidores.
\end{blockquote}

Así como nadie lee los créditos de cine, tampoco nadie se va a poner a entrar a ver a todos tus seguidores para saber
 quién recomiendas.  Es más, este tipo de follow friday tampoco sirve como saludo ya que el que no leyó tu tweet en su
 momento no se enterará que lo saludaste.
\item \textbf{No hagas Follow Friday en la madrugada del viernes.}A esas horas son muy pocos los que van a leer tu recomendación.  Deberías esperar a una hora de mayor audiencia.

\end{enumerate}

Finalmente recuerda, tú usas tu Twitter como quieras.  Si quieres usar tu Follow Friday como un simple saludo has como
 quieras.  Si lo quieres usar como recomendación asegúrate de que tus recomendados lleguen a la mayor audiencia posible.

\par% p
Y yo que hoy fui culpable de incumplir la regla 5, repito aquí a mis follow friday del 10 de septiembre de 2010:

\begin{enumerate}% {'id': 'timeline'}

\item
\begin{blockquote}
Mi primer Follow Friday del dí es a todos los blogueros que participan en el \anchor[http://twitter.com/search?q=\%23bloggersecreto]{\#bloggersecreto}.  Este fue mi primer \anchor[http://twitter.com/search?q=\%23FF]{\#FF} en bloque.
\end{blockquote}

\item
\begin{blockquote}
Y no recomendaré individualmente a ningún \anchor[http://twitter.com/search?q=\%23bloggersecreto]{\#bloggersecreto} para evitar suspicacias. \anchor[http://twitter.com/search?q=\%23FF]{\#FF}
\end{blockquote}

\item
\begin{blockquote}
Así que Follow Friday a @\anchor[http://twitter.com/rosacris]{rosacris} por dejarse hacer llave de un gato. \anchor[http://twitter.com/search?q=\%23FF]{\#FF}
\end{blockquote}

\item
\begin{blockquote}
También Follow Friday a @\anchor[http://twitter.com/viviangilro]{viviangilro} porque su \anchor[http://twitter.com/search?q=\%23BB]{\#BB} le impide comunicarnos. \anchor[http://twitter.com/search?q=\%23FF]{\#FF}
\end{blockquote}

\item
\begin{blockquote}
Un Follow Friday a @\anchor[http://twitter.com/tifis]{tifis} por ser una gran mujer y por el apoyo que me ha dado. \anchor[http://twitter.com/search?q=\%23FF]{\#FF}
\end{blockquote}

\item
\begin{blockquote}
Follow Friday a @\anchor[http://twitter.com/wambaly]{wambaly} porque nos invita a enviar muchos \anchor[http://twitter.com/search?q=\%23FF]{\#FF} hoy.
\end{blockquote}

\item
\begin{blockquote}
Follow Friday a @\anchor[http://twitter.com/ApoloDuvalis]{ApoloDuvalis} por ser un bacán y leer de mi formcosito incluso lo largo que él no pregunta. \anchor[http://twitter.com/search?q=\%23FF]{\#FF}
\end{blockquote}

\item
\begin{blockquote}
Follow Friday a @\anchor[http://twitter.com/elpalabrista]{elpalabrista} por sus palabras, sus lecturas, y su foto diaria de la National Geographic. \anchor[http://twitter.com/search?q=\%23FF]{\#FF}
\end{blockquote}

\item
\begin{blockquote}
Follow Friday a @\anchor[http://twitter.com/AForqueray]{AForqueray} por ser una gran persona y muy valiosa. \anchor[http://twitter.com/search?q=\%23FF]{\#FF}
\end{blockquote}

\item
\begin{blockquote}
Follow Friday a @\anchor[http://twitter.com/jastridb]{jastridb} por ser una de las muchas personas que se ha tomado el trabajo de aterrizarme. \anchor[http://twitter.com/search?q=\%23FF]{\#FF}
\end{blockquote}

\item
\begin{blockquote}
Follow Friday a @\anchor[http://twitter.com/\_BBra\_]{\_BBra\_} por sus alkaseltzer diarios. \anchor[http://twitter.com/search?q=\%23FF]{\#FF}
\end{blockquote}

\item
\begin{blockquote}
Follow Friday al poscolombiano \anchor[http://twitter.com/search?q=\%23juglardelzipa]{\#juglardelzipa}, precisamente por eso, por acercarnos a la poscolombianidad. \anchor[http://twitter.com/search?q=\%23FF]{\#FF}
\end{blockquote}

\item
\begin{blockquote}
Follow Friday a @\anchor[http://twitter.com/taymarquartdt\_]{taymarquartdt\_} porque también exige papas en fosforito con su crema de ahuyama. \anchor[http://twitter.com/search?q=\%23FF]{\#FF}
\end{blockquote}

\item
\begin{blockquote}
Follow Friday a @\anchor[http://twitter.com/anlugonz]{anlugonz} porque, aunque anda perdida, nos sigue escuchando desde Londres. \anchor[http://twitter.com/search?q=\%23FF]{\#FF}
\end{blockquote}

\item
\begin{blockquote}
Follow Friday a @\anchor[http://twitter.com/AcidaTata]{AcidaTata} quien nos viene recordando la música de los 90 mientras se la baila en streaming en la @\anchor[http://twitter.com/superestacion]{superestacion}.fm. \anchor[http://twitter.com/search?q=\%23FF]{\#FF}
\end{blockquote}

\item
\begin{blockquote}
Y un Follow Friday muy especial a @\anchor[http://twitter.com/ToryCarolita]{ToryCarolita} quien se preocupó de que yo no tomara a bien un regaño. \anchor[http://twitter.com/search?q=\%23FF]{\#FF}
\end{blockquote}

\item
\begin{blockquote}
[Por cierto, espero me disculpen por si haber hecho un \anchor[http://twitter.com/search?q=\%23FF]{\#FF} individual a un \anchor[http://twitter.com/search?q=\%23bloggersecreto]{\#bloggersecreto}.  [Se me escapó]]
\end{blockquote}

\end{enumerate}

\chapter{Mis creencias}
\begin{metadata}
	Published by \anchor[chlewey]{chlewey} on \anchor[http://ewey.co/B605]{Thu, 09 Sep 2010 12:39:59 +0000}\\
	\categories{agnosticismo, creencia, formspring, formspring-me, personal, religion}\\
	Shorthand: \anchor[http://blog.chlewey.net/2010/09/mis-creencias/]{mis-creencias}
\end{metadata}

\section{% {'class': ['formspringmeQuestion']}
\anchor[http://www.formspring.me/chlewey/q/1093901446]{Como defines tus creencias? Crees que hay diferencia entre tus creencias y las de una religion organizada?}}
\par% p% {'class': ['formspringmeAsker']}
por @\anchor[http://www.formspring.me/dvigitt]{dvigitt}

\par% p% {'class': ['formspringmeAnswer']}
Soy \anchor[http://es.wikipedia.org/wiki/Agnosticismo]{agnóstico}, es decir que creo que la realidad metafísica no  puede ser conocida por la ciencia y como tal cualquier afirmación al
  respecto no deja de ser una opinión o un deseo que no puede probarse.

La ciencia, y en particular el método científico, nos puede dar una  aproximación a cómo funciona el mundo y tal
 parece ser que la teoría  científica no precisa de soluciones mágicas o metafísicas para explicar  el mundo físico.

\par% p
La pregunta que intentan responder las religiones (organizadas o no) o  la religiosidad personal, más que el cómo es el
 porqué de las cosas.  El  propósito.  Si una religión se mete con el cómo muy probablemente  entrará en
 contradicciones con lo científicamente probable lo que  significará una de dos cosas: la religión miente o algún
 fenómeno  metafísico impide que el método científico conozca el cómo que nos dicta  ese dogma religioso.  El principio
 de la \anchor[http://es.wikipedia.org/wiki/Navaja\_de\_Ockham]{navaja de Ockham} me hace pensar  que es lo primero.

Por otro lado si alguien desde la ciencia intenta responder el porqué,  sin duda no entendió de qué trata el método
 científico y es aquí donde  hay espacio para las doctrinas religiosas o la religiosidad personal.

Es posible que exista un ser superior que haya creado el universo con un  propósito y que nuestra existencia en el
 mundo sea parte de ese  propósito.  O es posible que la conciencia colectiva de la humanidad sea  un ente metafísico
 superior, que se creó con la humanidad y que se  manifiesta a través de las religiones.  Hay muchas posibles
 conjeturas  que subordinan al hombre frente a un ente metafísico que da un propósito  a nuestras vidas.

O, tal vez, no haya ninguno y todas esas concepciones metafísicas no  sean más que un deseo más o menos colectivo de
 nuestras mentes de tener  un propósito.

La religión organizada, por regla general, no sería más que una doctrina  (no muy disímil a las doctrinas políticas)
 que procuran encausar la  necesidad humana de propósito dentro de unos parámetros que faciliten la  convivencia entre
 los semejantes.

Tal vez exista una religión organizada que parta de los principios agnósticos, pero tal no sería realmente una religión.

\horrule{}\textbf{Addenum}
Entrdas varias sobre creencias y religión
\begin{itemize}

\item \anchor[http://erikapao.blogspot.com/2010/09/de-esas-cosas-de-dios.html]{De esas cosas de Dios...} por @\anchor[http://twitter.com/ErikaPao86]{ErikaPao86}.
\item \anchor[http://carnavaltodalavida.com/decidiendo-creer/]{Decidiendo creer} por @\anchor[http://twitter.com/Turint]{Turint}.
\item \anchor[http://hijasdelatardecer.blogspot.com/2010/07/y-al-final-en-que-creo.html]{Y al final, ¿en qué creo?} por @\anchor[http://twitter.com/viviangilro]{viviangilro}.

\end{itemize}

\chapter{\#bloggersecreto}
\begin{metadata}
	Published by \anchor[chlewey]{chlewey} on \anchor[http://ewey.co/B614]{Thu, 16 Sep 2010 20:40:20 +0000}\\
	\categories{amigo-secreto, blogger-secreeto, bloggersecreto, blogs, information}\\
	Shorthand: \anchor[http://blog.chlewey.net/2010/09/bloggersecreto/]{bloggersecreto}
\end{metadata}

\par% p
Este artículo es una antesala del que, espero, escribiré mañana con motivo del \#\anchor[http://carnavaltodalavida.com/el-blogger-secreto/]{bloggersecreto}.

\par% p
Pero, ¿qué es el blogger secreto?
 Es una iniciativa que surgió en Twitter entre unos blogueros de jugar al amigo secreto entre ellos y convocando a la comunidad bloguera colombiana en Twitter.
 El \anchor[http://es.wikipedia.org/wiki/Amigo\_invisible]{amigo secreto} es una de esas tradiciones colombianas que se juega con motivo del mes del amor y la amistad: septiembre.
 Sí, en Colombia el \anchor[http://es.wikipedia.org/wiki/D\%C3\%ADa\_de\_San\_Valent\%C3\%ADn]{día de San Valentín} se celebra en septiembre mientras que el 14 de febrero celebran que a los empleados les consignan los intereses a las \anchor[http://es.wikipedia.org/wiki/Cesant\%C3\%ADas]{cesantías}.

\par% p
Pues bien, una vez convocada la iniciativa por \anchor[http://carnavaltodalavida.com/acerca-de/]{Túrin} (@\anchor[http://twitter.com/Turint]{turint}) y aprovechando que tengo este blogcito y que no tenía muchas más ocupaciones en septiembre, que no tengo oficina
 donde el departamento de recursos humanos organizara su propia versión del amigo secreto, etc. pues me uní a la
 convocatoria.

Así que henme aquí.
 Tratando de endulzar con comentarios a mi \#bloggersecreto en su blog, y esperando a ver cómo me endulzan.
 Dejando pistas, jugando a desinformar, y preparando el artículo de mañana el cual será mi regalo de blogger secreto.

\chapter{Cuestión de hormonas}
\begin{metadata}
	Published by \anchor[chlewey]{chlewey} on \anchor[http://ewey.co/B622]{Fri, 17 Sep 2010 17:00:33 +0000}\\
	\categories{blogger-secreeto, bloggersecreto, blogs, personal}\\
	Shorthand: \anchor[http://blog.chlewey.net/2010/09/cuestion-de-hormonas/]{cuestion-de-hormonas}
\end{metadata}

\begin{wrapfigure}{r}{240\px}\centering% {'width': '240', 'align': 'alignright', 'id': 'attachment_625', 'caption': 'Naty Marenco'}
\anchor[http://www.facebook.com/naty.marenco]{\includegraphics[width=240\px,height=297\px]{blog/natymh.jpg}}
\caption{Naty Marenco}
\end{wrapfigure}
\anchor[http://noescuestiondehormonas.blogspot.com/]{\emph{\textbf{No es cuestión de hormonas}}} nació el 19 de marzo de 2009.
 Un blog relativamente joven que está a punto de cumplir 18 meses, pero ya es uno de los sitios más reputados de la
 blogósfera colombiana, tanto que ganó el premio Twitter Blog en los \#\anchor[http://www.entretengo.com/tecnologia/resultados-premiostwt09-colombia.html]{PremiosTwt09}.

\par% p
La historia del nombre nos lo trae eMe \anchor[http://emepg.com/blog/2010/09/08/por-que-el-blog-no-es-cuestion-de-hormonas-se-llama-asi-blogonimia/]{aquí}:

\begin{blockquote}
El  nombre del blog sale de una larga conversación telefónica con mi mejor  amiga en la que debatíamos sobre los dramas
 masculinos versus los  femeninos, y concluimos que no todo en estos temas entre hombres y  mujeres puede ser atribuido
 a las \textbf{hormonas}… como pretenden minimizar los hombres cuando le dicen a uno que si todo el drama se debe a que “\textbf{estás en tus días}“
\end{blockquote}

Y uno de los temas recurrentes son las historias en las que se muestran distintos aspectos de la relación
 hombre-mujer. Mujeres seguras e inseguras, con miedos y esperanzas, que se envidean y comenten errores o crecen frente
 a sus relaciones con los hombres.

Y son hombres inseguros, casi todos medio perros o completamente perros, pero con los que logro algún grado de
 identificación porque son reales y en muchos casos atraviesan por los mismos dilemas con los que yo he vivido.

\par% p
¿Quien escribe?~ Natalia Marenco Hurtado, Naty o @\anchor[http://twitter.com/state\_0f\_mind]{state\_0f\_mind}, es un comunicadora que actualmente trabaja en la \anchor[http://www.reintegracion.gov.co/]{ACR}.

\begin{wrapfigure}{r}{240\px}\centering% {'width': '240', 'align': 'alignright', 'id': 'attachment_629', 'caption': '@state_of_mind'}
\anchor[http://twitter.com/state\_0f\_mind]{\includegraphics[width=240\px,height=161\px]{blog/state_0f_mind.jpg}}
\caption{@state\_of\_mind}
\end{wrapfigure}
Le gustan las ballenas (a quienes recientemente conoció personalmente \relax{% {'style': 'color: #888888;'}
\#queenvida}), los dulces (por quienes arriesgué que me descubriera ante los lentos ascensores de la ACR), la música (me \anchor[http://blip.fm/profile/Secret0fMind/props]{favoriteo en blib.fm} desde Calle 13 hasta Carmina Burana, pasando por Dressden Dolls, Vanessa Paradise, Pedro Guerra, los Sinatra, Patricio
 Cobarde y, desde luego, ese descubrimiento tan especial para mí como fue Natalia Lafourcade cantando \anchor[http://blip.fm/~w4pwu]{Piel Canela}) \relax{% {'style': 'color: #888888;'}
[insisto, si no me \anchor[/de-quereres-y-gustos/]{enamoro} de la Marenco me enamoro de la Lafourcade]}.

\par% p
Le gusta contar historias y logra algunas exquisitas como UNA ENCRUCIJADA (\anchor[http://noescuestiondehormonas.blogspot.com/2010/05/una-encrucijada.html]{Parte I}, \anchor[http://noescuestiondehormonas.blogspot.com/2010/05/una-encrucijada-ii.html]{II}, \anchor[http://noescuestiondehormonas.blogspot.com/2010/05/una-encrucijada-iii.html]{III}, \anchor[http://noescuestiondehormonas.blogspot.com/2010/05/una-encrucijada-iv.html]{IV}, \anchor[http://noescuestiondehormonas.blogspot.com/2010/05/una-encrucijada-v.html]{V}, \anchor[http://noescuestiondehormonas.blogspot.com/2010/05/una-encrucijada-vi.html]{VI}, \anchor[http://noescuestiondehormonas.blogspot.com/2010/05/la-encrucijada-vii.html]{VII}, \anchor[http://noescuestiondehormonas.blogspot.com/2010/06/una-encricijada-viii.html]{VIII}, \anchor[http://noescuestiondehormonas.blogspot.com/2010/06/la-encrucijada-ix.html]{IX}, \anchor[http://noescuestiondehormonas.blogspot.com/2010/06/una-encrucijada-x.html]{X}, \anchor[http://noescuestiondehormonas.blogspot.com/2010/06/una-encrucijada-xi.html]{XI}, \anchor[http://noescuestiondehormonas.blogspot.com/2010/06/una-encrucijada-xii.html]{XII}, \anchor[http://noescuestiondehormonas.blogspot.com/2010/06/una-encrucijada-xiii.html]{XIII}, \anchor[http://noescuestiondehormonas.blogspot.com/2010/06/encrucijada-xiv.html]{XIV}, \anchor[http://noescuestiondehormonas.blogspot.com/2010/06/una-encrucijada-xv.html]{XV} y \anchor[http://noescuestiondehormonas.blogspot.com/2010/06/una-encrucijada-xvi.html]{XVI}).~ Pero también cuenta de su vida como sus indagaciones por descubrir el origen de su apellido y cómo estos llegaron a \anchor[http://toolserver.org/~geohack/geohack.php?pagename=Campo\_de\_la\_Cruz\&params=10\_23\_N\_74\_53\_W\_region:CO\_type:city\_source:GNS-enwiki]{Campo de la Cruz} (Atlántico); las \anchor[http://noescuestiondehormonas.blogspot.com/2009/12/el-ultimo-post-del-ano.html]{tradiciones de año nuevo} en su familia; o cómo los hombres Marenco decidieron un día \anchor[http://noescuestiondehormonas.blogspot.com/2010/06/un-metro-o-una-regla.html]{medírsela}.

\par% p
A Naty la conocí en el \anchor[http://www.campus-party.com.co/]{Campus Party Bogotá 2010} y me pareció una hermosa mujer.
 No recuerdo si la seguía con su antiguo nick @natymh, pero por esas historias desagradables que últimamente he escuchado sobre Twitter tuvo que cerrar esa cuenta.
 Empecé a seguira con su nick actual y ahí al poco tiempo descubrí su blog al ver cómo otros felicitaban a la autora
 por sus escritos, y sí, entré y me encantó.

\par% p
Cuando entré a concursar al \anchor[/bloggersecreto/]{Blogger Secreto}, me tocó de suerte a Naty y su \textbf{\emph{No es cuestión de hormonas}} y eso me motivó a leermelo todo (bueno, casi todo) y a buscar quien es la autora para endulzarla y complacerla y me atrevo a decir que esto ha sido supremamente divertido.
 Este post es el regalo final para Naty, de acuerdo a las \anchor[http://carnavaltodalavida.com/el-blogger-secreto/]{reglas establecidas} por Túrin y sólo espero que le agrade.

\horrule{}
\begin{blockquote}
\anchor[http://noescuestiondehormonas.blogspot.com/2010/09/mis-regalos-de-bloggersecreto.html]{(recuento de endulzadas por Naty)}
\end{blockquote}

\chapter{Una enredada sociedad civil}
\begin{metadata}
	Published by \anchor[chlewey]{chlewey} on \anchor[http://ewey.co/B638]{Wed, 22 Sep 2010 14:27:08 +0000}\\
	\categories{activismo, information, libertad, web}\\
	Shorthand: \anchor[http://blog.chlewey.net/2010/09/una-enredada-sociedad-civil/]{una-enredada-sociedad-civil}
\end{metadata}

\par% p
Hace poco publiqué \anchor[/aunque-nos-llamen-piratas/]{comentado el borrador del manifiesto} del Partido Pirata de Colombia.
 Personalmente creo que muchos de los principios allí expuestos son principios que nos interesa a todos los que usamos Internet para trabajar o divertirnos y que tales propuestas son propuestas que trascienden a un proyecto político en particular.
 Estamos hablando de lo que nos afecta a todos los que usamos la Internet y servicios adjuntos.

\anchor[http://commons.wikimedia.org/wiki/File:Internet.svg]{\begin{wrapfigure}{r}{250\px}\centering% {'src': 'http://upload.wikimedia.org/wikipedia/commons/thumb/5/56/Internet.svg/250px-Internet.svg.png', 'title': 'Internet', 'height': '114', 'width': '250', 'alt': '[A map of the world by the countries with the most internet hosts]', 'class': ['alignright']}
\includegraphics[width=250\px,height=114\px]{blog/250px-Internet_svg.png}
\end{wrapfigure}
}Internet funciona gracias a los esfuerzos de los gobiernos, empresarios, organizaciones internacionales y nacionales y
 todos los usuarios que producimos contenidos.

\par% p
Los gobiernos ejecutan políticas de inclusión digital —\anchor[/la-retorica-de-la-libertad/]{o de restricción digital}—, establecen leyes que regulan la actividad, convocan licitaciones para prestadores de servicios y, en muchos casos,
 financian directamente parte del servicio.

La empresa, pública o privada, se manifiesta prestando servicios de acceso o de carrier, creando el hardware y el software que mueven a Internet a lo largo de todas sus capas y caminos.
 También, desde luego, creando la necesidad que permite que Internet se desarrolle.

Todos nosotros, cuando reenviamos mensajes inspiradores en correo electrónico, publicamos en Facebook o en Twitter, cuando escribimos un blog, subimos nuestras fotos a Flickr, jugamos en línea pocker o Farmville, etc. estamos creando Internet. Tenemos nuestro perfil en Facebook porque es la forma de ver lo que hacen nuestros amigos de la infancia y compartir con ellos nuestra granja virtual o las fotos de nuestros hijos o de la farra del viernes.
 Usamos Internet porque encontramos en Internet nuestros insumos de trabajo o nuestro entretenimiento: música, videos,
 porno, gatos tiernos, gatos graciosos, el último paseo del compañero de kínder, un estudio sobre los dípteros que
 viven en un palo de mango, la reseña de la película que queremos ver, detalles sobre el petardo que acaba de explotar
 o del último escándalo de esa figurita pública que no es de nuestros afectos.

Gran parte del Internet que usamos son contenidos producidos por otros usuarios como nosotros mismos.
 Y producimos contenidos para nuestros amigos.
 Internet existe y nos está llegando cada vez más gracias a estos contenidos.

Pero gran parte de lo que hace Internet útil es porque hay varias organizaciones que se encargan de que Internet
 funcione estableciendo normas y lineamientos que, cuando todos nos ponemos de acuerdo, permiten que la comunicación
 sea realmente efectiva.

\par% p
La \emph{Internet Engineering Task Force} (\anchor[http://www.ietf.org/]{IETF}), por ejemplo, es una de esas organizaciones.
 Es un foro donde participa la industria y los usuarios y donde se discuten los estándares técnicos que permiten que Internet funcione.
 Ahí se discute, por medio de documentos RFC (\emph{Request for comments}: solicitud de comentarios) la forma como Internet debe funcionar desde el punto de vista de los detalles técnicos y
 cuando las empresas que hacen Internet: la industria, los prestadores de servicio, etc. adhieren esos estándares es
 que Internet es útil para todos nosotros.

\par% p
La \emph{Internet Corporation for Assigned Names and Numbers} (\anchor[http://www.icann.org/]{ICANN}), cuya \anchor[http://cartagena39.icann.org/]{trigésima novena reunión} se llevará a cabo este diciembre en Cartagena, define y asigna los nombres y números que identifican cada cosa en
 Internet.

¿Quiénes participan en ICANN o IETF?
 Pues, desde luego, la industria interesada.
 Fabricantes como Cisco y Microsoft participan en la elaboración de estándares que afectarán a sus productos. —Otra discusión es si tales empresas adhieren a esos estándares.— Pero también participan universidades, asociaciones profesionales como la IEEE, gobiernos y, desde luego, cualquiera de nosotros.
 Nosotros somos la sociedad civil en Internet y así como hacemos Internet usándola y produciendo contenidos, también
 podemos hacer Internet participando de sus definiciones a través de estos organismos internacionales.

\anchor[http://www.isoc.org/]{\begin{wrapfigure}{r}{150\px}\centering% {'src': 'http://www.isoc.org/graphics/isoc/isoc_logo.gif', 'title': 'Internet Society', 'height': '60', 'width': '150', 'alt': '[ISOC]', 'class': ['alignright']}
\includegraphics[width=150\px,height=60\px]{blog/isoc_logo.png}
\end{wrapfigure}
}La \emph{Internet Society} (\anchor[http://www.isoc.org/]{ISOC}) es una de esas manifestaciones de la sociedad civil.
 Yo podría invitarte a que si te interesan los temas de la red te hagas partícipe del Partido Pirata, pero hay muchas razones por las cuales entiendo que no te interese: no compartes nuestro manifiesto o ya estás comprometido con otra propuesta política.
 Finalmente, como \textbf{partido}, el Pirata es una \textbf{parte} del espectro.
 Una organización como la Internet Society no.
 Es un foro, es un diálogo, una forma de manifestarte, de influir, de proponer por encima de un gobierno o de un
 partido.

Nuestro gobierno puede tener políticas de inclusión digital; nuestras universidades pueden tener programas de educación a través de la red; nuestros empresarios pueden querer llegar a más colombianos por medio de sus servicios; nosotros podemos seguir creando contenidos, desde la fotos de nuestro último paseo hasta ese artículo en nuestro blog que cambiará radicalmente nuestras percepciones.
 Y tal vez todo funcione.~ Bueno, realmente todo funciona.

\par% p
Pero en todo ello, esta sociedad civil que vive en y por la red, esta sociedad civil enredada, no se encuentra organizada o representada de una forma visible.
 La idea de la Internet Society es esa: visibilizar a la sociedad civil y servir de foro de discusión de los asuntos que nos competen a todos.
 De cómo un estándar como IPv6 es necesario pero conlleva riesgos de privacidad.
 De cómo un programa como Compartel y Computadores para Educar realmente está potenciando desarrollo en las comunidades o no.
 De cómo nuestras \anchor[/proyeccion-comunitaria/]{lenguas indígenas} se pueden proteger o si es necesario que se renuncie a ellas para lograr la inclusión de esas comunidades al mundo de
 la Internet.

\par% p
Todo ello podemos discutirlo y tener voceros no partidistas dentro de iniciativas como el \anchor[http://colombia.isoc.pro/]{Capítulo Bogotá} de la Internet Society.

\anchor[http://colombia.isoc.pro/]{Participa.}

\chapter{Sobre el regocijo por la muerte}
\begin{metadata}
	Published by \anchor[chlewey]{chlewey} on \anchor[http://ewey.co/B646]{Thu, 23 Sep 2010 14:02:33 +0000}\\
	\categories{actualidad, farc, guerrilla, opinion}\\
	Shorthand: \anchor[http://blog.chlewey.net/2010/09/sobre-el-regocijo-por-la-muerte/]{sobre-el-regocijo-por-la-muerte}
\end{metadata}

Ya varias veces lo he dicho: rechazo completamente a las FARC y su pretensión de ser nuestras fuerzas armadas y nuestro ejército.
 No creo en la economía planificada, ni la dictadura del proletariado, ni la lucha de clases, ni ninguna de esos otros
 ideales de izquierda.

Soy humanista.
 Creo en el valor de cualquier vida humana, aún la de aquellos que desprecian la vida de los demás.
 Y espero que se entienda que lo que voy a decir lo digo desde esta perspectiva y no como una admirador de las
 guerrillas.

No me alegro por la muerte de alias el Mono Jojoy.
 No felicito a las Fuerzas Armadas.~ No comparto la alegría colectiva de todos los que veo en mi timeline de Twitter.

Creo que si se requirieron 50 bombas sobre el campamento de alias el Mono Jojoy para darle de baja es porque faltaron
 más recursos, más imaginación y más determinación para garantizar su captura o su rendición incondicional.

\anchor[http://lachaiza.blogspot.com/2010/10/43-anos-de-su-partida-che-guevara.html]{\begin{wrapfigure}{r}{240\px}\centering% {'src': 'http://2.bp.blogspot.com/_FSLTVdTikps/TLCoS_8s-HI/AAAAAAAAHDA/t81xI8bV1W8/s1600/che-soldiers.jpg', 'title': 'Che muerto', 'height': '174', 'width': '240', 'alt': '', 'class': ['alignright']}
\includegraphics[width=240\px,height=174\px]{blog/che-soldiers.jpg}
\end{wrapfigure}
}En estos momentos tengo el recuerdo de dos fotografías en mi cabeza.
 En una el ejército boliviano muestra el cadáver abatido de Ernesto Guevara.
 Muchos aún aseguran que no fue una muerte en combate sino una ejecución extrajudicial.
 Sin embargo y a pesar de la importancia histórica de esta foto permanece opacada frente a la de ``Guerrillero heróico'' de Alberto Korda.
 La imagen que la historia y la cultura popular recuerda de ese asesino que fue Ernesto Guevara no es su muerte sino su
 altivez y su lucha.

\anchor[http://paulita24.wordpress.com/2010/04/26/yo-tambien-voy-a-pelar-las-nalgas/]{\begin{wrapfigure}{l}{227\px}\centering% {'src': 'http://paulita24.files.wordpress.com/2010/04/rebelde.jpg', 'title': u'Abigael Guzm\xe1n', 'height': '348', 'width': '227', 'alt': '', 'class': ['alignleft']}
\includegraphics[width=227\px,height=348\px]{blog/rebelde.jpg}
\end{wrapfigure}
}La otra foto que recuerdo es la de Abigael Guzmán, alias el Presidente Gonzalo, en un uniforme a rayas y tras las rejas.  Como un león enjaulado, vociferando como una forma de ocultar su derrota. Porque, aunque hay quienes insisten en ver a un guerrillero todavía rebelde en esas imágenes, lo que todos vemos es a una fiera que perdió la batalla.
 Sabemos que está en una prisión en Perú ya sin inspirar a nadie y, efectivamente, Sendero Luminoso está lejos de tener
 la relevancia o el poder que alguna vez tuvo.

La revolución cubana aún no ha desaparecido y se sigue lucrando de la imagen del ``Guerrillero heróico''.
 Sendero Luminoso ya no existe para efectos prácticos.
 Al ``Che'' Guevara muchos lo recuerdan como un mártir de la revolución por su muerte en Bolivia.
 Sólo los más mamertos recuerdan a alias el Presidente Gonzalo como alguien que no claudicó en su lucha a pesar de las
 rejas.

Por eso no me alegra la muerte de alias Raúl Reyes, ni la muerte de alias el Mono Jojoy.~ Por eso no las celebro.

Desde un punto de vista práctico, sí celebro cualquier cosa que disminuya la capacidad de lucha de las FARC y que acerquen a nuestro país a nuestra anhelada paz.
 Si la muerte de alias Reyes y alias Jojoy contribuyen a que menos colombianos mueran me parece muy bien.
 Pero no sus muertes en sí.

\horrule{}

\subsection{\anchor[\#addenum]{Addenum}}
\par% p% {'class': ['date']}
\begin{small}
27 de septiembre de 2010
\end{small}

Navegando por el regocijo de muchos, entiendo algunos elementos.

\par% p
Primero sentí que sólo una persona que odia a otro ser humano sentiría alegría por su muerte —y sin duda el \emph{Mono Jojoy} sembró muchas razones para ser odiado— pero creo que va más allá del odio: basta con desear, con creer que la muerte
 del otro es conveniente.  No es mi caso, como lo expliqué arriba; pero no siento tener una superioridad moral por la
 cual juzgue a quienes piensan diferente.

\par% p
Segundo, no creo que ni el \emph{Mono Jojoy} ni \emph{Raúl Reyes} se conviertan en mártires de una causa.  No tienen ni el carisma del \emph{Che} Guevara, ni éste es el momento histórico que idealice ese tipo de lucha armada.

\chapter{Hacen su esfuerzo}
\begin{metadata}
	Published by \anchor[chlewey]{chlewey} on \anchor[http://ewey.co/B651]{Wed, 13 Oct 2010 13:37:49 +0000}\\
	\categories{blogs, information, spam}\\
	Shorthand: \anchor[http://blog.chlewey.net/2010/10/hacen-su-esfuerzo/]{hacen-su-esfuerzo}
\end{metadata}

\par% p
Los siguientes son algunos comentarios que Akismet sospecha que son spam, pero que no está lo suficientemente seguro como para eliminarlo de una.
Nótese que no hay URLs en el mensaje, pero sí los hay acompañando al supuesto remitente.

\begin{blockquote}
I so enjoyed every bit of this site and I’ve bookmarked your blog to keep up with the new topics you will post in the
 future.
\par% div% {'class': ['signature']}
— \textbf{Barry Glynne “The Forex” Guy} en respuesta a \anchor[/una-enredada-sociedad-civil/]{Una enredada sociedad civil}

\end{blockquote}

¿Me querrá invirtiendo en Forex?
\begin{blockquote}
What I dont comprehend is how youre not even a lot more popular than you might be now. Youre just so intelligent. You
 know so significantly about this topic, created me consider it from so many diverse angles. Its like men and women
 arent interested unless it has some thing to accomplish with Lady Gaga! Your stuffs great. Continue to keep it up!
\par% div% {'class': ['signature']}
— \textbf{amazon} en respuesta a \anchor[/una-enredada-sociedad-civil/]{Una enredada sociedad civil}

\end{blockquote}

En principio el URL no tiene mucho que ver con Amazon.com
\begin{blockquote}
Hey – nice blog, just looking around some blogs, seems a pretty nice platform you are using. I’m currently using
 WordPress for a few of my sites but looking to change one of them over to a platform similar to yours as a trial run.
 Anything in particular you would recommend about it?
\par% div% {'class': ['signature']}
— \textbf{Sam “reverse phone look up” Hicks} en respuesta a \anchor[/una-enredada-sociedad-civil/]{Una enredada sociedad civil}

\end{blockquote}

Otro con un \emph{nick} que parece sugerir los servicios que ofrece.
\begin{blockquote}
Hey, observed your site by accident undertaking a search on Yahoo but I?ll undoubtedly be coming back again. ? How can
 I feel in God when only final week I obtained my tongue caught inside the roller of an electric typewriter?
\par% div% {'class': ['signature']}
— \textbf{Cheap} en respuesta a \anchor[/bloggersecreto/]{\#bloggersecreto}

\end{blockquote}

Para mí hay algo WTF en este mensaje.
\begin{blockquote}
Since researching for quite a while just for a smart content articles with reference to this one point . Checking in
 Bing I at last noticed this page. After reading these details I’m just seriously happy to express that I have a great
 uncanny feeling I stumbled onto the very things I needed. I’ll make certain to remember this blog and give it a look
 constantly.
\par% div% {'class': ['signature']}
— \textbf{nursing cover} en respuesta a \anchor[/una-enredada-sociedad-civil/]{Una enredada sociedad civil}

\end{blockquote}

No es que no crea que mis contenidos son \emph{smart content articles} pero que no haga referencia al lenguaje de mi blog muestran que no es más que un mensaje genérico.
\begin{blockquote}
The second volume of Brother’s licensed embroidery designs features the lovable, high-octane characters from the Disney
\par% div% {'class': ['signature']}
— \textbf{disney embroidery designs} en respuesta a \anchor[/sobre-el-regocijo-por-la-muerte/]{Sobre el regocijo por la muerte}

\end{blockquote}

Bueno, este ni hace el intento de simular que comenta sobre el contenido.
\begin{blockquote}
Thank you very much for posting, it’s really useful for me
\par% div% {'class': ['signature']}
— \textbf{proxy} en respuesta a \anchor[/una-enredada-sociedad-civil/]{Una enredada sociedad civil}

\end{blockquote}

\emph{'nuf said.}
\begin{blockquote}
The first thought that goes through my mind each and every time college begins is how much I crave for uni to finish.
 When vacation finally begins, I go home. I’m content for less then one week then I’m getting sad and I want uni to
 begin again. When school begins then I’m keeping track of the days until vacation starts again. My mother says I’m
 going crazy and I should seek help so I thought it would be a really good idea to ask here: Do you guys think I’m
 lunatic ? Waiting for an answer, I didn’t really know or wanna ask somewhere else.
\par% div% {'class': ['signature']}
— \textbf{Hazel Land} en respuesta a \anchor[/una-enredada-sociedad-civil/]{Una enredada sociedad civil}

\end{blockquote}

¿Y como por qué creyó que este era el mejor sitio para preguntar?
\begin{blockquote}
The only idea that goes through my brain each and every time college begins is how much I crave for school to end. When
 vacation finally begins, I go to my home friends. I’m really happy for one week or so then I’m getting tired and I
 want school to begin again. When university begins then I’m keeping track of the days until vacation starts again. My
 cousin says I’m not normal and I should ask for help so I thought it would be a safe idea to ask here: Do you people
 think I’m lunatic ? Looking forward for an answer, I didn’t really know or wanna ask somewhere else.
\par% div% {'class': ['signature']}
— \textbf{Jonah Bench} en respuesta a \anchor[/una-enredada-sociedad-civil/]{Una enredada sociedad civil}

\end{blockquote}

Nótese el gran parecido de estas dos últimas, sin embargo firmado por dos personas diferentes.

\horrule{}Y no acabo de escribir esto y siguen:
\begin{blockquote}
Hi. I wanted to drop you a quick note to express my thanks. I have been following your blog for a month or so and have
 picked up a heap of good information as well as enjoyed the way you’ve structured your site.
\par% div% {'class': ['signature']}
— \textbf{Barry Glynne “The Forex” Guy} en respuesta a \anchor[/cuestion-de-hormonas/]{Cuestión de hormonas}

\end{blockquote}

\chapter{El tirano de los minutos}
\begin{metadata}
	Published by \anchor[chlewey]{chlewey} on \anchor[http://ewey.co/B655]{Thu, 25 Nov 2010 18:26:19 +0000}\\
	\categories{comunicaciones, opinion}\\
	Shorthand: \anchor[http://blog.chlewey.net/2010/11/el-tirano-de-los-minutos/]{el-tirano-de-los-minutos}
\end{metadata}

\par% p
A veces me decía que cómo es posible esto si ya no estamos en los ochenta... luego recordaba que en los ochenta los
 buses de \anchor{colegio} tenían su propio servicio de radio.

\par% p
El pasado martes 23 de noviembre de 2010 cayó en Bogotá la lluvia represada de un fin de semana relativamente seco en
 medio de esta temporada \emph{% {'title': 'lluviosa'}
invernal}.~ Inundaciones en varios puntos de la ciudad y uno de los sitios afectados fue el paso de la \anchor[http://maps.google.com/?ie=UTF8\&ll=4.696,-74.086\&t=h\&z=17]{Avenida Boyacá sobre el río Juan Amarillo}.~ Esta inundación se interpuso en medio de la ruta de regreso de mi hijo del \anchor{colegio}.~ Un recorrido que normalmente toma 75 minutos tardó este día 190.~ No vale la pena quejarnos del \anchor[http://es.wikipedia.org/wiki/La\_Ni\%C3\%B1a\_\%28clima\%29]{Fenómeno de la Niña}, o de la falta de infraestructura adecuada o de otros temas estructurales y complejos, porque eso se llevaría muchos bytes y palabras que uds. no me irán a leer.
 Quiero enfocarme en el problema de la comunicación.

\par% p
Como padres entregamos a mi hijo a las 7:30 a la responsabilidad del colegio y reasumimos la responsabilidad a las 16:30 cuando el colegio nos lo entrega en la puerta de nuestra casa.
 En esas nueve horas uno espera que el colegio sea responsable y con el fin de no interferir con los procesos de formación ni causar caos innecesario a los esquemas programados no sabemos realmente que está pasando.
 Esperamos que el colegio en su responsabilidad nos informe de cualquier incidente importante tales como si el niño se
 enferma (y ellos tienen nuestros números telefónicos de casa, oficina y \anchor{celular}).
 Los horarios a los que pasan los buses del colegio no son muy exactos pues estos están sujetos a condiciones de
 tráfico cambiantes en la ciudad, a que uno de los niños tardare en salir de clase retrasando a los demás o a que un
 padre no esté atento a recoger al niño retrasando el recorrido mientras lo esperan; y en más de una ocasión la hora de
 entrega se ha acercado a las 17:00... un retraso entendible, por el que nisiquiera vale la pena pedir explicaciones.

\par% p
Pero el martes el retraso exedió por mucho la media hora y nosotros, como padres, no sabíamos qué era lo que sucedía.
 Ya era preocupante, y nuestra primera reacción fue comenzar por llamar al colegio.
 El problema, es que la jornada termina a las 17:00 y después de esa hora no había quién atendiera los teléfonos.
 Finalmente y tras una larga cadena de llamadas, sobre las 18:00 nos enteramos que la ruta estaba buscando vías alternas al paso de la Boyacá con Juan Amarillo y que se reportaba que los tres niños faltantes estaban bien.
 Hoy en día, en que en Colombia \anchor[http://www.asocel.org.co/prensa.php]{hay casi tantos teléfonos móviles como habitantes}, parecería inconcebible que no fuera posible contar con información sobre este tipo de incidentes que pudieran
 tranquilizar a los angustiados padres o avisarles de algún tipo de inconveniente.

Desde luego que hay problemas de protocolos y descuidos.
 El colegio no permite que los padres tengan el número del coordinador de rutas para evitar abusos tales como aquellos padres que se la pasan llamando para cambiar el lugar en donde el bus deberá entregar al hijo.
 A nosotros nos dieron el número celular de la directora de grupo (profesora) de nuestro hijo para inquietudes como esta, pero ese número se quedó perdido junto con el teléfono de mi esposa cuando se lo robaron.
 En fin, varios elementos conspiraron para que la comunicación no fluyera de una forma efectiva para nuestra pronta
 tranquilidad.

Sin embargo sí había protocolos para estos casos.
 El coordinador de rutas es un empleado del colegio que debe estar atento a cualquier incidente que suceda con los buses del colegio y como tal no está sujeto a los horarios de oficina del conmutador principal del colegio.
 Cuando sucede un incidente, la monitora de ruta (el adulto responsable de los niños mientras el conductor maneja) debe reportarlo al coordinador de rutas y tiene también el número telefónico de los papás.
 La monitora podría llamar directamente a los papás para anunciarles del motivo del retraso o pedir al coordinador de rutas que se encargue de estas comunicaciones.
 Parece que todo está solucionado.

Hasta que nos enfrentamos al tirano de los minutos.
 La monitora de ruta no podía llamar ni a los papás ni al coordinador de rutas para anunciar la novedad porque no tenía saldo en su teléfono móvil.
 No tenía minutos.

Como padres no podíamos llamar a la monitora de ruta (porque por los protocolos destinados a evitar abusos por parte de los padres, el número de su teléfono móvil no nos lo dejan saber) ni la monitora podía llamar a los padres porque no tenía saldo.
 No tenía minutos.
 Como padres tampoco pudimos llamar al coordinador de rutas, porque los mismos protocolos nos prohiben su número, ni el coordinador de rutas podía llamar a los padres porque no estaba enterado, porque la monitora de ruta no se había comunicado porque no tenía saldo.
 No tenía minutos.

\par% p
Mi experiencia me lleva a pensar que los planes de telefonía prepago o los \emph{postpago control} no son convenientes en un gran número de casos.
 No sirven para una persona que necesita en cualquier momento tener que generar una llamada.
 Una llamada que no puede estar sujeta a la excusa de que no tenía minutos.
 Mientras esperaba este martes a que llegara mi hijo, y aun sin saber qué pasaba, reflexionaba sobre el problema de la comunicación.
 Cuando yo era estudiante los buses de mi colegio contaban con tranceptores de radio con frecuencias privadas del operador de los buses.
 Si hubiera habido un incidente el chofer podía comunicarlo a su empresa, esta al colegio y este a los padres, sin estar pendientes del saldo del teléfono móvil (aunque no sé si los protocolos existían e, igual, como no existían los celulares, nuestros padres no estaban tan acostumbrados como nosotros a que si algo pasa podemos saberlo YA).
 Hoy en día muchas personas están sujetas al tirano de los minutos.
 A que sus comunicaciones dependen de cuántos minutos llevan gastados en el mes o a si la recarga de su teléfono
 prepago ha sido suficiente.

Yo nunca tuve teléfono prepago, aunque por algún tiempo tuve un plan control, esto es un plan donde yo acuerdo con el operador que este no me deje pasar de un determinado saldo mensual y que, una vez lo supere, o no podré seguir haciendo llamadas o tendré que comprar una recarga de saldo prepago.
 No fueron todos los meses (yo soy una persona que hablo poco por celular), pero los meses que me exedí sufrí del tirano de los minutos.
 No poder hacer una llamada, o tomar la decisión de comprar minutos extra, minutos que probablemente no gastaré en su totalidad.
 Aun cuando yo no tengo una obligación legal de tener que efectuar una llamada en cualquier momento, estoy convencido
 de una cosa: uno debería tener un plan de telefonía que le permita hacer la llamada que necesite, cuando la necesite y
 sin estarse preguntando cuánto le costará.

En mi caso tengo el plan postpago (crédito) más bajo de Movistar, y pocas veces lo excedo.
 Símplemente hablo poco por celular, pero cada vez que necesito hablar por celular simplemente llamo, sin preocupaciones, sin tener que estar pendiente de cual es el operador de mi conferente (porque esa es la otra, las personas a las que se les vence el saldo para llamar a otros operadores, así no tengan restricciones para el propio), sin estar pidiendo que me devuelvan la llamada.
 Para mí eso es tranquilidad.

Pero igual no tengo obligaciones legales de tener que llamar.
 Lo que me funciona para mí, como usuario particular, podría no serlo para otro tipo de usuarios particulares.
 Usuarios que prefieren controlarse por medio de prepagos o de postpagos controlados.

Pero personas que sí tienen esa obligación legal o contractual de llamar definitivamente no pueden estar sujetas al
 tirano de los minutos.

Habiendo trabajado anteriormente en coordinación de obras se presentaba otro problema: llegaba una cuadrilla de trabajo a un sitio remoto y encontraba que no estaban los materiales listos.
 Con su teléfono prepago y sin minutos el jefe no tenía otra que sentarse a esperar con toda la cuadrilla a que terminara el día o a que lo llamaran y poder así comunicar la situación, reportar la falta de materiales y que los coordinadores pudieramos gestionar ante proveedores y clientes los retrasos generados.
 Tras varias jornadas perdidas por ese tipo de causas  (el presupuesto no nos daba para pagarles de la cooperativa un
 plan abierto a todos los jefes de cuadrilla) finalmente nos tocaba a los coordinadores ponernos a hacer rondas de
 llamadas de sondeo para enterarnos de las novedades de los diferentes sitios; llamadas que en su gran mayoría eran
 para saber que todo iba bien.

\par% p
Por ley, los operadores celulares no pueden restringir las llamadas a los servicios de emergencia, como el \texttt{% {'title': u'polic\xeda'}
112} o el \texttt{% {'title': 'emergencias'}
123}, así sea de un teléfono prepago sin minutos; o incluso un teléfono sin SIM o un postpago con la cuenta vencida.
 En caso de un accidente, la monitora de ruta de mi hijo pudo haber llamado a la línea de emergencias del distrito \texttt{123} y reportarlo, aún sin saldo.  Pero la línea de emergencias del distrino no se va a poner a llamar a los padres para avisarles que sus hijos se encuentran bien en un trancón.
 Esa es la labor del coordinador de rutas, pero el número del coordinador de rutas del colegio de mi hijo no es un
 servicio de emergencias.

Hay varios tipos de soluciones alternativas a que una empresa pague a sus empleados planes abiertos o los obligue a que se lo paguen ellos mismos.
 Una es que el operador permita llamadas con cobro revertido (llamadas por cobrar) originadas aún desde teléfonos sin
 saldo, así sea a ciertos números específicos.

¿Alguna otra idea?

\chapter{Por el motivo incorrecto}
\begin{metadata}
	Published by \anchor[chlewey]{chlewey} on \anchor[http://ewey.co/B658]{Mon, 06 Dec 2010 17:09:37 +0000}\\
	\categories{depresion, muerte, personal}\\
	Shorthand: \anchor[http://blog.chlewey.net/2010/12/por-el-motivo-incorrecto/]{por-el-motivo-incorrecto}
\end{metadata}

Uno de los descubrimientos personales recientes es un don que carezco: no soy capaz de pensar en forma práctica: planear, prevenir, estar pendiente de las cuentas por pagar y por cobrar.
 Haciendo un trabajo termino concentrado en detalles de forma, perfeccionando un pixel que no es el que al cliente le
 interesa, y por más que intento enfocarme mi foco no es aquel que los demás esperan.

\begin{wrapfigure}{r}{150\px}\centering% {'src': 'http://blog.chlewey.net/wp-content/uploads/2010/12/colsa-var11-09-150x150.jpg', 'title': 'mural', 'height': '216', 'width': '150', 'alt': '[]', 'class': ['alignright', 'size-thumbnail', 'wp-image-678']}
\includegraphics[width=150\px,height=216\px]{blog/colsa-var11-09-150x150.jpg}
\end{wrapfigure}
Así que termino patinando.
 Sin lograr avanzar en mis objetivos y mucho menos en los objetivos de quien me requiere, sea este mi familia o mi
 contratante.

Y es frustrante.
 Terminan haciéndome sentir, termino sintiéndome, un estorbo; un lastre.
 Sentir que es más lo que empeoro la situación que lo que sirvo para resolverla.
 Es entonces cuando me deprimo.
~ No sé que tan clínicamente correcto sea aquí hablar de depresión, pero sí es una sensación de baja de ánimo, de
 perder mi mirada en divagaciones que no son pensamientos coherentes (y mucho menos útiles o prácticos).
 Llego a sentir que lo mejor es que yo no existiera, que dejara de existir.
 No ser más un lastre.~ No generar más expectativas en los demás que los lleve a una nueva desilución conmigo.

Para que aprendan.

\par% p
Pero no pasará.~ Más allá de mi \anchor[http://blog.chlewey.net/2010/08/need-the-rush/]{actitud temeraria} no estoy pensando en culminar el fin de mi existencia.
 Además porque sería por el motivo incorrecto, porque nadie sacará una lección de mi muerte y no solucionaré el
 problema de nadie.

Y no estaré ahí para saber si el sacrificio funcionó o no.

\chapter{Momentos decisivos}
\begin{metadata}
	Published by \anchor[chlewey]{chlewey} on \anchor[http://ewey.co/B662]{Mon, 29 Nov 2010 16:05:45 +0000}\\
	\categories{familia, fechas, personal, proyeccion-y-carrera}\\
	Shorthand: \anchor[http://blog.chlewey.net/2010/11/momentos-decisivos/]{momentos-decisivos}
\end{metadata}

\begin{wrapfigure}{r}{320\px}\centering% {'src': 'http://blog.chlewey.net/wp-content/uploads/2010/11/muchos.jpg', 'title': 'Somos parte de muchos', 'height': '180', 'width': '320', 'alt': '', 'class': ['alignright', 'size-full', 'wp-image-665']}
\includegraphics[width=320\px,height=180\px]{blog/muchos.jpg}
\end{wrapfigure}
Leyendo \anchor[http://carnavaltodalavida.com/el-mismo-hijo-de-puta-de-siempre/]{el último post} de \anchor[http://carnavaltodalavida.com/]{\emph{Carnaval toda la vida}} recordaba una teoría que un profesor nos exponía: cuando de relaciones se trata las mujeres entregan todo a su
 presente mientras que los hombres viven de momentos decisivos.

\par% p
Y mi vida está llena de eventos decisivos.~ Hoy recuerdo cada una de las mujeres que en su momento \anchor{me movieron el piso} y aún cuando tengo la firme convicción de dedicar el resto de mi vida a honrar el pacto que hace cerca de 10 años
 suscribí, cada una de estas otras mujeres son y seguirán siendo parte de mi vida.

\begin{object}% {'classid': 'clsid:d27cdb6e-ae6d-11cf-96b8-444553540000', 'width': '240', 'style': 'width: 240px; height: 205px;', 'codebase': 'http://download.macromedia.com/pub/shockwave/cabs/flash/swflash.cab#version=6,0,40,0', 'height': '205'}
\param{% {'name': 'src', 'value': 'http://www.youtube.com/v/pHTtRcux2GY?fs=1&hl=es_ES&rel=0'}
}\param{% {'name': 'align', 'value': 'right'}
}\embed{% {'src': 'http://www.youtube.com/v/pHTtRcux2GY?fs=1&hl=es_ES&rel=0', 'style': 'width: 240px; height: 205px;', 'align': 'right', 'height': '205', 'width': '240', 'type': 'application/x-shockwave-flash'}
}
\end{object}
\anchor{Hoy} cumplo 38 años.
 La cifra no es particularmente representativa como suelen serlo los múltiplos de 10, pero para mí tiene un significado
 muy especial por todo lo ocurrido durante el año transcurrido.

\par% p
En diciembre de 2009 se cumplieron 20 años desde que \anchor[http://chlewey.net/colsalle]{me gradué} \anchor{del colegio}.
 Esa celebración me tomó en medio de una crisis laboral que aún no supero, un tema que trataré más adelante.
 También fue la oportunidad de reencontrarme con quien fuese mi amor platónico, mi gran traga, de los últimos años de \anchor{mi bachillerato}.

Claramente no espero (ni esperaba ese diciembre) que se reviviera algo que alguna vez existió en mí.
 Pero no puedo decir tampoco que no me importó.
 Ella seguirá siendo una parte importante de mi historia y por ende de mi vida, aún cuando no sepa mayormente en qué anda o qué pasa con su vida.
 Un momento decisivo.

Recordaba mi crisis laboral, que en gran parte ha sido una crisis vocacional.
 Me he dado cuenta que no me veo realmente como un ingeniero electrónico trabajando en ingeniería electrónica.
Que el concepto de ser empleado no es mi propio ideal.
 El problema, sin embargo, es que no tengo respuestas hacia qué quiero hacer en su lugar.

\par% p
¿Comenzar un \emph{\anchor[http://es.wikipedia.org/wiki/Compa\%C3\%B1\%C3\%ADa\_startup]{startup}}? ¿Dedicarme a invertir con el objetivo de vivir de la renta? ¿Trabajar a destajo? Y ¿En qué? ¿Ser un \anchor[http://www.trabajadorweb.com/]{trabajador web}? ¿Un comerciante? ¿Un consultor de \anchor[http://blog.chlewey.net/tag/social-media/]{Social Media}? \anchor[http://blog.chlewey.net/2010/08/me-gustaria/]{¿Escribir? ¿Dictar cursos?}

El problema es que ya llevo más de dos años en esta incertidumbre y lo poco que he hecho como independiente en ese tiempo no ha sido suficiente.
 Y eso hace mella.~ Mella, entre otros en la familia y la relación de pareja.

Durante este año que concluye hoy, hubo varios momentos en los que sentí que mi matrimonio se había acabado.
 Que sólo faltaba formalizar esa terminación.
 No fue algo que yo deseara, pero sí algo que estaba pasando y que no podía desconocer.
 Y a falta de otro tipo de amigos, mi refugio, mi grupo de apoyo, el lugar para buscar consejos fue Internet.
 Twitter y su espejismo de anonimato, principalmente.

\par% p
Y sucedió.~ Permití que mi alma se acercara demasiado a otra persona, \anchor[http://tumblr.chlewey.net/post/1200649606/no-te-puedes-tener-una-relacion-complice]{una cómplice} de las noches que parecía capaz de leer mis desvaríos.
 Nunca supe que quise yo realmente con ella, ni a qué estaba dispuesto a jugar; pero la necesité.
 Me permití cambiar de perspectiva.
 Me permití buscarla e, incluso, bajar completamente cualquier concepto de prudencia.

No sé a qué pude haber llegado y aún me queda esa duda.
 Nada pasó finalmente.
 Tal vez pude haber presionado demasiado por encontrar una respuesta, lo que al final significó que la respuesta no
 pudo darse.

Pero siento que encontré la respuesta que era realmente importante.

Hoy hay otro de esos momentos decisivos.
 Una persona más que es y será parte de mi historia y de mi vida.
 Una persona en la que no estaré pensando todos los días pero a la que me será imposible ignorar.

Pero, al igual que mi primera traga en el colegio, siento que no me es importante.
 Siento que lo fue, y que ese haber sido no se borrará de mi historia ni de mi vida.
 Siento que aún hay muchas preguntas sin respuestas, respuestas que aún quisiera encontrar.
 Pero respuestas que en últimas no importan.

Lo que importa es renovar mi pacto.
 El pacto con una persona que se convirtió hace trece años y medio en otro de esos momentos decisivos y con quien hace diez formalizamos ese pacto en nuestros votos matrimoniales.
 Una persona con la que he compartido más de la tercera parte de mi vida, y con la que compartimos dos vidas más.

Una decisión difícil que debe renovarse día a día.
 Que debe buscar esa parte femenina de mi ser de ser capaz de entregar todo a mi presente para que sea mi futuro.
 Es lo que quiero, es lo que deseo, es por lo que estoy trabajando día a día.

\chapter{De tragedia en tragedia}
\begin{metadata}
	Published by \anchor[chlewey]{chlewey} on \anchor[http://ewey.co/B676]{Tue, 07 Dec 2010 18:37:50 +0000}\\
	\categories{actualidad, opinion, prevencion, tragedias}\\
	Shorthand: \anchor[http://blog.chlewey.net/2010/12/de-tragedia-en-tragedia/]{de-tragedia-en-tragedia}
\end{metadata}

\anchor[http://es.wikipedia.org/wiki/Archivo:Armerotragedy2.png]{\begin{wrapfigure}{r}{252\px}\centering% {'src': 'http://upload.wikimedia.org/wikipedia/commons/b/b7/Armerotragedy2.png', 'title': 'Tragedia de Armero, Tolima (1985)', 'height': '173', 'width': '252', 'alt': '[]', 'class': ['alignright']}
\includegraphics[width=252\px,height=173\px]{blog/Armerotragedy2.png}
\end{wrapfigure}
}La primera vez que supe que había un pueblo llamado Armero fue el 13 de noviembre de 1985, cuando escuchaba esa
 madrugada en la radio los primeros reportes de las poblaciones afectadas por el deshielo del volcán Nevado del Ruiz.
 No había reportes de Armero hasta que un piloto que sobrevolaba el área dijo al aire desconcertado ``Armero no existe''.

Armero era una ciudad próspera.  El eje del desarrollo del norte del Departamento del Tolima, una región de gran
 riqueza agrícola y ganadera.  Riqueza que en gran parte se debe al fértil suelo producto de erupciones anteriores del
 volcán Nevado del Ruiz y otros volcanes aledaños.  En 1985, pronto a cumplir mis 13 años, ignoraba todo esto.  Supe de
 Armero por la tragedia y tendrían que pasar varios años para entender que gran parte de esa tragedia pudo haber sido
 evitada aunque, tal vez, fue necesaria para mostrar las fallas que permitirían más adelante prevenir otras tragedias.

\anchor[http://www.elnuevodia.com.co/nuevodia/nacional/notas-nacionales/36435-el-invierno-causo-tragedia-en-antioquia.html]{\begin{wrapfigure}{l}{262\px}\centering% {'src': 'http://blog.chlewey.net/wp-content/uploads/2010/12/bello13.jpg', 'title': 'Tragedia de Bello, Antioquia (2010)', 'height': '180', 'width': '262', 'alt': '', 'class': ['alignleft', 'size-full', 'wp-image-687']}
\includegraphics[width=262\px,height=180\px]{blog/bello13.jpg}
\end{wrapfigure}
}25 años después Colombia vive otra tragedia.  Tragedia en muchos casos evitable, pero que, al igual que en Armero, es muy fácil decir ex post facto cuales fueron todas las fallas, pero muy difícil determinar cómo se hubieran podido prevenir.
 Año tras año en Colombia hay tragedias producto del \anchor{invierno}.
 Tragedias que por su regularidad no deben atribuirse a los caprichos de la naturaleza sino a la imprevisión humana.
 Las lluvias causan que los ríos y aroyos se extiendan a sus máximos pero muchas personas insisten en construir sus viviendas y sembrar sus cultivos al margen de la extensión media de su caudal.
 Cuando el río recupera su máximo se tragará estas viviendas y esos cultivos.
 En algunos casos, como en Barranquilla, se llega incluso a pavimentar a los aroyos secos e incluirlos en la
 nomenclatura urbana como calles.

La prevención tiene varios aspectos.
 El primero es conocer los riesgos de los diferentes lugares: dónde es probable que un curso de agua salga de su media para extenderse a su máximo; dónde bajará un lahar volcánico; dónde se presentan mayores riesgos sísmicos y de qué clase.
 Sin embargo es iluso suponer que conocer los riesgos nos lleve a construit y cultivar donde tales riesgos no existan.
 La cantidad de tierra existente libre de todo riesgo sería extremadamente escasa, para empezar.
 Pero, al igual que ocurrió con Armero, y que se repite a lo largo de todo el mundo, esas tierras con riesgo tienen sus
 atractivos.

\anchor[http://en.wikipedia.org/wiki/File:ModernEgypt,\_Festival\_of\_the\_Nile\_at\_Cairo,\_BAP\_24819.jpg]{\begin{wrapfigure}{r}{250\px}\centering% {'src': 'http://upload.wikimedia.org/wikipedia/commons/c/c8/ModernEgypt%2C_Festival_of_the_Nile_at_Cairo%2C_BAP_24819.jpg', 'title': 'Festival del Nilo en El Cairo', 'height': '128', 'width': '250', 'alt': u'[inundaci\xf3n del Nilo]', 'class': ['alignright']}
\includegraphics[width=250\px,height=128\px]{blog/ModernEgypt__Festival_of_the_Nile_at_Cairo__BAP_24819.jpg}
\end{wrapfigure}
}La civilización egipcia prosperó porque el Nilo inundaba su delta dejando tras de sí unas fértiles tierras aptas para el cultivo a gran escala.
 Tras muchas generaciones los habitantes del delta aprendieron a predecir las crecidas y a controlar aspectos de la inundación y así surgió una de las primeras civilizaciones.
 Otro tanto sucedería en las llanuras indundables de Mesopotamia, India y China.
 La ciudad de Nápoles creció a la sombra del volcán Vesubio; y surgió y creció y sigue creciendo a pesar de que la historia de las ciudades de Pompeya y Herculano era y es conocida.
 Si el Vesubio explota nuevamente y con la fuerza del año 79 d.C. será poco lo que el hombre pueda oponerle para salvar a la ciudad de Nápoles.
 Muchas generaciones de napolitanos se hicieron prósperas gracias al volcán.
 Alguna generación lamentará esto y se quejará de la falta de previsión de sus antepasados.

Conocer los riesgos es importante, más que para no desarrollar un sitio, para desarrollarlo inteligentemente.
 Para que las normas de construcción sismorresistentes sean más estrictas donde hay mayor riesgo sísmico, para exigir mejores simientos donde haya riesgos de avalancha, para constuir sobre pilotes donde haya riesgos de inundación, etc.
 Conocer los riesgos también es importante para monitorear sus causas.
 Al igual de Italia, Japón convive con muchos volcanes activos y todos ellos son monitoreados día y noche para predecir cuándo uno de ellos despertará.
 Y, ojalá, con la suficiente anticipación para que la población afectada pueda evacuar.
 Islandia es otro país que se acostumbró a convivir con sus volcanes y si bien el Eyjafjallajökull fue noticia mundial,
 para los islandeses mismos fue rutina, no mayor que cuando el Eldfell amenzó en 1973 toda una de sus islas pobladas.

\anchor[http://es.wikipedia.org/wiki/Archivo:Parque\_Narino\_Pasto.JPG]{\begin{wrapfigure}{r}{250\px}\centering% {'src': 'http://upload.wikimedia.org/wikipedia/commons/2/20/Parque_Narino_Pasto.JPG', 'title': u'Volc\xe1n Galeras visto desde Pasto', 'height': '164', 'width': '250', 'alt': '', 'class': ['alignright']}
\includegraphics[width=250\px,height=164\px]{blog/Parque_Narino_Pasto.jpg}
\end{wrapfigure}
}En Colombia hay un control permanente de varios volcanes activos, el principal de ellos es el Galeras, cerca de la población de San Juan de Pasto en el Departamento de Nariño.
 Ese control nos permite saber con una buena anticipación cuándo el volcán va a rugir.
 El problema, sin embargo, es que el Galeras ruge mucho pero aún no se despierta del todo.
 Lo que nos lleva al tercer peligro de una tragedia: la gente se acostumbra.
 En términos geológicos, un volcán que no explota en cientos de años es todavía un volcán activo.
 Pero en términos de la experiencia humana, cientos de años son muchas generaciones que siguen creciendo con la certeza de que nunca ha pasado nada y, por lo tanto, no pasará.
 Cada que el Volcán Galeras va a rugir, las autoridades advierten a la población y cada vez menos la población hace caso de la advertencia de las autoridades.
 Cuando el Galeras se decida finalmente a despertar, los pobladores a su sombra ignorarán el peligro.
 Aún no conocemos lo suficiente para distinguir un simple rugido de una erupción mayor... y tal vez necesitaremos una
 tragedia con varios muertos para aprender.

\par% p
En la actual emergencia invernal en Colombia vemos cierta falta de previsión en nuestras autoridades (nacionales y locales) que no han establecido completamente los riesgos y han permitido asentamientos donde se sabe que hay peligro.
 Pero también una falta de previsión en la misma población que ignora a las autoridades y construye en la falda de
 montañas que se derrumbarán o en la ronda de ríos que \relax{% {'style': 'text-decoration: line-through;'}
se desbordarán }recuperarán su máximo.
 También hay problemas de empresarios inescrupulosos que explotan los recursos sin la suficiente previsión para evitar
 tragedias; costumbres de cultivar que favorecen la deforestación y con ellos ciertos amortiguadores naturales del
 clima, y urbanizadores piratas que promueven asentamientos donde los riesgos son muy altos.

Pero aun cuando sepamos todo, aún cuando conozcamos los riesgos y seamos capaces de preveerlos y de planear teniendolos
 en cuenta, hay otro peligro: los recursos.

Prevenir cuesta.

Cuesta el salario de los investigadores que levantarán los mapas de riesgo; la adquisición de herramientas adecuadas para que hagan su trabajo.
 Cuesta los años de estudio y los honorarios de los arquitectos e ingenieros civiles que diseñarán de acuerdo con esos riesgos para prevenir las tragedias.
 Cuesta construir con calidad sismorresistente o con normas que permitan una infrastructura a prueba de lluvias e
 inundaciones.

La prevención tiene un costo social también, porque menos personas podrán encontrar un techo si exigimos que todos los
 techos tengan una calidad mínima.

Y la prevención tiene un costo político, porque significa gastar recursos en riesgos intangibles, recursos que podrían
 ser usados para cosas más vistosas o en burocracia innecesaria en la práctica pero políticamente lucrativa.

Reparar también cuesta.

Y el costo de la reparación puede se mayor que el de la prevención aún si nuestra única medida de costo es la plata.
 Porque, desde luego, hay cosas que ni la plata repara como son las vidas perdidas.

Pero, políticamente y en la psique de la mayoría de nosotros, es más rentable gastar más en reparar que lo que se gasta en prevenir.
 Porque el costo de reparar es un gasto tangible porque el riesgo ya se manifestó y no el intangible de un riesgo que
 podría no manifestarse.

Es fácil, ex post facto, decir que pudimos prevenir una tragedia.
 Si tan solo los pobres no construyeran en la orilla de los ríos, o si tan solo las corporacionea autónomas regionales hubieran hecho su trabajo, o si tan solo el gobierno hubiera construido los jarillones, o si tan solo las autoridades hubieran atendido las voces de alarma y hubieran obligado a la gente a evacuar.
 Pero, la verdad, me temo que necesitaremos muchas más tragedias para entender que prevenir es realmente una inversión
 que necesitamos hacer.

\chapter{Entre la legalidad y la legitimidad}
\begin{metadata}
	Published by \anchor[chlewey]{chlewey} on \anchor[http://ewey.co/B684]{Thu, 17 Feb 2011 18:20:59 +0000}\\
	\categories{estado, legitimidad, opinion, organizacion, politica}\\
	Shorthand: \anchor[http://blog.chlewey.net/2011/02/entre-la-legalidad-y-la-legitimidad/]{entre-la-legalidad-y-la-legitimidad}
\end{metadata}

Parto de la existencia de un concepto muy preciso: la legalidad.  Lo legal es aquello que permite u ordena la ley
 vigente de un estado, dentro de la jurisdicción de tal estado.  Algo se puede hablar de leyes internacionales, pero
 estos son más acuerdos entre los estados que se convierten en legislación interna.

\anchor[http://commons.wikimedia.org/wiki/File:Buenos\_Aires-Plaza\_Congreso-Pensador\_de\_Rodin.jpg]{\begin{wrapfigure}{r}{240\px}\centering% {'src': 'http://upload.wikimedia.org/wikipedia/commons/2/2c/Buenos_Aires-Plaza_Congreso-Pensador_de_Rodin.jpg', 'title': 'El Pensador de Rodin frente a la Plaza del Congreso', 'height': '180', 'width': '240', 'alt': '[]', 'class': ['alignright']}
\includegraphics[width=240\px,height=180\px]{blog/Buenos_Aires-Plaza_Congreso-Pensador_de_Rodin.jpg}
\end{wrapfigure}
}La legitimidad, por otro lado, es un concepto abstracto que no se refleja necesariamente en las leyes, sino que es
 dictado por la ética, entendiéndo como ética a un conjunto de comportamientos y normas sobre lo que pensamos que es
 correcto o no.  Aunque hay quienes hayan intentado establecer una ``ética universal'', la verdad es que existen
 diferentes escuelas éticas y formas de ver la ética lo que conlleva a que existan diferentes formas de interpretar qué
 es legítimo y qué no.

En algunos casos la legitimidad y la legalidad van de la mano.
 La mayor parte de las filosofías éticas condenan el homicidio y el robo, siendo estos así medios ilegítimos para obtener resultados e, igualmente, la mayor parte de las legislaciones procriben el homicidio y el robo como actos ilegales penalizables.
 Homicidio y hurto hablados en términos generales, porque hay casos en los cuales filosofías y legislaciones permiten
 que un ser humano pierda la vida en manos de otro ser humano, o que permita que una propiedad sea transferida sin
 contraprestación ni la voluntad de ceder del propietario original.

Ejemplo de estos últimos son la pena de muerte, las muertes en combate o el embargo de bienes.

\par% p
Una conversación que tuve durante mis vacaciones en enero planteaba un interesante dilema.
 Muchas ideas políticas actuales plantean que un estado debe funcionar como una empresa.
 Los bienes públicos deben ser bien administrados.
 Las actividades poco lucrativas deben ser abandonadas y las actividades que pueden hacer mejor empresas externas
 pueden entregarse en \emph{outsourcing}.
 Algunos de estos planteamientos son base del así llamado neoliberalismo.
 Pero llevados estos planteamientos al extremo podríamos tener un estado con una corrupción muy baja y con unas finanzas impecables, pero que ha abandonado algunas de las premisas de ser estado.
 Por ejemplo, la capacidad de servir a la población vulnerable.
 (Aunque, desde luego, hay filosofías políticas que consideran a las poblaciones vulnerables como pesos muertos que
 evitan el progreso del resto de la sociedad.)

Entonces, si no podemos llegar a un estado que se autofinancie completamente, es necesario entonces que el estado imponga tributos sobre los súbditos o ciudadanos.
 El ciudadano se ve obligado a pagar estos impuestos y en contraprestación tiene un estado que le provee de vías,
 servicios públicos, bienestar y, sobre todo, de seguridad.

\par% p
Pero los estados no son los únicos que hacen eso.~ El \anchor[http://blog.chlewey.net/2011/01/de-bandas-mafias-e-insurgencias/]{crímen organizado} también lo hace.~ Las bandas de crímen organizado, llámense \abbr{% {'style': 'font-variant: small-caps;'}
bacrim}, autodefensas, pandillas, guerrillas, mafia, etc. exigen contribuciones obligadas a la población con el fin de financiar su causa y a cambio ofrecen algún servicio, sobre todo, de seguridad.
 Quien no acceda a pagar su contribución puede ser sancionado.~ El estado promete carcel a los evasores.~ Las \abbr{% {'style': 'font-variant: small-caps;'}
Farc} promete \del{secuestros}\ins{retenciones}.
 Otros, simplemente advierten que algo le podría pasar a su negocio si no paga la contribución.
 La seguridad que suelen prometer las bandas mafiosas son, principalmente, seguridad ante ellas mismas; aunque algunas mafias también ofrecen protección ante delincuentes comunes y otros indeseables.
 También ofrecen protección ante competencia desleal (como quiera que se quiera definir competencia desleal, por
 ejemplo un productor colombiano puede considerar un producto importado de China como competencia desleal y pedir \del{al estado}\ins{a su mafia favorita} que evite esas importaciones o en su defecto que imponga altos aranceles).

La legitimidad puede verse como aquello que hace aceptable que un estado imponga tributos e inaceptable que una mafia lo haga.
 Para muchos libertarios, ningún estado es legítimo y como tal no son más que las mayores mafias de crimen organizado existentes.
 Pero lo que ofrece el estado: vias, servicios públicos, bienestar y seguridad, bien lo pueden ofrecer empresas
 privadas que operan en un mercado competitivo ofreciendo al consumidor los mejores precios.

\par% p
Cuando el estado (o una mafia) monopoliza un servicio de estos (vías, servicios públicos, bienestar, seguridad), es probable que termine abusando.
 Incluso aunque diga tener un compromiso social, es claro que si el estado no puede cubrir todos sus costos los
 trasladará, vía peajes, tarifas o impuestos, al \del{consumidor}\ins{ciudadano} quien no tendrá más opciones que pagar.  Por el contrario, una empresa con ánimo de lucro en un mercado competitivo
 buscará medios alternativos para cuadrar su caja sin comprometer los precios al público, porque estos precios son los
 que le permiten competir.

La filosofía socialista contrasta con esta filosofía libertaria.  Según esta filosofía el objetivo de una empresa
 capitalista no es competir mejor en un mercado libre sino lograr el mayor lucro posible.  Maximizar sus ganancias.  Y
 una forma de mazimizar ganancias es entrar en acuerdos con la competencia para subir los precios y absorber a la
 pequeña competencia.  Se forman así trusts, cárteles, surge el dumping y otras prácticas que distorcionan el libre
 mercado.

Los servicios básicos sólo están garantizados cuando existe un estado y este no está al servicio de los grandes capitales sino de los ciudadanos.
 Es un estado que controle a las grandes empresas para que estas no abusen. Es un estado que no delega en la empresa privada asuntos sensibles como la seguridad ciudadana sino que monopoliza la fuerza.
 Este estado socialista ideal es incluso lo suficientemente sabio para no permitir que los propios ciudadanos,
 encantados por los cantos de sirenas, lleguen a escoger limitaciones a su propia libertad.

Estos son dos extremos: la sociedad libertaria sin estado, o el sabio y omnipresente estado socialista.
 Ambas filosofías de una forma u otra deslegitiman al estado democrático liberal: el uno porque sí es un estado, el
 otro porque no garantiza poner al ciudadano por encima de los intereses privados (olvidando que cada ciudadano tiene
 sus propios intereses privados).

Hay variaciones.
 Los anarcosocialistas coinciden con los libertarios en que ningún estado es legítimo sólo que cambian el concepto de empresas capitalistas compitiendo en un mercado libre por empresas cooperativas autogestionadas.
 Otros proponen estados que sean básicamente reguladores o, incluso, no más que estados árbitros: en estos estados la libre competencia entre empresas privadas prestan todos los servicios salvo la resolución de disputas que pueden ser prevenidas de antemano (estado regulador) o a posteriori (estado árbitro).
 Los minarquistas proponen que el estado sólo controle a la fuerza pública, p. ej. manteniendo un ejército que defienda
 a la población ante ataques por otros estados.

En fin, hay muchos modelos políticos como escuelas filosóficas sobre la legitimidad del estado existen.

¿Qué hace al estado legítimo y a una mafia no?

¿Qué diferencia a un estado de una mafia?

Si bien no hay una respuesta universalmente aceptable, sí hay indicios.
 En un estado liberal democrático no hay un solo dueño del estado sino que los funcionarios lo son gracias al pueblo soberano que los eligió o eligió a sus partidos.
 Los funcionarios de carrera sólo llegan hasta cierto nivel, por encima del cual está el pueblo o las instituciones que
 representan al pueblo como un filtro para seguir ascendiendo.

Si bien eso suena muy ideal sí hay un contraste grande con la mayoría de las empresas de crímen organizado.
 En estas sólo los funcionarios de carrera pueden llegar a las áltas dignidades donde compran su cupo por medio de alianzas y balas.
 Ninguna mafia que dice representar a una población consulta a esa población cómo se debe administrar la mafia.
 Sólo se impone.

El estado democrático liberal puede no ser tan idílico como lo acabo de describir, pero se legitima en la medida en la que el pueblo acepta participar y participa sin temores y tiene una capacidad real de elegir.
 Así sea eligiendo los cantos de sirena de un político corrupto.
 Los estados socialistas, minarquistas, etc. tienen sus propias legitimidades que, usualmente, son más válidas y más
 universalmente aceptables que las legitimidades de los grupos de delincuencia organizada.

\chapter{Yo creo en el amor}
\begin{metadata}
	Published by \anchor[admin]{admin} on \anchor[http://ewey.co/B695]{Fri, 10 Dec 2010 20:10:48 +0000}\\
	\categories{amor, guest, matrimonio}\\
	Shorthand: \anchor[http://blog.chlewey.net/2010/12/yo-creo-en-el-amor/]{yo-creo-en-el-amor}
\end{metadata}

\par% div% {'style': 'margin: 0 1em 1ex; font-size: 0.91em;'}
Bloguero invitado: José Luis Peñaredonda\anchor[\#i\_jlp]{${}^\textrm{*}$}

''Hasta que la muerte los separe'', dice el cura. Ellos sonríen. La muerte y el  tiempo que los separa de ella es la
 última de sus preocupaciones. Ahora viene la  ceremonia y la mamá de la novia está con los pelos de punta debajo del
 litro de  laca que fija su peinado. Vienen los hijos, sacrificios, presupuestos y  préstamos de vivienda, pero ya
 habrá tiempo para pensar en todo eso. Primero los  gozosos.

Es que toda la gente que yo he conocido se ha casado ebria. No  de alcohol al estilo Las Vegas, por supuesto. Ebrias de
 amor, de sueños. O, como  dicen los neurocientíficos, de endorfinas. Si uno les hace caso a estos seres  extraños de
 bata blanca que dicen que tienen todas las respuestas, estar  enamorado es más o menos como haber consumido
 anfetaminas por un tiempo  relativamente largo. Pero si uno los toma en serio, la vida se convierte en algo  tan
 emocionante como una visita a la droguería.

Por eso yo prefiero creer  en una versión cursi del asunto. Crecí soñando con una versión del amor eterno.  que
 consiste en tener a alguien con el quien a pasear de la mano cuando tengamos  90 años y nuestra vida sexual sea un
 recuerdo borroso. Y ese sueño es  horriblemente problemático, no tengo que decir por qué. Especialmente cuando
 renunciar a ese sueño es difícil. Muy difícil. Puedo construir argumentos  racionales y filosos como una espada, pero
 esa vocecita molesta se las arregla  para no callarse.

Y eso que solo he hablado del amor. No he hablado del  matrimonio, una institución cuya cantidad de defectos –todos lo
 sabemos– es  enorme. Otro sueño de esos que muchos cultivan toda la vida es tener una boda  grande, bonita y costosa
 con todos los amigos. Y para ellas, con todas las  amigas. Creo que, para la mayoría de las mujeres, las bodas son el
 equivalente  de los carros para muchos hombres. Entre más grande y ostentoso, mejor. Sospecho  que esa es una de las
 muchas rivalidades que las mujeres creen que mimetizan con  la melosería y la 'queridura'. El sueño no es solo tener
 una boda bonita. Es  tener la boda más bonita entre las bodas de las amigas.

El asunto es que  el amor eterno, para muchas personas, implica al matrimonio. Para ellos, cumplir  el sueño del amor
 eterno los obliga a casarse. Pero ese no es el punto. El punto  es: ¿A qué se debe que persigamos sueños que no nos
 convienen, por decirlo  suavemente? Porque, creo, necesitamos de los sueños para darles sentido a  nuestras
 realidades. Yo creo en el amor porque –lo confieso– sueño con el amor  para toda la vida. Si no creyera en él no me
 hubiera dado la oportunidad de  enamorarme. Pero enamorarme no me ha hecho olvidar todos los defectos y  problemas del
 amor. Insisto en ello porque ese sueño del amor eterno me ayuda a  levantarme todas las mañanas y a sobrellevar todas
 esas cosas malas.

\par% p
Pero  yo no creo en el matrimonio. Nunca he soñado con casarme, ni creo en la relación  de necesidad que algunos trazan
 entre casarse y amar a alguien por toda la vida.  Pero para alguien que cree en el matrimonio y sueña con estar casado
 con alguien  para toda la vida, estar casado es vital. El matrimonio es difícil. Pero no  claudicar en él es necesario
 para tener razones para seguir viviendo.

\par% div% {'style': 'margin: 1ex 1em;'}
Por \textbf{\anchor{José Luis Peñaredonda}}. (@\anchor[http://twitter.com/noalsilencio]{noalsilencio})
Autor de \anchor[http://elrestodelcorcho.wordpress.com/]{\textbf{\emph{El resto del corcho}}}.

\chapter{Por qué vale la pena casarse}
\begin{metadata}
	Published by \anchor[admin]{admin} on \anchor[http://ewey.co/B700]{Fri, 10 Dec 2010 20:20:04 +0000}\\
	\categories{guest, matrimonio}\\
	Shorthand: \anchor[http://blog.chlewey.net/2010/12/por-que-vale-la-pena-casarse/]{por-que-vale-la-pena-casarse}
\end{metadata}

\par% div% {'style': 'margin: 0 1em 1ex; font-size: 0.91em;'}
Bloguero invitado: Hacemeun14\anchor[\#i\_jmca]{${}^\textrm{*}$}

Hace unos años cuando estaba entre mis 20 y 25 años pensaba que el matrimonio era algo que nunca iba a estar en mis
 planes; como cualquier ser humano de esa edad me creía inmortal y temerario y dado que mi hígado estaba en perfecto
 funcionamiento y mi colon no se manifestaba de ninguna manera pues me la pasaba de fiesta en fiesta.

Eran tiempos bonitos, una novia joven y atractiva y muy pocas responsabilidades, ¡oh juventud, divino tesoro! Pero esos
 años pasan tan rápido como se disfrutan. De pronto uno deja de ser el audaz jovencito que frecuenta los sitios de moda
 y tiene miles de amigos, para convertirse en un adulto joven con barriga incipiente.

Las visitas al médico son ahora frecuentes y el vendedor de la droguería ya sabe que uno va por una caja de Omeprazol.
 El inevitable paso del tiempo y el reloj biológico le pellizcan a uno el culo y bueno, de repente la idea de casarse
 se considera seriamente.

Entonces uno dice ¿por qué no? Casarse con la novia de toda la vida, esa mujer que lo hace a uno sonreír, conoce sus
 gustos, sus más oscuros secretos y la ruta indicada para una segura y deliciosa ‘petit morte’ conjunta.

Seguramente ya no serán los fines de semana de amigos, juegos de mesa, asados y borracheras memorables; pero estarán la
 inversión millonaria en la boda, la conversión inmediata al catolicismo para hacer el cursillo y poder casarse, la
 fiesta y el recibir regalos repetidos. Estarán las idas al supermercado para hacer mercado, contratar a una señora
 para que cocine y haga el aseo, pagar la cuota de la casa – carro – posgrado, y por supuesto, hacer un avance con la
 tarjeta de crédito… otro… otro… el último de este mes.

Ya no habrá que lidiar con los amigos borrachos ni ir a recoger los controles del Xbox a la casa de Daniel; ahora habrá
 que discutir con el plomero porque ¡Cómo va a costar tanto ese arreglo! Y así sucesivamente, una novedad tras otra,
 una nueva responsabilidad y una noche más en la que uno se da cuenta de que ¡nos dormimos sin tener sexo!

Pero en el fondo de todo uno busca eso, tranquilidad, saber que tiene donde llegar a dormir y donde comer, la sensación
 placentera del abrazo y los besos honestos de la otrora novia joven y atractiva, ahora esposa despelucada y ‘gordita’.

El matrimonio vale la pena, pues como diría un amigo ‘así uno siempre tiene qué hacer los viernes por la noche’.

@hacemeun14

\par% p
Nota: ¡yo si me quiero casar!

\par% div% {'style': 'margin: 1ex 1em;'}
Por \textbf{\anchor{}Hacemeun14}. (@\anchor[http://twitter.com/hacemeun14]{hacemeun14})
Autor de \anchor[http://hacemeun14.blogspot.com/]{\textbf{\emph{¿Me podés hacer un 14? }}}

\chapter{Cuestión de género}
\begin{metadata}
	Published by \anchor[chlewey]{chlewey} on \anchor[http://ewey.co/B703]{Thu, 13 Jan 2011 18:02:24 +0000}\\
	\categories{activismo, genero, gramatica, opinion}\\
	Shorthand: \anchor[http://blog.chlewey.net/2011/01/cuestion-de-genero/]{cuestion-de-genero}
\end{metadata}

\anchor[http://tumblr.chlewey.net/post/2826834275/que-tiene-intrinsecamente-un-clavo-que-lo-haga]{\begin{wrapfigure}{r}{160\px}\centering% {'src': 'http://blog.chlewey.net/wp-content/uploads/2011/01/clavo-120.png', 'title': 'clavo y puntilla', 'height': '120', 'width': '160', 'alt': '[clavo y puntilla]', 'class': ['size-full', 'wp-image-721', 'alignright']}
\includegraphics[width=160\px,height=120\px]{blog/clavo-120.png}
\end{wrapfigure}
}¿Qué tiene intrínsecamente un clavo que lo haga macho y una puntilla que la haga hembra?

\par% p
Últimamente se ha vuelto común hablar de\emph{ cuestiones de género} para referirse a los asuntos de sexo.
 Una breve revisión de lo que la Real Academia Española nos define en su Diccionario de la lengua, nos aclara:

%\begin{tabular}{lllllllll}% FIX

\begin{description}
 \item[género]\anchor[http://buscon.rae.es/draeI/SrvltConsulta?TIPO\_BUS=3\&LEMA=g\%C3\%A9nero]{.}  (Del lat. \emph{genus}, \emph{genĕris}).
\begin{enumerate}

\item \relax{% {'style': 'color: #666;'}
m.} Conjunto de seres que tienen uno o varios caracteres comunes.
\item \relax{% {'style': 'color: #666;'}
m.} Clase o tipo a que pertenecen personas o cosas. \emph{% {'style': 'color: #80d;'}
Ese género de bromas no me gusta}
\item \relax{% {'style': 'color: #666;'}
m.} En el comercio, mercancía.
\item \relax{% {'style': 'color: #666;'}
m.} Tela o tejido. \emph{% {'style': 'color: #80d;'}
Géneros de algodón, de hilo, de seda}
\item \relax{% {'style': 'color: #666;'}
m.} En las artes, cada una de las distintas categorías o clases en que se pueden ordenar las obras según rasgos comunes de
 forma y de contenido.
\item \relax{% {'style': 'color: #666;'}
m.} \emph{% {'style': 'color: #00d;'}
Biol.} Taxón que agrupa a especies que comparten ciertos caracteres.
\item \relax{% {'style': 'color: #666;'}
m.} \emph{% {'style': 'color: #00d;'}
Gram.} Clase a la que pertenece un nombre sustantivo o un pronombre por el hecho de concertar con él una forma y,
 generalmente solo una, de la flexión del adjetivo y del pronombre. En las lenguas indoeuropeas estas formas son tres
 en determinados adjetivos y pronombres: masculina, femenina y neutra.
\item \relax{% {'style': 'color: #666;'}
m.} \emph{% {'style': 'color: #00d;'}
Gram.} Cada una de estas formas.
\item \relax{% {'style': 'color: #666;'}
m.} \emph{% {'style': 'color: #00d;'}
Gram.} Forma por la que se distinguen algunas veces los nombres sustantivos según pertenezcan a una u otra de las tres clases.

\end{enumerate}

\end{description} %&
\begin{description}
 \item[sexo]\anchor[http://buscon.rae.es/draeI/SrvltConsulta?TIPO\_BUS=3\&LEMA=sexo]{.}  (Del lat. \emph{sexus}).
\begin{enumerate}

\item \relax{% {'style': 'color: #666;'}
m.} Condición orgánica, masculina o femenina, de los animales y las plantas.
\item \relax{% {'style': 'color: #666;'}
m.} Conjunto de seres pertenecientes a un mismo sexo. \emph{% {'style': 'color: #80d;'}
Sexo masculino, femenino.}
\item \relax{% {'style': 'color: #666;'}
m.} Órganos sexuales.
\item \relax{% {'style': 'color: #666;'}
m.} Placer venéreo. \emph{% {'style': 'color: #80d;'}
Está obsesionado con el sexo.}

\end{enumerate}

\end{description} %\\

%\end{tabular}

Básicamente el \emph{género} \relax{% {'style': 'color: #444;'}
(acepciones 7 a 9 de género)} es un asunto gramatical el cual biológicamente \relax{% {'style': 'color: #444;'}
(acepción 6 de género)} no corresponde al \emph{sexo} \relax{% {'style': 'color: #444;'}
(acepciones 1 y 2 de sexo)}.~ Así que ¿cuál es el género del que tratan las \emph{cuestiones de género}?~ ¿Sería a la clase \relax{% {'style': 'color: #444;'}
(acepción 2 de género)} de sexo \relax{% {'style': 'color: #444;'}
(acepción 2 de sexo)}?~ Una forma muy vaga de referirnos a una de las muchas formas como podemos tipificar o dividir a los seres humanos.

\par% p
Hay dos motivos por el cual creo que se ha popularizado hablar de \emph{cuestiones de género} a los asuntos sociales que afectan de forma diferente a hombres y mujeres y es porque hablar de \emph{cuestiones de sexo} tiene dos problemas.~ El primero es que, fuera de contexto, la palabra \emph{sexo} bien puede ser interpretada por su acepción 4: \emph{placer venéreo}.
 Las cuestiones de sexo parecerían sugerir en el discurso moderno lo que se habla o no sobre sexualidad.
 El otro motivo sería que al hablar de \emph{cuestiones de genero} y no de \emph{cuestiones de sexo} se quiere dar un mayor énfasis a los aspectos sociales y no biológicos que diferencian a hombres y mujeres.

En la mayoría de países, incluso en los más igualitarios, un número importante de mujeres se encuentran subordinadas en los hogares a los hombres y por unas similares condiciones de trabajo laboral, el promedio salarial de la mujer es inferior a la del hombre.
 Esas cuestiones pretenden ser tratadas como un problema social (cuestión de género) y no de biología (cuestión de sexo): aunque claramente tiene un origen biológico: en promedio el cuerpo de la hembra humana adulta es más pequeño que el del macho humano adulto y la reproducción le roba a la hembra meses de trabajo en contraste con tan solo unos minutos al macho.
 Esas diferencias biológicas han contribuído a la noción tradicional de la mujer realizando tareas en el hogar,
 mientras el hombre sale a cazar el sustento y pelea por el territorio.

Pero no quiero hablar de esas cuestiones de género sexual, sino de las cuestiones del género gramatical.

\par% p
El género es una característica que tienen ciertas lenguas por medio de la cual los substantivos generan diferencias en cómo se declinan o conjugan otras palabras.
 Si en la oración \textbf{% {'style': 'font-weight: normal; font-style: italic; color: #80d;'}
Juan tiene un clavo negro}, cambio \emph{clavo} por \emph{puntilla}, el cambio se reflejará en otros dos lugares: \textbf{% {'style': 'font-weight: normal; font-style: italic; color: #80d;'}
Juan tiene un\emph{% {'style': 'text-decoration: underline;'}
a} puntilla negr\emph{% {'style': 'text-decoration: underline;'}
a}}.
 El artículo y el adjetivo cambian y no porque hayamos pasado de un macho clavo a una puntilla hembra.
 Ahora, si pregunto por la longitud del clavo me dirán que \textbf{% {'style': 'font-weight: normal; font-style: italic; color: #80d;'}
\emph{% {'style': 'text-decoration: underline;'}
Él} es larg\emph{% {'style': 'text-decoration: underline;'}
o}}, mientras que de la puntilla me dirán que \textbf{% {'style': 'font-weight: normal; font-style: italic; color: #80d;'}
\emph{% {'style': 'text-decoration: underline;'}
Ella} es larg\emph{% {'style': 'text-decoration: underline;'}
a}}.
 Ahora bien, aunque querramos establecer diferencias entre clavos y puntillas, en la gran mayoría de los casos no es
 más que haber escogido una palabra u otra para el mismo objeto.

\par% p
Igualmente podríamos decir que \textbf{% {'style': 'font-weight: normal; font-style: italic; color: #80d;'}
Juan es \emph{% {'style': 'text-decoration: underline;'}
una} persona muy lind\emph{% {'style': 'text-decoration: underline;'}
a}}, o que \textbf{% {'style': 'font-weight: normal; font-style: italic; color: #80d;'}
María es \emph{% {'style': 'text-decoration: underline;'}
un} \anchor{bollito} precios\emph{% {'style': 'text-decoration: underline;'}
o}}, sólo para expresar que \textbf{% {'style': 'font-weight: normal; font-style: italic; color: #80d;'}
Juan es muy lind\emph{% {'style': 'text-decoration: underline;'}
o} como persona} o que \textbf{% {'style': 'font-weight: normal; font-style: italic; color: #80d;'}
María es precios\emph{% {'style': 'text-decoration: underline;'}
a} y atractiv\emph{% {'style': 'text-decoration: underline;'}
a}}.~ Claramente aquí el género por el cual declina el adjetivo sigue al género de la palabra y no al sexo del sujeto.

Las lenguas indueropeas, que incluyen al español, el inglés y el alemán, entre otras, tradicionalmente manejan tres géneros.
 La mayoría de palabras que se refieren a varones y a animales machos toman el género gramatical 1; la mayoría de palabras que se refieren a mujeres y a animales hembra toman el género gramatical 2.
 La mayoría de palabras que no se refieren a seres vivos se reparten entre los géneros 1, 2 y 3.
 Esto ha motivado a que el género 1 se llame ``masculino'', el género 2 se llame ``femenino'' y el género 3 termina como
 ``neutro''.

\par% p
En inglés, este sistema de géneros ha desaparecido casi completamente.~ Sólo se conserva en el uso de los pronombres (\emph{he}, \emph{she}, \emph{it}) y casi reservado al sexo de las personas.~ Una persona normal que se refiera en inglés a una vaca (\emph{cow}: hembra) utilizará el pronombre neutro \emph{it}.~ Si una persona usa el pronombre femenino \emph{she} para referirse a la vaca estará denontando una relación especial con la vaca o una personificación de la misma.
 Podríamos incluso usar el pronombre masculino \emph{he} para referirnos a una vaca como \anchor[http://www.nick.com/shows/back-at-the-barnyard/characters/otis.html]{Otis}, de \emph{La Granja} (\emph{\anchor[http://www.nick.com/shows/back-at-the-barnyard]{Back at the Barnyard}}), que si bien es biológicamente una vaca hembra, su personificación es masculina.

Por otro lado el inglés ha desarrollado otro sistema de clases de substantivos que suele tener una relevancia más importante en la gramática: substantivos contables y no contables.
 He visto gramáticos que consideran esta división como los verdaderos géneros del idioma inglés moderno.

\par% p
En español, el género neutro quedó fusionado con el masculino en casi todos los aspectos.
 El género neutro sólo se usa para distinguir conceptos como \emph{el negro} (el color negro, el hombre negro, el clavo negro) de \emph{lo negro} (conjunto abstracto de cosas que se caracterizan por ser negras).
 Por ello, aunque lo neutro no ha desaparecido, el neutro si es un género casi inexistente en el español.

\par% p
El alemán sí conserva este esquema de tres géneros, y esto nos permite observar cosas como que \emph{Mensch} (ser humano) es \del{% {'datetime': '2011-01-16T16:37:32+00:00'}
femenino} \ins{% {'datetime': '2011-01-16T16:37:32+00:00'}
masculino} así se refiera a un varón o a una mujer, mientras que \emph{Kind} (niño) es neutro, bien se trate de niña o varoncito.~ Bueno, en español \emph{persona} es femenino, también.

\par% p
Palabras masculinas son miembro, orangután, gorila, \anchor{gurre}, \anchor{bollito}.  Palabras femeninas son persona, cebra, girafa, res, lombriz, águila.~ Por ejemplo \textbf{% {'style': 'font-weight: normal; font-style: italic; color: #80d;'}
\emph{% {'style': 'text-decoration: underline;'}
una} cebra macho es tan rayad\emph{% {'style': 'text-decoration: underline;'}
a} como la hembra}.~ El sexo de la cebra no cambia ni el artículo ni los adjetivos.~ Hay casos que confunden, por ejemplo \textbf{% {'style': 'font-weight: normal; font-style: italic; color: #80d;'}
\emph{% {'style': 'text-decoration: underline;'}
un} águila calv\emph{% {'style': 'text-decoration: underline;'}
a} macho será tan cabeciblanc\emph{% {'style': 'text-decoration: underline;'}
a} como la hembra}.
 El adjetivo sigue el género femenino del substantivo mientras que el artículo masculino no es poque el águila sea macho, sino por cuestiones fonéticas.
 El águila hembra también usa el artículo masculino.

\par% p
En cuanto a los gorilas, es igualmente válido decir \emph{% {'style': 'color: #80d;'}
el gorila hembra} que \emph{% {'style': 'color: #80d;'}
la gorila}.~ En muchos casos un substantivo invariante podrá declinar con el artículo y el adjetivo: \emph{% {'style': 'color: #80d;'}
el gorila} / \emph{% {'style': 'color: #80d;'}
la gorila}; en otros casos el cambio de género de un substantivo invariante puede acarrear diferencias de fondo.~ Por ejemplo: \emph{% {'style': 'color: #80d;'}
la escolta} es el séquito que garantiza la seguridad de una persona mientras que \emph{% {'style': 'color: #80d;'}
el escolta} es un miembro masculino de tal séquito.~ Un miembro femenino de una escolta es también \emph{% {'style': 'color: #80d;'}
una escolta}.

\par% p
Así las cosas, me gusta ver el género masculino y femenino exclusivamente como un componente gramatical de la lengua
 española, y no como una cuestión social que hay que corregir con adefesios tales como ese \emph{lenguaje no excluyente}, que nos invita a que los escoltas y las escoltas del español y de la lengua española se encarguen de cuidar la manera
 como nosotros y nosotras escribimos para resarcir por medio de los lápices y las plumas los problemas y las
 inequidades sociales de las relaciones entre hombres y mujeres en nuestro medio.

\chapter{Matrimonio: un buen negocio}
\begin{metadata}
	Published by \anchor[admin]{admin} on \anchor[http://ewey.co/B704]{Mon, 13 Dec 2010 21:39:06 +0000}\\
	\categories{empresa, guest, matrimonio}\\
	Shorthand: \anchor[http://blog.chlewey.net/2010/12/matrimonio-un-buen-negocio/]{matrimonio-un-buen-negocio}
\end{metadata}

\par% div% {'style': 'margin: 0 1em 1ex; font-size: 0.91em;'}
Bloguero invitado: Andrés Meza Escallón\anchor[\#i\_ame]{${}^\textrm{*}$}

Yo soy de los que piensa que casarse es como montar una empresa en sociedad: no es para todo el mundo y lo más
 importante es que la socia le inspire a uno una confianza de proporciones bíblicas.

\par% p
En efecto, montar una empresa no es para todo el mundo, por lo que la mayoría de la gente se siente cómoda en el papel
 de empleado de alguien más. Claro, también están los independientes o \emph{FreeLancer}, quienes padecen las desventajas de los empresarios y las desventajas de los empleados con casi ninguno de sus
 beneficios, pero esa es otra historia. Para propósitos de nuestra analogía, sigamos con los casados / empresarios.

\par% p
Primero, es clarísimo que casarse es como tener una empresa con una socia y una asamblea de socios chiquitos pero cansones. ¿O es que creen que los suegros, padres, tíos, hijos no quieren meter la cucharada en sus decisiones “de pareja“?
 La sabiduría popular ya advierte que “\emph{uno se casa con la esposa y la familia de la esposa}” y viceversa.

Segundo, montar una empresa o casarse implica tener recursos o estar en condiciones para generarlos. También implica
 obligaciones (de ahí el tradicional “el que tiene tienda, que la atienda”) que van desde las más obvias (como
 cumplirle a los proveedores, empleados y clientes) hasta las que uno desde afuera no ve (tributarle al Estado o llevar
 registros contables). En el caso del matrimonio, obviamente se espera de los cónyuges que tengan tanto sexo como sea
 posible y sostengan los gastos de la casa, pero también que respeten el contrato que firmaron.

También cuando se monta una empresa se debe tener muy clara la razón social, o el propósito para el que creó. Si es una fundación, debe tener clara su función social, si es con ánimo de lucro, de tener muy claro cómo espera obtener beneficios. Obviamente ambos enfoques no son excluyentes, pero se debe tener muy claro cuál es el prioritario. Por el lado del matrimonio, se debe tener claro para qué se casan los cónyuges:
 ¿para criar una familia? ¿Para consolidar un patrimonio? ¿Para emprender un proyecto de vida conjunto? Nuevamente,
 estos propósitos no son excluyentes, es más, se espera que se complementen, pero si no se tiene en mente ninguno de
 esos tres, ¿para qué casarse?

\par% p
Cuando tenemos empresas también se debe tener claro que a veces se gana y a veces se pierde. Pero si después de un tiempo uno se da cuenta de que la relación beneficio / costo ya no mayor que cero, ¿para qué
mantener la empresa?
 Con los matrimonios pasa lo mismo: si el balance entre las cosas positivas y las negativas ya no es positivo, pues
 ambos deben hacer una reingeniería para rectificar el rumbo o irse cada uno por su lado. Bien lo decía Chiquinquirá
 Blandón: “\emph{un amor que sirva o un adiós que libere}”.

Por otro lado, cuando monta una empresa en sociedad, más importante que la socia sea una genio de las finanzas, es que
 inspire confianza. Si sospecha que la socia le puede robar o que no es competente para encargarse de la empresa
 temporalmente si llega a faltar, ¿para qué arriesgarse?

\par% p
Después de todas estas consideraciones, quienes todavía quieran casarse pueden estar haciendo un mejor negocio que
 estando solteros. Un matrimonio es una plataforma que da suficiente estabilidad para arriesgarse en proyectos de largo
 plazo (tener casa propia, criar hijos y nietos, desarrollar una carrera, etc.). Y para muchos, esos son los proyectos
 que valen la pena.

\par% div% {'style': 'margin: 1ex 1em;'}
Por \textbf{\anchor{Andrés Meza Escallón}}. (@\anchor[http://twitter.com/ApoloDuvalis]{ApoloDuvalis})
Autor de \anchor[http://apoloduvalis.blogspot.com]{\textbf{\emph{La cantera de palabras}}}

\chapter{De bandas, mafias e insurgencias}
\begin{metadata}
	Published by \anchor[chlewey]{chlewey} on \anchor[http://ewey.co/B714]{Wed, 19 Jan 2011 17:30:45 +0000}\\
	\categories{actualidad, guerrilla, information, narcotrafico, paramilitarismo}\\
	Shorthand: \anchor[http://blog.chlewey.net/2011/01/de-bandas-mafias-e-insurgencias/]{de-bandas-mafias-e-insurgencias}
\end{metadata}

\par% p
Anoche en \anchor[http://www.caracol.com.co/programa.aspx?id=130992]{Hora 20} se daba una discusión sobre la naturaleza de las ahora llamadas \abbr{% {'style': 'font-variant: small-caps;'}
bacrim} (por \emph{\relax{% {'style': 'text-decoration: underline;'}
ba}ndas \relax{% {'style': 'text-decoration: underline;'}
crim}inales}), término acuñado por el gobierno anterior para distinguir estas nuevas modalidades de violencia de los hasta entonces llamados paramilitares o autodefensas.
 Según \anchor[http://es.wikipedia.org/wiki/Mar\%C3\%ADa\_Jimena\_Duz\%C3\%A1n]{María Jimena Duzan}, las \abbr{% {'style': 'font-variant: small-caps;'}
bacrim} no son más que paramilitares, mientras que otros panelistas buscaban otras formas de denominarlos, sin faltar la
 comparación con los \anchor[http://es.wikipedia.org/wiki/Pandilla\#Maras]{maras}.

\par% p
Difiero en cierta forma de la apreciación de Duzan, no tanto en el fondo sino más bien en la forma.
 Yo creo que las Autodefensa Unidas de Colombia (\abbr{AUC}), sus grupos previos y adjuntos no eran más que bandas criminales con una pretención de discurso político, y que
 técnicamente no eran paramilitares.

\anchor[http://commons.wikimedia.org/wiki/File:Cali\_chart2.jpg]{\begin{wrapfigure}{r}{250\px}\centering% {'src': 'http://upload.wikimedia.org/wikipedia/commons/c/c3/Cali_chart2.jpg', 'title': 'Organigrama del Cartel de Cali', 'height': '160', 'width': '250', 'alt': '', 'class': ['alignright']}
\includegraphics[width=250\px,height=160\px]{blog/Cali_chart2.jpg}
\end{wrapfigure}
}La mayor parte de los grupos de delincuencia organizada poseen ciertos elementos en común.
\begin{enumerate}

\item Poseen una base social con la que pretenden algún tipo de identificación y de la que muchas veces esta identificación es real.
 Esta base social es también base del reclutamiento.
\item Se convierten en el estado de facto dentro de esta clase social, permeando también al estado de jure por intimidación,
 compra o infiltración de los agentes estatales formales.
\item Usan la violencia letal como una forma de preservar el poder tanto al interior como frente a grupos rivales de
 delincuencia organizada y frente a la autoridad formal.
\item Se financian por medio de actividades ilegales como la extorsión, el contrabando, el hurto, la creación de monopolios y
 contratos fraudulentos con el estado.

\end{enumerate}

Dentro de estos elementos, si bien habrá grupos de delincuencia organizada que no pretenden justificarse, casi siempre estos grupos buscan algún tipo de justificación.
 No se autoconsideran criminales sino como una estructura de poder necesaria dentro de las circumstancias de su
 comunidad.

Cuando hablamos de crimen organizado, Holiwood nos lleva a pensar en la mafia siciliana en los EE.UU.
 En la Sicilia rural la mafia era una forma de vida, un estado de facto, basado en leyes de lealtad que funcionaban ante la ausencia de un estado formal impuesto desde Palermo o desde Roma.
 La fuerza letal era una forma de justicia interna y una forma de definir las diferencias entre los distintos señores mafiosos.
 Cuando los sicilianos inmigraron a los EE.UU. mantuvieron esos esquemas sociales mafiosos, pero lo extendieron a casi toda la comunidad inmigrante italiana.
 La seguridad otorgada a la población base derivó en extorsión y frente a las políticas puritanas en EE.UU. que prohibían el juego, la prostitución y el alcohol, pronto estas mafias econtraron que estos negocios ilegales eran muy lucrativos.
 Ahí están los cuatro elementos del crimen organizado, en este caso autojustificado en tradiciones ancestrales
 sicilianas.

\par% p
Un ejemplo de justificación son los grupos guerrilleros autodenominados marxista-leninistas como las Fuerzas Armadas
 Revolucionarias de Colombia, Ejército del Pueblo (\abbr{% {'style': 'font-variant: small-caps;'}
Farc-EP}).
 Ellos tienen un discurso de revolución popular comunista, se autodenominan defensores del pueblo colombiano y justifican estar en armas como su imperioso deber de combatir las estructuras burguesas del poder en Colombia.
 Poseen los cuatro elementos de las delincuencias comunes: una base social (el pueblo colombiano entendiéndose como los
 pobres sometidos a las clases burguesas, bien sea que estos existan realmente o se los inventen), funciones de estado
 paralelo, violencia letal al interior, frente a otras guerrillas y bandas criminales y frente al estado, y
 financiación con actividades ilícitas.

\par% p
No me interesa indagar cuánto de su discurso es auténtico ni cuanto una mera \anchor[http://blog.chlewey.net/2008/01/injusticia-social-bullshit/]{pretención de justificación}.~ Las \abbr{% {'style': 'font-variant: small-caps;'}
Farc}, el \abbr{ELN} y otras guerrillas son una modalidad de delincuencia organizada.
 Si eliminásemos el punto de financiación con actividades ilícitas podríamos presentar un caso de que los grupos insurgentes sean algo diferente, pero no es lo que sucede: son delincuencia.
 Delincuencia con discurso político, pero delincuencia al fin y al cabo.
 Sus defensores dirán que esa financiación por actividades ilícitas es necesaria, o no es muy disímil a la forma como
 los estados \emph{formales} se financian, pero no hablaré de justificaciones sino de hechos.

\par% p
Los así llamados paramilitares o autodefensas (integrados o no a las \abbr{AUC}) también fueron delincuencia organizada, con todos sus cuatro puntos.

\par% p
No me gusta el término ‘paramilitares’ para referirnos a ese fenómeno (y mucho menos el prefijo ‘para-’), porque esto es una distorsión del concepto de paramilitar.
 El prefijo ‘para-’ viene del griego \GR{παρά}-, \emph{pará} y significa ‘al margen de’ o ‘junto a’. De ahí se deriva ‘paralelo’ y se aplica a términos como ‘paranormal’, ‘paraestatal’, etc.
 En el sentido etimológico del término ‘paramilitar’ es cualquier cosa al margen de lo militar o junto a lo militar, y se refiere a cualquier tipo de organización que posea estructuras similares a los militares pero que no son militares.
 Esto incluye desde los \emph{\del{Boy} Scouts} y las defensas civiles hasta los cuerpos de bomberos, las policías, las insurgencias, etc.

\par% p
El uso más usual del término ‘paramilitar’ por fuera de Colombia se refiere a grupos armados organizados por un estado y que se mantienen por fuera de las fuerzas militares de la nación o de los departamentos de policía.
 Estos grupos suelen responder directamente al gobierno de turno por lo que también se les denomina ‘policía política’.
 Ejemplos son las \emph{\anchor[http://es.wikipedia.org/wiki/Schutzstaffel]{Schutzstaffel}} (\abbr{SS}) del Tercer Reich o los \anchor[http://es.wikipedia.org/wiki/Comit\%C3\%A9s\_de\_Defensa\_de\_la\_Revoluci\%C3\%B3n]{Comités de Defensa de la Revolución} en Cuba.
 Muchos de estos grupos son legales dentro de sus respectivos estados, amparados por decretos, leyes y constituciones.

\par% p
En Colombia, la presencia de las guerrillas comunistas empezó a crear grupos de reacción o autodefensa ante las mismas.
 Algunos de los grupos que se organizaron tenían un origen completamente ilegal como el grupo Muerte A Secuestradores (\abbr{MAS}), creado por el Cartel de Medellín como respuesta a los secuestros perpetrados por el M-19 contra sus familiares.
 Otros grupos como las \abbr{% {'style': 'font-variant: small-caps;'}
convivir} fueron amparados por leyes que pretendían organizar y dar un marco legal a la cooperación civil en la lucha contrainsurgente.
 Por mucho tiempo las fuerzas militares constitucionales de Colombia toleraron, o incluso colaboraron, con estos grupos
 contrainsurgentes, pues dentro de una lógica de guerra, cualquier colaboración frente a un enemigo común está
 justificada.

\par% p
[Por lógica de guerra me refiero a que las Fuerzas Militares constitucionales cambiaron el foco de \textbf{proteger} a la población colombia a \textbf{combatir} a cierta amenaza a la población.]

Por esta tolerancia y colaboración (o tal vez simplemente de oficio), los defensores del discurso guerrillero
 comenzaron a catalogar a todos estos grupos contrainsurgentes, legales o ilegales, como ‘paramilitares’ y ese es el
 término que terminó acuñado frente a los medios y la población en general para referirse a todos esos grupos,
 grupúsculos, bandas, etc. sin importar si estaban al servicio de los terratenientes, los comerciantes o los
 narcotraficantes.

\par% p
Pronto estos ‘paramilitares’ empezaron a adquirir en mayor o menor escala los elementos de la delincuencia organizada: base social, paraestatalidad, violencia letal y financiación ilegal.
 Algunos porque lo eran de orignen: ejércitos privados al servicio del narcotráfico.
 Otros por evolución.~ Parten de una base social: \emph{la gente de bien} en alguna región, entendiéndose a los propietarios de tierras, los comerciantes legales, etc.
 Recurren a la violencia letal como una reacción a la violencia letal de los grupos insurgentes.
 Pasan a la financiación ilegal cuando los aportes de la gente de bien involucrada dejan de ser suficientes y, dentro
 de esta lógica de ilegalidad y violencia letal, terminan convertidos en paraestados (estados paralelos) y buscan
 reemplazar o cooptar al estado formal.

\par% p
En los años 1990, Carlos Castaño, con la muy probable colaboración de personalidades \emph{de bien} de Colombia, emprende el proyecto de organizar a todos los grupos de autodefensas y reunirlos bajo la pretensión de un
 mando único con fuerza de negociación frente al estado: las \abbr{AUC}.~ las \abbr{AUC} incluyeron a grupos de \abbr{% {'style': 'font-variant: small-caps;'}
convivir} ilegalizados, ejércitos privados de narcotraficantes y otras bandas de crimen organizado que supuestamente compartían la lucha contrainsurgente.
 Escudados en la ausencia del estado, empezaron a formarse como un paraestado en sus zonas de influencia y a ampliar su proyecto en zonas donde las autodefensas no habían surgido espontáneamente.
 Salvatore Mancuso quiso ir más allá en su proyecto de paraestatalidad: tomarse al estado formal desde adentro; por lo
 menos en el grado suficiente para garantizar una interlocución con el estado que permitiera legalizar eventualmente el
 proyecto.

\par% p
En su afán de crecer las AUC (que nunca fueron más que una federación de bandas de crímen organizado) y ante la eventual negociación con el estado presidido por Álvaro Uribe Vélez, muchos grupos de narcotraficantes con sus ejércitos privados se unieron a esa federación.
 Finalmente se presentó la negociación, se intentaron legalizar algunas de las actividades criminales, y la
 organización \abbr{AUC} dejó de existir.

\par% p
Pero no todas las autodefensas se federaron a las \abbr{AUC} ni todos los federados se sometieron al estado formal.
 Los negocios vinculados al narcotráfico, entre los muchos otros métodos ilegales de financiación, continuaron y hoy en
 día hay muchos grupos que guardan afinidad en métodos con las \abbr{AUC}, varios de ellos que continúan con un discurso contrainsurgente, pero todos no son más que bandas de crimen
 organizado, como fueron bandas de crimen organizado los afiliados a las \abbr{AUC} antes y durante la vigencia de esta federación.

\par% p
Las ahora denominadas \abbr{% {'style': 'font-variant: small-caps;'}
bacrim} son eso: crimen organizado. No se diferencia de los ‘paramilitares’ no porque las \abbr{% {'style': 'font-variant: small-caps;'}
bacrim} sean paramilitares sino porque los ‘paramilitares’ también eran bandas de crimen organizado.
 La diferencia está en la intensidad del discurso contrainsurgente.~ Hablar de las \abbr{% {'style': 'font-variant: small-caps;'}
bacrim} como herederos del ‘paramilitarismo’ tiene cierto sentido, pero hablar de que las \abbr{% {'style': 'font-variant: small-caps;'}
bacrim} son paramilitares es una imprecisión mayor a decir que las \abbr{AUC} (o sus grupos conformantes) eran paramilitares.

Por otro lado considero que es un error craso decir que los maras son los herederos de las guerrillas centroamericanas.

\par% p
Podemos ver a las pandillas como una forma de crimen organizado.
 Tienen una base social: poblaciones urbanas marginadas con una fuerte base de reclutamiento entre jóvenes que no tienen mayores oportunidades.
 Hasta ahí van las similitudes con las pandillas de \emph{West Side Story}.
 Las pandillas son grupos ilegales que recurren a la violencia letal, tanto al interior de la pandilla como frente a otras pandillas, y a la financiación con actividades ilegales entre las que se incluye la extorsión y el menudeo de drogas ilegales, y en ese proceso se convierten en paraestados.
 Estas pandillas han estado surgiendo en varias ciudades.~ La película \emph{Pandillas de Nueva York} nos recuerda que no es un fenómeno reciente.
 En los años 1980, varias pandillas urbanas negras en los EE.UU. empesaron a reunirse en dos franquicias principales:
 los \emph{Blood} y los \emph{Creeps}.~ Como reacción surgen franquicias como los \emph{Latin Kings} de origen mexicano.~ Muchos grupos de supremacía aria se forman como pandillas con franquicias nacionales en los EE.UU.

\par% p
En este ambiente, una importante población de origen centroamericano, principalmente salvadoreño, desplazados por la
 guerra de los años 1980, terminan viviendo en las barriadas de Los Ángeles Este aprendiendo a convivir entre Latin
 Kings, Bloods y Creeps y frente a ellos forman su propia pandilla: la Mara Salvatrucha número 13 (\abbr{MS13}).
 Terminada la guerra en El Salvador, el gobierno de los EE.UU. empieza a deportar masivamente a los pandilleros salvadoreños presos, quienes llegan a un país devastado por la guerra civil y que no ofrecía oportunidades legales, pero sí una oportunidad ilegal: cientos de desmovilizados de las guerrillas y los paramilitares, armados con armas automáticas, que pronto se unirían a los pandilleros.
 Pandilleros que conocían los negocios ilegales en EE.UU. y tenían experiencia en manejar franquicias de pandillas
 extendidas por varias ciudades.

\par% p
No veo a Colombia con su propia versión de maras.
 En Colombia hay pandillas.
 Muchas de esas pandillas fueron cooptadas por el narcotráfico o por las guerrillas (milicias urbanas), varias se
 unieron a las \abbr{AUC} (p. ej. Bloque Metro) y varias se han alimentado de los ex guerrilleros y ex paramilitares desmovilizados.
 Pero esas pandillas no las importamos de los EE.UU.
 Estas pandillas de crímen organizado se unen a las bandas rurales ahora conocidas como \abbr{% {'style': 'font-variant: small-caps;'}
bacrim} y se unen a la insurgencia (\abbr{% {'style': 'font-variant: small-caps;'}
Farc}) y se unen a los carteles de narcotraficantes privados, en un fenómeno complejo y muy propio de Colombia.

Lo que vemos en Colombia son bandas rurales y urbanas que fueron forjados en la guerra insurgente y contrainsurgente y en las guerras urbanas de pandillas y con el nivel de recursos que dejaron las mafias narcotraficantes que importaban materia prima de Bolivia y Perú y la entregaban en las calles de los EE.UU.
 Son mafias que se dedican al narcotráfico, a la extorsión, al contrabando, al cooptar el estado regional y obterner fraudulentos negocios en la salud, los transportes, los juegos de azar, etc.
 Grupos atomizados (todavía) pero fuertemente armados.
 Y varios de ellos con un apoyo internacional
 de soñadores despistados que creen aún que las insurgencias en Colombia son una lucha legítima de un pueblo en contra
 de un estado opresor.

\chapter{El tal ofiuco}
\begin{metadata}
	Published by \anchor[chlewey]{chlewey} on \anchor[http://ewey.co/B727]{Tue, 25 Jan 2011 18:19:20 +0000}\\
	\categories{astrologia, horoscopo, information, ofiuco, alter, serpentario}\\
	Shorthand: \anchor[http://blog.chlewey.net/2011/01/el-tal-ofiuco/]{el-tal-ofiuco}
\end{metadata}

El año tropical es el que transcurre entre dos veces la misma estación (p. ej. entre equinoxio vernal y equinixio
 vernal).  El año sideral es cuando la tierra le da la vuelta al sol con respecto a las otras estrellas.  Ambos años
 difieren en unos pocos minutos, pero tras miles de años esos minutos se convierten en días.

\anchor[http://commons.wikimedia.org/wiki/File:Ecliptic\_path.jpg]{\begin{wrapfigure}{r}{300\px}\centering% {'src': 'http://blog.chlewey.net/wp-content/uploads/2011/01/Ecliptic_path-300x225.jpg', 'title': u'Ecl\xedptica', 'height': '225', 'width': '300', 'alt': u'[ecl\xedptica]', 'class': ['alignright', 'size-medium', 'wp-image-737']}
\includegraphics[width=300\px,height=225\px]{blog/Ecliptic_path-300x225.jpg}
\end{wrapfigure}
}Cuando los babilonios describieron el zodiaco, hace unos 2500 años, (y desde una perspectiva geocéntrica), el sol
 recorría una zona del firmamento conocido hoy como ``la eclíptica'' (por ser donde ocurren los eclipses), y en esa zona
 ellos veían trece constelaciones: desde aries (donde el sol estaba en el equinoxio de primavera) hasta piscis.  Las
 constelaciones tenían tamaños diferentes, pero, para efectos astrológicos, dividieron el año tropical en doce períodos
 más o menos iguales.

Dos de estas constelaciones: Escorpio (escorpión) y Ofiuco (serpentario) ocupaban más o menos el mismo lugar pero la
 eclíptica pasaba más cerca del centro de Escorpio, y como 13 es un número incómodo, dejaron al serpentario por fuera y
 quedar así con el mágico número 12.

La precesión de los equinoxios (la diferencia entre el año tropical y el año sideral), trae dos efectos.  Uno, desde
 hace 2000 años, el equinoxio de primavera no ocurre con el sol en aries sino con el sol en piscis (Era de Piscis),
 dentro de unos 100 años el equinoxio ocurrira con el sol en acuario (Era de Acuario, la tal nueva era de la que hablan
 los que hablan de la Nueva Era).

\anchor[http://blog.chlewey.net/wp-content/uploads/2011/01/Ophiuchus.jpg]{\begin{wrapfigure}{l}{300\px}\centering% {'src': 'http://blog.chlewey.net/wp-content/uploads/2011/01/Ophiuchus-300x267.jpg', 'title': 'Ophiuchus', 'height': '267', 'width': '300', 'alt': '[el serpentario]', 'class': ['alignleft', 'size-medium', 'wp-image-740']}
\includegraphics[width=300\px,height=267\px]{blog/Ophiuchus-300x267.jpg}
\end{wrapfigure}
}Pero no sólo el equinoxio se desplaza con respecto a las estrellas.  Con el equinoxio se desplaza también la eclíptica,
 y en estos años la porción de la eclíptica que pasa por el serpentario (ofiuco) es más relevante que la que pasa por
 escorpión.

Hoy hay dos corrientes astrológicas.  La astrología tropical toma las casas creadas por los babilonios hace 2500 años,
 las cuales no corresponden a constelaciones reales sino a los doce signos arquetípicos que conocemos y que dividen al
 año tropical en 12 partes más o menos iguales.  La astrología tropical es la más extendida en occidente y la que todos
 conocemos.

La astrología sideral tiene a su vez dos corrientes.  Una astrología sideral arquetípica que sigue dividiendo al año
 sideral en 12 signos con períodos más o menos iguales, y una astrología sideral astronómica, que se basa en la
 constelación real sobre la cual veríamos pasar al sol.  La primera, la sideral arquetípica, no incluye al nuevo signo
 zodiacal, sino que es básicamente la misma astrología tropical pero atrasada unos 23 días (a la fecha actual).

La astrología sideral astronómica... bueno, esa es la del tal ofiuco.

\horrule{}
\section{Anexo}
\begin{tabular}{lllllllll}% FIX

Cosntelación &
Tropical arquetípico &
Sideral arquetípico &
Sideral astronómico \\

\anchor[http://commons.wikimedia.org/wiki/File:Aries.svg]{\includegraphics[width=20\px,height=19\px]{blog/20px-Aries_svg.png}} &
Aries &
21 de marzo a 19 de abril &
15 de abril a 15 de mayo &
17 de abril a 13 de mayo \\

\anchor[http://commons.wikimedia.org/wiki/File:Taurus.svg]{\includegraphics[width=20\px,height=21\px]{blog/20px-Taurus_svg.png}} &
Tauro &
20 de abril a 21 de mayo &
16 de mayo a 15 de junio &
14 de mayo a 19 de junio \\

\anchor[http://commons.wikimedia.org/wiki/File:Gemini.svg]{\includegraphics[width=20\px,height=21\px]{blog/20px-Gemini_svg.png}} &
Geminis &
22 de mayo a 22 de junio &
16 de junio a 15 de julio &
20 de junio a 20 de julio \\

\anchor[http://commons.wikimedia.org/wiki/File:Cancer.svg]{\includegraphics[width=20\px,height=16\px]{blog/20px-Cancer_svg.png}} &
Cancer &
23 de junio a 22 de julio &
16 de julio a 15 de agosto &
21 de julio a 9 de agosto \\

\anchor[http://commons.wikimedia.org/wiki/File:Leo.svg]{\includegraphics[width=20\px,height=26\px]{blog/20px-Leo_svg.png}} &
Leo &
23 de julio a 22 de agosto &
16 de agosto a 15 de septiembre &
10 de agosto a 15 de septiembre \\

\anchor[http://commons.wikimedia.org/wiki/File:Virgo.svg]{\includegraphics[width=20\px,height=24\px]{blog/20px-Virgo_svg.png}} &
Virgo &
23 de agosto a 23 de septiembre &
16 de septiembre a 15 de octubre &
16 de septiembre a 30 de octubre \\

\anchor[http://commons.wikimedia.org/wiki/File:Libra.svg]{\includegraphics[width=20\px,height=17\px]{blog/20px-Libra_svg.png}} &
Libra &
24 de septiembre a 23 de octubre &
16 de octubre a 15 de noviembre &
16 de octubre a 22 de noviembre \\

\anchor[http://commons.wikimedia.org/wiki/File:Scorpio.svg]{\includegraphics[width=20\px,height=22\px]{blog/20px-Scorpio_svg.png}} &
Escorpio &
24 de octubre a 22 de noviembre &
16 de noviembre a 15 de diciembre &
23 de noviembre a 28 de noviembre \\

\anchor[http://commons.wikimedia.org/wiki/File:Ophiuchus\_zodiac.svg]{\includegraphics[width=20\px,height=20\px]{blog/20px-Ophiuchus_zodiac_svg.png}} &
Ofiuco &
N/A &
29 de noviembre a 17 de diciembre \\

\anchor[http://commons.wikimedia.org/wiki/File:Sagittarius.svg]{\includegraphics[width=20\px,height=20\px]{blog/20px-Sagittarius_svg.png}} &
Sagitario &
23 de noviembre a 12 de diciembre &
16 de diciembre a 14 de enero &
18 de diciembre a 17 de enero \\

\anchor[http://commons.wikimedia.org/wiki/File:Capricorn.svg]{\includegraphics[width=20\px,height=19\px]{blog/20px-Capricorn_svg.png}} &
Capricornio &
22 de diciembre a 20 de enero &
15 de enero a 14 de febrero &
18 de enero a 15 de febrero \\

\anchor[http://commons.wikimedia.org/wiki/File:Aquarius.svg]{\includegraphics[width=20\px,height=13\px]{blog/20px-Aquarius_svg.png}} &
Aquario &
20 de enero a 19 de febrero &
15 de febrero a 14 de marzo &
16 de febrero a 11 de marzo \\

\anchor[http://commons.wikimedia.org/wiki/File:Pisces.svg]{\includegraphics[width=20\px,height=25\px]{blog/20px-Pisces_svg.png}} &
Piscis &
20 de febrero a 20 de marzo &
15 de marzo a 14 de abril &
12 de marzo a 16 de abril \\

\end{tabular}

P.S. Nací con el sol entre escorpión y serpentario, pero siempre me he identificado con el signo tropical de
 sagitario.  Igual, nunca leo el horozcopo ni me he hecho una carta astral.

\chapter{Escandalitos e indignaciones}
\begin{metadata}
	Published by \anchor[chlewey]{chlewey} on \anchor[http://ewey.co/B739]{Mon, 31 Jan 2011 21:18:18 +0000}\\
	\categories{derechos-de-los-animales, escandalo, opinion, sociedad, twitter}\\
	Shorthand: \anchor[http://blog.chlewey.net/2011/01/escandalitos-e-indignaciones/]{escandalitos-e-indignaciones}
\end{metadata}

La última gran indignación nacional, cual la refleja Twitter, fue el perricidio cometido por tres agentes de la policía
 y que quedara grabado en un video aficionado.  Y la twynch mob se vino con todo: descrédito a la policía como
 institución; recordar las torturas a los toros; renegar de la especie humana...

Hay varias reflexiones que surgen.

\par% p
La primera es esa capacidad de armar escándalos.  Porquerías suceden todos los días.  Niños mueren de hambre, crían y
 cazan a zorros por sus pieles, lluvias excesivas generan la pérdida de hábitat de especies salvajes y domésticas, los
 leones matan leopardos y hienas, nos enfermamos, evitamos enfermarnos asesinando bacterias con \del{armas de destrucción masiva}\ins{antibióticos}, mueren estrellas engulléndose a civilizaciones de las que nunca supimos: guerras, desastres naturales, gaia
 expresándose.

Pero nos encanta indignarnos por escándalos creados.
 Un video grabado con nuestro celular y compartido por una red social, y vemos como un episodio, realmente
 insignificante frente al gran orden de las cosas, se convierte en el tema, en la tendencia, en el asunto del que hay
 que hablar y del cual yo estoy hablando en este artículo.

\par% p
El segundo punto es esa actitud de masa que nos lleva a ver a los otros como otros y como masa.
 Unos taxistas agreden a unos usuarios y todos tenemos que emprenderla contra el gremio de los taxistas.
 Personalmente la mayor parte de las veces que he usado taxi me ha ido bien.
 Personas atentas, que siguen sugerencias, que esperan a que mi suegra se suba y se baje sin presionar, que, avisados
 con tiempo, pasan por una \anchor{bomba} a cambiar un billete cuando no tienen vueltas, etc.~ No siempre, pero nos quedamos con el escándalo del ``no siempre''.

\par% p
Quienes mataron a la perrita fueron unos agentes de la policía.~ No fue la Policía.~ Pero no importa.~ Es \textbf{nosotros} (los que sí pensamos porque tenemos Twitter, o Blackberry, o Facebook) \textbf{contra ellos} (los taxistas, sin excepciones; los policías, sin matices; los curas, etc.)
 Siempre hay una nueva causa.
 Un nuevo escándalo.~ Una nueva indignación en contra de un grupo siempre que no seamos nosotros.

\par% p
Pero no siempre es de nosotros contra ellos.
 A veces es símplemente nosotros.
 Convertimos el escándalo como una forma de decir que como colombianos damos asco, o que damos asco como seres humanos.
 Nos quedamos con la teoría de Mr. Smith en \emph{Matrix} de que los seres humanos no somos mamíferos sino virus, como si tal hipótesis tuviese el menor sustento en alguna realidad.
 Los seres humanos somos una porquería porque maltratamos a otras especies.~ Porque somos predadores.

Pues sí.
 Somos predadores.
 Ser predadores nos hizo humanos, nos hizo lo suficientemente concientes como para que seamos nosotros mismos quienes
 nos manifestemos frente a nuestra condición de predadores.

\par% p
Un león no tiene esa opción.~ Un león (o una manada) \anchor[http://blog.chlewey.net/2007/09/no-me-envien/]{matará un búfalo} o un ñu.~ \anchor[http://www.google.com/\#q=lions+attack+hyena\&tbs=vid:1]{Matará a una hiena}, o a un leopardo, y no por hambre sino porque también son predadores, porque son competencia.
 Un león macho matará a cualquier cachorro de león que no le pertenezca.
 Y no lo cuestionan.
 Hace poco veía un video aficionado de cómo una manada de leonas atacaba a una hiena rompiéndole los huesos y dejándola viva para que muriera desangrada.
 Un gran acto de crueldad animal.
 Si los leones fueran humanos entre ellos habría una indignación porque era un acto completamente gratuito de crueldad.
 Sin un propósito útil.

Pero son leones.

Nosotros terminamos perdonando ese acto de crueldad porque los leones no pueden decidir por ellos mismos.
 Porque es parte de la naturaleza.
 Porque, en últimas, las hienas son los villanos favoritos de las películas de Disney así que hasta justificado está.

Pero cuando es un ser humano el agresor, entonces es imperdonable.
 Los seres humanos somos lo peor que le pudo haber pasado a la naturaleza.

La naturaleza está llena de predadores viciosos.
 Una manada de elefantes es capaz de destruir bosques.
 Una pareja de castores puede destruir el hábitat de otras especies en su afán de crear un lago propio.
 Pero de todos los predadores viciosos que ha producido la naturaleza ésta creó finalmente a una especie que es capaz de cuestionarse.
 Que es capaz de establecer límites a su propio comportamiento.

Eso somos nosotros.~ Seres maravillosos que podemos reprobar los errores que cometemos.

Pero es más fácil criticar y reprobar, indignarnos por los escandalitos que creamos de la nada que reconocernos a
 nosotros mismos nuestro propio valor.

\chapter{Reflexiones de movilidad}
\begin{metadata}
	Published by \anchor[chlewey]{chlewey} on \anchor[http://ewey.co/B933]{Tue, 01 Feb 2011 17:07:25 +0000}\\
	\categories{movilidad, opinion, restriccion-vehicular, trasnporte-publico}\\
	Shorthand: \anchor[http://blog.chlewey.net/2011/02/reflexiones-de-movilidad/]{reflexiones-de-movilidad}
\end{metadata}

En ocasiones cuando releo este blog me encuentro con que lo que opinaba en ese entonces no es exactamente lo que sigo
 pensando.  Eso está bien pues entre ese entonces y hoy han ocurruido nuevos sucesos y he aprendido cosas nuevas.

\anchor[http://commons.wikimedia.org/wiki/File:Londres\_04\_07\_160.JPG]{\begin{wrapfigure}{r}{255\px}\centering% {'src': 'http://upload.wikimedia.org/wikipedia/commons/d/d5/Londres_04_07_160.JPG', 'title': u'Transporte p\xfablico en Londres', 'height': '170', 'width': '255', 'alt': '[]', 'class': ['alignright']}
\includegraphics[width=255\px,height=170\px]{blog/Londres_04_07_160.jpg}
\end{wrapfigure}
}He querido ponerme a hacer artículos de \emph{actualización} de post viejos, y tres situaciones de actualidad me llevan a que el turno sea hoy sobre \anchor[http://blog.chlewey.net/2007/08/mejorar-el-transito/]{\emph{Algunas ideas para mejorar el tránsito en Bogotá}}.~ No porque piense diferente sino, más bien, porque me reafirmo en ciertos puntos.

\par% p
Primero, este jueves 3 de febrero se celebra una nueva jornada del \emph{Día sin carro}, una iniciativa de reflexión sobre la movilidad ciudadana que pretendía mostrarnos que la ciudad puede funcionar sin el automóvil particular.
 Aunque no ha faltado quienes quieran repensarlo como \anchor[http://patton.blogdeldia.com/item/933]{un acto ecológico}, o recreativo, o una excusa para no ir a trabajar.

Segundo, hoy, 1 de febrero, unos padres de familia en respuesta a un problema particular, decidieron tomar las vías de
 hecho y bloquear a Transmilenio, el sistema de bus de tránsito rápido de Bogotá que es lo más parecido a transporte
 público masivo que tenemos, afectando a cientos de miles de usuarios.

\par% p
Tercero, este 14 de febrero se vencía la extensión de la restricción vehicular en Bogotá, mal llamada \emph{Pico y placa}, pues la restricción ya no se reducía a las horas pico.
 Tras supuestos estudios técnicos que se basaron principalmente en el caos vehicular que se presentaba en las tardes la
 segunda semana de enero, las autoridades capitalinas decidieron que continuarían con la restricción de todo el día
 (bueno, de 14 horas) por uno o dos años más.

\par% p
Estos hechos me llegan junto con una reciente reflexión de mi hermana sobre el \anchor[http://luzelenathompson.blogspot.com/2011/01/transporte-publico-bogotano.html]{transporte público colectivo} en Bogotá.

Veo que muchas veces el tema de movilidad se reduce al trancón y se mira como principal culpable al automóvil particular.
 La movilidad es la necesidad que una población tiene de desplazarse entre su lugar de vivienda, el de estudio, el de trabajo, el esparcimiento y los de aprovisionamiento, entre otros.
 Adicionalmente se incluye todo lo que es el transporte de mercancías.

La movilidad personal puede realizarse a pie, en automómvil particular, en taxi, en bicicleta, en transporte público colectivo o masivo, en moto, a lomo de caballo, en carreta, en patines, etc.
 La decisión que cada persona toma es una relación entre la disponibilidad, el costo y la comodidad percibida.
 En la comodidad percibida juegan factores como la facilidad de encontrar dónde parquear el vehículo, las
 conglomeraciones, la existensia de rutas y facilidad de conexiones, el tiempo, etc.

Si es fácil parquear en la acera del frente del sitio al que te diriges, el automóvil se convierte en un medio muy conveniente de transporte puerta a puerta.
 Así, el automovilista utilizará el automóvil particular en todas las ocasiones posibles y sólo mediante la prohibición explícita de usar su carro considerará utilizar otro medio de transporte.
 Si el transporte público colectivo sufre de los mismos trancones que el transporte particular, disminuyen los
 insentivos para usar un bus en lugar de un automóvil.

En el transporte personal individual, la mejor relación costo/velocidad la suele dar la bicicleta en la mayoría de ciudades modernas, incluída Bogotá.
 Sin embargo cuando una persona tiene problemas de locomoción la bicicleta no es una opción.
 Y no es una opción (o no la más cómoda) cuando se transportan niños, o cuando se transportan ciertos tipos de cargas como el mercado semanal o el proyecto de clase.
 Tampoco es muy cómoda la bicicleta frente a ciertos estados del clima.

No todo el mundo está dispuesto a usar la bicicleta, aun con el estado de salud suficiente para poder usarla.
 El transporte público colectivo o masivo debería ser una muy buena opción.
 Bien diseñado, será mucho más cómodo que utilizar vehículos individuales, y efectivamente en ciudades como Nueva York
 o Tokyo muchos de los desplazamientos son más rápidos y económicos utilizando un transporte público existente, ubicuo
 y con facilidades de conexiónes entre rutas y medios, que estar en medio de trancones por utilizar un transporte
 público individualizado.

Mi propuesta se mantiene: la ciudad debe diseñarse para que el transporte público colectivo y masivo funcione.
 Funcione bien y sea fácil de utilizar.
 Y que ese transporte público esté blindado frente a los trancones que afecten al automóvil particular.

En estos años me confirmo en la idea de que las principales amenazas a la movilidad; los principales generadores de trancón; son los vehículos mal parqueados más la gente que no sabe conducir.
 Si a esto se suman taxis y buses deteniendose a dejar o recoger  pasajeros en cualquier lugar, incluyendo en doble y
 triple línea, el tráfico será más lento de lo que debería ser.

Debería estar completamente prohibido que en una vía arteria en una hora pico un automóvil o un camión estacione al lado del andén ocupando un carril de circulación.
 Y debería sancionarse con rigor cualquier comportamiento que estorbe el tránsito.
 Tomando fotos y sancionando al responsable del vehículo (p. ej. el dueño, o la empresa que lo afilia) así no se pueda
 individualizar al conductor (el dueño del vehículo debería ser responsable por quien lo conduce).

El transporte público colectivo debe integrarse al masivo.
 Debe ser conducido por verdaderos profesionales que no cometan infracciones y que respondan con su licencia cuando se equivoquen; como un médico que nunca entutelará aduciendo el derecho al trabajo cuando lo sancionen por negligencia.
 Que respeten los paraderos.
 Pero, también, que circulen por carriles exclusivos en los cuales un automóvil particular no puede circular y mucho
 menos detenerse.

\par% p
Finalmente, el autmóvil particular es una libertad ciudadana.
 Una libertad que conlleva responsabilidades y que requiere cierta dignidad de quien la ejerce, pero una libertad que a criterio del ciudadano puede tomar en cualquier momento.
 Sin restricciones artificiales (como el \emph{pico y placa}).

Tal vez quien decida utilizar su automóvil encontrará más trancones.
 Pero lo hará por una decisión propia.
 Porque sopesó y sus necesidades específicas de transporte le sugirieron utilizar mejor su autmóvil a caminar, usar bicicleta o transporte público.
 Porque preferirá dar vueltas en medio de trancones para buscar un parqueadero más o menos cercano a su destino porque
 sabe que no podrá dejar el carro al frente del mismo.

Debe pensarse la movilidad como la forma de lograr que quienes quieran movilizarse lo puedan hacer.
 Ofreciendo que lo puedan hacer cómodamente.
 Y favoreciendo positivamente las modalidades de transporte más ecológicas o más civilmente responsables y no
 prohibiendo usar las modalidades legales que a criterio del alcalde no convengan.

En cuanto al transporte de carga: un detalle que sin duda favorecería mucho la movilidad es obligar que los camiones repartidores funcionen en las horas valle nocturnas.
 Cuando al detenerse frente al local que van a surtir no afecten la movilidad de la hora pico.

\chapter{De personas y avatares}
\begin{metadata}
	Published by \anchor[chlewey]{chlewey} on \anchor[http://ewey.co/B962]{Wed, 09 Mar 2011 16:56:04 +0000}\\
	\categories{information, precencia, social-media, twitter}\\
	Shorthand: \anchor[http://blog.chlewey.net/2011/03/de-personas-y-avatares/]{de-personas-y-avatares}
\end{metadata}

Persona viene del latin per-sonare y se refiere a quien tiene voz propia.  A quien puede decir ``yo soy''.  Los animales,
 por ejemplo (y sin restarles valor), no son personas.  Ni es persona una piedra.

\anchor[http://commons.wikimedia.org/wiki/File:Avatars.jpg]{\begin{wrapfigure}{r}{255\px}\centering% {'src': 'http://upload.wikimedia.org/wikipedia/commons/a/a0/Avatars.jpg', 'title': 'Avatars', 'height': '320', 'width': '255', 'alt': '', 'class': ['alignright']}
\includegraphics[width=255\px,height=320\px]{blog/Avatars.jpg}
\end{wrapfigure}
}Por otro lado, y de acuerdo a la legislación de un país dado, una empresa podría ser una persona.  Una persona jurídica
 que le da derechos de representarse a si misma como algo diferente a sus socios, y que le permite suscribir contratos
 en nombre propio.

Uno de los misterios del cristianismo trinitario es (y de ahí su nombre) el misterio de la Santísima Trinidad.  Este
 nos dice que hay tres personas distintas y un solo Dios verdadero.  En cierta forma el hinduísmo tiene un concepto
 parecido: el avatar.  Un avatar es una manifestación de una entidad superior.  Una persona.

Sin entrar en sutilezas teológicas, el concepto de persona nos muestra ese aspecto.  No hay una relación biunívoca
 entre los seres humanos y las personas.  Ser humano y persona no son sinónimos.  Los códigos colombianos, por ejemplo,
 no reconoce como personas a los no nacidos ni a los sordomudos que no sepan escribir.  Pero si los reconoce como seres
 humanos.

Por otro lado, y a nivel jurídico, hay personas que nos son seres humanos: las empresas, corporaciones, edificios con
 administración, partidos políticos, etc. son personas jurídicas.

Pero también son personas los dioses y demonios, los duendes y ogros, y cualquier ser consiente de sí mismo que exista
 dentro de la realidad o dentro de alguna mitología u obra de ficción.  Seres que no son, necesesariamente humanos.

Muchos de nosotros manejamos personas distintas.  Nos presentamos como Pepito Pérez ante nuestra familia y amigos y
 como Pepón Iracundo frente a nuestras amistades de Internet.  Ahora, bien sea por consideraciones de privacidad o por
 las menores restricciones sociales que conlleva el anonimato en la red, hay diferencias sutiles entre Pepito Pérez y
 Pepón Iracundo.  Aun cuando pretendamos no tenerlas.

Pepón Iracundo es un avatar.
 El la forma como Pepito Pérez vive en un mundo algo diferente al suyo. Habrá personas con las que Pepito interactua
 diariamente que deconocerán o no le importará su nombre real: para ellos su amigo es Pepón Iracundo.

Hay quienes establecen una separación abrupta entre ambas personas, más por cuestiones de privacidad que por cuestiones de personificación.
 En nuestro ejemplo es como si Pepito Pérez nunca colocara su nombre al lado de Pepón Iracundo; pero todo lo que
 escribe bajo su pseudónimo es lo que realmente piensa o como realmente siente Pepito Pérez.

También hay quienes hacen lo contrario.
 No temen que su nombre real se relacione con su persona digital, pero hay diferencias de carácter.
 Por ejemplo, Pepón Iracundo podría ser más extrovertido o más huraño que Pepito Pérez.
 Esto puede ser intencional o no.

Desde luego que separando ambas cosas: la personalidad del carácter y la relación entre las personas digital y de carne
 y hueso, se puede llegar a juegos complejos, en los que la persona digital sea realmente un personaje de ficción con
 poco que ver con la persona de carne y hueso.

También hay quienes pretenden que no hay diferencia.

\anchor[http://blog.chlewey.net/wp-content/uploads/2011/03/chlewey-cp10.jpg]{\begin{wrapfigure}{r}{255\px}\centering% {'src': 'http://blog.chlewey.net/wp-content/uploads/2011/03/chlewey-cp10-255x300.jpg', 'title': 'chlewey', 'height': '300', 'width': '255', 'alt': '', 'class': ['alignright', 'size-medium', 'wp-image-977']}
\includegraphics[width=255\px,height=300\px]{blog/chlewey-cp10-255x300.jpg}
\end{wrapfigure}
}En mi caso he llegado a manejar hasta cuatro personas en Twitter que no son marcas personales ni personajes ficticios.
 Chlewey y Carlos Thompson \anchor[http://blog.chlewey.net/2009/06/sobre-chlewey/]{son una misma persona} y por ello @\anchor[http://twitter.com/chlewey]{chlewey} se supone que soy yo.~ Con mi personalidad, con mi nombre, con todo lo que yo soy.

Pero no es del todo verdad.
 Me quejo más como @chlewey y soy, en general, más extrovertido en Twitter de lo que soy en el mundo tridimensional.

\par% p
Mis otras personas en twitter son @\anchor[http://twitter.com/carlos\_thompson]{carlos\_thompson} que es una cuenta de seguridad por si en algún momento necesito mostrar una faceta más seria.~ @\anchor[http://twitter.com/epeady]{epeady} fue creada para mantener separada mi privacidad con el resultado de haber creado una persona aún más extrovertida y emocional que @chlewey.
 Finalmente @\anchor[http://twitter.com/rataflechera]{rataflechera} fue un experimento que quice conservar.~ Básicamente ha terminado en un @epeady II pero mucho menos locuaz.

\par% p
Entre las personas que he visto en Twitter, Usenet, y otros foros virtuales hay personajes ficticios, tanto salidos de
 obras de ficción como Sheldon Cooper de \emph{The Big Bang Theory}, como personas cuyo propósito es ocultar al ser humano que los maneja.
 Hay parodias, cuentas falsas (que pretenden hacerse pasar por una persona real extraña al creador del perfil, e
 igualmente, esa persona real puede ser ficticia, ficcional o de carne y hueso, generalmente una celebridad), marcas
 (incluyendo marcas personales: una forma de asumir el perfil de Twitter o Usenet como una marca y no como un ser
 humano), etc.

Somos presencias en los lugares en los que nos presentamos.
 A veces nuestra presencia es más o menos homogénea.
 Otras veces las discrepancias son grandes e intencionales.
 Podemos hablar de nuestras conquistas ficticias frente a los compinches de parranda y ser un esposo fiel y amoroso en la casa.
 O quejarnos todo el día en Twitter mientras en Facebook sólo mostramos la cara feliz de nuestras fiestas y vacaciones,
 mientras somos un trabajador responsable en la empresa.

Somos más de una persona.~ Tal vez somos tantas personas como espacios de interacción vivamos.

Y eso está bien.

\chapter{Derechos de los niños}
\begin{metadata}
	Published by \anchor[chlewey]{chlewey} on \anchor[http://ewey.co/B963]{Thu, 24 Feb 2011 17:29:04 +0000}\\
	\categories{adopcion, derechos, homosexuales, opinion}\\
	Shorthand: \anchor[http://blog.chlewey.net/2011/02/derechos-de-los-ninos/]{derechos-de-los-ninos}
\end{metadata}

\par% p
A raiz de un \anchor[http://www.elespectador.com/noticias/politica/articulo-252460-procurador-negativa-adopcion-gay-no-conviccion-biblica]{reciente pronunciamiento} del Producador Ordóñez vuelve a ponerse sobre el tapete el tema de la adopción de niños por parejas homosexuales: uno
 de los caballitos de batalla de quienes dicen luchar por la igualdad de derechos de los homosexuales.

\anchor[http://blog.chlewey.net/wp-content/uploads/2011/02/Die\_Adoption.jpg]{\begin{wrapfigure}{r}{300\px}\centering% {'src': 'http://blog.chlewey.net/wp-content/uploads/2011/02/Die_Adoption-300x240.jpg', 'title': u'Die Adoption, por Ferdinand Georg Waldm\xfcller', 'height': '240', 'width': '300', 'alt': u'[Die Adoption, por Ferdinand Georg Waldm\xfcller]', 'class': ['alignright', 'size-medium', 'wp-image-968']}
\includegraphics[width=300\px,height=240\px]{blog/Die_Adoption-300x240.jpg}
\end{wrapfigure}
}El tema de la adopción no es un asunto de que unos adultos adquieran a un niño para convertirse en padres y tener un nuevo juguete o una nueva responsabilidad.
 El tema de la adopción es permitir que un niño (\anchor[http://blog.chlewey.net/2011/01/cuestion-de-genero/]{varón o hembra}) tenga una familia que lo quiera y que se comprometa a educarlo y a criarlo.
 La ley permite así que un adulto soltero o un matrimonio de adultos pueda adoptar a un niño.

Desde que la legislación colombiana reconoce la unión marital de hecho, no sé si se ha pronunciado al respecto de si un
 hombre y una mujer, que conviven pero no están casados, puedan adoptar a un hijo como padre y madre respectivamente.

Desde el punto de vista de otorgar una familia al menor, no habría mayor inconveniente si un sólo miembro de la pareja adopta a un niño y el otro miembro lo cría en igualdad de condiciones.
 Eso pasa a cada rato sin que hayan adopciones de por medio, p. ej. cuando un padre se separa o enviuda y luego vuleve
 a casarse: el nuevo cónyuge asume el papel del padre o madre faltante sin que la ley otorgue una patria potestad.

Cuando vamos a las relaciones homosexuales, por ley no hay impedimento a que uno de los miembros de la pareja adopte y
 el otro ejerza de facto como un segundo padre o una segunda madre.

El problema es la figura jurídica en caso de que el padre adoptivo falte.
 ¿Quien tiene la patria potestad sobre el menor?
 ¿Quien ha ejercido como padre o madre (o segundo padre o segunda madre) pero que no tiene derechos según la ley?

Son comunes los casos de custodia donde un padre separado muere y el hijo se lo disputan entre la concubina (quien ejerció como madre) y la ex esposa (madre natural del hijo pero que habia perdido la custodia frente al padre).
 [Adáptese el sexo de acuerdo al caso.]
 En este caso esta madrastra concubina, si bien nunca fue oficialmente madre del menor, tienen algún chance de obtener la custodia legal del hijo, particularmente si la madre natural no está en condiciones de mantener al niño (razón por la cual perdió la custodia en primer lugar).
 No así sucede si el concubino del padre tenía el mismo sexo que este y convivían en una relación homosexual.

No reconocer la adopción en uniones maritales de hecho, heterosexuales u homosexuales, implica que el compañero no adoptante no puede ejercer legalmente como padre cuando el padre adoptivo falta.
 Es complicar la situación jurídica del niño.

Así que si vemos a la adopción como el derecho del niño a tener una familia que lo críe y que lo eduque y,
 principalmente que lo quiera, debería permitirse que el niño sea adoptado por cuantos adultos quieran hacerse cargo de
 él, así sean parejas homosexuales.

Esto no se trata de derechos de los gay.~ Se trata de derechos de los niños.

\chapter{Naturaleza de la frustración}
\begin{metadata}
	Published by \anchor[chlewey]{chlewey} on \anchor[http://ewey.co/B992]{Wed, 23 Mar 2011 16:14:09 +0000}\\
	\categories{castigo, educacion, familia, personal, premio, proyeccion-y-carrera}\\
	Shorthand: \anchor[http://blog.chlewey.net/2011/03/naturaleza-de-la-frustracion/]{naturaleza-de-la-frustracion}
\end{metadata}

No tengo ningún estudio formal sobre la neturaleza humana, pero me gusta observar lo que sucede a mi alrededor.
 Hay dos sujetos de prueba que me han interesdo particularmente en los últimos meses, si no años, casos que tal vez un
 psicólogo o un psiquiatra podrían verlos como típicos o no particularmente anormales pero que para mí son importantes.

\anchor[http://blog.chlewey.net/wp-content/uploads/2011/03/n532809841\_1520558\_2992.jpg]{\begin{wrapfigure}{r}{195\px}\centering% {'src': 'http://blog.chlewey.net/wp-content/uploads/2011/03/n532809841_1520558_2992-195x300.jpg', 'title': u'Carlos Thompson de ni\xf1o', 'height': '300', 'width': '195', 'alt': '[me]', 'class': ['alignright', 'size-medium', 'wp-image-993']}
\includegraphics[width=195\px,height=300\px]{blog/n532809841_1520558_2992-195x300.jpg}
\end{wrapfigure}
}Criar a un hijo es una tarea complicada.
 Siempre hay un momento en el que riñen lo que el niño quiere hacer con lo que el niño tiene que hacer: un ‘tiene que hacer’ que es dictado por los grandes, por los adultos, por uno.
 El niño no quiere someterse a esa voluntad adulta por lo que uno, como adulto, debe obligarlo.
 Debe cambiarle lo que el niño quiere hacer por lo que tiene que hacer: comer, estudiar, cuidar su salud, etc.
 Todo lo que ello implica son refuerzos positivos a largo plazo y no las satisfacciones inmediatas de jugar con sus
 juguetes, ver monos en la televisión o jugar un videojuego.

¿Cómo convencerlo que esas satisfacciones a largo plazo son más importantes que las satisfacciones a corto plazo?
 No es que la paciencia del niño le de para experimentar la verdadera importancia del refuerzo a largo plazo, p. ej. una buena salud o una buena educación; menos aún cuanto tales refuerzos no se perciben como tales, porque, en el mundo ideal, no habría con qué compararlos en carne propia; porque los buenos hábitos de higiene no garantizan estar libre de enfermedades, sólo aumentan la probabilidad de que así sea.
 En cambio la satisfacción inmediata es eso.
 Es esa descarga de endorfinas que produce la actividad placentera.~ El refuerzo es inmediato y fácil de percibir.

A una mente de 6 años hablarle de lo importante, cuando lo único que experimenta es lo inmediato, parece una pérdida de tiempo.
 Entonces toca reforzar lo importante con premios y castigos inmediatos.
 Negociaciones y amenazas.
 No parece sano.
 No parece sano que el único motivo del niño para cuidar su salud sea el dulce o el muñeco que recibirá como premio por
 tomarse su medicamento, o evitar la palmada o que le apaguen el televisor si no pasa a la mesa.

En un adulto de 38 años la situación es distinta.

Ya tiene una mente que es capaz de extrapolar la gratificación distante aun cuando no tenga la experiencia.
 El adulto ya entiende el concepto de lo importante sobre lo inmediato.

Pero también tiene experiencia.
 Una experiencia que podría decirle en muchos casos que lo distante, por importante que sea, es incierto.
 La gratificación distante no depende sólo de lo que hoy haga para lograrla, sino de otra serie de acciones propias y
 de terceros, muchas de las cuales que se escapan del control de uno.

La gratificación inmediata sigue siendo inmediata.

En este choque entre la razón y el impulso, el niño de 6 años tiene una guía clara en los adultos que comprenden la razón y orientan al niño, por medio de negociaciones de premios y castigos, en la dirección correcta.
 El adulto de 38 años ya cuenta con un mejor raciocionio y con una mayor experiencia.
 Y esta experiencia puede ser positiva o negativa.

En mi caso particular, esta experiencia ha sido negativa.
 Tal vez porque nunca desarrollé el hábito de la constancia, o porque la suerte no me acompañó, mi vida está llena de frustraciones.
 Cosas que quice y que creí importantes nunca se dieron.
 Esto me pone en una situación donde siento que no vale la pena luchar, donde a pesar de ser capaz de ver lo importante
 que tengo por delante, en la práctica he abandonado mis deseos de luchar.

No es que no sepa qué es lo importante.
 Tengo la capacidad mental para verlo, para entenderlo.
 Para saber que tengo que luchar.
 Para hacer planes.
 No tengo es la energía interior para ejecutarlos.
 El sentimiento de impotencia supera la racionalidad y esto me lleva a preferir esas gratificaciones inmediatas de una
 mente ávida de absorver y recrear información.

Entonces están los demás.

Como mi comportamiento se parece más al de un niño de 6 años, en que vivo más pendiente de las gratificaciones inmediatas que en lo importante; no veo raro que en muchos casos los demás me traten como a un niño de 6 años.
 Entonces interactuar conmigo se convierte en una negociación de premios y castigos.

Pero el problema ya es estuctural y, desafortunadamente, pienso más allá de los premios y castigos.
 Los premios no me motivan positivamente y más si siento que son un intento de que yo haga lo que sé que no puedo hacer —o más que lo que sé no puedo hacer, lo que ya abandoné querer hacer—.
 El castigo termino viéndolo como una consecuencia que estoicamente soportaré.

Y es aquí donde surge el sermón.
 En su frustración los demás que aún se preocupan por mí terminan descargando su impotencia en palabras que intenten hacerme reaccionar.
 El problema es que cuando sé que tienen razón en sus palabras eso no hace más que reforzar mi sentimiento de fracaso en la vida que es la primera razón por la cual ya renuncié a luchar.
 Entonces soporto estoicamente este castigo que es el sermón, sintiéndome cada vez más frustrado y menos motivado a
 seguir luchando.

Pero, cuando percibo que mi interlocutor, que mi sermonero de turno se equivoca, o que pasa de la frustración a la intención de herir con sus palabras, fácilmente ese odio paralizante que siento hacia mí se transforma inmediatamente en rabia hacia esta persona.
 Ya nisiquiera escucho.
 Ya no me importa el mensaje que me quería transmitir.
 Simplemente quiero que se calle y descargar mi rabia en una huída antes de descargarla sobre los demás.

\par% p
Por eso sigo convencido de que \anchor[http://tumblr.chlewey.net/post/149913679/solo-se-que-no-necesito-sermones-comprobado-no]{el sermón no sirve}.
 No logra despertar conciencia sino, por el contrario, a sepultar la conciencia bajo el odio y la frustración.
 Es contraproducente.

Pero no es solo contraproducente el sermón con adultos de 38 años.
 Lo es también con niños de 6.
 Mi hijo tiene problemas de comportamiento.
 Algunos son probablemente derivados de cómo él ve a su padre frustrado e inactivo y víctima de ataques de los demás miembros de su familia.
 Otros son, sin duda, producto de que yo, en mis crisis existenciales, lo descuido, tal vez porque entre las luchas que he abandonado se encuentra él.
 Pero, en gran parte, creo que muchos de sus problemas se deben también a los refuerzos que nosotros, los adultos, le
 inculcamos.

Siento que mi hijo reacciona a la negociación de premios y castigos y al sermón de la misma forma que yo lo hago, guardadas las proporciones por la diferencia de edades y de experiencia.
 Y aquí siento que yo contribuyo en hacer con él lo que sé que no funcionó ni funciona conmigo.

Me asusta que estoy contibuyendo a formar a otro fracasado como yo, y no por lo que pasivamente hago (dejarme llevar)
 sino por lo que activamente intento hacer (corregirlo).

\chapter{Ciberacoso}
\begin{metadata}
	Published by \anchor[chlewey]{chlewey} on \anchor[http://ewey.co/B997]{Tue, 29 Mar 2011 13:01:26 +0000}\\
	\categories{ciberacoso, information, twitter}\\
	Shorthand: \anchor[http://blog.chlewey.net/2011/03/ciberacoso/]{ciberacoso}
\end{metadata}

Por ahí me entero de un nuevo caso de ciberacoso.  Un twittero al que sigo casi que desde que entré a Twitter ha sido
 amenazado por su actividad en la red.  Con más de 8700 seguidores y 94000 tweets es una persona reconocida en la
 twittósfera colombiana y sólo por ello es una persona admirada por muchos y detestada por aun muchos más.

\anchor[http://www.madcoversite.com/mad210.html]{\begin{wrapfigure}{r}{209\px}\centering% {'src': 'http://www.madcoversite.com/mad210backprintid.jpg', 'title': 'Yes, me worry!', 'height': '280', 'width': '209', 'alt': '[Alfred E. Newman]', 'class': ['alignright']}
\includegraphics[width=209\px,height=280\px]{blog/mad210backprintid.jpg}
\end{wrapfigure}
}No faltan quienes creen que es un sobreactuado.  Que es alguien que no es capaz de resistir el trolleo.  Por lo que
 conozco de otros casos, estoy seguro que lo que desencadenó su última reacción no fue un caso de trolleo sino un caso
 real de amenazas serias y sustentadas orquestadas por la red.

Muchos celebran, pues la victima en este caso es una persona que no cae bien a todo el mundo.
 Se divierten con la situación.~ Sus antipatías personales se imponen sobre la el análisis serio de la situación.

Personalmente este twittero no es la persona que más me simpatiza, por las mismas razones que otros lo odian.
 Trina mucho y pretende ser chistoso en gran parte de lo que dice.
 Pretención que no siempre cae bien.
 Lo he conocido personalmente y siento que esa imagen de prepotencia que inspira esconden a una persona como uno:
 normal, algo tímida, con principios.

\par% p
Creo que no he sido de las personas que celebraban sus twitteractividad.
 En un par de ocasiones habré lanzado puyas sobre lo que dice o como lo dice.~ Mal podría yo considerarme un fan de \anchor[http://twitter.com/hyperconectado]{hyperconectado}; y es, desde este punto de vista, que siento una formal protesta contra la celebración, la ligereza y la jocosidad con
 la que algunos tratan el tema del ciberacoso.

\chapter{Sobre derechos de autor y derechos de copia}
\begin{metadata}
	Published by \anchor[chlewey]{chlewey} on \anchor[http://ewey.co/B1008]{Thu, 07 Apr 2011 15:47:48 +0000}\\
	\categories{activismo, derechos-de-autor, derechos-de-copia, information}\\
	Shorthand: \anchor[http://blog.chlewey.net/2011/04/derechos-autor-y-copia/]{derechos-autor-y-copia}
\end{metadata}

\par% p
No me gusta comprar pirata, pero me han regalado películas.
 Prefiero usar software libre.
 Procuro cuidar que las imágenes en mi blog estén debidamente referenciadas (si es que no son propias) y, preferiblemente, cuenten con licencias libres.
 Pero rebloggeo en \anchor[http://tumblr.chlewey.net/]{Tumblr} cosas interesantes que a veces ni sé de dónde vienen.
 No puedo decir que estoy completamente libre de infracciones a los derechos de autor, pero procuro hacerlo.

Los derechos de autor son, como su nombre lo dice, el derecho que un autor tiene sobre su obra.
 Estos derechos incluyen los derechos morales y los derechos patrimoniales.
 El primero de los derechos morales es el derecho al reconocimiento, esto es que el autor sea reconocido como tal.
 También el autor tiene el derecho de no ser vinculado como responsable de cualquier modificación que otra persona hace sobre su obra, sea autorizada o no.
 Puede haber una discusión sobre si los derechos morales son renunciables.
 Yo puedo crear una obra y donarla al dominio público (renunciar a mis derechos) pero igualmente nadie puede luego decir que fue su autor.
 Tampoco podría decir que no fui el autor si realmente lo fui y mi obra (o mi pieza de código) representan una responsabilidad penal o civil.
 Esta responsabilidad se basa en que esos derechos morales no son renunciables.

\par% p
Los derechos patrimoniales son los que determinan que el titular puede beneficiarse económicamente de la obra.
 Los derechos patrimoniales se pueden ceder o se puede renunciar a ellos.
 Una forma de renunciar a ellos es poner tu contenido en el dominio público o, simplemente, rehusándote a cobrar por ellos.
 Un ejemplo clásico de lo segundo es la foto \emph{''Gerrillero heróico''} que \anchor[http://en.wikipedia.org/wiki/Alberto\_Korda]{Korda} tomó de Ernesto Guevara.
 Korda conserva todos sus derechos pero se ha negado a cobrar por ellos y sólamente una vez ejerció sus derechos
 morales y patrimoniales para sancionar el uso de su foto en una campaña comercial que consideró contraria a los
 ideales del \emph{Che}.

Entre los derechos morales y patrimoniales, el autor puede determinar cómo se distribuye su obra, quién puede
 beneficiarse de las mismas y qué tipo de regalías recibe por el beneficio económico o no que otros obtengan por usar o
 distribuir.

Un licenciamiento Copyleft, es una forma como el autor de una obra puede ceder parte de sus derechos, renunciando
 parcialmente a los derechos patrimoniales, para que su obra pueda ser distribuída y reutilizada bajo la premisa de que
 cualquier modificación siga siendo Copyleft.

\par% p
El sistema de licencias de \anchor[http://creativecommons.org/]{Creative Commons} permiten que un autor permita la libre distribución de una obra y defina o no si conserva o renuncia a una serie de derechos.
 Primero en la Creative Commons los derechos morales son irrenunciables\begin{wrapfigure}{l}{50\px}\centering% {'src': 'http://creativecommons.org/images/deed/by.png', 'title': 'by', 'height': '50', 'width': '50', 'alt': '[by]', 'class': ['alignleft']}
\includegraphics[width=50\px,height=50\px]{blog/by.png}
\end{wrapfigure}
: el autor siempre debe ser referenciado como éste lo indique.
 La segunda: la distribución debe ser libre (no se necesita pedir permisos ni pagar regalías por distribuir, ni para
 adaptar la obra a las necesidades específicas\begin{wrapfigure}{r}{50\px}\centering% {'src': 'http://creativecommons.org/images/deed/remix.png', 'title': 'remix', 'height': '50', 'width': '50', 'alt': '[remix]', 'class': ['alignright']}
\includegraphics[width=50\px,height=50\px]{blog/remix.png}
\end{wrapfigure}
\begin{wrapfigure}{r}{50\px}\centering% {'src': 'http://creativecommons.org/images/deed/share.png', 'title': 'share', 'height': '50', 'width': '50', 'alt': '[share]', 'class': ['alignright']}
\includegraphics[width=50\px,height=50\px]{blog/share.png}
\end{wrapfigure}
) siempre y cuando se respeten los otros tres derechos opcionales.

\par% p
Los derechos opcionales en Creative Commons son el derecho patrimonial a lucrarse de la obra y el derecho moral a controlar las modificaciones.
 Una licencia Creative Commons puede incluir la opción ``no comercial''\begin{wrapfigure}{r}{50\px}\centering% {'src': 'http://creativecommons.org/images/deed/nc.png', 'title': 'non-comertial', 'height': '50', 'width': '50', 'alt': '[nc]', 'class': ['alignright']}
\includegraphics[width=50\px,height=50\px]{blog/nc.png}
\end{wrapfigure}
 que indica que nadie puede lucrarse de la obra sin el permiso (y el posible pago de regalías) del autor.
 Si no incluye explícitamente la opción ``no comercial'' se asume que el autor renunció a ese control y a ese patrimonio.

\par% p
En cuanto a las modificaciones posteriores hay dos opciones: ``sin derivados'' y ``compártase igual'' (mutuamente excluyentes pero no complementarias).
 La opción de sin derivados\begin{wrapfigure}{l}{50\px}\centering% {'src': 'http://creativecommons.org/images/deed/nd.png', 'title': 'no derivatives', 'height': '50', 'width': '50', 'alt': '[nd]', 'class': ['alignleft']}
\includegraphics[width=50\px,height=50\px]{blog/nd.png}
\end{wrapfigure}
 indica que el autor prohíbe usar su obra como base para crear una obra nueva.
 Se permiten las adaptaciones al formato para el uso específico, pero la obra debe ser escencialmente la misma si la piensa redistribuir.
 La opción de compartase igual\begin{wrapfigure}{r}{50\px}\centering% {'src': 'http://creativecommons.org/images/deed/sa.png', 'title': 'share alike', 'height': '50', 'width': '50', 'alt': '[sa]', 'class': ['alignright']}
\includegraphics[width=50\px,height=50\px]{blog/sa.png}
\end{wrapfigure}
 implica que si pueden derivarse nuevas obras pero cada nueva creación derivada debe mantener las mismas opciones: ser Creative Commons, conservar la opción de comercialización (aunque la modificación podría cerrarla, pero no abrirla), y seguir siendo ``compártase igual''.
 Si no se especifica, signifia que pueden crearse modificaciones y estas modificaciones utilizar licencias cerradas.

\anchor[http://copysouth.org/portal/]{Copysouth} es un tipo de licenciamiento que usa una forma diferencial: es más restrictivo (p. ej. un copyright completo con todos los derechos reservados) en países ricos y menos restrictivo (p. ej. un Copyleft o un dominio públic) en países pobres.
 Hay muchas otras opciones.~ En el caso del software se conocen, entre otras, licencias \anchor[http://www.gnu.org/copyleft/gpl.html]{GPL}, \anchor[http://www.gnu.org/copyleft/lesser.html]{LGPL}, \anchor[http://www.freebsd.org/copyright/freebsd-license.html]{FreeBSD} y muchas otros sistemas abiertos, además de las licencias cerradas y el dominio público.

¿Qué significa esto?

Que como autor tengo ciertos derechos a los cuales puedo renunciar total o parcialmente.
 Uno de esos derechos que puedo reservarme o no es a controlar la distribución.

En el esquema tradicional de distribución de música, el músico tiene en teoría un derecho a controlar la distribución, derecho que casi siempre cede a su casa disquera.
 Este derecho consiste en grabar material mecánico (p. ej. un LP), magnético (p. ej. cassette o disco duro), óptico (p. ej. CD) o electrónico (p. ej. una memoria flash) el cual vende sólo por medio de canales autorizados.
 Cualquier venta no autorizada de los medios controlados y la creación de nuevos medios no autorizados para vender su música es una violación a los derechos de copia.
 También es, en teoría, la copia no autorizada para compartir o, si así lo especifica, para reprodución privada.
 En este último caso, por ejemplo, si yo compro un CD y lo rippeo para utilizarlo en mi reproductor de MP3, podría
 estar violando un derecho de copia aún cuando no tenga intención de compartir mi música.

Cuando se controlan los medios hay dos efectos claros.
 El primero es que es fácil determinar que un medio no controlado es una copia ilegal.
 Pero lo segundo es que quien compra el medio cree haber comprado la obra.
 Por ello ha sido práctica común la copia y adaptación de medios para la reproducción privada sin ningún tipo de
 sanción, por la dificultad del dueño de los derechos de verificar que esas copias existan.

Pero en Internet sucede otra realidad.~ Dos realidades, realmente.

La primera es que los formatos digitalizados pueden ser copiados y redistribuidos de una forma muy fácil.
 La segunda, que toda distribución involucra canales teóricamente controlables.

\par% p
Lo primero implica que los artistas o sus editoriales no pueden controlar el medio.~ El medio \anchor[http://blog.chlewey.net/2010/03/la-retorica-de-la-libertad/]{es abierto y es libre}.~ Lo segundo implica que podría haber una forma de controlar al usuario vigilando su conexión.

Cuando un artista, o más que el artista: su casa editorial, ve que su esquema de negocio basado en controlar el medio
 de distribución se ve amenazado por Internet, su primera reacción es a querer controlar Internet, y el punto para
 hacerlo es entrometerse en la conexión del usuario final.

\par% p
Personalmente creería que es más inteligente tratar de replantear el negocio.
 Un artista puede, por ejemplo, vender 10.000 copias de un CD a \abbr{% {'style': 'font-variant: small-caps;'}
usd} 10 cada copia.
 O puede liberar su música, que sea copiada por 1 millón de usuarios de los cuales tan solo el 10\% pagarán en promedio \abbr{% {'style': 'font-variant: small-caps;'}
usd} 1 por copia.
 Más personas escuchan su música y la ganancia neta es la misma.
 Más personas que escuchan y gustan de su música significa mayores regalías por ejecución pública y más personas
 interesadas a pagar una boleta para un concierto y, muy probablemente, en este último esquema hay menos intermediarios.

Sí, de haber vendido un millón de CD habría habido una mucho mayor ganancia.
 Pero esto es un sofisma.
 La gran mayoría de esos usuarios que descargaron gratis un tema no hubieran comprado el CD: simplemente no habrían
 sido seguidores de tu música.

\par% p
Así que, como artista, antes de apoyar leyes mal diseñadas que buscan coartar las libertades civiles tales como \anchor[http://www.youtube.com/watch?v=goo-iQ\_SiLk]{el derecho a la privacidad}, deberías plantearte qué pretendes con tu obra.
 ¿Llegar a más gente y buscar oportunidades de negocio en ello? ¿O controlar la venta de medios (p. ej. vender CD)?

\&nbsp;

\chapter{Por el verde que soñamos}
\begin{metadata}
	Published by \anchor[chlewey]{chlewey} on \anchor[http://ewey.co/B1021]{Mon, 16 May 2011 18:11:51 +0000}\\
	\categories{actualidad, elecciones, information}\\
	Shorthand: \anchor[http://blog.chlewey.net/2011/05/por-el-verde-que-sonamos/]{por-el-verde-que-sonamos}
\end{metadata}

Cuando se discutía la reforma constitucional que permitiría la reelección de Uribe en 2006, no faltaban algunos personajes que se declaraban como uribistas no reeleccionistas.
 El más notable de ellos fue el columnista de El Tiempo Juan Manuel Santos.
 Pero, finalmente, el congreso aprobó la reforma constitucional y la Corte Constitucional declaró exequible esa reforma.
 Sin ningún reparo Santos se convirtió en el jefe de un partido político nuevo, el Partido Social de Unión Nacional,
 que reuniría a varios políticos afines al presidente a reelegir Álvaro Uribe Vélez.

Para las elecciones de 2010, muchos políticos afines al uribismo se declararon en contra de otra reforma más para permitir una segunda reelección.
 Varios de ellos incluso expresaron abiertamente aspiraciones presidenciales que los llevarían a competir contra Uribe si la reelección se permitía.
 Entre estos uribistas que se atreverían a competir con Uribe estaban Marta Lucía Ramírez (senadora por el Partido de
 la U), Enrique Peñalosa (ex candidato al senado por el partido Por el país que soñamos), Germán Vargas Lleras (senador
 por el partido Cambio Radical) y Noemí Sanin (embajadora ante el Reino Unido y ex candidata presidencial en 2002 por
 el movimiento político Sí, Colombia).

En ese escenario político que se avecinaba, Ramírez y Peñalosa se unieron con otros tres personajes políticos con
 aspiración presidencial: el ex candidato presidencial Antanas Mockus (quien compitió en 2006 contra Uribe), el ex
 alcalde de Bogotá Luis Eduardo Garzón (quien en 2002 compitiere contra Uribe) y el ex alcalde de Medellín Sergio
 Fajardo, quien de los cinco era el único que abiertamente declaraba que sí sería candidato presidencial en 2010.

Dentro de estos así llamados quíntuples, y ante la falta de decisión de qué querían hacer, Fajardo decidió que montaría
 una campaña política por sus propios medios creando un movimiento político llamado Convergencia Ciudadana y en alianza
 con el movimiento Alianza Social Indígena quien había avalado a Fajardo como candidato a la alcaldía de Medellín.

Ramírez se apartó también del grupo y buscó luego su aspiración presidencial dentro del Partido Conservador, junto con
 Noemí Sanín, Andrés Felipe Arias (quien siempre declaró que su precandidatura dentro del conservatismo era una forma
 de guardarle el cupo a su presidente Álvaro Uribe), José Galat (de los pocos conservadores que consideraban que Uribe
 no era lo suficientemente godo) y Álvaro Leyva (cuya posición uribista no recuerdo).

De ese grupo de quíntuples quedaron entonces los tres ex alcaldes de Bogotá: Enrique Peñalosa, Antanas Mockus y Luis
 Eduardo Garzón.

Peñalosa venía del Partido Liberal.
 En el congreso liberal de 2005 con miras a las elecciones de 2006 quiso proponer que 1) el Partido Liberal no se declarara como partido de oposición ante Uribe, y 2) que se celebrara una consulta interna en marzo de 2006.
 Los cálculos políticos le daban una mejor opción de ser candidato presidencial por el Partido Liberal en esa fecha, pero el grueso del liberalismo, de clara tendencia serpista, se encargó de que Peñalosa abandonara el partido.
 Sin un partido que avalara su aspiración presidencial, Peñalosa creó el movimiento político Por el país que soñamos,
 con el que aspiraría al senado como parte de la coalición uribista.

Con el mismo apoyo uribista y representanto una coalición del Partido Liberal y los partidos uribistas Cambio Radical y el Partido de la U, Enrique Peñalosa se presentó en 2007 para la alcaldía de Bogotá.
 La cual perdió frente a Samuel Moreno del Polo Democrático Alternativo y apoyado por una serie de liberales disidentes
 que se convirtieron en mayoría.

Antanas Mockus aspiró a la presidencia de la República en 2006 junto con un movimiento político denominado Visionarios por Colombia y avalado por la Alianza Social Indígena.
 Su aspiración fue proclamada antes de que la reelección de Uribe tuviese luz verde y en su campaña no quiso denominarse antiuribista sino posturibista.
 La campaña fue un fracaso y tras ella continuó trabajando bajo el nombre de Visionarios en un grupo que nunca se
 definió si aspiraba a ser un partido político o un centro de pensamiento.

Luis Eduardo Garzón fue candidato presidencial en 2002 como un acuerdo de partidos de izquierda.
 Se presentó como candidato de Alternativa Democrática y al recibir otros apoyos de otros partidos de izquierda se creó el Polo Democrático Independiente.
 Garzón obtuvo la tercera votación en esas elecciones y con ese caudal político aspiraría en 2003 a la alcaldía de Bogotá donde se presentó por Alternativa Democrática con el apoyo del Polo Democrático Independiente y de un amplio sector del Partido Liberal quien renunció a llevar candidato propio.
 Como alcalde estaba impedido a tomar posición en las elecciones legislativas y presidenciales de 2006, para las cuales
 Alternativa Democrática y el Polo Democrático Independiente se unieron finalmente para conformar el Polo Democrático
 Alternativo.

Al terminar su alcaldía, Garzón no se sintió identificado con el Polo Democrático Alternativo, ni estos con Garzón.
 Ante esta perspectiva Garzón había declarado a modo de chiste que fundaría el partido de la calle.

Así que los tres ex alcaldes de Bogotá.
 Enrique Peñalosa (uribista), Antanas Mockus (posturibista) y Lucho Garzón (de origen antiuribista) terminaron juntos y con la perspectiva de formar algo que llevara a alguno de ellos a la presidencia de la República en 2010.
 Ni Por el país que soñamos, ni Visionarios por Colombia ni el partido de la calle tenían personería jurídica.

Por otro lado, el Partido Verde Opción Centro, que se había constituido para las elecciones regionales de 2003 con base
 en la antigua Alianza Democrática M-19, tenía una personería pero no tenía una fuerte campaña a nivel nacional.

Así se unieron.
 El Partido Verde Opción Centro y los tres precandidatos presidenciales y formaron el Partido Verde.
 El cual, en las elecciones legislativas de marzo de 2010 logró 5 senadores y 3 representantes, y en esa misma fecha,
 en consulta abierta, Antanas Mockus obtuvo un respaldo mayoritario sobre los otros dos precandidatos.

El eje de la campaña presidencial del Partido Verde en 2010 se centró sobre los principios de Visionarios por Colombia
 de Antanas Mockus, por encima de lo que venía de Opción Centro, Por el país que soñamos y el partido de la calle.

Por otro lado, ante un fracaso en las legislativas y en las encuestas, Sergio Fajardo declinó sus aspiraciones presidenciales y se unió al Partido Verde como candidato vicepresidencial de Antanas Mockus.
 La campaña, sin embargo, se siguió centrando en las banderas de Visonarios y su visión posturibista, aun cuando el
 programa real era una amalgama de Opción Centro, Por el país que soñamos, Convergencia Ciudadana, Visionarios y el
 partido de la calle.

De los quíntuples originales, Marta Lucía Ramírez logró un honorable tercer lugar en la consulta interna conservadora que se había convertido en un referendo entre la postura más uribista de Arias y la más idependiente de Sanín.
 El triunfo de Sanín, sin embargo, evitó que el Partido Conservador se uniera al Partido de la U.
 Al final, la corriente más uribista del Partido Conservador, encabezada por Arias, terminó apoyando al candidato de la
 U, Juan Manuel Santos y abandonando a su candidata Sanín.

Con una importante segunda votación en la primera vuelta presidencial, pero sin espacio para crecer políticamente (pues
 la campaña de Mockus había sido muy crítica del Polo Democrático Alternativo y los demás partidos ya habían mostrado
 coqueteos hacia el candidato de la U), el debate presidencial para la segunda vuelta se centró en las fallas de Mockus
 por un lado y un discurso anticorrupción por el otro.

La Unión Nacional propuesta por el candidato del Partido de la U, y al cual se unieron el Partido Conservador, el Partido Liberal y Cambio Radical, fue más exitosa en su campaña y obtuvo 9 millones de votos en la segunda vuelta presidencial.
 En mi opinión, no todos ellos fueron votos uribistas.
 El Partido Verde no sumo muchos votos más, pues el Polo decidió marginarse y el discurso anticorrupción y pro derecho
 a la vida no tuvo mayor acogida frente a la realidad de que Santos sería el próximo presidente de la República.

Pero eso ya pasó y hoy el discurso se centra en las elecciones regionales de 2011.
 Por una especie de acuerdo tácito, Antanas Mockus no se opuso a que Enrique Peñalosa fuese proclamado como candidato a la Alcaldía de Bogotá.
 El mismo Enrique Peñalosa que en 2007 recibió el apoyo del entonces presidente Álvaro Uribe Vélez y que en 2006 se
 había presentado como cabeza de lista de un movimiento político de la coalición uribista.

Así que en ningún momento debe sorprendernos que Uribe siga apoyando a Peñalosa y que Peñalosa siga aceptando el apoyo
 de Uribe.

Por otro lado, si bien la campaña presidencial de 2010 se intentó enfocar en las toldas verdes como una campaña en pro
 de la transparencia política, debemos recordar que esas eran las banderas de Visionarios por Colombia, no
 necesariamente las banderas de Opción Centro, Por el país que soñamos, el partido de la calle o Convergencia Ciudadana.

Hoy veo evidente la ruptura entre los Visionarios que ganaron la consulta interna de 2010 y el resto del pragmatismo
 político del Partido Verde que fue lo que en primer lugar hizo posible al Partido Verde.

Ahora.
 Yo nunca voté por Uribe.
 Ni en 2002 (cuando voté por Garzón), ni en 2006 (cuando voté por Mockus).
 Probablemente hubiera votado por Uribe en una segunda vuelta frente a Horacio Serpa o frente a Carlos Gaviria Díaz.
 Nunca he querido denominarme antiuribista.
 Me opuse a la reelección para 2006 y me opuse para 2010 no por oponerme a Uribe sino por oponerme al caudillismo y al cambio de las reglas de juego democrático en nombre propio.
 Por otro lado siempre fui afín a los Visionarios y a lo que representa Antanas Mockus.
 Critiqué aspectos de la campaña verde en 2010.

Hoy, sin embargo, creo que Antanas Mockus se equivoca.
 Como Visionario creo que la línea visionaria del Partido Verde se equivoca al querer imponer las banderas visionarias
 como las banderas del Partido Verde, pero, sobre todo, en desprestigiar al resto de la dirigencia del Partido que no
 comulga con esta visión de la política.

También creo que Peñalosa se equivoca al quererse mostrar como uribista.
 No desde el punto de vista de estrategia electoral, pues probablemente el apoyo de Uribe y del partido de la U le
 sirva para sumar votos, sino desde el punto de vista de lo que se quería construir.

Peñalosa no necesita a Álvaro Uribe subido en la misma tarima.

Tampoco necesita a Mockus.

\chapter{Remix-Redo.  El caso de De música ligera.}
\begin{metadata}
	Published by \anchor[chlewey]{chlewey} on \anchor[http://ewey.co/B1026]{Wed, 18 May 2011 20:28:17 +0000}\\
	\categories{musica, odio, uncategorized}\\
	Shorthand: \anchor[http://blog.chlewey.net/2011/05/remix-redo-el-caso-de-de-musica-ligera/]{remix-redo-el-caso-de-de-musica-ligera}
\end{metadata}

\par% p
La última \anchor[http://twitter.com/\#!/search/\%23celedondaasco]{indignación en Twitter} fue el atrevimiento del cantante vallenato \anchor[http://es.wikipedia.org/wiki/Jorge\_Celed\%C3\%B3n]{Jorge Celedón} de hacer un homenaje a \anchor[http://es.wikipedia.org/wiki/Gustavo\_Cerati]{Gustavo Cerati} interpretando \anchor[http://es.wikipedia.org/wiki/De\_m\%C3\%BAsica\_ligera]{\emph{De música ligera}} a su estilo.~ Para quienes no lo han escuchado dejo aquí un link a Youtube (clic en la imagen):

\par% p% {'style': 'text-align: center;'}
\anchor[http://www.youtube.com/watch?v=9QiFy-qQdHk]{\includegraphics[width=266\px,height=176\px]{blog/celdonligera.png}}

Comenzaré con las aclaraciones de rigor: me gusta \anchor[http://es.wikipedia.org/wiki/Soda\_Stereo]{Soda Stereo} y, en general, me gusta lo que ha hecho Gustavo Cerati después de Soda.~ No escucho \anchor[http://es.wikipedia.org/wiki/Vallenato]{vallenatos} más allá de lo que me toca y nunca he seguido la carrera de Jorge Celedón.
 Así que mal puedo considerarme un defensor de oficio de este guajiro de Villanueva.

Dicho esto, me parece bien lo que ha hecho Celedón.
 Tal vez este fragmento cantado con solo voz y acordeón e interpretado en vivo por televisión no sea el mejor material posible (me gustaría saber cómo suena eso tras un trabajo serio de estudio), pero lo que escuché me gusta (bueno, sin el ``Ay hombe'').
 Hay algo interesante ahí.
 Un sonido interesante del acordeón vallenato que me suena a ciertas interpretaciones de bandoneon (nunca dije que yo tuviera buen oído musical).
 Sí.~ Me gusta.

No digo que me guste lo suficiente como para ir a comprar (o a bajar) el nuevo disco de Jorge Celedón.
 Pero sí lo suficiente como para considerar risible esa enorme indignación twittera.

\par% p
Recuerdo también cuando \anchor[http://es.wikipedia.org/wiki/Shakira]{Shakira} se atrevió a cantar \anchor[http://es.wikipedia.org/wiki/Nothing\_Else\_Matters]{\emph{Nothing Else Matters}} de \anchor[http://es.wikipedia.org/wiki/Metallica]{Metallica} en Montreal. (\anchor[http://www.youtube.com/watch?v=JSEmyYOxqMg]{ver en Youtube}).
 Otra gran indignación.
 Y a juzgar por los comentarios en Youtube veo que la mayoría de los indignados son colombianos y no otros seguidores
 de Metallica.

No dejo de pensar que existe cierto sentimiento de snobismo.
 Terminamos denigrando de nuestros artistas sólo porque creemos que Shakira está sobrevalorada o porque odiamos el
 vallenato, al mismo tiempo que ensalsamos glorias foráneas hasta convertir a personas como Cerati en poco menos que
 dioses.

Consideramos una gran afrenta que nuestros ídolos sean tocados por estos artistuchos nacionales.
 No toleramos esa apropiación de los altares.~ Es una profanación religiosa a nuestras más sentidas creencias.

\par% p
Por mi parte, yo confieso que me gusta esto.
 Me gusta cuando se toman temas de mis artistas favoritos y se les da otro aire.~ Cuando Guayacán tomó el clásico \emph{Yolanda} de Pablo Milanés y nos lo entregó en salsa.~ Cuando La 33 interpreta la \emph{Pantera Mambo}.~ Cuando Marylin Manson interpreta \emph{Personal Jesus} o \emph{Tanted Love}.~ Cuando Celia Cruz interpreta \emph{Obladi Oblada}.~ Cuando Natalia Lafourcade nos trae \emph{Piel canela}.

\par% p
Habrá desaciertos, sí.
 Sin duda hay versiones que no me gustan (y como mi memoria es mala para lo que no me gusta, pues no las recuerdo), pero para mí no serán más que versiones desafortunadas que no escucharé.
 Si a alguien no le gusta Jorge Celedón, no tiene por qué escuchar su \del{tributo}\ins{homenaje} a Cerati.~ Bueno, será inevitable escucharlo, pero no más de lo que le es evitar escuchar \emph{Qué bonita es esta vida}.

Aplaudo lo que hizo Jorge Celedón.
 Ojalá hubiera más personas que se arriesgaran a ello.
 Algunos harán cosas más interesantes que otros, y yo escucharé lo que me guste.
 Sin ningún sentimiento religioso herido.

\chapter{Una nueva estrategia}
\begin{metadata}
	Published by \anchor[chlewey]{chlewey} on \anchor[http://ewey.co/B1032]{Fri, 03 Jun 2011 19:54:23 +0000}\\
	\categories{activismo, derechos-de-autor, derechos-de-copia, opinion}\\
	Shorthand: \anchor[http://blog.chlewey.net/2011/06/una-nueva-estrategia/]{una-nueva-estrategia}
\end{metadata}

Aunque el activismo político no pague mis servicios, ni la matrícula de los niños, ni la matrícula del colegio, ni el
 seguro del carro, por alguna razón siento que es importante pensar sobre estas cosas que poco tienen que ver con mi
 vida práctica.

Recientemente se han presentado en este país algunos proyectos de ley que atentan contra el espíritu liberal de nuestros principios constitucionales.
 El más sonado de ellos, por su potencial de afectar a nosotros los internautas, es la denominada \#LeyLleras; un proyecto de ley que busca mecanismos rápidos extrajudiciales para eliminar contenidos que supuestamente violen los derechos de autor y derechos conexos.
 Pero junto a ellos, aunque menos sonados, está el proyecto de ley de inteligencia y contrainteligencia y el proyecto
 de ley de seguridad ciudadana.

\par% p
El proyecto de seguridad ciudadana (aunque creo que ya es ley) agrega en su artículo 44 un artículo al Código Penal
 (ley 599 de 2000) que reza así:

\begin{blockquote}
\textbf{Artículo 353A.} Obstrucción a vías públicas que afecten el orden público. El que por medios ilícitos incite, dirija, constriña o
 proporcione los medios para obstaculizar de manera temporal o permanente, selectiva o general, las vías o la
 infraestructura de transporte de tal manera que atente contra la vida humana, la salud pública, la seguridad
 alimentaria el medio ambiente o el derecho al trabajo, incurrirá en prisión de veinticuatro (24) a cuarenta y ocho
 meses (48) y multa de trece (13) a setenta y cinco (75) salarios mínimos legales mensuales vigentes y pérdida de
 inhabilidad de derechos y funciones públicas por el mismo término de la pena de prisión.
\end{blockquote}

En otras palabras, la protesta ciudadana que recurra a bloqueos será penalizada, salvo permiso previo de la autoridad
 competente.

Personalmente (y así lo saben quienes me conocen) rechazo el bloqueo de la vía pública como medio de protesta.
 Pero una cosa es que yo no crea en ese medio y otra que ese mecanismo esté prohibido por la ley.
 Sí, el artículo establece que es punible sólo si este bloqueo afecta ciertos derechos, pero esta será una situación
 que siempre se cumplirá, sobre todo por la forma tan vaga como en Colombia se establece el derecho al trabajo.

El proyecto de ley estatutaria 195 de 2011 sobre inteligencia y contrainteligencia en su artículo 35 incluye
 modificaciones al código penal que elevan de multa a prisión varias conductas sobre divulgación de documentos secretos
 y establece que las penas se doblen cuanto esta divulgación se haga a favor de grupos ilegales o gobiernos extranjeros.

Asumiendo que esto sólo contemple documentos reservados en el uso legítimo de la inteligencia y contrainteligencia por
 parte del estado, algunas de estas modificaciones pueden tener sentido, pero cuando tenemos un precedente como el de
 las chuzadas del DAS, es que vemos la gravedad de lo que esto implica: una sanción efectiva en contra de que la prensa
 o canales alternos de veeduría ciudadana que puedan garantizar la transparencia del estado.

El proyecto de ley 246 de infracciones a derechos de autor en Internet no es más que un mecanismo para hace más rápido
 y por fuera de los canales judiciales so pena de hacer copartícipe a los proveedores de internet de las posibles
 infracciones a derechos de autor.

Este proyecto de ley (llamado Ley Lleras por los internautas) es inocuo para combatir la piratería en Internet, entendiéndose por piratería la distribución de contenidos amparados por derechos de autor reservados por un precio que no reconoce regalías a los tenientes de los derechos de autor patrimoniales.
 Esta piratería que es un claro delito fácilmente se salta los canales contemplados por la Ley Lleras, tanto por
 mecanismos técnicos como por el hecho de que sus principales ingresos no están en Internet.

Entonces esta ley ¿a quién sanciona?

Existe siempre el peligro del abuso, del cual hay ejemplos reales.
 Un mecanismo extralegal como el que propone la Ley Lleras fácilmente puede ser usado por un ente censurador (no
 necesariamente un estado sino que puede ser una parti civil interesada) para bloquear rápidamente contenidos de
 rivales, de la competencia o, en general, de otros.

Pero la ley, en general, va contra el pequeño usuario en Internet.

El articulado de la ley no incluye las excepciones constitucionales que ya se reconocen en Colombia sobre derechos de
 copia para uso personal, ni las excepciones contempladas por acuerdos internacionales sobre fines educativos y
 protección a población vulnerable o en discapacidad, ni contempla los casos de emergencia social donde ya hay ejemplos
 claros de cómo la legislación de derechos de autor atenta contra el objetivo de rescatar vidas.

Al no contemplar estas excepciones y crear mecanismos extrajudiciales, la Ley Lleras atenta directamente contra el pequeño creador de contenidos: contra el que escribe un blog y lo alimenta con música.
 Contra el padre que sube un video a Youtube con su hijo bailando una canción de moda.
 Contra el profesor que comparte por Dropbox un texto con sus alumnos.

Los derechos de autor, tal y como los conocemos hoy en día, donde el registro es implícito y se extienden a 70 años
 después de la muerte del autor, son una construcción del siglo 20 utilizados por la industria de contenidos (industria
 editorial, fonográfica, del cine y de la televisión), para controlar un mercado que fue posible por la
 industrialización y amenazada por esta misma.

Pero este fenómeno de la industrialización que viene de mediados del siglo XIX y se extendió en el siglo XX, ha evolucionado en la revolución de Internet.
 La industria de contenidos fue posible por los adelantos tecnológicos, pero los nuevos adelantos tecnológicos se
 convierten en un nuevo desafío.

Y la respuesta de la industria al nuevo desafío ha sido endurecer una posición anacrónica.
 Seguir extendiendo los derechos de autor a límites absurdos, pretendiendo que la sociedad se prive, por ejemplo, de
 las así llamadas obras huérfanas (obras sin registro, sin autor, pero con derechos de copia no vencidos) y elevando
 demandas masivas contra miles de pequeños usuarios que se han hecho de copias privadas bajo la creencia de que son
 gratis.

A veces pareciera que ante una nueva realidad tecnológica que les impide seguir controlando un modelo de negocio, lo
 único que pretende ahora la industria es cambiar el modelo de negocio.

Ya no piensan vivir de las regalías que aseguran vendiendo el soporte físico de una obra cultural, sino vivir de las demandas a miles de pequeños usuarios.
 Y por ello buscan pasar leyes absurdas como la Ley Lleras, que no protege el modelo de negocio que dicen defender,
 sino crear el marco legal para su nueva estrategia.

\chapter{¿Qué no vale?}
\begin{metadata}
	Published by \anchor[chlewey]{chlewey} on \anchor[http://ewey.co/B1040]{Mon, 13 Jun 2011 21:32:18 +0000}\\
	\categories{antanas-mockus, elecciones, guerrilla, gustavo-petro, opinion}\\
	Shorthand: \anchor[http://blog.chlewey.net/2011/06/que-no-vale/]{que-no-vale}
\end{metadata}

\par% p
Hace poco tuve una breve discusión en Twitter sobre si el principio Visionario de \emph{''No todo vale''} era compatible o no con el pasado guerrillero de una persona como Gustavo Petro.

Antes de dejarme enredar yo solito en otra discusión sobre legitimidad de la fuerza pública o la lucha guerrillera
 sostenía y sostengo que no hay divergencia en el criterio siempre y cuando la discusión sea sobre el presente y no
 sobre el pasado.

\par% p
Repito primero que, en mi concepto, la lucha armada en los años 1960 y 1970 obedecen \anchor[http://ewey.co/B16]{más a la moda de una época que a causas objetivas}, y que la continuación de tal lucha en las décadas siguientes no sólo no se justifica por las condiciones del país, sino que no obedecen tampoco a un movimiento global.
 Con esto quiero decir que en ningún momento considero que alzarse en armas contra el estado colombiano es un medio
 válido para acceder al poder político en Colombia.

Por otro lado sí creo en la capacidad y necesidad de transformación de las personas.
 Así como no es válido tomar las armas para acceder al poder político, si creo que es válido dejar las armas a favor de
 nuevas formas de hacer la política.

\par% p
Lo primero, como parte de la estrategia de \emph{combinación de todas las formas de lucha}, es claramente una manifestación del \emph{''todo vale''}.~ Lo segundo es un reconocimiento de que no todo vale.

¿Es suficiente?

No.

Si tomar las armas para dejarlas después hace parte de un cálculo político, dejar las armas no constituye un propósito de enmienda sino parte de una estrategia.
 Esta tesis de la estrategia también aplica aunque el cálculo no se haya hecho antes de tomar las armas.
 Si el motivo de dejar las armas no corresponde al reconocimiento de un error sino apenas a una relectura de la
 situación.

No sé qué tan sincero haya sido el arrepentimiento de Gustavo Petro.
 No sé qué tan sincero fue el arrepentimiento de Carlos Pizarro León-Gómez como el líder del M-19 que en algún momento consideró que la lucha armada no tenía sentido y pretendió continuar por la senda política.
 ¿Fué un arrepentimiento? ¿fué una relectura de la situación geopolítica?
 ¿Qué hay de los otros líderes del M-19?~ ¿Se arrepintieron igual y sinceramente o sólo le llevaron el juego a Pizarro?

Sin embargo y en mi opinión, personas como Antonio Navarro Wolff y Gustavo Petro han hecho méritos desde 1990 en la
 vida pública para creer que su pasado guerrillero es eso: un pasado.

\par% p
Legalmente lo es.
 Hubo una ley de anmistía que no sólo condonó los delitos de rebelión y conexos en que hubieren incurrido los militantes desmovilizados del M-19 sino que les garantizaba plena amnistía política.
 Así que si mi único criterio de \anchor[http://blog.chlewey.net/2011/02/entre-la-legalidad-y-la-legitimidad/]{legitimidad es lo legal}, es legítimo elegir a ex militantes del M-19.

No sé que tan sinceros hayan sidos los actos de Petro hace 21 años cuando se desmovilizó, pero votar por él hoy es
 legal y su pasado reciente ha sido acorde con una serie de principios constitucionales y democráticos.

\par% p
Tal vez Gustavo Petro no sea la persona idónea para representar el \emph{''no todo vale''} de Antanas Mockus; pero esto no es por su pasado guerrillero, que quedó en el pasado, sino por hechos recientes.
 Personalmente objeto más la justificación de Petro de haber votado por Alejandro Ordóñez para procurador.
 Petro manifestó públicamente que no votaba por Ordóñez por considerarlo la persona idónea al cargo sino por las
 repercusiones políticas de haber apoyado su elección.

\subsection{El maldito argumento que no debí desarrollar}
Como parte de mi argumentación presenté un desafortunado símil.
 ¿Por qué no le perdonaríamos a Petro un pasado guerrillero pero sí le perdonaríamos a un soldado constitucional su
 participación en un homicidio que haya sido exculpado legalmente?

Las guerrillas colombianas, y el M-19 no fue la excepción, participó de asesinatos, secuestro y extorsión.
 Estos podremos llamarlos delitos graves y es por esos delitos por los cuales descalificamos a ex militantes como
 Gustavo Petro.

Pero existen muchas circunstancias en las cuales producir la muerte de otra persona (asesinato), retener a una persona
 en contra de su voluntad (secuestro) o amenzar su integridad, libertad o propiedades a cambio de contraprestaciones
 económicas (extorsión) no sólo no es punible sino sancionada por el estado.

Un miembro de la fuerza pública constitucional está habilitado a usar fuerza letal en contra de un delincuente.
 Incluso si comete un error de apreciación puede ser exculpado.
 Aún si tal error de apreciación ocurre en cumplimento de órdenes enmarcadas dentro de una ley o política cuestionable, el servidor público está exculpado.
 (Ojo, no hablo de obediencia debida, donde el soldado debería tener el criterio de que la orden impartida es ilegal,
 sino de un error de apreciación, p. ej. disparar contra un manifestante desarmado pero con actitud desafiante o contra
 un campesino inocente que ignoró las advertencias de detenerse e identificarse en un retén legal.)

Pero no sólo los miembros de la fuerza pública pueden ser exculpados de homicidio.
 Cualquiera de nosotros seríamos exculpados de matar a otra persona cuando defendemos nuestras vidas o frente a
 accidentes (de caza, automovilísticos, etc.) cuando no existen méritos para ser imputados legalmente de la culpa.

En cuanto al secuestro, no hay mayores diferencias en la definición entre el secuestro y la detención legal con fines de encarcelación de un presunto delincuente, salvo porque los agentes de un caso son ilegales y los agentes del otro operan bajo una normativa legal.
 Podríamos decir, también, que justificamos encarcelar al delincuente porque éste cometió una falta previa legalmente sancionada con pérdida de la libertad.
 Pero si las autoridades procesan a un inocente por una falsa denuncia, o por un error de identificación, ni el juez ni
 los policías que ejecutan la orden serán imputables de secuestro.

Técnicamente también es secuestro cada vez que obligamos a un familiar o amigo a estar donde él o ella no quieren estar.
 Sólo que ese hecho se queda en familia y rara vez será de conocimiento de un juez.

En cuanto a la extorsión, la DIAN y por extensión sus funcionarios o los jueces que actúan bajo sus denuncias son
 culpables de proferir amenazas contra la libertad o la propiedad de una persona a cambio de pagos.

El homicidio, el secuestro y la extorsión pueden ser legales o exculpables.
 Las personas que trabajan para organizaciones que, como el estado, matan, secuestran y extorsionan, tampoco se
 consideran culpables de estas prácticas; ni siquiera en el caso de que agentes del estado maten, secuestren o
 extorsionen sin justificación legal (salvo que sean partícipes directos de esos hechos ilegales).

Ahora.
 Si una persona trabajó en el departamento de cobros de cartera de una institución financiera, secuestró a un amigo por
 un par de horas como preámbulo a una fiesta sorpresa o por evadir una volqueta que lo cierra mató a un ciclista, no
 por ello recibirá una muerte política.

La diferencia entre perdonar unas acciones y condenar otras no está en la definición de la acción sino en las
 circunstancias legales que rodean la acción; y por ello mismo la ley puede variar y convertir hechos punibles en casos
 amnistiados.

\par% p
En ningún momento pretendí equiparar los hechos y mucho menos condenar la labor de los soldados.

\subsection{Y del principio verde ¿qué?}
El principio del ``no todo vale'' no es un principio fundacional del Partido Verde, sino uno de los principios de
 Visionarios por Colombia que se convirtió en caballo de batalla cuando el pre candidato Visionario Antanas Mockus ganó
 la consulta interna del Partido Verde.${}^\textrm{\anchor[http://blog.chlewey.net/2011/05/por-el-verde-que-sonamos/]{[véase aquí]}}$

Por ello mismo la colectividad Partido Verde no estaba obligada a adherirse por siempre a tal principio.

Consideraría en estos momentos un gran error de Antanas Mockus que, retirado del Partido Verde, lance su propia candidatura o se adhiera sin mayores criterios a otra campaña como la de Gustavo Petro o la de Gina Parody.
 Particularmente sería un error si esto obedeciera a un cálculo político para obtener beneficios burocráticos frente a
 un posible desdibujamiento de la campaña de Enrique Peñalosa.

Ya lo dije, el pasado guerrillero de Petro es irrelevante frente al principio del ``no todo vale'', pues si bien adherirse al M-19 iba en contra del principio, participar de la construcción política del país tras abandonar las armas es afín al principio.
 (Consideraciones de sinceridad aparte.)

Álvaro Uribe Vélez y el Partido Social de Unión Nacional (Partido de la U) nunca han expresado un rechazo a las prácticas cuestionables ocurridas durante su administración.
 Se excusan en que no fueron ellos o que siempre había ocurrido, pero no las han rechazado.

Más que las prácticas en sí (cuya responsabilidad es cuestionable) ha sido la ausencia del condena a lo que se han
 opuesto Mockus y sus seguidores.

El M-19 sí pidió perdón por su pasado guerrillero.
 Un perdón que a la luz de la legislación de 2011 no es penalmente suficiente, pero lo fue bajo la legislación de 1990.
 No estoy seguro de cual sea la posición de Petro hoy frente a su pasado guerrillero: si lo condena, si lo enorgullece,
 si se arrepintió o no.

En aras de la coherencia debe ser la postura de Petro de hoy frente a su pasado guerrillero y no el hecho mismo de
 haber sido guerrillero lo que determine si está al lado del todo vale o al lado del no todo vale.

\par% p
Pero indagar esto es un desgaste que no conviene a Mockus ni a sus seguidores.

\subsection{Y yo...}
Varias veces he expresado mi admiración por Mockus. Informalmente fui Visionario antes de ser Verde. He votado por
 Mockus dos veces para alcalde de Bogotá y dos veces para Presidente (2006 y 2010).

Hoy, sin embargo, me margino de ser Verde, Verde disidente, Visionario zanahoria, o cualquier postura que establezca
 Antanas Mockus.

\par% p
Hoy \anchor[http://blog.chlewey.net/2010/09/aunque-nos-llamen-piratas/]{estoy en otro proceso político} y en aras a la coherencia ideológica con mi nuevo proyecto político no tomaré postura frente a la alcaldía de Bogotá
 ni frente a la novela del Partido Verde, salvo que los partícipes sienten una postura frente a cuatro principios
 básicos:

\begin{itemize}

\item Transparencia en la función pública.
\item Derechos y libertades civiles.
\item Libre acceso a la cultura.
\item Universalidad y neutralidad en la red.

\end{itemize}

Salvo un pronunciamiento de cualquier campaña al respecto, mi voto permanecerá secreto.

\chapter{Con patente de corso}
\begin{metadata}
	Published by \anchor[chlewey]{chlewey} on \anchor[http://ewey.co/B1054]{Fri, 01 Jul 2011 18:48:00 +0000}\\
	\categories{activismo, derechos-de-autor, information, partido-pirata}\\
	Shorthand: \anchor[http://blog.chlewey.net/2011/07/con-patente-de-corso/]{con-patente-de-corso}
\end{metadata}

¿Qué motivación necesita un creador para crear? ¿Qué motiva a un investigador a hacer ciencia?
 ¿De dónde surge que alguien quiera desarrollar un nuevo invento?

Por mucho tiempo la humanidad creó sin que existieran patentes o derechos de autor asimilables a derecho a la propiedad.
 En el último siglo, sin embargo, la tasa de creación se incrementó considerablemente.
 También, durante este tiempo, las patentes y los derechos de autor y de copia fueron más importantes.
 Esta correlación bien podría explicar como que lo segundo haya contribuido a lo primero, que las legislación sobre la
 así llamada propiedad intelectual hayan impulsado la creación.

Esta lectura de la situación puede no ser completamente cierta.
 Hoy creamos más no porque las creaciones intelectuales estén mejor protegidas sino porque existen los medios técnicos
 que facilitan seguir creando.

Un investigador ha tenido en las últimas décadas más información, más literatura, más teorías científicas que nunca antes.
 Un inventor tiene más ideas.
 Un intérprete tiene mejor tecnología que le permite crear y darse a conocer.
 En muchos aspectos la legislación sobre propiedad intelectual tiene más bien una intención de frenar esta facilidad de
 construir sobre lo construido.

Una patente, por ejemplo, puede ser un incentivo para un investigador porque podrá lucrarse de lo descubra.
 Esta es particularmente útil porque permite que empresas inviertan en investigación porque podrán asegurar un monopolio con los resultados útiles de la investigación.
 Pero las patentes también privatizan el conocimiento y, repito, forman monopolios.
 Un investigador afín no puede construir sobre el conocimiento reservado en espera de patentes y deberá pagar luego un
 canon por el uso del conocimiento patentado.

Las regalías son un incentivo para que los autores se dediquen a crear.
 Sin duda muchos autores e intérpretes basan sus carreras artísticas en recibir regalías y vender libros y discos.
 ¿Es esta la motivación de cada autor?

La historia está llena de profesiones que surgieron en un momento dado debido a una tecnología o una filosofía y desaparecieron después.
 Un ejemplo relativamente reciente es el ingeniero de vuelo en un avión.
 Cuando los aviones jet empezaron a tomarse el transporte intercontinental y transcontinental de personas, era necesario un tripulante que se encargara de que la máquina funcionara correctamente. Un profesional que le indicara al capitán cómo funcionaba el avión para que este lo pudiera pilotear correctamente y llevar a los pasajeros seguros a sus destinos.
 La tecnología había creado así una nueva profesión: el ingeniero de vuelo.

Pero los aviones se volvieron más sofisticados y la computación llegó a las cabinas.
 Hoy en día un computador puede vigilar la máquina y darle al piloto información más precisa y oportuna de lo que un ingeniero de vuelo podría darle.
 La tecnología mató entonces a esa profesión.
 Hoy en día no existen ingenieros de vuelo.
 O, bueno, existen aun unos pocos, muy pocos, volando ciertos aviones militares complejos.

La industria editorial se creó tras una serie de invenciones tecnológicas.
 Y la industria editorial, junto con las profesiones que había creado, bien puede desaparecer frente a las nuevas
 tecnologías.

Así como no tiene sentido detener la tecnología para conservar el empleo de los ingenieros de vuelo, no tiene sentido detener el avance tecnológico para conservar ciertos empleos.
 Autores de best sellers, estrellas de rock, productores musicales, editores y muchas otras profesiones derivadas de la
 industria del entretenimiento están amenazadas por los principios propios del Internet y de las tecnologías que se han
 venido desarrollando alrededor de la red.

La misma red y sus nuevas aplicaciones han venido creando, por otro lado, nuevas profesiones.

El mundo avanza, y cada vez avanza más rápido.
 La revolución industrial hizo posible la industria del entretenimiento como la conocemos hoy en día, y la revolución
 informática la está cambiando, quitándole el poder a los productores y editores que monopolizaban la distribución del
 contenido empoderando más al creador independiente pero, sobre todo, al usuario final.

Podemos seguir legislando para proteger ciertos paradigmas y proteger ciertos modelos de producción.
 Es una opción política.
 Pero nosotros proponemos una nueva opción política.~ Nuestra opción política consiste en empoderar al usuario.

En el siglo XVIII y venideros los liberales defendieron la idea de que los individuos podían desarrollarse por fuera de la tutela del rey y de la iglesia.
 En el siglo XIX los conservadores se opusieron a ciertos modelos deshumanizantes de libertinaje.
 En el siglo XX socialistas y comunistas defendieron al trabajador frente a modelos industriales creados por el liberalismo económico y a finales del siglo los partidos verdes se opusieron a que la tecnología y los negocios acabaran ciegamente con el mundo.
 Los cambios en los modelos de producción han creado nuevos retos políticos.

Y hoy hay un nuevo cambio en la forma de producir, en la forma de interactuar, en la forma de desarrollarnos como personas y como sociedad.
 Y el Partido Pirata es una visión frente a esas nuevas formas, frente a estos nuevos retos.

Una visión donde la tecnología nos sirve para vigilar a nuestros gobernantes.
 Una visión donde la tecnología nos permite escoger lo que consumimos en términos de cultura y entretenimiento.
 Una visión donde las ideas fluyen y empoderamos al usuario a que seleccione lo que le sirva o lo que desee.
 Una visión donde el usuario no tenga miedo de asumir su poder.
 Una visión donde esa tecnología debe fluir y hacerse universal.

Esta es la visión del Partido Pirata.

\chapter{Caminando ciudades}
\begin{metadata}
	Published by \anchor[chlewey]{chlewey} on \anchor[http://ewey.co/B1057]{Sun, 07 Aug 2011 18:25:08 +0000}\\
	\categories{caminar, ciudades, olores, personal}\\
	Shorthand: \anchor[http://blog.chlewey.net/2011/08/caminando-ciudades/]{caminando-ciudades}
\end{metadata}

\anchor[http://blog.chlewey.net/wp-content/uploads/2011/08/DSC03295.jpg]{\begin{wrapfigure}{r}{300\px}\centering% {'src': 'http://blog.chlewey.net/wp-content/uploads/2011/08/DSC03295-300x225.jpg', 'title': 'Puente de Brooklyn, NYC', 'height': '225', 'width': '300', 'alt': '[Puente de Brooklyn]', 'class': ['alignright', 'size-medium', 'wp-image-1072']}
\includegraphics[width=300\px,height=225\px]{blog/DSC03295-300x225.jpg}
\end{wrapfigure}
}Caminaba por las calles de Nueva York. Fue una visita relámpago, tenía que pernoctar una noche en la gran manzana en
 camino a visitar a mis papás en Japón. Traté de contactar a mis amigos de Facebook pero sólo hasta esa mañana, ya con
 el hotel reservado alguien me respondió positivamente. Una amiga boliviana a quien no veía en 18 años y quien vivía en
 Staten Island. Así que de mi hotel en Jamaica (Long Island) me fui a la ciudad a ver qué se alcanzaba a conocer.

Me apeé donde el metro quiso dejarme, saliendo a la calle a ver letreros en chino y salí a caminar la ciudad rumbo a lo que intuí era el sur. Me topé con una ardilla frente al puente de Brooklyn pero no alcancé a preparar la cámara para registrarla antes de que saliera corriendo.
 Crucé la zona cero.~ Bordeé Battery Park y finalmente llegué al muelle del ferry a Long Island.

Ya en Battery el olor era claro.
 Ese olor a ciudad marina.
 He tratado de recordar exactamente de dónde lo reconocía porque el olor de las ciudades caribeñas es diferente.
 Es mar de trópico.
 Por otro lado Estocolmo no huele a mar.
 No lo suficiente por la baja salinidad del Báltico.
 Pero el olor era claro.
 Esos olores que te dicen que ya has estado ahí, que el ambiente te es familiar, así sea la primera vez que estás ahí.

\anchor[http://blog.chlewey.net/wp-content/uploads/2011/08/P9030258.jpg]{\begin{wrapfigure}{l}{300\px}\centering% {'src': 'http://blog.chlewey.net/wp-content/uploads/2011/08/P9030258-300x225.jpg', 'title': u'Cuervo asaltando un picnic en Yoyogi - T\u014dky\u014d', 'height': '225', 'width': '300', 'alt': u'[Cuervo en T\u014dky\u014d]', 'class': ['size-medium', 'wp-image-1071', 'alignleft']}
\includegraphics[width=300\px,height=225\px]{blog/P9030258-300x225.jpg}
\end{wrapfigure}
}Hay otros dos tipos de olores que por alguna razón me recuerdan a ciudades: los olores de la comida en la calle y el olor ácido de los buses que funcionan con alcohol.
 Cuando era niño recuerdo que en las calles de Madrid vendían castañas.
 No podría recordar con exactitud a qué huelen las castañas pero siento que cuando me encuentre nuevamente con ellas me transportarán a Madrid.
 Un destello tuve cuando comí castañas en Japón pero ese olor estaba contaminado con el azufre de los termales.

Más claro es el olor de los puestos de salchichas que de alguna forma me transportan a Estocolmo.
 Aunque no cualquier puesto de perros calientes.
 Es más, el típico puesto de perros calientes no tiene ese olor que logra una buena bratwurst y es por ello que cuando
 siento el olor correcto logro transportarme a una calle en Estocolmo o a un paseo de verano por Alemania.

En Bogotá hay un olor que a veces me sorprende al caminar, pero al caminar de noche.
 Si mal no recuerdo me lo presentaron como sándalo pero bien podría ser el cestro o zorrillo, mejor conocido como galán de la noche.
 Símplemente vas caminando por la calle y de repente aparece ese olor.

\anchor[http://blog.chlewey.net/wp-content/uploads/2011/08/P9030300.jpg]{\begin{wrapfigure}{r}{300\px}\centering% {'src': 'http://blog.chlewey.net/wp-content/uploads/2011/08/P9030300-300x225.jpg', 'title': u'T\u014dky\u014d - Shinjuku', 'height': '225', 'width': '300', 'alt': u'[T\u014dky\u014d]', 'class': ['alignright', 'size-medium', 'wp-image-1070']}
\includegraphics[width=300\px,height=225\px]{blog/P9030300-300x225.jpg}
\end{wrapfigure}
}Pero los olores no es lo único que te da la calle.
 Caminar es encontrarse con muchas cosas.
 Ver como peatones y automóviles se comportan frente a un paso peatonal o qué tan bien hecha está la ciudad para ser paseada.
 Hacerte preguntas sobre el local de comidas frente al cual acabas de pasar.
 Ver las personas solitarias o en grupos que deambulan.

Cuando caminas calles desconocidas o ciudades desconocidas la experiencia se multiplica.
 A veces es bueno simplemente caminar sin rumbo, perderte y luego averiguar cómo encuentras el camino a casa.
 Bogotá tal vez no es la ciudad más segura para hacerlo pero no me arrepiendo cuando me he atrevido, como lo hiciere en
 Venecia, Medellín o Yokohama.

Es difícil ver la cara de los demás cuando conduces.
 O sentir los aromas de la calle cuando viajas en un bus o el vagón de un metro.
 La ciudad se ve diferente desde las diversas situaciones pero de una forma u otra me gusta conocer y sentir las
 ciudades mientras me muevo por ellas.

\chapter{De redes sociales y otras geekadas}
\begin{metadata}
	Published by \anchor[chlewey]{chlewey} on \anchor[http://ewey.co/B1059]{Fri, 08 Jul 2011 16:09:59 +0000}\\
	\categories{facebook, google, opinion, social-media, twitter, web}\\
	Shorthand: \anchor[http://blog.chlewey.net/2011/07/de-redes-sociales-y-otras-geekadas/]{de-redes-sociales-y-otras-geekadas}
\end{metadata}

\anchor[http://blog.chlewey.net/wp-content/uploads/2011/07/redsocial.jpg]{\begin{wrapfigure}{r}{300\px}\centering% {'src': 'http://blog.chlewey.net/wp-content/uploads/2011/07/redsocial-300x168.jpg', 'title': 'Red social', 'height': '168', 'width': '300', 'alt': '', 'class': ['alignright', 'size-medium', 'wp-image-1063']}
\includegraphics[width=300\px,height=168\px]{blog/redsocial-300x168.jpg}
\end{wrapfigure}
}No me gusta usar el término “red social” para designar servicios web.
 La red social es el conjunto de interacciones que tenemos con los demás, bien medie un espacio físico como el colegio o la oficina, bien medie una aplicación en Internet como Facebook o Twitter, bien medie un intercambio epistolar con cartas o con email, etc.
 Pero creo que la batalla la tengo perdida y por redes sociales nos referiremos exclusivamente a las aplicaciones en
 Internet que pretenden mediarlas, \anchor[http://www.formspring.me/r/tiene-sentido-hablar-de-red-social-sin-internet-como/192331392619678112]{aunque la gente sea conciente}.

\par% p
La reflexión viene a raíz de que desde la semana pasada Google pretende ser una red social.~ Lo intentó adquiriendo \anchor[http://www.orkut.com/]{Orkut} y servicios como \anchor[http://www.youtube.com/]{YouTube}, \anchor[http://www.blogger.com/]{Blogspot/Blogger} y \anchor[http://picasaweb.google.com]{Picasa}.~ Lo intentó lanzando \anchor[http://www.google.com/buzz]{Buzz} integrado a \anchor[http://mail.google.com/]{GMail} y causando una reacción adversa en sus usuarios de email.~ Ahora lo intenta nuevamente con \anchor[http://plus.google.com/]{Google+} (gúguel plus) y mi reflexión parte de lo que he visto en Google+ y conversaciones que he tenido y lo que yo esperaría
 que fuese una red social.

\par% p
Mi \anchor[/palceres-culposos/]{incursión en la web social} fue mediante Hi5.
 Parecía tan chévere, tener a todas las personas que conozco en un sistema de intercambio de información: ver sus fotos y su actividad reciente, aunque a la larga no fue más que un espacio para entrar a ver fotos de las amigas de mis amigos.
 Creo recordar servicios anteriores que prometían mantener el contacto con los ex compañeros de la universidad o el colegio y recordar los cumpleaños y mantener las actualizaciones de teléfonos e emails.
 Me topé con LinkedIn, con un perfil mucho más profesional de Hi5 y poco a poco empezaron a llegarme invitaciones de
 otros servicios, al tiempo que yo enviaba invitaciones a Hi5 y LinkedIn.

Ya era evidente un problema.
 La fragmentación.
 Cada amigo quería estar en su web social favorita e invitar a los demás a ella, y yo no tenía interés de andar probando la web social favorita de los demás: quería tener a todos en Hi5 y LinkedIn.
 Hasta que finalmente caí en las garras de Facebook y se presentó un primer fenómeno: todos mis amigos estaban entrando a Facebook.
 Finalmente un sistema que los uniría a todos.
 Y quienes no entraban a Facebook tampoco entrarían a otros servicios.
 Tarde o temprano todos mis contactos de Hi5 (al menos todos los que realmente me interesaban) terminaron en Facebook.

Y luego conocí a Twitter.
 Twitter no era una web social sino un servicio de microblogging, así que entrar a Twitter no era abrir una web social más sino incursionar en nuevos métodos de comunicación, que eventualmente se convertiría en mi principal medio de comunicación (por encima del chat, el correo electrónico y la interacción en Facebook).
 Y finalmente uno va estableciendo conexiones personales a través de las conexiones en línea y fue así que Twitter se convirtió, en concepto, en una web social más.
 Incluso dentro del propio Facebook interactúo más con mis contactos de Twitter que con mis ex compañeros de colegio,
 universidad o trabajo.

Incursionando así en otros sabores de la web 2.0, entré a Flickr, Tumblr, YouTube, Formspring, etc. y empiezo a notar un patrón: el grueso de mis conexiones son twitteros.
 Así como twitteros fueron mis contactos en Buzz y twitteros son mis contactos en Google+.
 Podría pensar que hay básicamente tres tipos de personas en mi vida: aquellos que no les interesa o no quieren nada que ver con la web social, aquellos a quienes les basta Facebook y posiblemente algún servicio extra como Flickr, y aquellos que probamos de todo.
 [Desde luego, no todos son encasillables.]

Pero también conozco mucha reacción adversa hacia Facebook.
 No sólo de quienes rechazan el concepto de web social en sus vidas, sino de las personas que encuentro en Twitter y en otras web sociales.
 Los términos de servicio de Facebook, la exposición, el rechazo a las invitaciones a jugar vampiros, guerras de mafia y a la granja, etc.
 los amigos que te etiquetan en los anuncios de artículos para vender, el hecho de que lo que publiques lo lea tu mamá,
 tu primito adolescente, tu profesor de matemáticas en el colegio y tu ex compañero de la universidad que quiere que
 todos renovemos nuestra fe, y muchos otros fenómenos hacen de Facebook un territorio incómodo.

\par% p
Y llega ahora Google+.
 Muy aclamado por muchos (aunque, y también por ello, ya criticado por muchos otros).
 Una red social más.
 Otro sitio más donde sigo viendo las caras que veía en Buzz que no fueron diferentes a las caras que ya veo en Twitter, pretendiendo que estamos haciendo algo distinto mientras encerramos a nuestros contactos en círculos.
 \anchor[https://plus.google.com/101044403353657024129/posts/Y7htuUuAFcC]{\begin{wrapfigure}{r}{300\px}\centering% {'src': 'http://blog.chlewey.net/wp-content/uploads/2011/07/evolution-300x152.jpg', 'title': 'evolution', 'height': '152', 'width': '300', 'alt': '', 'class': ['alignright', 'size-medium', 'wp-image-1060']}
\includegraphics[width=300\px,height=152\px]{blog/evolution-300x152.jpg}
\end{wrapfigure}
}Tratando de pensar qué diferencia a Google+ del resto de redes sociales y compartiendo gráficos donde comparamos la evolución humana con la evolución de la web social.
 A la hora de la verdad sigue siendo la misma división de personas: las que no les interesan las web sociales, las que
 les basta Facebook, y las que probamos otras cosas.

\par% p
El correo electrónico lleva décadas funcionando.
 Su gran éxito es que es un estándar abierto.
 No importa si mi correo electrónico era originalmente el sistema de memos de mi empresa, o la mensajería de American
 Online \emph{(You've got mail!)}; no importa si evolucionó de los servidores Unix de la universidad, el sistema de Exchange de la empresa, de Lotus Notes o de los servicios de correo web como Hotmail o GMail.
 No importa si lo provee nuestra ISP o si es una extensión de Facebook.
 Todos podemos intercambiar información entre todos por medio del correo electrónico.
 Bueno, no es que todos tengan correo electrónico.~ \anchor[http://en.wikipedia.org/wiki/Donald\_Knuth]{Donald Knuth} \anchor[http://www-cs-faculty.stanford.edu/~uno/email.html]{dejó de usarlo hace décadas} y ya hay generaciones que obvian tener email porque con la mensajería de Facebook o de Blackberry tienen lo que necesitan.
 Pero el email funciona a través de servicios y plataformas porque se ha hecho uno.

Mi visión sería que la web social se normalizara y se volviera una.
 Cada uno de nosotros escogería el servicio que más nos agrade (Facebook, Hi5, Google+, Lotus Notes, Exchange, corporativo, privado, etc.) por cuestiones de interfaz, aplicaciones o privacidad; y desde nuestro servicio de web social podríamos interactuar con todos los demás: actualizando fotos, haciendo videollamadas, enviando mensajes, actualizando nuestro estado, compartiendo enlaces, compartiendo mensajes cortos o artículos completos de nuestra cosecha, invitando a que visiten nuestra granja o a que pongan su amor en la foto con la que competimos en un concurso.
 Y no sólo publicar, también obtener.
 Ver y filtrar lo que publican nuestros amigos, nuestros contactos profesionales, nuestras celebridades favoritas,
 nuestras marcas preferidas, etc.

No sé qué tan posible sea.
 Dentro del concepto hay algunos aspectos técnicos que resolver.
 Hay aspectos de interfaz que resolver.
 Hay aspectos de privacidad que resolver.
 Pero, principalmente, hay aspectos políticos que resolver.
 En mi visión cada usuario es dueño de sus propios datos, un esquema que (letra menuda aparte) comparte Google pero no comparte Facebook.
 El modelo de negocio de Facebook es ser el dueño de los datos de sus usuarios y convencer a los usuarios de que todas sus necesidades en la web las suple Facebook.
 Es así como Facebook acaba de integrar videollamadas como ya antes lo hiciera con el chat y con el email.
 Google es más abierto siempre y cuando ese tráfico le permita hacer una mejor minería de datos.
 Y Google también incursiona en querer ser la web, incluyendo el querer ser el navegador de la web y el sistema
 operativo de los dispositivos móviles conectados a Internet.

Intercambiar información entre plataformas web de redes sociales es un proceso tedioso pues cada quien tiene sus estándares y quiere convencer a los demás de no abandonar su pedazo de la tajada.
 Fácilmente puedes exportar tus mensajes de Twitter a Facebook pero lo contrario es engorroso.
 Y la incursión de Google+ está haciendo esto difícil pues las demás web sociales están reaccionando como si Google
 quisiera robarles su información.

\par% p
Facebook tiene la posición dominante.
 Y Facebook es el sistema más cerrado y le conviene mantenerlo cerrado.
 Y no veo a mis amigos de Facebook con el más mínimo interés de \anchor[https://plus.google.com/114362517137530736470/posts/EyLAHuSENB4]{probar algo más}.
 Mucho menos veo a mis amigos que están fuera de Facebook a querer entrar en cualquier otra plataforma de web social.
 Tal vez en algún futuro la web social sea un estándar como hoy lo es el correo electrónico, pero ese futuro está lejos
 de ser un futuro inmediato.

\chapter{Violencia de Género}
\begin{metadata}
	Published by \anchor[chlewey]{chlewey} on \anchor[http://ewey.co/B1066]{Thu, 11 Aug 2011 19:53:34 +0000}\\
	\categories{activismo, derechos-de-la-mujer, opinion, violencia}\\
	Shorthand: \anchor[http://blog.chlewey.net/2011/08/violencia-de-genero/]{violencia-de-genero}
\end{metadata}

Escribe El Tiempo en su editorial impreso que cientos de mujeres han sido asesinadas en lo corrido del año. Con
 respecto al episodio protagonizado por Hernán Darío Gómez, el periódico busca que estemos conscientes con respecto a
 la realidad colombiana en cuanto a violencia de género.

La palabra femicidio o feminicidio ha sido acuñada para referirse al homicidio de una mujer ocurrido por su condición
 de mujer. Una pregunta que siempre me he hecho es qué determina si el homicidio de una mujer ocurrió dada su condición
 de mujer o si esta condición fue más bien circunstancial. El problema que tengo con el concepto no es sólo con su vaga
 definición sino también por su relevancia estadística.

El Tiempo no contrasta la gran cantidad de mujeres asesinadas frente al número total de personas asesinadas en Colombia
 en lo corrido del año. Son muchos más los hombres asesinados. El problema de la violencia en Colombia, como casi en
 todo el mundo, es un problema principalmente masculino donde los hombres son los principales victimarios y las
 principales víctimas y en gran medida la violencia letal contra la mujer no se debe a su sexo sino a los mismos
 motivos que la violencia letal contra los hombres: plata, venganza, guerra y riñas públicas.

No tengo datos sobre la violencia no letal (p. ej. lesiones personales) pero me atrevería a apostar que las principales
 víctimas de lesiones personales motivadas por riñas callejeras y robo son igualmente hombres.

En donde sí creo que las cifras se invierten es en las relacionadas a violencia intrafamiliar. Al menos en cuanto a los
 casos más graves (podría imaginarme que hay más hombres cacheteados por mujeres que mujeres golpeadas a puño por
 hombres, pero los puños suelen ser causa de lesiones físicas y psicológicas mayores).
¿Si el hombre es más proclive a
 ser la víctima, por qué los casos que escandalizan son los casos contra las mujeres?

Hay tres factores que entran en consideración. Uno es el carácter excepcional de la cuestión. No son la norma sino las
 excepciones las que escandalizan. La violencia pública contra la mujer es relativamente rara y cuando ocurre nos
 indigna.

Un segundo factor son las diferencias físicas e histórico-sociales que ponen al hombre en un nivel superior (más
 grande, más fuerte y en posiciones de poder). Mientras que la violencia entre hombres puede verse como una lucha entre
 iguales, la violencia del hombre contra la mujer adquiere un carácter de abuso.

El tercer factor de indignación es el carácter principalmente doméstico de la violencia del hombre sobre la mujer. Al
 ser doméstico pasa desapercibido en la esfera pública y es necesario visibilizarlo y eso hace que las personas
 preocupadas por la violencia intrafamiliar se encarguen de quererlo hacer visible.

No conozco los detalles de lo acontecido con Hernán Darío Gómez, ni si es una persona proclive a la violencia en
 general y a la violencia contra la mujer en particular. Con base en la información que tengo tal parece que se trató
 de un acto relativamente aislado donde no se ha determinado el factor condicional o circunstancial de que la víctima
 haya sido mujer. Este hecho público, aparentemente excepcional, ahora se ha convertido en el caballo de batalla de
 quienes luchan por la visibilización de la violencia doméstica endémica.

Propagandísticamente el hecho es muy bueno. El problema que tengo es que es propaganda. Mal me queda defender a Gómez
 ante tantos interrogantes sueltos, pero creo igualmente que mal hacemos todos al apresurarnos a condenarlo.

\chapter{Viviendo en prisiones}
\begin{metadata}
	Published by \anchor[chlewey]{chlewey} on \anchor[http://ewey.co/B1078]{Mon, 05 Sep 2011 14:09:35 +0000}\\
	\categories{opinion}\\
	Shorthand: \anchor[http://blog.chlewey.net/2011/09/viviendo-en-prisiones/]{viviendo-en-prisiones}
\end{metadata}

\par% p
Reciéntemente veía partes del capítulo de \anchor[http://www.natgeo.tv/us/especiales/no-le-digan-a-mi-madre]{\emph{No le digan a mi madre}} con Diego Buñuel en Johannesburgo. Una mujer de un comité civil de vigilancia hablaba de la necesidad de recuperar el
 derecho a dormir con la puerta abierta, en lugar de convertir a nuestros hogares en prisiones motivados por la
 inseguridad.

Hace unos días comenzaron a hacer una obra en el edificio de al lado.
 Están cerrando la bahía de parqueo con una reja.
 De este tipo de protección ya se había hablado en nuestro edificio pero no se había emprendido la obra por falta de presupuesto.
 El caso es que mis vecinos perciben como una necesidad colocar rejas que protejan nuestro ambiente más cercano de los
 peligros de la ciudad.

\relax{% {'style': 'width: 246px;', 'class': ['alignright']}
\anchor[http://blog.chlewey.net/wp-content/uploads/2011/09/bahia-1.png]{\begin{wrapfigure}{r}{246\px}\centering% {'src': 'http://blog.chlewey.net/wp-content/uploads/2011/09/bahia-1.png', 'title': u'bah\xeda original', 'height': '120', 'width': '246', 'alt': u'[bah\xeda original, sin divisiones] ', 'class': ['alignright', 'size-full', 'wp-image-1089']}
\includegraphics[width=246\px,height=120\px]{blog/bahia-1.png}
\end{wrapfigure}
}\anchor[http://blog.chlewey.net/wp-content/uploads/2011/09/bahia-2.png]{\begin{wrapfigure}{r}{246\px}\centering% {'src': 'http://blog.chlewey.net/wp-content/uploads/2011/09/bahia-2.png', 'title': u'bah\xeda, primera modificaci\xf3n', 'height': '120', 'width': '246', 'alt': u'[primera modificacion: primera parcelaci\xf3n] ', 'class': ['alignright', 'size-full', 'wp-image-1090']}
\includegraphics[width=246\px,height=120\px]{blog/bahia-2.png}
\end{wrapfigure}
}\anchor[http://blog.chlewey.net/wp-content/uploads/2011/09/bahia-3.png]{\begin{wrapfigure}{r}{246\px}\centering% {'src': 'http://blog.chlewey.net/wp-content/uploads/2011/09/bahia-3.png', 'title': u'bah\xeda, segunda modificaci\xf3n', 'height': '120', 'width': '246', 'alt': u'[bah\xeda completamente parcelada] ', 'class': ['alignright', 'size-full', 'wp-image-1091']}
\includegraphics[width=246\px,height=120\px]{blog/bahia-3.png}
\end{wrapfigure}
}\anchor[http://blog.chlewey.net/wp-content/uploads/2011/09/bahia-4.png]{\begin{wrapfigure}{r}{246\px}\centering% {'src': 'http://blog.chlewey.net/wp-content/uploads/2011/09/bahia-4.png', 'title': u'bah\xeda, primer cerramiento', 'height': '120', 'width': '246', 'alt': u'[bah\xeda con primer cerramiento] ', 'class': ['alignright', 'size-full', 'wp-image-1092']}
\includegraphics[width=246\px,height=120\px]{blog/bahia-4.png}
\end{wrapfigure}
}\anchor[http://blog.chlewey.net/wp-content/uploads/2011/09/bahia-5.png]{\begin{wrapfigure}{r}{246\px}\centering% {'src': 'http://blog.chlewey.net/wp-content/uploads/2011/09/bahia-5.png', 'title': u'bah\xeda, cerramiento en curso', 'height': '120', 'width': '246', 'alt': u'[bah\xeda con segundo cerramiento] ', 'class': ['alignright', 'size-full', 'wp-image-1093']}
\includegraphics[width=246\px,height=120\px]{blog/bahia-5.png}
\end{wrapfigure}
}}La bahía de parqueo estaba originalmente diseñada para suplir la necesidad de parquaderos de visitantes a siete edificios, pero en algún momento la administración de uno de los edificios decidió levantar muros y abrir su propia entrada y mantener así privada su parcela de la bahía.
 Pronto todos los demás edificios siguieron el ejemplo (salvo los dos del extremo occidental) motivo por el cual una
 gran bahía se conviertió en seis bahías más pequeñas.

Cuando aún no había cesado el debate sobre si las bahías podían cercarse o no, el edificio del extremo oriental decidió
 hacer el cerramiento de su bahía instalando una reja a su alrededor para tranquilidad de sus residentes.

En distintas asambleas de propietarios y reuniones de consejo de administración nunca me he opuesto frontalmente a la
 idea, pero varias veces he expresado que no me gusta.

Actualmente la bahía de parqueadero de visitantes, con todos los obstáculos impuestos por muros y cadenas destinados a
 la parcelización, se convierte en una suave transición entre la ciudad y el edificio.

Recuerdo una vez que llegué a la madrugada después de una noche de trabajo.
 Venía con el láptop de la empresa y en el trayecto entre donde me dejó mi taxi y la portería del edificio se me acercó un tipo.
 Se acercó como si quisiera preguntar algo, pero apenas vio al portero al interior de mi edificio se devolvió.
 (Hablando días después con el portero, éste no se dio cuenta de nada).

A veces me pregunto qué hubiera sido si hubiera tenido que esperar afuera de una reja a que el portero la abriera.

Creo que en nuestro afán de hacer más seguro nuestro entorno inmediato estamos abandonando el resto de la ciudad a nuestra responsabilidad.
 Hacemos más segura nuestra bahía de parqueadero de visitantes y hacemos más insegura la calle por fuera de la bahía.

Nos imponemos obstáculos a nosotros mismos pero, una vez adentro, nos sentimos más tranquilos.

Separamos más claramente una ciudad allá, insegura, amenazante, cruel y un acá, la comodidad y la seguridad del hogar.
 Olvidando, a veces, que llegar a esa comodidad del hogar implica atravezar esa ciudad que nos empecinamos en abandonar
 y hacerla más insegura.

\par% p
A la teoría de Giuliani y sus \anchor[http://es.wikipedia.org/wiki/Teor\%C3\%ADa\_de\_las\_Ventanas\_Rotas]{ventanas rotas} (Wilson, Kelling, 1982) me gusta pensar también en la teoría de las bahías encerradas.
 En la medida en la que nos encerremos en nuestras casas, convirtiéndolas en pequeños castillos o, más exactamente, en
 pequeñas prisiones y abandonamos la ciudad, más insegura estamos haciendo esa ciudad.

No construimos prisiones para encerrar a los antisociales y sacarlos de las calles, sino prisiones para encerrarnos nosotros y dejar que los antisociales se apropien de las calles que hemos abandonado detrás de la reja.
 Y al tiempo que renunciamos a la ciudad, la hacemos y percibimos más insegura y por ello mismo sentimos más necesario
 seguir recluyéndonos en nuestras propias prisiones.

\chapter{Planeando bodas}
\begin{metadata}
	Published by \anchor[chlewey]{chlewey} on \anchor[http://ewey.co/B1098]{Fri, 16 Sep 2011 16:57:18 +0000}\\
	\categories{blogger-secreto, bloggersecreto, blogs, bodas, matrimonio, personal}\\
	Shorthand: \anchor[http://blog.chlewey.net/2011/09/planeando-bodas/]{planeando-bodas}
\end{metadata}

\par% p
Falta poco más de una semana para que se case la menor de mis primas.
 Entre todos mis primos, primas y mi propio núcleo familiar es la tercera boda.
 Estaba pensando si el número es pequeño pues igual tampoco es que tenga t\relax{% {'style': 'font-stretch: wider;'}
a\relax{% {'style': 'font-stretch: wider;'}
a}a}ntos primos (dos de lado de mi madre, cuatro de parte de mi papá, mi hermana y yo) aunque ya todos pasamos de los 30.

\anchor[http://plannersyco.blogspot.com]{\begin{wrapfigure}{r}{250\px}\centering% {'src': 'http://blog.chlewey.net/wp-content/uploads/2011/09/bodas.jpg', 'title': 'Bodas', 'height': '1079', 'width': '250', 'alt': '[bodas] ', 'class': ['alignright', 'size-full', 'wp-image-1099']}
\includegraphics[width=250\px,height=1079\px]{blog/bodas.jpg}
\end{wrapfigure}
}Como no soy de las personas a las que invitan sus amigos y compañeros de trabajo a sus propias bodas, pues lejos estoy de ser un experto en bodas.
 La mía propia fue bastante sencilla.
 Un poco a las carreras porque tocaba aprovechar que mis papás llegaban de vacaciones y el cura que mi esposa quería no
 se fuera aún de la ciudad.

A veces creo que aquello de la boda está bastante mitificado.
 Veo una tendencia creciente de la gente en no creer en el matrimonio y creo que es claro que en el mundo de hoy ya la
 realización de una mujer no es casarse y tener hijos, pero creo todavía percibir ese mito del día de la boda como el
 día más especial con el que muchas niñas crecen añorando.

\par% p
La boda perfecta.
 Sé, tras diez años de matrimonio, que lo importante es lo que vives con esta persona elegida.
 Más de la mitad de mi vida adulta la he pasado al lado de esta persona y, a pesar de mi \anchor[http://es.wikipedia.org/wiki/S\%C3\%ADndrome\_de\_Peter\_Pan]{síndrome de Peter Pan}, no puedo dejar de pensar en mí como el esposo de Beatriz y el papá de Sebastián y Ana María.
 La boda fue un incidente que oficializó lo primero.
 Algo que bien pudo ocurrir frente a un notario o en una ceremonia religiosa privada.

El primer día de lo que serían mis días.~ Por ello es importante.

Pero es un día.~ Apenas el primero de lo que se esperan sean muchos días.

Pero esa boda perfecta también tiene su razón de ser.

Es una razón ceremonial.
 Si el matrimonio ha de definirnos, es claro que el inicio del mismo sea un momento para compartir y así como celebramos los grados (p. ej. el inicio de nuestra vida adulta y profesional) bien vale la pena celebrar los casamientos.
 Y, ¿por qué no? Celebrarlos en grande.

\anchor[http://blog.chlewey.net/wp-content/uploads/2011/09/cakes.jpg]{\begin{wrapfigure}{l}{250\px}\centering% {'src': 'http://blog.chlewey.net/wp-content/uploads/2011/09/cakes.jpg', 'title': 'Bodas', 'height': '740', 'width': '250', 'alt': '[bodas] ', 'class': ['size-full', 'wp-image-1100', 'alignleft']}
\includegraphics[width=250\px,height=740\px]{blog/cakes.jpg}
\end{wrapfigure}
}\anchor[https://twitter.com/YelenaWP]{Yelena Castañeda} organiza bodas.
 Aún no se qué tan rentable sea organizar bodas en Yopal, no porque la gente no se case en la capital del Casanare sino
 porque no es una ciudad tan grande como Bogotá o Cali, pero la idea podría funcionar.

\par% p
Mientras tanto, y aunque no tenga que planear ninguna boda pronto, pasarse por el blog de \anchor[http://plannersyco.blogspot.com/]{Planners y Co.}, es una buena forma para distraernos viendo cómo se casa la gente en otros lugares, las fiestas que organizan los
 famosos, ideas para ponqués de bodas que bien podrían repensarse para cumpleaños y grados, etc.

¿Por qué no tomar, por ejemplo, la costumbre francesa de invitar a ciertos conocidos únicamente para el postre?

Por ahora tengo por lo menos 15 años (bien podrían ser 25 o más) para planearle la boda a Ana María.
 Eso si no es de las que se unen al club de las que nunca se casan o de las que se casan a lo sencillo o por su cuenta.

Pero las primeras comuniones no están tan lejos, y vendrán los 15 años (¿preferirá un viaje?) y los grados....

\par% p
Siento no haber indagado más sobre mi \anchor[https://twitter.com/search/\%23bloggersecreto]{\#bloggersecreto} de este año.~ A diferencia \anchor[http://blog.chlewey.net/2010/09/bloggersecreto/]{del año pasado} mi tiempo ha estado algo ocupado entre el trabajo y un proyecto personal.~ Por el momento les comparto \anchor[http://secretplanning.blogspot.com/]{lo que hice de endulzada}.

\chapter{Estados fallidos}
\begin{metadata}
	Published by \anchor[chlewey]{chlewey} on \anchor[http://ewey.co/B1108]{Fri, 16 Sep 2011 13:54:10 +0000}\\
	\categories{actualidad, colombia, estado, estados-unidos, haiti, opinion, somalia}\\
	Shorthand: \anchor[http://blog.chlewey.net/2011/09/estados-fallidos/]{estados-fallidos}
\end{metadata}

\anchor[http://commons.wikimedia.org/wiki/File:2010\_Haiti\_earthquake\_relief\_efforts\_by\_the\_US\_Army.jpg]{\begin{wrapfigure}{r}{300\px}\centering% {'src': 'http://blog.chlewey.net/wp-content/uploads/2011/09/2010_Haiti_earthquake_relief_efforts_by_the_US_Army-300x217.jpg', 'title': '100115-N-4774B-445', 'height': '217', 'width': '300', 'alt': u'[Soldados estadounidenses en Hait\xed] ', 'class': ['alignright', 'size-medium', 'wp-image-1110']}
\includegraphics[width=300\px,height=217\px]{blog/2010_Haiti_earthquake_relief_efforts_by_the_US_Army-300x217.jpg}
\end{wrapfigure}
}Tras el terremoto de Haití en enero de 2010 escuché muchas voces denunciando el hecho de que el gobierno de facto (quien garantizaba el orden público, distribuía las ayudas, organizaba a los escuadrones de rescate, etc.) era el ejército de los Estados Unidos y no las autoridades haitianas.
 Pero esas críticas no llegaban a la descalificación del hecho.
 Las críticas venían de una declaración de principios antiimperialistas pero parecían razonables en reconocer que de
 otra forma no hubiera funcionado.

Tal vez las críticas más duras venían del propio interior de los Estados Unidos y no por el temor imperialista sino por
 los recursos gastados, -aunque, igual, esas personas parecían reconocer que si no se actuaba así tendrían luego a sus
 puertas un problema de refugiados—.

\anchor[http://commons.wikimedia.org/wiki/File:Oxfam\_East\_Africa\_-\_A\_family\_gathers\_sticks\_and\_branches\_for\_firewood.jpg]{\begin{wrapfigure}{l}{300\px}\centering% {'src': 'http://blog.chlewey.net/wp-content/uploads/2011/09/Oxfam_East_Africa_-_A_family_gathers_sticks_and_branches_for_firewood-300x199.jpg', 'title': 'Familia recoge ramas parara una fogata - Oxfam East Africa', 'height': '199', 'width': '300', 'alt': '[Familia kenyata] ', 'class': ['alignleft', 'size-medium', 'wp-image-1109']}
\includegraphics[width=300\px,height=199\px]{blog/Oxfam_East_Africa_-_A_family_gathers_sticks_and_branches_for_firewood-300x199.jpg}
\end{wrapfigure}
}En estos últimos días nos enfrentamos a un nuevo reto humanitario.
 Una larga racha de sequía ha afectado el oriente africano afectando principalmente a Kenya, Sudán, Etiopía, Eritrea y Somalia.
 A pesar de sus recientes guerras (incluyendo la independencia de Sudán del Sur), en Sudán, Etiopía y Eritrea la
 comunidad internacional ha podido ofrecer su ayuda a través de distintas ONG y de los propios estados que administran
 su presencia.

La situación en Somalia es cláramente diferente.
 No hay un estado como tal que goce de suficiente control, estabilidad y reconocimiento.
 Hay estados de facto comandados por caudillos guerreros que han desplazado a las ONG internacionales y a cualquier
 estructura de organización estatal interna.

\anchor[http://commons.wikimedia.org/wiki/File:Somali\_Pirates.jpg]{\begin{wrapfigure}{r}{300\px}\centering% {'src': 'http://blog.chlewey.net/wp-content/uploads/2011/09/Somali_Pirates-300x208.jpg', 'title': u'Piratas somal\xedes', 'height': '208', 'width': '300', 'alt': u'[Piratas somal\xedes] ', 'class': ['alignright', 'size-medium', 'wp-image-1111']}
\includegraphics[width=300\px,height=208\px]{blog/Somali_Pirates-300x208.jpg}
\end{wrapfigure}
}Antes de la hambruna, las principales noticias de Somalia venían por cuenta de sus piratas.
 Ante la ausencia de un estado formal con su respectiva armada, los propios caudillos así como otros aventureros se
 lanzaban al mar en proyectos que combinaban proteger las aguas nacionales de Somalia y extorcionar a buques
 extranjeros para lucro personal.

Sin embargo, a pesar de esta reciente piratería y a pesar de la presente emergencia humanitaria de la hambruna, la
 comunidad internacional no se atreve a intervenir en Somalia.

Hay dos grandes diferencias entre Somalia y Haití.
 El primero es que Haití ha estado en años recientes en un proceso de consolidación institucional y, aunque sus recursos internos sean insuficientes, hay un gobierno, hay un estado con suficientes garantías de estabilidad, cierto grado de control y con reconocimiento internacional.
 Segundo, ese estado permitió, aceptó o incluso invitó a las fuerzas militares de los EE.UU. para que apoyaran y
 lideraran las labores de rescate y reconstrucción.

Tal vez Haití sea un estado fallido.~ Pero hay un estado.

Somalia ni siquiera es un estado fallido, y cualquier cosa que medio pretenda ser un estado allí rechaza fuertemente la intervención extranjera.
 Cualquier fuerza internacional, sean los EE.UU., la OTAN, la Unión Europea, incluso los mismos cascos azules de la
 ONU, serán rechazados por los caudillos guerreros.

Ya los países del primer mundo tienen suficientes guerras pretendiendo combatir el terrorismo en Irak y Afganistán, o
 acudiendo a la ayuda de los hasta hace poco rebeldes y hoy reconocidos líderes libios como para entrar a una aventura
 militar con el objetivo de imponer ayudas humanitarias a una población que se está muriendo de hambre.

Es una idea difícil de vender al interior de sus propios países y difícil de vender frente a una comunidad
 internacional que está harta de intervencionismo.

A veces me pregunto si el concepto de «estado fallido» no es más que una justificación para un nuevo colonialismo.
 Claramente el caso de Haití nos muestra que hay estados que no tienen los recursos suficientes para atender las necesidades de sus propias poblaciones y que la intervención internacional puede ser un alivio necesario cuando es dirigida por países que sí tienen esos recursos.
 Tal vez pensar en un nuevo colonialismo sea una propuesta políticamente incorrecta pero no del todo absurda e
 injustificable.

Pero del término también se abusa.

\anchor[http://www.stanford.edu/group/commwiki/cgi-bin/mediawiki/index.php?title=Image:Colombia\_farc-rebel.jpg]{\begin{wrapfigure}{l}{300\px}\centering% {'src': 'http://blog.chlewey.net/wp-content/uploads/2011/09/Colombia_farc-rebel-300x194.jpg', 'title': 'Rebeldes de las Farc', 'height': '194', 'width': '300', 'alt': '[Rebeldes de las Farc] ', 'class': ['alignleft', 'size-medium', 'wp-image-1112']}
\includegraphics[width=300\px,height=194\px]{blog/Colombia_farc-rebel-300x194.jpg}
\end{wrapfigure}
}Muchos han querido ver en la Colombia pre-Uribe a un estado fallido y justifican su tesis en que en un gran número de municipios las autoridades civiles habían sido desplazadas por amenazas de las Farc.
 En que en gran parte del territorio las autoridades de facto eran las guerrillas o los paramilitares.

Sin embargo, estoy seguro que si en 2001 hubiera habido una emergencia humanitaria como la de Somalia en 2011, o la de
 Haití en 2010 (o como la reciente emergencia invernal aquí mismo en 2010–2011) hubiera sido el estado constitucional
 apoyado por sus fuerzas militares quienes hubieran administrado las ayudas y no caudillos rebeldes o fuerzas armadas
 extranjeras.

Esta tesis no implica que la existencia de un estado funcional sea garantía suficiente para evitar el drama humanitario.
 Los problemas que ha tenido la administración de Santos para subsanar los problemas de la reciente emergencia invernal en Colombia no implican que Colombia sea un estado fallido.
 Los propios EE.UU. que cumplieron un papel aceptable (tal vez incluso bueno) administrando las ayudas en Haití tuvo
 problemas administrando la emergencia causada por Katrina en Nueva Orleans.

A pesar de la corrupción y la ineficiencia del estado (nacional o regional).
 A pesar del aparente abandono.
 A pesar de la existencia todavía de caudillos rebeldes, aún los colombianos no hemos caído en un estado de abandono y
 desesperanza.

Aun no se llega al estado de que las madres tengan que abandonar a sus hijos más pequeños a que mueran de innanición con la esperanza de que sus hijos mayores logren llegar a un campamento de refugiados al otro lado de la frontera.
 (No descarto que pase en casos muy puntuales, solo que está lejos de ser algo generalizado como sí pasa en Somalia.)

Quienes en 2002 creían que Colombia era un estado fallido encontraron dentro de la misma institucionalidad colombiana
 una solución de su agrado.

\chapter{Tiramisú gratis}
\begin{metadata}
	Published by \anchor[chlewey]{chlewey} on \anchor[http://ewey.co/B1129]{Tue, 11 Oct 2011 15:09:03 +0000}\\
	\categories{aborto, activismo, legalizacion, opinion, tiramisu}\\
	Shorthand: \anchor[http://blog.chlewey.net/2011/10/tiramisu-gratis/]{tiramisu-gratis}
\end{metadata}

\anchor[http://commons.wikimedia.org/wiki/File:6439\_-\_Luzern\_-\_Tiramisu.JPG]{\begin{wrapfigure}{r}{300\px}\centering% {'src': 'http://blog.chlewey.net/wp-content/uploads/2011/10/Tiramisu.jpg', 'title': u'Tiramis\xfa adornado con fresas, uchuva y carambolo.', 'height': '225', 'width': '300', 'alt': u'[Tiramis\xfa] ', 'class': ['alignright', 'size-full', 'wp-image-1130']}
\includegraphics[width=300\px,height=225\px]{blog/Tiramisu.jpg}
\end{wrapfigure}
}Vuelve y suena el asunto de la penalización del aborto.
 Hace varios años la Corte Constitucional en una sentencia determinó que en tres casos excepcionales: en caso de
 violación o inseminación no consentida, en caso de deformidad en el feto no compatible con la vida y en el caso en que
 la vida de la madre corra peligro.

Este aborto despenalizado debería realizarse dentro de los primeros tres meses de gestación.
 Un término en el que más o menos se considera que el desarrollo neuronal del embrión/feto no se ha desarrollado lo
 suficiente para ser realmente considerado como un ser que siente e interactúa con su entorno.

Ahora, como abortar (en esos tres casos) ya no se penaliza, muchas voces salieron a decir que ahora el aborto es un derecho amparable de la mujer.
 Se exigía a las EPS a realizar los abortos.
 Se obligaba a hospitales, incluso aquellos regentados por comunidades religiosas opuestas al aborto, a que tenían que
 practicar la terminación de la vida del embrión/feto, bajo el argumento de que la objeción de conciencia es individual
 y no institucional.

En algún momento me perdí en la argumentación: una sentencia que planteaba la despenalización de una práctica bajo
 ciertas condiciones (en mi opinión tal vez insuficientes), se convertía en todo un derecho amparable.

El tiramisú es uno de los postres más ricos que existen en mi opinión, y que yo sepa consumir tiramisú no está prohibido.
 Es decir, comer tiramisú es un derecho.
 Y como es un derecho yo apelo al estado a que garantice mi derecho y que ningún repostero alegue que no le pagué para
 negarme mi derecho a consumir tiramisú.

Exijo que el estado ampare mi derecho al tiramisú.

Retomando la seriedad del argumento, la interrupción voluntaria del embarazo tiene varios puntos de vista.

Muchos buscan ver esto como un conflicto entre el derecho de una mujer de decidir sobre su cuerpo v/s el derecho de un ser humano en potencia de existir.
 A veces creo que este es un enfoque equivocado tanto de quienes defienden el derecho al aborto como de quienes
 promueven su penalización.

Si a mí no me gusta mi mano, bien podría pensar en cercenármela.
 Exigir al sistema de salud que me provea el servicio de amputación y luego, directa o indirectamente, obligar a la sociedad a que se adapte a mi invalidez.
 Es sensato pensar que el estado se oponga a semejante disparate a pesar de que es mi derecho sobre mi cuerpo.
 Sin embargo, reconozco, podría haber una mayor carga a la sociedad cuando una mujer decide tener un hijo que cuando
 decide abortarlo, en contraste con el que se amputa su mano.

Hay muchas instancias donde el estado o la sociedad se oponen a mi derecho a decidir sobre mi cuerpo.
 El estado penaliza los intentos de suicidio y la asistencia al suicidio.
 El estado me obliga a usar cinturón de seguridad.
 Se mete en contra de mi voluntad cuando quiero poner en peligro mi vida.~ Algunas de estas intromisiones las aceptamos.

¿Por qué el estado o la sociedad no podrían decir algo cuando mi derecho sobre mi cuerpo además involucra otra vida
 humana?

No tengo estadísticas, pero podría pensar que la mayoría de los casos de embarazos abortados no corresponden a las tres
 excepciones de ley.

Veamos cada una de las excepciones y comencemos con el riesgo a la vida de la madre.

Hay muchas razones por las cuales el personal médico puede tener que decidir entre la vida de dos personas.
 Bien por la escasez de recursos (medicamentos, personal, tiempo) bien por razones médicas, es posible que el personal
 médico tenga que dejar morir a un paciente con el fin de que otro paciente viva.

Ese dilema bien puede presentarse cuando un feto amenaza la vida de su madre y el evaluador médico considera que no es
 posible salvar a ambos.

Este es un dilema que bien puede presentarse dentro de los primeros tres meses de gestación, o bien puede presentarse más adelante.
 Incluso cerca al momento del parto natural.

Si el personal médico decide salvar la vida del ser más viable (p. ej. la madre) no debería ser penalizado.
 Pretender salvar la vida del embrión o del feto de pocas semanas podría incluso ser un absurdo si la madre muere.

Volver a penalizar este caso es un absurdo jurídico.

En cuanto a la deformación incompatible con la vida, el gran dilema es definir qué es una malformación incompatible con la vida.
 Casos como el síndrome de Down nos muestra que ciertas personas pueden tener limitaciones, una expectativa de vida reducida, y sin embargo ser personas plenas.
 No toda malformación sería incompatible con la vida.
 ¿Dónde se establece el límite?~ ¿Qué diferencia hay entre esto y la eutanasia?

El argumento de que Dios así lo quiere y de que no podemos oponernos a la voluntad de Dios es un argumento teológico, no jurídico.
¿Cuál sería el argumento jurídico?
 Es más.
 ¿Qué tan fácil es determinar una malformación incompatible con la vida durante los tres primeros meses de gestación?

Declaro abierto este punto.

El tercer caso: violación o inseminación no consentida.

En muchos de los argumentos pro-abortistas se habla del derecho de la mujer v/s el derecho de quien aun no hemos determinado si sí es o no una persona.
 El derecho de una mujer sobre su cuerpo v/s el derecho de un parásito.

En muchos casos ese parásito está ahí por decisiones de la misma mujer.
 No la decisión de embarazarse sino la decisión de tener sexo con protección nula, insuficiente o inadecuada.
 Muchas decisiones mal tomadas acarrean consecuencias que afectan a muchas personas: un accidente por conducir ebrio, por ejemplo.
 Pretender que el aborto sea un simple procedimiento médico para enmendar errores es algo que en mi opinión atenta
 contra la responsabilidad social.

Pero es claro que frente a la violación y frente a la inseminación artificial no consentida, no se trata de un acto de irresponsabilidad de la mujer.
 No podemos convertir a la mujer en doble víctima: primero la victima del acto en sí y luego en la víctima del hijo de
 su violador.

El embarazo humano está lejos de ser ese acto perfecto que los defensores del diseño inteligente quieren vendernos.
 El cuerpo de la mujer no está perfectamente diseñado para albergar a un feto.
 Antes de la medicina moderna la principal causa de muerte de la mujer estaba relacionada con el embarazo y el parto lo
 que claramente demuestra que está lejos de ser un hecho para el cual el ser humano esté perfectamente diseñado.

Un hijo no nato exige mucho a su madre.
 Le exprime la vida.
 La pone en riesgo.
 Limita su vida social y laboral.
 El peso extra afecta la columna.
 Los órganos internos se desplazan.
 Respirar se dificulta cuando el espacio que ocupa el feto comprime hacia arriba los pulmones.

Claramente, cuando no parte de una decisión libre, exigir a la mujer que continúe con un embarazo no deseado es una
 segunda victimización (así una vez nacido el parásito, este se convierta en el centro de su vida y afirme convencida
 de que es lo mejor que le pudo haber pasado).

Particularmente, negarle el poder de decidir a esta mujer victimizada cuando no existen argumentos biológicos (sino tan
 solo teológicos) de que esa masa de células sea ya una persona, no tiene mayor sentido.

Mi opinión sigue siendo que el aborto no me gusta.
 Mi opinión sigue siendo que el tema del aborto no es un caso cerrado y que hay argumentos fuertes en ambas posturas.
 Mi opinión sobre el aborto es, sin embargo, que desde el punto de vista jurídico el estado no puede obligar a continuar un embarazo bajo los tres casos (y posiblemente más) contemplados por la sentencia de la Corte y que es un exabrupto legislar para convertir tesis teológicas en leyes.
 También es mi opinión que no podemos convertir una despenalización en un derecho amparable.

Personalmente preferiría que no se penalizara en ningún caso, pero que tampoco se incluyera en el POS.
 Que así como hay reposterías donde puedo ir a pagar por mi tiramisú, existan clínicas donde bajo todos los estándares
 de seguridad para la mujer, esta pueda ir a practicarse un aborto cuando su conciencia así lo dicte y que sea
 responsabilidad de nuestros legisladores encontrar los mecanismos para que estos casos sean tan infrecuentes como sea
 posible.

Infrecuentes porque la mujer tiene los mecanismos para prevenir un embarazo no deseado, desde la capacidad de decirle
 no a su novio, hasta la capacidad de protegerse adecuadamente.

Infrecuentes porque la medicina avanza hacia situaciones en la que la vida de la madre no esté en riesgo antes de que
 el feto sea viable.

Infrecuentes porque los padres conocen la probabilidad de que existan malformaciones genéticas antes de ser padres y
 asuman responsablemente sus decisiones.

Infrecuentes porque un violador sabe que no saldrá impune.

En fin.

\chapter{Imágenes de conejitos}
\begin{metadata}
	Published by \anchor[chlewey]{chlewey} on \anchor[http://ewey.co/B1133]{Thu, 24 Nov 2011 20:41:49 +0000}\\
	\categories{imagen, marcas, opinion}\\
	Shorthand: \anchor[http://blog.chlewey.net/2011/11/imagenes-de-conejitos/]{imagenes-de-conejitos}
\end{metadata}

No sé hasta qué punto mi mente relaciona algunos hechos de formas extrañas o si hay también otras personas que hacen
 las mismas analogías. Una de mis más recientes discusiones mentales viene a raíz de la imagen que Almacenes Éxito
 escogió para su papel higiénico de la casa: un conejito blanco.

\anchor[http://jorgemlopez.tumblr.com/post/11333762666/los-conejos-que-desaparece-el-mago-se-convierten]{\begin{wrapfigure}{r}{224\px}\centering% {'src': 'http://blog.chlewey.net/wp-content/uploads/2011/11/conejoexito-224x300.jpg', 'title': u'Papel higi\xe9nico \xc9xito', 'height': '300', 'width': '224', 'alt': u' [Papel higi\xe9nico \xc9xito] ', 'class': ['alignright', 'size-medium', 'wp-image-1137']}
\includegraphics[width=224\px,height=300\px]{blog/conejoexito-224x300.jpg}
\end{wrapfigure}
}La marca Scott (de Kimberly-Clark) escogió como su imagen institucional a unos cachorritos de labrador. ¿Por qué no
 podría escoger Éxito otro animalito tierno?

Bueno, no sé. Los perritos de Scott son unos seres tiernos y juguetones que agarran el papel y juegan con él, corren
 con él, arman un nido con él y duermen sobre él. La imagen que parece transmitir la marca es que es un papel
 resistente (para resistir el juego), rendidor (para desenrollarse en un largo juego de cachorros), suave (para servir
 de cama para los animalitos). Por otro lado genera recordación de marca (el papel de los perritos).

Claramente uno no relaciona a los cachorritos de Scott como si estos animalitos fuesen el papel.

\par% p
En cambio la imagen de Éxito parece ser precisamente eso. El papel es blanco como el conejito y es suavecito como el
 conejito. El conejito \textbf{es} el papel.

Y es ahí donde mis relaciones mentales empiezan a jugar. La misma suavidad que sugiere el conejito me hace imaginarme
 sentir esa suavidad cuando uso el papel… pero para lo que uno usa el papel higiénico es para limpiarse las nalgas
 untadas de popó.

El popó que queda en las nalgas después de defecar es una substancia pegachenta (por eso hay que limpiarla) y lo que
 hace suave al conejito son sus pelos. Esos pelos y algo pegachento e inmediatamente se me viene a la mente de qué será
 sentir en la raja del trasero una mezcla de popó y pelos de conejo. Esta imagen mental no me es del todo agradable.

La siguiente imagen es que justo después de usar al conejito no lo dispondré como dispongo el papel (a la papelera o al
 inodoro) sino que será ahora una cosa ahí untada de mierda saltando por toda la casa. Esa imagen mental tampoco me es
 del todo agradable.

Por otro lado la imagen del conejito como algo suavecito tal vez funcione en la foto. O tal vez funcione con los
 conejos de mago de fiestas infantiles que son unos seres completamente dóciles y bobos. Pero he visto conejos de
 verdad y algo que me han dejado claro los conejos de verdad es que debajo de esa figura suave y tierna tienen patas
 con uñas.

Y sólo imaginarme un conejo de esos reales en el acto de ser usado como papel higiénico me hace pensar que el animalejo
 ese querrá usar sus uñas.

Y esa es otra imagen mental que tampoco me es del todo agradable.

En otras palabras no. No me gustan las imágenes que la bendita foto del conejo en los empaques de papel higiénico marca
 Éxito pone en mi mente.

\par% p
Pero tal vez en otras personas forme \anchor[http://www.formspring.me/r/un-lindo-y-tierno-conejito-blanco-si-es-la-imagen-mas-certera-para-representar-a-una-linea-de-papel-higienico/261591311058995334]{un imaginario diferente} y esas imágenes logren que alguien quiera comprar “el papel de los conejitos”. Vaya uno a saber.

\chapter{Reflexiones afrocolombianas}
\begin{metadata}
	Published by \anchor[chlewey]{chlewey} on \anchor[http://ewey.co/B1135]{Fri, 09 Dec 2011 17:14:51 +0000}\\
	\categories{indignacion, movilidad-social, opinion, racismo}\\
	Shorthand: \anchor[http://blog.chlewey.net/2011/12/reflexiones-afrocolombianas/]{reflexiones-afrocolombianas}
\end{metadata}

Entre las muchas cosas que he estado fallando como padre ha sido en lograr involucrar más a mi hijo en asumir sus
 propias responsabilidades. Cuando mi hijo tiene una tarea que requiera investigar algo en revistas, libros o Internet
 solemos ser mi esposa y yo (y sobre todo mi esposa) quienes googleamos, imprimimos y le entregamos todo hecho para que
 nuestro hijo cumpla con su deber escolar.

En una tarea reciente le pidieron buscar fotos de afrocolombianos. Muy diligente mi esposa se puso a buscar fotos que
 mostrara a los afrocolombianos y su cultura. Muchos trajes coloridos usados en fiestas en la región pacífico y otros
 estereotipos. Me sentí mal por no haber participado más en la investigación y, sobre todo, en la discusión sobre el
 tema. Hay una foto que me hubiera encantado incluir pero que no sabía cómo buscar: en un almanaque interno que
 Ericsson publicó para 1976 o 1979 aparecía en uno de los meses un colombiano de ascendencia africana estudiando en
 Estocolmo y aplicando las nociones de telefonía digital que vendría a implementar en Colombia. Sin duda la foto de
 este pionero de la telefonía digital en Colombia sería por mucho más significativa que mostrar negritos bailando en
 alguna festividad en el Chocó, y no sólo porque rompía el estereotipo del afrocolombiano sino porque el ingeniero que
 aparece en al foto es el abuelo de mi hijo.

\anchor[http://blog.chlewey.net/wp-content/uploads/2011/11/revista-hola.jpg]{\begin{wrapfigure}{r}{300\px}\centering% {'src': 'http://blog.chlewey.net/wp-content/uploads/2011/11/revista-hola-300x212.jpg', 'title': u'El Beverly Hills cale\xf1o', 'height': '212', 'width': '300', 'alt': '[la foto]', 'class': ['alignright', 'size-medium', 'wp-image-1140']}
\includegraphics[width=300\px,height=212\px]{blog/revista-hola-300x212.jpg}
\end{wrapfigure}
}Esto que sería una reflexión interna lo hago público a raíz de la polémica causada recientemente por la foto que publicó la revista española Hola.
 En ella cuatro generaciones de una prestigiosa familia caleña posan en su mansión mientras que las empleadas de
 servicio aparecen como parte de la decoración ante el fondo de la ciudad.

Se habla de que las Zarzur, o el fotógrafo, o el periodista, o la revista (qué sé yo) son racistas.
 Que las disculpas como que en la familia quieren a las susodichas empleadas son hipócritas y un burdo intento de
 esconder ese racismo o, como escuché insinuaciones, de perpretar formas de esclavitud modernas.

¿Hay racismo en Colombia?~ ¿Hay un conflicto racial?

La periodista colombo-española Salud Hernández-Mora insinuaba que el conflicto es latente.
 Que la casi totalidad de desplazados por el conflicto son afrodescendientes como se puede observar en cualquier
 reunión de desplazados en Bogotá.

No creo tener más o mejores datos que Hernández-Mora pero creo que hay varias cosas que podrían viciar la apreciación.
 Creo que a esos negros no los desplazaron por negros sino porque eran una población vulnerable que vivía en tierras ricas y apetecidas sin ser ellos ricos, que por medio de la forma como se crearon los palenques y comunidades negras en Colombia les tocó esa suerte.
 Algunos indígenas también corren esa suerte, así como muchos campesinos blancos y mestizos pero el grueso de los
 indígenas que vivían en zonas ricas fueron desplazados en siglos anteriores y los que hoy sobreviven y son víctimas
 del desplazamiento podrían tener otro tipo de relaciones con la tierra que los hace menos proclives de venir a Bogotá.

Podría ser también que los grandes depredadores de tierras sean más proclives a convertir a los campesinos blancos, mestizos e indios despojados en jornaleros de sus minas o plantaciones y, en contraste, a los negros despojados simplemente los corren de sus tierras.
 Lo cual sí sería un motivo racista.

Las empleadas de Sonia Zarzur son negras.
 ¿Por qué?
 Desconozco si es una política de la familia Zarzur de sólo contratar negros como empleados de servicio doméstico pero
 creería que es más una cuestión de oferta y demanda.

\par% p
Lo \anchor[http://youtu.be/RawqWDpV6Z8]{exponía} Josefina Klinger en TEDxBogotá 2010: una de las principales aspiraciones de las jóvenes de Nuquí (y otras poblaciones
 pobres, olvidadas y negras) es trabajar en el servicio doméstico en Medellín, Bogotá y Cali (ver video minuto
 5:18-5:50).

Hay desigualdades.
 Hay racismo.
 Pero hay mucho más que simplificaciones de que los negros son pobres y que los blancos sólo pueden ver a los negros
 como sirvientes o desplazados.

\par% p
No sé si mi padre perdió oportunidades por ser negro, pero el ha sido un profesional exitoso.
 Hijo de un panadero de La Dorada (un imigrante de las Antillas) y una campesina del Altiplano.
 Huérfano de padre desde los 10 años.
 No solo se graduó de ingeniero y trajo la telefonía digital a Colombia y hoy, ya pensionado, lo llaman a Guatemala a
 migrar al nuevo sistema de \anchor[http://en.wikipedia.org/wiki/AXE\_telephone\_exchange]{AXE}.
 Su hermana (mi tía) se graduó de médica y desarrolló su carrera en neonatología cuando aun no existía la neonatología
 como especialidad médica, convirtiéndose en una de las fundadoras de tal especialidad en la Universidad Nacional.

\par% p
Identificandome racialmente como negro, no veo mayor racismo en la foto de Hola.
 Es una foto que refleja una realidad mucho más compleja que una lectura racial o clasista.
 No es que las empleadas de las Zarzur sean negras.~ Soy de los que creen que si fuesen de piel clara sin duda \anchor[http://alexisivansocco.wordpress.com/2011/12/08/la-hipocresia-suma-una-polemica-a-la-revista-hola/]{habría habido menos polémica}, sin que la problemática hubiera sido diferente.

El hecho simple de que existan familias que viven en mansiones y personas que sirvan en esas mansiones (independiente
 de si las unas son blancas y las otras negras) es un punto a reflexionar pero no necesariamente un punto que yo
 condene.

Hay y habrá desigualdad social.
 En gran medida gran parte del progreso que hemos logrado como humanidad se debe a la aspiración de personas de superarse por encima de los demás.
 Hay personas que tienen la suerte de tener mejores oportunidades y personas que, como mi papá y mis tíos o como Josefina Klinger crean oportunidades y personas que tienen otro tipo de vocaciones o, símplemente, una falta cultural de aspiraciones.
 Hay problemas de mentalidad pero también de ética de trabajo.
 Hay quienes asumen el servicio personal como una vocación y no como la única aspiración.

\par% p
Y, retomando el tema de mi post anterior sobre la publicidad y la identidad de marca, veo los últimos comerciales de Límpido, la marca de cloro limpiador de JGB.
 Ahí sale la \emph{Blanquita} de siempre, el personaje interpretado por \anchor[http://www.soho.com.co/testimonio/articulo/yo-soy-blanquita-la-de-limpido/5619]{Alicia García}, quien siempre fue mostrada como la empleada.
 Una mujer negra en uniforme que muestra su condición de subalterna en un hogar.
 Pero ahora no sale como tal.~ Sale como el ama de casa.~ Nuevos tiempos de corrección política, supongo.

No veo que la foto de Hola refleje algo que no suceda.
 Por mucho, la principal crítica que veo que le cabe es que muestren al servicio doméstico como decoración de la
 mansión.

\chapter{Pensando en Navidad}
\begin{metadata}
	Published by \anchor[chlewey]{chlewey} on \anchor[http://ewey.co/B1142]{Sat, 24 Dec 2011 15:29:31 +0000}\\
	\categories{fechas, information, navidad}\\
	Shorthand: \anchor[http://blog.chlewey.net/2011/12/pensando-en-navidad/]{pensando-en-navidad}
\end{metadata}

\begin{blockquote}% {'style': 'float: right; padding: 0 1ex; margin: 0 0 1ex 1em;'}

\par% p% {'style': 'font-style: italic;'}
La Virgen se está peinando
entre cortina y cortina.
Sus cabellos son de oro
y su peine de plata fina.

\par% p% {'style': 'font-style: normal; text-align: right; font-size: .9em;'}
Los peces en el río
Tradicional villancico

\end{blockquote}

Pasaba esta mañana por la sacristía de la Iglesia de la Concepción y su tienda de imágenes y figuras sagradas y junto
 con muchas otras reflexiones que hay en esta época me fijaba en la iconografía actual que tiene Jesús Cristo y la
 sagrada familia.

La historia nos cuenta que Jesús nació en una tradicional familia judía, descendiente de Jesé y de David. La imagen que
 tenemos hoy en día de los judíos actuales está basada principalmente en los asquenazíes y sus rasgos fisonómicos más
 notables tienen posible origen en Asia Central y el Cáucaso, más que en Palestina. En la época de Jesús, sin embargo,
 es posible que sus rasgos sean más afines a lo que hoy pudiéramos esperar de alguien proveniente de Siria, Jordania o
 Palestina.

Algo de especulación hay aquí también, porque finalmente la Media Luna Fértil del Medio Oriente fue una zona de
 permanente tránsito humano. La misma tradición bíblica nos dice que en lo que hoy es Israel se asentó Abraham
 proveniente de Mesopotamia y su descendencia fue a parar al Egipto faraónico de donde regresarían a Tierra Santa para
 luego ser invadidos por los babilonios, los griegos y los romanos, entre otros.

Después de Jesús esa área fue conquistada por los árabes, los francos (europeos), los turcos, los ingleses y,
 finalmente, los judíos asquenazíes; y durante todas estas conquistas no sólo llegaban las élites de los nuevos
 gobernantes sino las migraciones internas dentro de cada imperio.

\anchor[http://commons.wikimedia.org/wiki/File:Sgo\_coraz\%C3\%B3n16.jpg]{\begin{wrapfigure}{l}{200\px}\centering% {'src': 'http://upload.wikimedia.org/wikipedia/commons/3/3b/Sgo_coraz%C3%B3n16.jpg', 'title': u'Sagrado Coraz\xf3n', 'height': '267', 'width': '200', 'alt': u'[Sagrado Coraz\xf3n] ', 'class': ['alignleft']}
\includegraphics[width=200\px,height=267\px]{blog/Sgo_coraz__n16.jpg}
\end{wrapfigure}
}No sé cómo lucía Jesús, pero muy probablemente la imagen que hoy tenemos de un señor de cabellos y barba rubia y ojos
 azules sea bastante errada.

A veces cuando observamos figuras para el pesebre realizadas por indígenas del Altiplano andino, o por comunidades
 negras en África, observamos que copian los rasgos de su propia raza en las figuras de José, María, Jesús, los
 pastores y los reyes magos. Y cuando observamos esas figuras muchos de nosotros consideramos tales adaptaciones como
 pintorescas.

Pues bien: la imagen de una sagrada familia con fisionomía europea y cabellos rubios es una también representación
 pintoresca de origen germano y nor-italiano, que se popularizó durante el renacimiento pues representaba los ideales
 de belleza de los artistas florentinos.

\anchor[http://commons.wikimedia.org/wiki/File:Arsen\_Bulmaisimisze.qriste.jpg]{\begin{wrapfigure}{r}{324\px}\centering% {'src': 'http://upload.wikimedia.org/wikipedia/commons/7/79/Arsen_Bulmaisimisze.qriste.jpg', 'title': u'Jes\xfas por Arsen Bulmaisimisze', 'height': '391', 'width': '324', 'alt': u'[imagen de Jes\xfas en el siglo 13] ', 'class': ['alignright']}
\includegraphics[width=324\px,height=391\px]{blog/Arsen_Bulmaisimisze_qriste.jpg}
\end{wrapfigure}
}Si observamos la iconografía anterior al renacimiento, es común una visión de Jesús con rasgos más medioorientales,
 ojos oscuros y cabello y barba negros. Pero estas iconografías igual son bastante posteriores a la época en la que
 vivió el Jesús histórico.

\par% p
Y ya que mencioné a Jesé arriba, me remito a uno de los gozos que los colombianos rezamos todos los años en la Novena
 de Aguinaldos:

\begin{blockquote}
¡Oh raíz sagrada de Jesé
que en lo alto
presentas al orbe
tu fragante nardo!
¡Dulcísimo niño
que has sido llamado
lirio
 de los valles
bella flor del campo!
\end{blockquote}

Muchas versiones hablan de José en lugar de Jesé en lo que parece ser una contracorrección (tratar de corregir algo
 que estaba bien). Se conoce como el Árbol de Jesé a la descendencia de Jesé o Isaí, padre de David. No repito toda la
 línea de David a Jesús porque Mateo y Lucas presentan algunas diferencias aunque igual, ambas genealogías terminan en
 José de Nazaret quien supuestamente es padre putativo y no biológico de Jesús.

\chapter{Tomando SOPA}
\begin{metadata}
	Published by \anchor[chlewey]{chlewey} on \anchor[http://ewey.co/B1148]{Fri, 20 Jan 2012 19:31:58 +0000}\\
	\categories{activismo, derechos-de-autor, opinion}\\
	Shorthand: \anchor[http://blog.chlewey.net/2012/01/tomando-sopa/]{tomando-sopa}
\end{metadata}

\anchor[http://www.imdb.com/title/tt0063462/]{\begin{wrapfigure}{r}{209\px}\centering% {'src': 'http://blog.chlewey.net/wp-content/uploads/2012/01/theproducers-209x300.jpg', 'title': 'The Producers', 'height': '300', 'width': '209', 'alt': '[The Producers poster] ', 'class': ['alignright', 'size-medium', 'wp-image-1154']}
\includegraphics[width=209\px,height=300\px]{blog/theproducers-209x300.jpg}
\end{wrapfigure}
}Un productor musical es una persona que acompaña al artista y lo complementa en sus falencias. Provee al cantante de
 una orquesta y provee a la banda de otros instrumentos que harán que su música suene más profesional. Graba, edita,
 crea un producto musical más allá de lo que el artista es capaz de hacer por sus propios medios. El productor acompaña
 al artista. Le aconseja qué temas tocar. Le sugiere nuevos autores. Conoce lo que el público quiere y aconseja al
 artista a lograrlo. Crea esa conexión entre el artista y el público. Trabaja de mano con las emisoras para que roten
 la música del artista. Lleva los CD a las tiendas de música. Conoce los canales de consumo para llegar a ellos
 liberando al artista de esa tediosa promoción y lograr que estos puedan pensar mejor en seguir creando y mejorando su
 presentación.

Un artista sin productor tiene que costear la grabación de su propia música. Tiene que usar sus propias uñas para
 promocionar su creación. Llamar a las emisoras, a las tiendas y a los demás canales de distribución para que su música
 se conozca. Sin poder contratar músicos que lo acompañen instrumentalmente ni editores musicales profesionales su
 trabajo no tendrá nunca el sonido de estudio profesional que el público requiere.

El productor es necesario en la industria y si desaparece frente a las nuevas amenazas tecnológicas los artistas y los
 consumidores perderán. El artista perderá oportunidades y el consumidor de música perderá calidad.

¿O no?

Hoy en día un buen computador personal puede convertirse en un estudio de grabación y un puesto de edición con calidad
 profesional. Los artistas pueden compartir pistas entre sí y un cantante puede recibir la colaboración de un
 guitarrista en algún otro lugar del ciberespacio. Las estaciones de radio y la televisión pública abierta no son los
 únicos espacios de promoción. Un buen producto o un producto que llegue al corazón de un público se puede promocionar
 a sí mismo gracias al voz a voz de las redes sociales.

En los años precedentes a la última revolución digital el sueño de un artista musical era contratar con una casa
 disquera, con un productor que hiciera todo lo que describí en el primer párrafo. Pero ese sueño estaba reservado para
 pocos. El negocio del productor consistía en tener pocos productos que pudiera comercializar masivamente. Tener
 estrellas. Fabricar estrellas.

El productor trabajaba también con unos pocos medios cerrados y controlados de distribución: emisoras de radio,
 televisión, tiendas de música.

El grueso de los músicos se veía rezagado a trabajar por su propia cuenta. Sin un productor. Como se describió en el
 segundo párrafo. Sin acceso casi a los pocos medios de distribución existentes. Sin sonar en la radio ni vender en las
 tiendas.

Las herramientas provistas tras la última revolución digital han abierto la posibilidad de que más artistas puedan
 acceder a medios alternos de distribución. Muchos artistas, incluso, han redescubierto que su posibilidad de obtener
 una subsistencia a partir de su música no viene de la cantidad de temas vendidos sino de la capacidad de ser
 contratados para tocar en vivo. Y con las nuevas tecnologías y los nuevos medios ya pueden obviar a la disquera para
 lograr ese reconocimiento.

Desde luego que el productor pierde.

O, más bien, el productor que basaba su modelo de negocio en la capacidad de controlar unos pocos y escasos medios
 pierde.

También pierde la estrella musical. Aquel que, gracias a sus productores, lograba un alto reconocimiento y contratos
 cautivos y que, gracias a ello, podía renegociar las condiciones para obtener un poco más de lo que el productor
 reserva para el grueso de sus artistas.

Esta tecnología que permite a los nuevos artistas tener canales alternos de distribución permite también otra cosa: que
 los usuarios generales compartan. Desafortunadamente para la industria fonográfica tradicional, los usuarios no se
 limitan a compartir la música independiente: también comparten la música de estrellas musicales.

\par% p
Desde el inicio de la industria editorial los consumidores siempre han compartido. Los libros impresos se prestan entre
 amigos. Cuando la música empezó a grabarse y a distribuirse por medio de discos, los usuarios se prestaban e
 intercambiaban los acetatos. El cine logró un mejor control porque no vendía un producto que el público pudiera
 conservar para reproducir más tarde. \anchor[http://blog.chlewey.net/wp-content/uploads/2012/01/home\_taping\_is\_killing\_music.png]{\begin{wrapfigure}{r}{280\px}\centering% {'src': 'http://blog.chlewey.net/wp-content/uploads/2012/01/home_taping_is_killing_music.png', 'title': 'Home taping is killing music', 'height': '231', 'width': '280', 'alt': '[HTIKM campaign poster] ', 'class': ['alignright', 'size-full', 'wp-image-1156']}
\includegraphics[width=280\px,height=231\px]{blog/home_taping_is_killing_music.png}
\end{wrapfigure}
}Con la aparición de cassette (originalmente de audio y luego de video) el usuario tuvo la capacidad de hacer copias
 tanto para uso personal como para compartir con familiares y amigos. Y sí, también vinieron empresarios independientes
 que decidieron hacer copias para comercializarlas por fuera del control de las disqueras y los estudios.

Pero la industria de contenidos se mueve. Ese cassette que amenazaba la venta de un disco de acetato también provocó la
 aparición del walkman y la industria musical descubrió que era rentable vender los LP también en cinta. Los estudios
 descubrieron que después de la presentación en teatros de sus películas podían seguir obteniendo ganancias por el
 alquiler de VHS en las videotiendas. (Antes habían descubierto que la televisión no solo era una amenaza sino que
 podían venderle a las cadenas los derechos de transmisión de películas que ya habían terminado su ciclo en los
 teatros.)

Hoy los consumidores tienen más y mejores formas de compartir e Internet ha logrado que los usuarios establezcan
 conexiones con personas que no conocen pero comparten intereses. Yo conozco pocas personas en mi circulo social
 inmediato que se interesen en la construcción de lenguas o la historia alterna, pero gracias a Internet pude
 establecer contacto con personas en todo el mundo que comparten esos intereses. Y un interés que muchas personas
 comparten es el consumo de contenidos culturales como la música.

Pero los productores temen.

El negocio se les está escapando.

La capacidad de la gente de compartir permite a los nuevos artistas promocionar más fácil sus nuevas creaciones y
 buscar colaboración para llevarlas a niveles más profesionales sin la necesidad del productor. Pero también visibiliza
 la cantidad de extraños que comparten entre sí y de forma anónima productos cuya comercialización antes controlaban.

Es una falacia llamar piratería a este esquema.

La industria editorial y fonográfica siempre ha sabido que los consumidores que comparten se convierten en promotores
 que generan nuevas ventas. Ayudando muchas veces a conquistar mercados nuevos. El Jazz y el Rock no se convirtieron en
 música universal porque las disqueras de EE.UU. hubieran promocionado sus discos en tiendas y emisoras de todo el
 mundo sino porque los consumidores viajaban con su música y llegaban a nuevas regiones donde la música empezaba a
 gustar, originalmente sin ganancia para los productores pero eventualmente creaban el mercado donde estos productores
 podían tener ganancias adicionales.

\par% p
El problema que enfrentaban los productores era cuando \relax{% {'id': 'independientes'}
comerciantes independientes} copiaban y vendían las copias dentro de los nuevos mercados o incluso en casa, capturando el mercado que el productor
 hubiera deseado.

Fueron estos empresarios independientes a quienes la industria editorial y fonográfica empezó a llamar piratas y a
 exigir a sus gobiernos y a gobiernos foráneos una legislación que protegiera sus propios intereses frente a estos
 piratas.

La industria no temía al usuario que compartía un libro o un disco, sino al que copiaba y vendía la copia al por mayor.

Pero llegamos ahora a una nueva revolución tecnológica. Una que favorece al usuario y al artista y productor
 independiente sobre una industria basada en controlar medios de distribución.

Esa industria que quiere llamarnos ahora a todos nosotros piratas. Una industria que pretende destruir cualquier medio
 de distribución que no pueda controlar. Y quiere hacerlo no porque sea el nido de empresarios independientes (piratas)
 que se lucren por fuera de su esquema de distribución sino porque ellos mismos se han quedado obsoletos.

Y de ahí viene la SOPA y la PIPA. Y viene el SINDE, el HADOPI, la \#LeyLleras y demás legislación que bajo el pretexto
 de proteger a los propietarios intelectuales y combatir la piratería no son más que lobistas quieren obtener ganancias
 por demandar a los usuarios que promocionan gratis sus productos.

Leyes que no hacen nada, siquiera, para combatir a los empresarios que distribuyen por lucro obras copiadas sin el
 permiso de la industria de contenidos. Esos empresarios que acaso serían lo que podríamos llamar propiamente piratas.

\chapter{Escepticando}
\begin{metadata}
	Published by \anchor[chlewey]{chlewey} on \anchor[http://ewey.co/B1152]{Tue, 07 Feb 2012 21:30:48 +0000}\\
	\categories{escepticismo, personal}\\
	Shorthand: \anchor[http://blog.chlewey.net/2012/02/escepticando/]{escepticando}
\end{metadata}

\par% p
Pasaba por \anchor[http://www.randi.org/]{un sitio web} dedicado al escepticismo y en los foros me encontraba con una discusión recurrente: ¿puede uno declararse cristiano (o
 creyente de cualquier religión) y a la vez escéptico? ¿O es el escepticismo una actitud exclusiva del ateísmo?

Crecí dentro de una tradición católica. Aprendí mis bases científicas y filosóficas de parte de los Hermanos de las
 Escuelas Cristianas (Hermanos Lasallistas). Me hice profesional en una universidad regida por la Compañía de Jesús.
 Tomé a conciencia mi confirmación a los 16 años. Me casé por la iglesia y bauticé a mis hijos quienes ahora estudian
 en un colegio laico de orientación católica. Me creo, sin embargo, escéptico. Un escepticismo que me lleva a
 considerarme agnóstico más que ateo entre otras cosas porque no dejo de considerarme católico pues, aunque no tomo el
 dogma con fe de carbonero, siento una relación fuerte con todo este sistema de valores que enmarca mi educación
 cristiana.

¿Qué es una religión? Una religión tiene un componente mitológico. Cuando hablo de mitología no me refiero a que sea
 falso sino en el sentido de aquella forma como dentro de la religión se explica cómo llegó a ser el mundo y cómo este
 es. Como escéptico pongo en duda la veracidad de la mitología cristiana y sé que hay personas mucho más creyentes que
 yo que hacen esto mismo.

La posición oficial del catolicismo es que la creación en siete días tiene un significado mucho más simbólico que
 literal. La falta de una estructura vertical hace que otras iglesias cristianas tomen mucho más literal la Biblia en
 contra de la evidencia científica, pero igual hay creacionistas dentro del catolicismo como hay evolucionistas que se
 consideran cristianos no papistas. Mis dudas sobre la mitología cristiana tocan al mismo Jesús histórico.

\par% p
La existencia misma del Jesús histórico es discutible. No hay mayores corroboraciones contemporáneas de la existencia
 de Jesús. No hay crónicas de su juicio y ejecución, o de los demás detalles que marcan su existencia. El historiador
 Josefo \anchor[http://es.wikipedia.org/wiki/Testimonio\_flaviano]{parece mencionar} a Jesús indirectamente en un párrafo de \emph{Antigüedades Judías}, párrafo del que \anchor[http://es.wikipedia.org/wiki/Testimonio\_flaviano\#Versiones]{hay evidencias de alteración} en las transcripciones subsiguientes. Todos los demás documentos antiguos sobre Jesús son documentos cristianos
 escritos después de su muerte.

Asumiendo la historicidad de Jesús:
 ¿Nació de una virgen?
 ¿Esta virgen es a su vez hija de otra virgen?
 ¿Resucitó en cuerpo y carne a los tres días de su muerte?
 ¿Ascendió al cielo? Hoy sabemos que no existe un lugar en las alturas llamado cielo ¿Exáctamente a dónde ascendió Jesús?
 ¿Ocurrieron realmente los milagros narrados en los evangelios?~ ¿Es hijo único de Dios?~ ¿Es Dios?

Además del componente mitológico, una religión es también una filosofía de vida, un conjunto de valores. Incluso he
 leído a miembros de la curia católica decir que elementos de la religión tales como la divinidad de Jesús no son
 realmente el eje de nuestra religión sino los valores que Jesús (histórico o literario) nos revela. Los valores
 propios del cristianismo, aquellos que lo hicieron original en su época, se centran en la fe, la esperanza y la
 caridad.

¿Son estos valores los que mejor nos convienen como seres humanos? Los romanos habían adoptado de la cultura griega
 valores como la razón que en gran medida se contrapone a la fe. Friedrich Nietzsche, entre muchos otros pensadores,
 nos lleva a rechazar valores como la fe y la esperanza por la forma como estos se oponen al progreso y la razón. Karl
 Marx llama a la religión el opio del pueblo porque estos valores de fe y esperanza han sido utilizados para someter al
 grueso de los súbditos y fieles. La caridad cristiana también recibe críticas de pensadores quienes ven ahí una falta
 de incentivos para el progreso humano.

Poner en duda los principios mitológicos y de valores de mi religión ¿me hacen un apóstata o un hereje? O, por el
 contrario, ¿no desecharlos me hace un mal escéptico?

En principio un escéptico es una persona que pone en duda todo y saca sus propias conclusiones a partir de la
 información que puede probar. Pero todos, por más escépticos que seamos, tenemos una serie de creencias.

Yo creo más en la evolución que en la creación del mundo. Tiene más sentido para mí con todo lo que he leído el
 universo antiguo propuesto por la biología evolucionista y la astrofísica que el universo joven descrito por la
 Biblia. Pero, la verdad, no tengo corroboración de lo uno o lo otro. Ante la imposibilidad de hacer mis propios
 experimentos tengo que recurrir a distintas autoridades de uno u otro campo, como Richard Dawkins y Stephen Hawking
 por el universo antiguo y la vida evolucionada o quien quiera que haya escrito el libro del Génesis en la Biblia o sus
 equivalentes en otras religiones a favor del universo joven y creado.

A decir verdad, desde mi punto de vista, el universo pudo haber aparecido hace cinco minutos y que todo lo que yo
 recuerdo que pasó en mis casi 40 años de vida no sean más que recuerdos implantados.

Pero ¿quién me implantó esos recuerdos? ¿Por qué? ¿Qué sentido tiene que todo el universo no sea más que algo que
 existe en mi mente y ha existido tan solo por unos pocos minutos? Si bien no tengo una prueba de que esto no sea así,
 las otras teorías, incluyendo la creación judeocristiana, tienen más sentido. Un mundo que percibo como más antiguo
 que mi propia existencia tiene más sentido que un mundo completamente imaginado por mí.

Y, bajo ese mismo criterio, prefiero creerle a un Dawkings y un Hawking que han pensado el universo con la ayuda de
 toda una comunidad científica que los nutre y los cuestiona que lo que dice un libro escrito hace más de dos mil años
 por personas sin credenciales científicas a menos que acepte la hipótesis de que un dios que lo sabe todo dictó tales
 libros.

Pero, que yo prefiera creerle a estos personajes científicos no me hacen necesariamente un creyente ciego de cierta
 interpretación de la ciencia.

Dawkings y Hawking bien pueden estar equivocados. Probablemente sí existe un dios todopoderoso que creó todo incluyendo
 la fabricación de las evidencias que Dawkings y Hawking usan como base de sus teorías. Ellos también han tenido
 críticas dentro de la propia comunidad científica y hay científicos reconocidos que ponen en duda sus conclusiones, no
 por un sesgo religioso sino porque la esencia de la ciencia es controvertir lo que se conoce para ponerlo a prueba y
 aproximarnos a la verdad.

Así que aunque creo en estos científicos no tengo una fe ciega en ellos. Si ellos niegan a Dios no quiere decir que
 Dios no existe. Es decir, Dios no deja de existir porque Nietzsche, Dawkings, Hawking y otros lo nieguen. Tampoco
 existe porque la Biblia lo diga. Pero, incluso, la interpretación de un universo autocontenido de Hawking, o de la
 existencia medible de genes y memes de Dawkings, o los valores que propone Nietzsche en contrapeso a la fe, esperanza
 y caridad tiene que ser así sólo porque ellos lo digan.

Para mí son simplemente interpretaciones más creíbles. Pero en últimas es eso, una creencia.

Creo que lo esencial de ser escéptico no está en renunciar a toda creencia sino en saber que lo que uno cree no es la
 verdad absoluta y por ende estar dispuesto a aceptar otras hipótesis como posibilidades.

\chapter{Nothing Worse Than Wasted Talent}
\begin{metadata}
	Published by \anchor[chlewey]{chlewey} on \anchor[http://ewey.co/B1162]{Thu, 08 Mar 2012 19:26:41 +0000}\\
	\categories{personal, proyeccion-y-carrera, razonamiento-circular, talentos}\\
	Shorthand: \anchor[http://blog.chlewey.net/2012/03/nothing-worse-than-wasted-talent/]{nothing-worse-than-wasted-talent}
\end{metadata}

Sé que estoy haciendo algo mal.
Lo sé.
 Si no no sabría cómo explicar que las personas a mi alrededor estén felices y contentas insertadas como miembros
 productivos de la sociedad mientras yo me estrello una y otra vez con mi incapacidad de lograr lo que se espera de mí.

\begin{wrapfigure}{r}{300\px}\centering% {'width': '300', 'align': 'alignright', 'id': 'attachment_1163', 'caption': u'The saddest thing in life is wasted talent. \u2014 Lorenzo Anello a su hijo C. en A Bronx Tale'}
\anchor[http://blog.chlewey.net/wp-content/uploads/2012/03/photo-Il-etait-une-fois-le-Bronx-A-Bronx-Tale-1993-4.jpg]{\includegraphics[width=300\px,height=196\px]{blog/photo-Il-etait-une-fois-le-Bronx-A-Bronx-Tale-1993-4-300x196.jpg}}
\caption{The saddest thing in life is wasted talent. — Lorenzo Anello a su hijo C. en A Bronx Tale}
\end{wrapfigure}

\par% p
Esta percepción de fracaso constante se ha ido apoderando cada vez más de mí.~ No siempre fue así.~ Cuando estaba en el \anchor{colegio} yo no era el mejor alumno, pero tenía mis fuertes en los que me destacaba y que me hacían ver como ese genio que triunfará en la vida por encima de sus compañeros.
 Un talento por el que me destacaba.~ Un talento que me haría grande.

Nunca estudié bien.
 Llegaba a los exámenes con lo que había retenido de clases y no con lo que había estudiado la noche anterior.
 No hacía la mitad de mis tareas ni el trabajo en clase.
 Me ponía a dibujar laberintos en el cuaderno en lugar de tomar apuntes.
 Pero no me iba tan mal.
 De alguna forma en los exámenes lograba sacar notas aceptables y buenas que me permitían finalmente pasar el año.
 Recordaba.~ Relacionaba.~ Analizaba.

Las matemáticas fueron mi fuerte probablemente porque allá pesa más la capacidad de pensar que la de recordar.
 O tal vez porque mi mente fue particularmente diestra en la forma de pensar de las matemáticas que en cierta forma se extendía a todo lo demás.
 Las matemáticas me dieron para viajar a otros países a representar a Colombia en las Olimpiadas y para haber obtenido
 medallas en esas competencias.

Un talento que me haría triunfar.

Pero un talento que no me ha servido de mucho en mi vida adulta.
 No me ha servido de mucho el talento de recordar, relacionar y analizar.

No sé si aún tengo la mente brillante que tenía cuando era joven.
 No he tenido muchas oportunidades de ejercitarla porque las exigencias de la vida adulta me piden otros talentos.
 Talentos que no tuve, que no crié, que no cultivé.~ El talento de cumplir.~ El talento de pensar en forma práctica.

Mi capacidad de analizar me lleva a darme cuenta de muchas de mis fallas.
 Darme cuenta de todas esas pequeñas decisiones que me hacen fracasar.
 Pero ser capaz de ver a posteriori esto; ser capaz incluso de predecirlo a veces; no me ayudan a lograrlo.

Para actuar correctamente uno no necesita pensar nada.
 No es más que una cuestión de entrenamiento.
 Es una cuestión de automatizar las respuestas.~ Una cuestión de olvidarte de lo que no conlleve a su fin.

\par% p
No tengo la disciplina de ser disciplinado y ante eso no existe fuerza de voluntad posible que me lleve a cambiar la situación.
 Puedo analizar todo hasta encontrar mis errores, pero eso no me sirve para prevenirlos.
 Y tal vez ni siquiera necesite tanto análisis para prevenirlo.
 Necesito disciplina para adquirir disciplina.~ Y ahí hay una trampa conceptual.~ Un \anchor[http://es.wikipedia.org/wiki/Trampa-22\_\%28libro\%29]{\emph{Catch-22.}}

\par% p
Es algo que no puedo hacer solo.
 Tampoco es algo que alguien pueda hacer por mí.
 Necesito un entrenador.
 Ni siquiera necesito un guía.
 Necesito alguien que me acostumbre a actuar correctamente sin pensar.
 Todo lo demás será carreta, lo tomaré como carreta y lo desecharé como carreta.

\begin{blockquote}
We have a misconception that if we only cared enough about something, we would do something about it. But that’s not
 true.
\end{blockquote}

Motivation is in the mind; follow-through is in the practice. Motivation is conceptual; follow-through is practical. In
 fact, the solution to a motivation problem is the exact opposite of the solution to a follow through problem. The mind
 is essential to motivation. But with follow through, it’s the mind that gets in the way.

\begin{address}
— \anchor[http://blogs.hbr.org/bregman/2012/01/your-problem-isnt-motivation.html]{Peter Bergman}
\end{address}

Uno de los problemas de este sobreanálisis es que empiezo a reconocer el fracaso de mis expectativas y mis acciones.
 Me siento cada vez más incapaz de actuar dentro de una sociedad a la que le importa muy poco mi talento y sí mucho los
 talentos que no tengo.

Recuerdo que, cuando trabajé en Huawei, mi supervisor inmediato me decía que apreciaba mucho esos momentos en los que mi talento había ayudado a resolver más eficazmente ciertos problemas.
 Esos momentos en los que utilicé mi habilidad para pensar por fuera de la caja y aplicar esos conocimientos que tenía, pero por los cuales no me habían contratado.
 Pero a pesar de esos momentos, en el día a día yo simplemente no era confiable.
 Y yo sabía entonces y sigo sabiendo ahora que mi jefe tenía razón.

Son pocos los empleadores, o posibles socios, que confíen en mí sólo por mis momentos de inspiración.
 Creo aun que tengo un gran talento para dar.
 Oxidado, tal vez.
 Pero la falta de otros talentos: la falta de disciplina, principalmente, me llevan a seguir desperdiciando mi
 potencial.

O tal vez sólo deba olvidarme de lo que podría ser capaz.

Emplearme en algo que no sólo no requiera pensar sino que me impida pensar.
 Finalmente pensar no me ha servido de mucho en mi vida adulta.

El problema es que sí creo que hay algo peor que el talento desperdiciado y es tener que matar tu único talento sólo
 para adaptarte a los demás.

\chapter{Mi derecho de pasar por encima tuyo}
\begin{metadata}
	Published by \anchor[chlewey]{chlewey} on \anchor[http://ewey.co/B1189]{Fri, 09 Mar 2012 22:38:18 +0000}\\
	\categories{activismo, bogota, filosofia-politica, opinion, transmilenio, trasnporte-publico}\\
	Shorthand: \anchor[http://blog.chlewey.net/2012/03/mi-derecho-de-pasar-por-encima-tuyo/]{mi-derecho-de-pasar-por-encima-tuyo}
\end{metadata}

Uno de los puntos de discusión interna dentro del Partido Pirata Colombiano (en formación) ha sido el tema de la utilización de situaciones de hecho como forma válida de expresar nuestra opinión.
 Algunos de nosotros creemos que si queremos ser un partido legítimo dentro del que reconocemos como un estado de derecho nuestro mensaje será más claro y coherente si nos atenemos a los canales de expresión legales.
 Queremos que nuestra voz se escuche en la calle pero también en el Congreso.
 Sin embargo, por muy de derecho que el estado pretenda ser, muchas personas en todo el mundo llegan a la conclusión de
 que de vez en cuando su legítima protesta tiene que hacerse escuchar ocupando Wall Street, bloqueando Transmilenio o
 tumbando la página web de una institución gubernamental.

\anchor[http://yfrog.com/o0ksbayj]{\begin{wrapfigure}{r}{300\px}\centering% {'src': 'http://blog.chlewey.net/wp-content/uploads/2012/03/tm-20120309-041-300x225.jpg', 'title': 'Protestas en Transmilenio - bloque frente a la Universidad Minuto de Dios', 'height': '225', 'width': '300', 'alt': '', 'class': ['alignright', 'size-medium', 'wp-image-1194']}
\includegraphics[width=300\px,height=225\px]{blog/tm-20120309-041-300x225.jpg}
\end{wrapfigure}
}Como ante toda causa, suelen ser más los elementos subjetivos que los objetivos que me llevan a simpatizar con ella o no o con sus métodos.
 He sentido cierta simpatía por los movimientos que en 2011 ocuparon las plazas de Tahrir en El Cairo o del Sol en Madrid.
 Igualmente suelo sentir cierto regocijo cuando Anonymous desfasa temporalmente la página de un promotor de censurar Internet bajo la excusa de unos derechos de autor que se quedan en manos de grandes corporaciones.
 Pero esa simpatía personal no debe nublar mi juicio sobre lo que es justo o válido.

Bueno.
Debo reconocer que lo que puede ser justo o válido bajo una filosofía política puede ser injusto o inválido por otra.
 Hay ciertos aspectos de la filosofía política con la cual simpatizo: el hombre debe ser libre y el estado, bajo unas
 reglas muy claras, debe entrar a mediar donde esas libertades entran en conflicto.

Está mi libertad de expresión, por ejemplo.
 Mi libertad de expresión bien me podría llevar a publicar un artículo en mi blog en defensa del régimen Nazi.
 O en contra de la comunidad LGBT.
 O a favor de la legalización de las drogas blandas.
 O en contra de Jota Mario Valencia.
 O a favor de Jota Mario Valencia.
 Pero mi libertad de expresión no solo me permite escribir en mi blog.
 Me permite publicar eso en un medio impreso, o decirlo públicamente en una calle.
O de tatuármelo en mi cuerpo.
 Pero no sólo eso.
 Yo puedo cantar a grito herido en una fiesta una canción con derechos de autor reservados.
 O mi libertad de expresión me permite pintar una pared.
 O desfasar una página web.
 O verter pintura indeleble sobre un abrigo de piel que porta otra persona. O… o tomarme Transmilenio y evitar que
 otros usuarios puedan desplazarse.

Está también mi libertad de locomoción.
 La libertad que yo tenga de moverme libremente en el mundo sin que haya muros o fronteras que me lo impidan.
 La libertad que tengo de entrar a tu casa y abandonarla cuando me plazca.
 O, por lo menos, el derecho que tengo de poder desplazarme entre el lugar que habito y el lugar donde trabajo por
 medio de vías públicas y servidumbres.

Está también la libertad de poseer.
 El derecho a la propiedad.
 El derecho a que yo tenga algo llamado mi casa donde yo decido quien entra o no.
 El derecho a considerar que toda una laguna es mía y que sólo pueden beber de ella las personas que me paguen por un derecho sobre mi propiedad.
 El derecho a que nos reunamos todos los de la aldea para establecer una línea imaginaria que los de la aldea vecina no puedan cruzar.
 El derecho a que si yo pongo siete palabras juntas puedo acusar de robo de propiedad intelectual a cualquier otra
 persona que use juntas esas siete palabras.

Todos estos derechos y libertades entran en conflicto en algún momento y las sociedades tienen una serie de reglas
 implícitas y explícitas para dirimir muchos de esos conflictos y mecanismos formales e informales para resolver otros
 casos donde tales reglas no existen.

Si dos suecos discuten sobre quién es el dueño de una cabra, irán ante un juez adjuntando testigos y documentos.
 Si dos yemenitas tienen la misma discusión se dirime por quién saca primero su fusil.
 Distintas sociedades tienen distintos mecanismos para resolver sus problemas.
 Pero dentro de mi filosofía política debo establecer cuáles son mis límites entre lo válido y lo que no lo es.

Esta mañana un personaje que se presentaba como representante de una asociación de usuarios de Transmilenio decía que no podía condenar el derecho de expresión de los usuarios que aburridos por el mal servicio bloqueaban las estaciones.
 La libertad de expresión, como cualquier otro derecho o libertad tiene algún límite cuanto se enfrenta a otras
 libertades y derechos, como la buena honra de una persona difamada, el derecho a obtener información de una página web
 desfasada o tumbada, o el derecho de un trabajador de regresar a su casa tras una jornada de trabajo.

Alguna sociedad o alguna filosofía política dirá que es más importante la libertad de expresión.

Pues bien.
 Dentro de mi filosofía, no toda forma de expresión es válida cuando se afectan los derechos de otras personas.
 Es válido decir que el servicio de Transmilenio es un asco cuando lo es (y porque lo es).
 Es válido buscar mecanismos para amplificar nuestra denuncia cuando nuestra denuncia es válida.
 Pero en mi libro no es válido que ese mecanismo para amplificar nuestra voz sea convertir en víctimas a las mismas
 personas que decimos representar.

Y esas divagaciones vinieron antes de que me enterara en qué habían terminado las protestas: en destrucción de
 propiedad pública (pública significa que pertenece al pueblo, no que no pertenece a nadie) y en vulgar robo de dinero.

\anchor[http://mypict.me/index.php?id=335833644]{\begin{wrapfigure}{r}{225\px}\centering% {'src': 'http://blog.chlewey.net/wp-content/uploads/2012/03/tm-20120309-02-225x300.jpg', 'title': u'Protestas en Transmilenio - Saqueo de las cajas en la Estaci\xf3n Calle 72', 'height': '300', 'width': '225', 'alt': u'Foto: Protestas en Transmilenio - Saqueo de las cajas en la Estaci\xf3n Calle 72', 'class': ['alignright', 'size-medium', 'wp-image-1191']}
\includegraphics[width=225\px,height=300\px]{blog/tm-20120309-02-225x300.jpg}
\end{wrapfigure}
}\anchor[http://mypict.me/index.php?id=335835409]{\begin{wrapfigure}{l}{225\px}\centering% {'src': 'http://blog.chlewey.net/wp-content/uploads/2012/03/tm-20120309-03-225x300.jpg', 'title': u'Protestas en Transmilenio - Destrucci\xf3n de estaci\xf3n Calle 72', 'height': '300', 'width': '225', 'alt': u'Foto: Protestas en Transmilenio - Destrucci\xf3n de estaci\xf3n Calle 72', 'class': ['alignleft', 'size-medium', 'wp-image-1192']}
\includegraphics[width=225\px,height=300\px]{blog/tm-20120309-03-225x300.jpg}
\end{wrapfigure}
}
\par% div% {'style': 'clear: both; background: #eee; font-size: 90%;'}
Nota: Ninguna de las fotos es mía. Fueron publicadas por usuarios de Twitter y haciendo clic sobre ellas llega a sus
 sitios originales de publicación.

\chapter{Bloqueando Transmilenios}
\begin{metadata}
	Published by \anchor[chlewey]{chlewey} on \anchor[http://ewey.co/B1200]{Mon, 12 Mar 2012 18:19:06 +0000}\\
	\categories{opinion, transmilenio, trasnporte-publico}\\
	Shorthand: \anchor[http://blog.chlewey.net/2012/03/bloqueando-transmilenios/]{bloqueando-transmilenios}
\end{metadata}

\anchor[http://www.sibrtonline.org/fichas-tecnicas/transmilenio/6]{\begin{wrapfigure}{r}{250\px}\centering% {'src': 'http://www.sibrtonline.org/img/upload/gra-logo-tm-ok-4e4a68d15da3e.jpg', 'title': 'Logo de Transmilenio', 'height': '200', 'width': '250', 'alt': '[logo TM] ', 'class': ['alignright']}
\includegraphics[width=250\px,height=200\px]{blog/gra-logo-tm-ok-4e4a68d15da3e.jpg}
\end{wrapfigure}
}Uno de los problemas de cómo transcurrieron las protestas del pasado viernes en contra de Transmilenio, es que los
 disturbios, la destrucción de propiedad pública y el saqueo de las taquillas, así como la posterior polémica sobre si
 la foto que El Tiempo publicó en primera plana el sábado fue \anchor[http://elmacarenazoo.es.tl/PRINCIPAL-MEDIO-DE-COMUNICACION-DE-COLOMBIA-CREA-FALSOS-POSITIVOS-EN-IMAGENES.htm]{un montaje} o \anchor[http://www.facebook.com/carlos.spoon/posts/328611830520709]{no}, desvían la atención sobre los motivos y las formas de la jornada:
\begin{enumerate}

\item los motivos de la propuesta, y
\item el recurso del bloqueo como método de protesta.

\end{enumerate}

Sobre el recurso del bloqueo expresé algunos puntos \anchor[http://blog.chlewey.net/2012/03/mi-derecho-de-pasar-por-encima-tuyo/]{en mi anterior artículo}.
 Entiendo que originalmente se había llamado a un boicot y no a un bloqueo.
 Un boicot es la decisión personal de no comprar o utilizar un artículo de origen ofensivo y de convencer a otros de hacer lo mismo.
 Un bloqueo es impedir que quienes quieran comprar usar el artículo o servicio lo hagan.

Pero analicemos hoy el otro punto: los motivos que produjeron la protesta (haya sido un boicot, un bloqueo, o
 vandalismo): el servicio que presta Transmilenio.

\anchor[http://www.radiosantafe.com/2010/01/15/cae-banda-de-asaltantes-en-transporte-publico-en-germania/]{\begin{wrapfigure}{r}{300\px}\centering% {'src': 'http://c0364889.cdn2.cloudfiles.rackspacecloud.com/wp-content/uploads/2009/07/septima2.jpg', 'title': u'Buses en Bogot\xe1 - subiendo al bus', 'height': '153', 'width': '300', 'alt': '[foto: subiendo al bus]', 'class': ['alignright']}
\includegraphics[width=300\px,height=153\px]{blog/septima2.jpg}
\end{wrapfigure}
}\anchor[http://m.eltiempo.com/colombia/los-buses-cebolleros-de-bogota-desapareceran-con-la-llegada-del-sistema-integrado-de-transporte/7835142/1/home]{\begin{wrapfigure}{r}{300\px}\centering% {'src': 'http://blog.chlewey.net/wp-content/uploads/2012/03/IMAGEN-7835767-1-300x152.jpg', 'title': u'Buses en bogot\xe1 - cebollero', 'height': '152', 'width': '300', 'alt': '[foto: bus cebollero]', 'class': ['alignright', 'size-medium', 'wp-image-1201']}
\includegraphics[width=300\px,height=152\px]{blog/IMAGEN-7835767-1-300x152.jpg}
\end{wrapfigure}
}\anchor[http://blog.chlewey.net/wp-content/uploads/2012/03/bogota-buses.jpg]{\begin{wrapfigure}{r}{300\px}\centering% {'src': 'http://blog.chlewey.net/wp-content/uploads/2012/03/bogota-buses-300x182.jpg', 'title': u'Buses en Bogot\xe1 - panorama', 'height': '182', 'width': '300', 'alt': '[foto: buses] ', 'class': ['alignright', 'size-medium', 'wp-image-1202']}
\includegraphics[width=300\px,height=182\px]{blog/bogota-buses-300x182.jpg}
\end{wrapfigure}
}Primero quiero aclarar que el servicio de transporte público colectivo tradicional es mucho peor que el servicio de
 Transmilenio, pero los usuarios no protestan ante este por varios motivos entre los que destaco:
\begin{enumerate}

\item Transmilenio es mucho más fácil de bloquear que el caos del transporte público colectivo tradicional.
\item Transmilenio es más reconocible como institución que el montón de instituciones que forman el transporte público
 colectivo tradicional.

\end{enumerate}

Se dice, por ejemplo, que Transmilenio es manejado por solo 20 familias.
 Pero si vamos a ver son las mismas 20 familias que manejan el transporte público colectivo tradicional.

\par% p
Hace poco más de 10 años \anchor[http://blog.chlewey.net/2001/08/el-negocio-del-servicio-publico/]{escribía sobre las diferencias entre el enfoque de negocio y el enfoque de servicio público }en cuanto al transporte público.
 Pero de una forma u otra todo servicio público debe buscar ser económicamente viable.
 Prestar el servicio produce costos (combustible, salarios, arreglos y reposición de bienes de capital, etc.) y estos
 costos serán cobrados a los usuarios por medio de las tarifas, o serán cobrados a personas interesadas por medio de
 publicidad, o serán cobrados los ciudadanos por medio de impuestos que entran como subsidios o como operación directa
 por el estado.

En el transporte público colectivo tradicional son operadores privados los que se encargan de todos los detalles de la operación y el estado (el Distrito) tan solo está para otorgar rutas.
 Para la empresa operadora el negocio está en tener muchos buses trabajando en cada una de sus rutas, pues cobra por afiliación del bus.
 Para el conductor el negocio está en recoger muchos pasajeros porque cobra una fracción del pasaje.
 Los otros conductores que cubren la misma ruta son competencia para el chofer del bus, y este debe planear los tiempos de recorrido para procurar que se acumulen muchos pasajeros pero que el siguiente conductor no lo rebase y los recoja.
 El dueño del bus, si no es operador ni conductor, maximiza sus ingresos cuando cada bus es rentable (recogiendo muchos pasajeros con pocos recorridos).
 La combinación de todos estos factores tienden a perjudicar más que a favorecer al usuario.

En Transmilenio hay otra serie de actores en el negocio.
 Está el Distrito como dueño de la infraestructura: vías, estaciones, equipos de comunicación, etc.
 Están los operadores privados, los dueños de los buses, los conductores, etc. Y el operador del recaudo.

Para el Distrito Transmilenio no es rentable.
 Para los operadores el negocio está en maximizar el número de pasajeros por bus, aunque tienen obligaciones de cumplir ciertos horarios.
 Para los conductores los detalles de la operación no deben ser negocio: ellos reciben un salario fijo.

Claramente para el operador sigue siendo mejor negocio no prestar el mejor servicio, aunque en un sentido contrario al
 caso del transporte público colectivo tradicional.

¿Qué pasa si el sistema no es rentable para el operador?
 Pues que muchos operadores preferirán invertir en otras cosas.
 ¿Cómo lograr que el operador se mantenga en el negocio?
 Subsidiando parte de la operación.
 La otra opción es dejar que el operador de vaya y sea el estado quien asuma la operación.

El principal problema de Transmilenio en cuanto a operación es que se requiere que los buses vayan llenos.
 (Eso también le pasa a los dueños de buses en el transporte público colectivo tradicional.)
 Y eso es percibido por el usuario como largas esperas, acumulación de personas en los vagones de las estaciones
 y buses llenos donde el usuario es estrujado.

Pero creo que hay otros problemas con Transmilenio que no obedecen a la operación.
 Uno de los peores: la cultura ciudadana.

\anchor[http://logiamisantropica.blogspot.com/2011/06/acerca-de-la-que-llaman-servicio.html]{\begin{wrapfigure}{r}{300\px}\centering% {'src': 'http://3.bp.blogspot.com/-lTNqH5MHioc/TgU1n-2gf5I/AAAAAAAAAFQ/u9i1jatvzGs/s1600/Transmilenio+03-11-10.jpg', 'title': 'Transmilenio: Usuarios subiendo a un articulado', 'height': '225', 'width': '300', 'alt': '[foto: subiendo al articulado]', 'class': ['alignright']}
\includegraphics[width=300\px,height=225\px]{blog/Transmilenio+03-11-10.jpg}
\end{wrapfigure}
}\anchor[http://www.mobocol.com/2010/10/las-manas-en-transmilenio-parte-1.html]{\begin{wrapfigure}{r}{300\px}\centering% {'src': 'http://farm3.static.flickr.com/2681/4407155662_a73696e30d.jpg', 'title': 'Transmilenio: Usuarios esperando articulado en horas pico', 'height': '200', 'width': '300', 'alt': '[foto: esperando articulado] ', 'class': ['alignright']}
\includegraphics[width=300\px,height=200\px]{blog/4407155662_a73696e30d.jpg}
\end{wrapfigure}
}En casi todos los sistemas de transporte masivo, los usuarios han aprendido que es mejor dejar salir a quienes se bajan
 del bus o del tren antes de subirse. Pero eso no funciona en Transmilenio en Bogotá. Al usar el sistema en horas pico,
 fácilmente uno comprueba que el civismo individual no sirve. Si uno no empuja a los demás no puede entrar. Si uno se
 pone a esperar a que los otros salgan, alguien más se colará al atiborrado bus. Y si uno es el que necesita salir
 tiene que salir empujando porque quienes quieren entrar, o quienes están esperando el siguiente bus, bloquean la
 puerta.

Algunos detalles de la operación contribuyen a este panorama: la utilización de las mismas puertas para recorridos diferentes, por ejemplo.
 Puedes estar de primero en la fila para entrar, pero si para el bus que no te sirve quedas estorbando.
 Diera la impresión de que el sistema no está diseñado para que los usuarios sean decentes entre ellos.

\par% p
El viernes hablaba que el bloqueo a Transmilenio afectaba a los otros usuarios.
 En cierta forma me parece absurdo que para exigir la mejora de un servicio tengamos que dañarle el servicio a los demás.
 Pero si somos los ciudadanos los culpables del caos, tal vez el bloqueo si estaba justificado.
El objetivo era molestar a uno de los principales agentes de que Transmilenio funcione mal: \textbf{el usuario de Transmilenio}.

\chapter{Reflexiones legalizadas}
\begin{metadata}
	Published by \anchor[chlewey]{chlewey} on \anchor[http://ewey.co/B1211]{Mon, 30 Apr 2012 19:51:47 +0000}\\
	\categories{drogas, narcotrafico, opinion}\\
	Shorthand: \anchor[http://blog.chlewey.net/2012/04/reflexiones-legalizadas/]{reflexiones-legalizadas}
\end{metadata}

\anchor[http://commons.wikimedia.org/wiki/File:A\_small\_cup\_of\_coffee.JPG]{\begin{wrapfigure}{r}{300\px}\centering% {'src': 'http://upload.wikimedia.org/wikipedia/commons/thumb/4/45/A_small_cup_of_coffee.JPG/320px-A_small_cup_of_coffee.JPG', 'alt': u'[taza de caf\xe9] ', 'class': ['alignright']}
\includegraphics[width=300\px]{blog/320px-A_small_cup_of_coffee.jpg}
\end{wrapfigure}
}Salvo cosas como el café o el chocolate no consumo regularmente \anchor[http://blog.chlewey.net/2007/09/placeres-culposos/]{ninguna substancia química} que sea considerada una droga.
 No fumo, salvo pasivamente y a regañadientes.
 Soy abstemio por gusto frente al alcohol: no me gusta tomar aunque muy ocasionalmente acompañe una comida con un vaso de cerveza (250 cc, no una pinta) o una copa de vino.
 Nunca he usado drogas recreativas ilegales y con cierta desidia me tomo las drogas medicinales que me han recetado.
 Detesto que fumen al lado mío.

Por otro lado soy padre.
 Tengo dos hijos que si bien hoy son lo suficientemente pequeños como para preocuparme en no pocos años podrían ser consumidores de substancias adictivas.
 No me gustaría verlos convertidos en alcohólicos, fumadores empedernidos o drogadictos.

Dicho esto creo que pertenezco al segmento demográfico que más se opondría a la legalización de las drogas.
 Mi libre desarrollo de la personalidad no se ve amenazada por la prohibición y, por el contrario, una legalización
 implicaría mayor disponibilidad de que mis hijos terminaran cayendo en la adicción.

Y, sin embargo, no me opongo.

\par% p
Pensemos en una droga ilegal media como la cocaína\anchor[http://en.wikipedia.org/wiki/File:Cocaine-from-xtal-1983-3D-balls.png]{\begin{wrapfigure}{r}{320\px}\centering% {'src': 'http://upload.wikimedia.org/wikipedia/commons/thumb/6/6b/Cocaine-from-xtal-1983-3D-balls.png/320px-Cocaine-from-xtal-1983-3D-balls.png', 'alt': '', 'height': '203', 'class': ['alignright'], 'width': '320'}
\includegraphics[width=320\px,height=203\px]{blog/320px-Cocaine-from-xtal-1983-3D-balls.png}
\end{wrapfigure}
} y pensemos en los problemas sociales que conlleva su consumo teniendo en cuenta que ni es la más adictiva ni la más inocua.
 Distintas drogas tienen distintos efectos y deberían estudiarse caso por caso.

La cocaína es un estimulante y como tal una persona bajo los efectos de la cocaína no tendría mayores dificultades para conducir, manejar maquinaria pesada u otras actividades sociales y productivas que requieran atención.
 Cosa que no sucede con alguien bajo la influencia del alcohol.
 El cocainómano puede ser más dado a un comportamiento temerario, pero eso también sucede con el alcohol.
 En términos generales el consumo y los efectos directos de la cocaína no son más peligrosos para la sociedad que el de
 otras substancias legales.

\anchor[http://lasdrogasyadicciones.blogspot.com/2011/12/el-alcohol-que-tomamos.html]{\begin{wrapfigure}{r}{250\px}\centering% {'src': 'http://2.bp.blogspot.com/-qWIkkSyIuQ8/Tut55yv60YI/AAAAAAAADe8/dinetyb5yJ0/s1600/El-alcohol-que-tomamos.jpg', 'alt': '', 'height': '223', 'class': ['alignright'], 'width': '250'}
\includegraphics[width=250\px,height=223\px]{blog/El-alcohol-que-tomamos.jpg}
\end{wrapfigure}
}A largo plazo, el consumo de cocaína produce problemas para la salud.
 Pero, igualmente, si lo comparamos con drogas legales como el alcohol o la nicotina, el efecto no es particularmente mayor.
 Así que desde el punto de vista de salubridad pública la cocaína tampoco sería una gran carga para la sociedad frente
 a lo que ya tenemos.

Los efectos estimulantes y temerarios de la cocaína son utilizados por ciertos delincuentes como una aliciente para cometer delitos como robos o asaltos, no relacionados con el tráfico o consumo del alcaloide.
 Podría pensarse que de no haber estado presente habría ciertos delitos que el potencial infractor no hubiera cometido
 pero es difícil de cuantificar esto cuando nuevamente el alcohol es más utilizado en estos casos, y más disponible
 dado su aspecto legal.

La mayor parte de los problemas de intoxicación relacionados con la cocaína están realmente relacionados con las sustancias que se utilizan para mezclar el alcaloide y no con el alcaloide en sí.
 La disponibilidad de una cocaína legal y regulada bien podría reducir una gran parte de este problema.

\anchor[http://en.wikipedia.org/wiki/File:Marlboro4wiki2.JPG]{\begin{wrapfigure}{r}{320\px}\centering% {'src': 'http://upload.wikimedia.org/wikipedia/commons/thumb/9/9c/Marlboro4wiki2.JPG/320px-Marlboro4wiki2.JPG', 'alt': '[Marlboro] ', 'height': '240', 'class': ['alignright'], 'width': '320'}
\includegraphics[width=320\px,height=240\px]{blog/320px-Marlboro4wiki2.jpg}
\end{wrapfigure}
}El efecto adictivo de la cocaína es principalmente psicológico.
 La abstinencia de la cocaína produce depresión, la cual puede evitarse con la continuación del consumo.
 La nicotina, presente en el tabaco legal, es mucho más adictiva haciendo que ciertos centros del placer sean incapaces
 de funcionar sin una dosis de nicotina.

Sin embargo la depresión producida por la abstinencia de la cocaína puede ser un problema social más notorio que la
 ansiedad que produce la ausencia de nicotina.

Los otros grandes problemas de la cocaína, fuera de los efectos del consumo en sí, están relacionados con la ilegalidad.
 Los productores de cocaína, al no poder disponer de una regulación estatal legal, tienen que acudir a su propia mano para garantizar su negocio.
 Esto fomenta la creación de cárteles y de prácticas monopolísticas, las cuales, frente a la ausencia de competencia legal, permiten aumentar los precios y las ganancias.
 En la producción y tráfico de cocaína hay mucha plata que se alimenta de la propia ilegalidad.

\anchor[http://blog.chlewey.net/wp-content/uploads/2012/04/Pablo-Escobar.jpg]{\begin{wrapfigure}{l}{300\px}\centering% {'src': 'http://blog.chlewey.net/wp-content/uploads/2012/04/Pablo-Escobar-300x225.jpg', 'alt': '[Pablo Ecobar] ', 'height': '225', 'class': ['size-medium', 'wp-image-1214', 'alignleft'], 'width': '300'}
\includegraphics[width=300\px,height=225\px]{blog/Pablo-Escobar-300x225.jpg}
\end{wrapfigure}
}Los altos precios hacen que el consumidor habitual gaste bastante dinero en mantener su vicio.
 En ocasiones dinero que no puede obtener de forma lícita lo cual hace que el adicto sea más proclive a cometer delitos (no por el efecto embriagante de la cocaína, sino para obtenerla).
 Estas ganancias terminan financiando estructuras ilegales que deben competir entre sí y que al no poder acceder a las estructuras de control de los estados, los lleva a utilizar el asesinato y otros métodos violentos como mecanismo de control.
 El dinero se convierte también en poder corruptor, llegando a cooptar al estado.
 Un funcionario íntegro podría no hacer mucha diferencia porque lo pueden matar y reemplazar por un funcionario menos
 escrupuloso o, por lo menos, que aprecie mejor su propia vida sobre sus posibles valores.

¿Por qué mantener la droga como ilegal?

\par% p
Es claro que una simple reforma en el código penal que elimine como delito la producción y el tráfico de drogas hoy ilegales no ayuda mucho.
 El negocio está hoy en manos del \anchor[http://blog.chlewey.net/2011/01/de-bandas-mafias-e-insurgencias/]{crimen organizado} y aunque se legalice la droga no cambiará su mentalidad mafiosa.
 La legalización bajo este escenario sería quitarle al estado una herramienta, un motivo más, para controlar al crimen
 organizado.

Mientras tanto habrá mayor oferta.
 Tal vez un poco más económica, pero mayor, sin que se garantice una mejor calidad.
 La muy legal y regulada industria tabacalera bien nos muestra cómo las ganancias obtenidas por el vicio de la
 población baja los escrúpulos de los empresarios quienes han buscado productos más adictivos y contratado científicos
 mercenarios para desvirtuar los estudios que muestran los riesgos del tabaco.

Más oferta, calidad igual de mala y la producción en manos de empresas con mentalidad mafiosa parece ser un caso peor
 que el actual de prohibición.

La legalización no puede hacerse sin una oferta completamente legal y comprometida que contrarreste cualquier
 interferencia mafiosa, pero para que esta oferta legal funcione debe haber una demanda que la haga rentable y ese
 tampoco es un escenario deseable.

Si bien la cocaína en sí no es mucho más peligrosa que el alcohol y el tabaco, está lejos de ser una substancia inocua.
 Aumentar el consumo no es deseable.
 Si no se aumenta el consumo hay menos incentivos para que un empresario legal quiera entrar a un negocio en el que
 tendrá que competir con mafiosos, así pueda en teoría contar con la protección del estado.

Las despenalizaciones parciales pueden llegar a ser soluciones aún peores: dejar completamente legal el consumo y la
 venta al detal pero dejar ilegal la oferta al por mayor (necesaria para suplir a los minoristas) desestimula
 completamente a cualquier empresario que quiera permanecer 100\% en la legalidad.

La persecución actual a la producción y tráfico de estupefacientes no es buena, pero es muy difícil encontrar una
 alternativa.

\&nbsp;

\chapter{Campuseando}
\begin{metadata}
	Published by \anchor[chlewey]{chlewey} on \anchor[http://ewey.co/B1222]{Fri, 11 May 2012 13:23:10 +0000}\\
	\categories{campus-party, gustavo-prieto, opinion}\\
	Shorthand: \anchor[http://blog.chlewey.net/2012/05/campuseando/]{campuseando}
\end{metadata}

\par% p
Una reciente polémica que ha surgido en \anchor[http://es.wikipedia.org/wiki/Medio\_social\_(social\_media)]{la comunidad “social”} colombiana ha sido la decisión del Distrito de Bogotá de no subvencionar el evento \anchor[http://www.campus-party.co]{Campus Party Colombia} y la consecuente decisión de los organizadores de este evento de buscar una sede diferente a la capital del país.

Cuando escribo esto los organizadores de la Campus Party no han cerrado aún la esperanza de seguir negociando con el
 Distrito pero me temo que la decisión tomada por la administración capitalina es definitiva y veo improbable un
 acuerdo.

\anchor[http://blog.chlewey.net/wp-content/uploads/2012/05/cpvsdc.png]{\begin{wrapfigure}{r}{250\px}\centering% {'src': 'http://blog.chlewey.net/wp-content/uploads/2012/05/cpvsdc.png', 'title': 'cpvsdc', 'height': '40', 'width': '250', 'alt': '', 'class': ['alignright', 'size-full', 'wp-image-1225']}
\includegraphics[width=250\px,height=40\px]{blog/cpvsdc.png}
\end{wrapfigure}
}La decisión del distrito parece tomada: Campus Party es un evento cerrado que no permite la participación de otros operadores de telecomunicaciones diferentes al patrocinador exclusivo Telefónica y que, por su carácter cerrado, no aporta mucho a la inclusión digital de los menos favorecidos.
 En su lugar la administración distrital prefiere organizar un evento propio: la \anchor[http://twitter.com/search/\%23semanatic]{Semana TIC}.~ La razón oficial \anchor[http://www.bogotahumana.gov.co/index.php/noticias/comunicados-de-prensa/956-alcaldia-mayor-de-bogota-explica-su-no-participacion-en-campus-party-2013]{dada por el Distrito}, sin embargo, es sobre un hecho más puntual: los organizadores de Campus Party no permiten la participación de \anchor[http://www.etb.com.co/]{ETB}, la empresa de telecomunicaciones de la cual el Distrito es el principal accionista.

\par% p
Por otro lado la razón social del Campus Party es supuestamente una empresa llamada Futura Networks.
 Encuentro que existe una empresa llamada \anchor[http://www.google.com.co/search?q=\%22futura+networks\%22+site:ccb.org.co]{Futura Networks de Colombia S.A.S.} (sociedad por acciones simplificada) pero no veo mayor información sobre la organización internacional ni sobre sus accionistas.
 A juzgar por mi experiencia \anchor[http://es.wikipedia.org/wiki/Telef\%C3\%B3nica]{Telefónica} no es solamente el patrocinador exclusivo sino, para efectos prácticos, el principal organizador de la Campus Party.
 Aun cuando no sea Telefónica la dueña del Campus Party, ser patrocinador exclusivo implica que hay un contrato para
 ello e implica el derecho a vetar a cualquier otra empresa que quiera aparecer como patrocinadora u organizadora.

En otras palabras: Futura Networks de Colombia no puede deshacerse de Telefónica y no puede permitir que la ETB participe, salvo que rescinda el contrato con Telefónica y encuentre alguien que pueda reemplazar su labor técnica, logística y económica.
 Cosa que, no sé siquiera si es posible ya que sospecho que Telefónica es también el principal accionista de Futura
 Networks.

Desde luego que una salida favorable sería que la subvención estatal que hasta ahora había brindado el Distrito fuere
 reemplazada por fondos privados, bien sean de la propia Telefónica si tal interés tiene en el evento o de parte de
 otros patrocinadores que no sean competencia de Telefónica.

La pregunta que viene es ¿está bien o está mal que las cosas estén sucediendo así?

\anchor[http://twitpic.com/20u31b]{\begin{wrapfigure}{r}{225\px}\centering% {'src': 'http://blog.chlewey.net/wp-content/uploads/2012/05/cpco3-chlewey-225x300.jpg', 'title': 'Campus Party Colombia 2010', 'height': '300', 'width': '225', 'alt': '[] ', 'class': ['alignright', 'size-medium', 'wp-image-1226']}
\includegraphics[width=225\px,height=300\px]{blog/cpco3-chlewey-225x300.jpg}
\end{wrapfigure}
}He sido campusero y como tal me gustaría que se siguiera celebrando el Campus Party en Bogotá. También me gustaría que hubiera otros Campus Party en otras ciudades colombianas, preferiblemente sin que sea una opción exclusiva.
 También me gustaría que Campus Party tuviera una mayor proyección social, que si bien siga siendo por esencia un
 evento exclusivo, pueda igualmente reportar que ha servido efectivamente a la innovación y a la inclusión digital y no
 un simple escenario donde un pequeño grupo de campuseros pagos y otro grupo similar de invitados especiales llegan a
 jugar videojuegos y bajar contenidos mientras escuchan una que otra conferencia.

\par% p
Por otro lado no me opongo a que el Distrito patrocine otro evento como la Semana TIC.
 Entre más eventos de tecnología mejor: \anchor[http://flisolbogota.info/]{FLISOL}, \anchor[http://www.bogotech.org/]{Bogotech}, \anchor[http://socialmediaweek.org/bogota/]{Social Media Week}, \anchor[http://barcamp.org/BarCampBogota]{Barcamp Bogotá}, \anchor[http://domingoenlamanana.com/]{Domingo en la mañana}, \anchor[http://www.tedxbogota.com/]{TEDxBogotá}, \anchor[http://www.tedxceiba.co/]{TEDxCeiba }y muchos más.~ Eventualmente con mejores especializaciones, diferentes objetivos y mayores ofertas.

\par% p
Ahora, que el distrito quiera amarrar un patrocinio a la promoción de una empresa como ETB me parece la peor excusa posible.
 Como tampoco me gusta que quiera vender la idea de que la Semana TIC reemplaza al Campus Party porque es más afín al
 proyecto político de la actual administración capitalina porque eso convierte a la Semana TIC en el proyecto de \anchor[http://es.wikipedia.org/wiki/Gustavo\_Petro]{Gustavo Petro} y su \anchor[http://www.progresistas.com.co/]{Partido Progresista} y no como el evento de Bogotá.

Si Telefónica y Futura Networks quieren seguir haciendo la Campus Party en Bogotá con seguridad que encontrarán dentro de sus propios bolsillos o en otros patrocinadores la viabilidad económica de lograrlo.
 Si se quieren ir de Bogotá, es parte de la libertad de empresa.

Y bienvenida la Semana TIC mientras esta no se convierta en propaganda de la actual administración distrital.

\chapter{El placer de especular}
\begin{metadata}
	Published by \anchor[chlewey]{chlewey} on \anchor[http://ewey.co/B1230]{Wed, 16 May 2012 15:55:08 +0000}\\
	\categories{derecha, farc, guerrilla, izquierda, opinion, uribismo}\\
	Shorthand: \anchor[http://blog.chlewey.net/2012/05/el-placer-de-especular/]{el-placer-de-especular}
\end{metadata}

\begin{wrapfigure}{r}{300\px}\centering% {'width': '300', 'align': 'alignright', 'id': '', 'caption': 'Encapuchados lanzaban papas explosivas, por @presidiario1728'}
\anchor[http://yfrog.com/oeexmkij]{\includegraphics[width=300\px,height=200\px]{blog/exmki.jpg}}
\caption{Encapuchados lanzaban papas explosivas, por @presidiario1728}
\end{wrapfigure}

\par% p
Inició el \anchor[http://es.wikipedia.org/wiki/Tratado\_de\_Libre\_Comercio]{Tratado de Libre Comercio} \anchor[http://es.wikipedia.org/wiki/Tratado\_de\_Libre\_Comercio\_entre\_Colombia\_y\_Estados\_Unidos]{entre Colombia y los Estados Unidos} y, como era previsible, un sector de la sociedad autodenominado como “sectores sociales” o etiquetado como “la izquierda” salió a protestar.
 Según reportes que oí hubo encapuchados en la Universidad Nacional amenazando el tráfico en la Carrera 30 y la Calle 26 y otros tantos en la Universidad Distrital haciendo disturbios por la Avenida Circunvalar.
~ Disturbios que incluían la explosión de las así llamadas “papas bomba”.~ Escuché también de disturbios similares en
 la Universidad de Antioquia.

\par% p
Por la mañana la \anchor[http://noticierovenevision.net/internacionales/2012/mayo/15/26632=policia-colombiana-desactiva-carro-bomba-en-bogota-]{noticia era} la desactivación de un carro-bomba dirigido al comando de la Policía.
 Por un aparente desperfecto mecánico la camioneta cargada de explosivos se averió en el céntrico barrio residencial y
 comercial \anchor[http://www.martires.gov.co/barrio-eduardo-santos]{Eduardo Santos}.~ La bomba fue completamente desactivada por las autoridades.

\begin{wrapfigure}{r}{300\px}\centering% {'width': '300', 'align': 'alignright', 'id': '', 'caption': u'Foto de los veh\xedculos afectados, por @sanchezjuma'}
\anchor[http://twitter.com/sanchezjuma/status/202445328413241344/photo/1]{\includegraphics[width=300\px,height=229\px]{blog/As869B8CMAAUsnC.jpg}}
\caption{Foto de los vehículos afectados, por @sanchezjuma}
\end{wrapfigure}

\par% p
El día se puso más “interesante” después.
 Antes del medio día explotó una bomba en la Calle 74 con Avenida Caracas en lo que a todas luces fue un atentado
 contra el abogado, ex Ministro de Interior y Justicia y conductor de radio \anchor[http://www.elpais.com.co/elpais/judicial/noticias/perfil-del-dexministro-del-interior-y-justicia-fernando-londono-hoyos]{Luis Fernando Londoño Hoyos}.~ Una “\anchor[https://www.google.com.co/search?q=bomba+lapa\&tbm=isch]{bomba lapa}” fue colocada sobre el capó de la camioneta blindada en la que se transportaba el exministro.
 Un escolta retiró el artefacto el cual explotó causando la muerte de este escolta, el conductor del vehículo y
 causando graves heridas al abogado, a un conductor de buseta que se encontraba al lado y a muchos otros transeúntes.

\begin{wrapfigure}{l}{94\px}\centering% {'width': '94', 'align': 'alignleft', 'id': '', 'caption': 'Carlos Fuentes'}
\anchor[http://commons.wikimedia.org/wiki/File:Carlos\_Fuentes.jpg]{\includegraphics[width=94\px,height=134\px]{blog/Carlos_Fuentes.jpg}}
\caption{Carlos Fuentes}
\end{wrapfigure}

\par% p
Por la tarde se conoció la noticia sobre la muerte del escritor e intelectual mexicano \anchor[http://es.wikipedia.org/wiki/Carlos\_Fuentes]{Carlos Fuentes}.~ Salvo esta última noticia vienen las especulaciones sobre si los hechos anteriores estaban conectados o no.

\begin{wrapfigure}{r}{100\px}\centering% {'width': '100', 'align': 'alignright', 'id': '', 'caption': u'Fernando Londo\xf1o'}
\anchor[http://www.hoy.com.ec/noticias-ecuador/atentado-bomba-revive-el-miedo-en-bogota-546849.html]{\includegraphics[width=100\px,height=133\px]{blog/londono.png}}
\caption{Fernando Londoño}
\end{wrapfigure}

Londoño Hoyos fue ministro del hoy expresidente Álvaro Uribe Vélez y trabaja actualmente como director y conductor del programa de opinión “La hora de la verdad” de la cadena Radio Super, desde donde defiende las ideas políticas y la gestión del gobierno pasado.
 Es reconocido como una de las voces más importantes de “la derecha” colombiana.
 Antes de ser ministro fue abogado litigante y representó en muchas ocasiones a demandantes contra el estado.
 En su actividad privada se hizo poseedor de acciones de la empresa Invercolsa que en su momento estaban disponibles sólo para empleados.
 Londoño, siendo contratista y no empleado, adquirió un importante paquete de acciones en lo que muchos, incluyendo la Procuraduría General de la Nación y el Juzgado 28 Civil de Circuito de Bogotá consideraron ilegal.
 Sin duda la lista de enemigos personales e ideológicos de Londoño Hoyos es bastante grande.

\par% p
Adicionalmente ayer se votaba en la Cámara de Representantes el proyecto de la ley \anchor[http://www.senado.gov.co/sala-de-prensa/noticias/item/12483-marco-legal-para-la-paz-se-votara-el-martes-proximo]{Marco Legal para la Paz}, de la cual el expresidente Uribe y varios de sus escuderos, incluido Londoño Hoyos, han sido férreos opositores.

\par% p
No han faltado las especulaciones sobre si el atentado contra Londoño vendría de sectores obscuros de la derecha para
 enturbiar el proyecto de ley que permitiría una eventual paz con las guerrillas que supuestamente Uribe y su gobierno
 estuvieron a punto de derrotar pues si el atentado fuere atribuido a las \begin{abbr}% {'style': 'font-variant: small-caps;', 'title': 'Fuerzas Armadas Revolucionarias de Colombia'}
Farc
\end{abbr}
, como enemigo natural del exministro, no tendría sentido premiar a las guerrillas con una ley de impunidad.

\par% p
También hay quienes dicen que el atentado no pudo provenir de las \begin{abbr}% {'style': 'font-variant: small-caps;', 'title': 'Fuerzas Armadas Revolucionarias de Colombia'}
Farc
\end{abbr}
 por ello mismo: porque sería un tropiezo para demostrar voluntad de paz.

\par% p
Desde finales de los años 1990 las \begin{abbr}% {'style': 'font-variant: small-caps;', 'title': 'Fuerzas Armadas Revolucionarias de Colombia'}
Farc
\end{abbr}
 abrazaron el terrorismo “puro” como una forma más de combate contra el estado.
 Antes de ello las guerrillas colombianas no se caracterizaban por ataques indiscriminados contra la sociedad civil o atentados con bombas contra figuras públicas, aunque para la definición de muchos el sólo hecho de alzarse en armas y combatir a la fuerza pública ya los convierte en terroristas.
 Con anterioridad los atentados con bombas fueron casi exclusivos de grupos narcoterroristas como la agrupación Los
 Extraditables del Cartel de Medellín.

\par% p
En los años 1990, las \begin{abbr}% {'style': 'font-variant: small-caps;', 'title': 'Fuerzas Armadas Revolucionarias de Colombia'}
Farc
\end{abbr}
 empezaron a usar dispositivos explosivos improvisados en cilindros de gas vacíos que podían ser arrojados (nunca supe cómo los arrojaban).
 Estos cilindros bomba eran arrojados principalmente contra puestos de policía, los cuales por ser primordialmente cuerpos de seguridad ciudadana (y no combatientes antiinsurgentes) se encuentran en los centros urbanos de pequeños municipios.
 Los cilindros bomba producen una destrucción indiscriminada lo que unido a los poco precisos mecanismos de lanzamiento
 producían un gran daño a la población civil cercana a los puestos de policía.

\par% p
Los defensores de turno de la guerrilla decían que no se les podía pedir a las \begin{abbr}% {'style': 'font-variant: small-caps;', 'title': 'Fuerzas Armadas Revolucionarias de Colombia'}
Farc
\end{abbr}
 armas de precisión como las que poseían las fuerzas armadas constitucionales y que era responsabilidad del estado de poner combatientes en medio de la población civil.
 (Repito, un Policía no es un combatiente sino un servidor civil.)

\par% p
En 1999, sin embargo, \anchor[http://www.eltiempo.com/archivo/documento/MAM-950641]{estalló una bomba} frente a las oficinas de la Federación Nacional de Fondos Ganaderos en Bogotá, causando daños en la casa que fungía como oficina y destrucción, heridos y muertos entre las personas que pasaban entonces por la calle.
 Las \begin{abbr}% {'style': 'font-variant: small-caps;', 'title': 'Fuerzas Armadas Revolucionarias de Colombia'}
Farc
\end{abbr}
 fueron identificadas como los autores de ese atentado y, desde entonces, no han sido tímidas en colocar bombas destinadas a causar daño en instituciones civiles no combatientes y terror en la sociedad.
 Uno de los casos más sonados la bomba en el Club El Nogal en pleno gobierno de Uribe y ministerio de Londoño Hoyos.
 Uno menos conocido (anterior incluso a la \begin{abbr}% {'title': u'Federaci\xf3n Nacional de Fondos Ganaderos'}
FNFG
\end{abbr}
), fue el atentado de Santo Domingo (Arauca) cuando en medio de combates con las fuerzas armadas las \begin{abbr}% {'style': 'font-variant: small-caps;', 'title': 'Fuerzas Armadas Revolucionarias de Colombia'}
Farc
\end{abbr}
 detonaron un carro bomba que mató a varios civiles, incluidos niños, y los hechos fueron luego atribuidos a bombas
 racimo lanzadas desde aviones de la Fuerza Aérea Colombiana FAC.

\par% p
La tesis de que el atentado contra Londoño Hoyos no favorece a las \begin{abbr}% {'style': 'font-variant: small-caps;', 'title': 'Fuerzas Armadas Revolucionarias de Colombia'}
Farc
\end{abbr}
 porque enturbiaría la aprobación de la ley Marco Legal para la Paz no tiene sentido a la luz de lo que las \begin{abbr}% {'style': 'font-variant: small-caps;', 'title': 'Fuerzas Armadas Revolucionarias de Colombia'}
Farc
\end{abbr}
 han demostrado en los últimos años.~ En mi opinión a las \begin{abbr}% {'style': 'font-variant: small-caps;', 'title': 'Fuerzas Armadas Revolucionarias de Colombia'}
Farc
\end{abbr}
 les vale huevo que aprueben o no esa ley.
 Como les vale huevo si con el atentado convertían a Londoño Hoyos en un mártir de la derecha colombiana.
 Tampoco creo que las \begin{abbr}% {'style': 'font-variant: small-caps;', 'title': 'Fuerzas Armadas Revolucionarias de Colombia'}
Farc
\end{abbr}
 fuesen tan brutas de creer que matando a Londoño callaban a la derecha.

\par% p
Sin embargo, desde la lógica que han mostrado las \begin{abbr}% {'style': 'font-variant: small-caps;', 'title': 'Fuerzas Armadas Revolucionarias de Colombia'}
Farc
\end{abbr}
 este atentado tiene sentido.~ A las \begin{abbr}% {'style': 'font-variant: small-caps;', 'title': 'Fuerzas Armadas Revolucionarias de Colombia'}
Farc
\end{abbr}
 como grupo extremista no les interesa callar a los extremistas del otro espectro político.~ Dentro de la lógica de las \begin{abbr}% {'style': 'font-variant: small-caps;', 'title': 'Fuerzas Armadas Revolucionarias de Colombia'}
Farc
\end{abbr}
 es más importante debilitar el centro porque con un centro debilitado las voces moderadas pierden audiencia y son más
 proclives a gravitar hacia cualquiera de los extremos.

E internacionalmente la extrema derecha, aún hoy, es más rechazada internacionalmente que la extrema izquierda.
 Colombia, dividida entre esos dos extremos, estaría en una guerra donde ambas partes recibirían un grado de apoyo moral, económico e, incluso, militar de la comunidad internacional.
 Una institucionalidad destruida y, eventualmente, una capacidad de tomarse el estado.

Por el contrario, una democracia fuerte bajo principios políticos liberales, donde el pueblo encuentre solución a sus
 problemas dentro de la propia institucionalidad, donde se pueda discutir abiertamente de las distintas formas de
 resolver los problemas sociales y con compromisos y posturas moderadas, deslegitima completamente la lucha armada de
 los extremistas.

\par% p
En una democracia funcional las \begin{abbr}% {'style': 'font-variant: small-caps;', 'title': 'Fuerzas Armadas Revolucionarias de Colombia'}
Farc
\end{abbr}
 no tendrían otra opción a acabar la guerra que rendirse, lo cual aún con leyes benévolas como las que sugiere el Marco
 Legal para la Paz, sigue siendo una derrota.

\par% p
Las \begin{abbr}% {'style': 'font-variant: small-caps;', 'title': 'Fuerzas Armadas Revolucionarias de Colombia'}
Farc
\end{abbr}
 nunca estuvieron ni medianamente cerca a ganar la guerra y no lo están hoy.
 Pero nunca, ni en los momentos más duros de la doctrina de la Seguridad Democrática del gobierno de Álvaro Uribe Vélez, estuvo cerca de ser completamente derrotada.
 Mientras ellos crean que aún tienen una esperanza de continuar su absurda lucha lo seguirán intentando.

Y dentro de ese sentido asesinar a una de las voces más activas de la derecha dura cumple varios objetivos: enviar el
 mensaje de que los enemigos de su revolución no están a salvo (y de paso declarar a ese tipo de derecha como su
 enemigo), crear confusión en la población civil (terrorismo) y polarizar el país: una justificación para continuar
 peleando.

\par% p
Pero las \begin{abbr}% {'style': 'font-variant: small-caps;', 'title': 'Fuerzas Armadas Revolucionarias de Colombia'}
Farc
\end{abbr}
 no son el único enemigo de Fernando Londoño Hoyos.
 En estos momentos, con la información que conozco y que han revelado las autoridades no hay plena certeza de que sí
 hayan sido las \begin{abbr}% {'style': 'font-variant: small-caps;', 'title': 'Fuerzas Armadas Revolucionarias de Colombia'}
Farc
\end{abbr}
 los perpetradores de ese atentado.
 En mi opinión sí son los más probables autores, pero afirmar que lo son y la justificación que acabo de dar no es más
 que especulación de mi parte.

\par% p
Igualmente hay indicios de que el atentado que intentaron por la mañana contra el comando de la Policía hayan sido las \begin{abbr}% {'style': 'font-variant: small-caps;', 'title': 'Fuerzas Armadas Revolucionarias de Colombia'}
Farc
\end{abbr}
 y si ambos atentados hubieran cumplido su objetivo el grado de caos que habría hoy en la ciudad sería bastante alto.

\par% p
En los pasados disturbios contra Transmilenio, \anchor[http://blog.chlewey.net/2012/03/bloqueando-transmilenios/]{de los que hablé en este blog}, han surgido evidencias de que las acciones más violentas fueron coordinadas por las \begin{abbr}% {'style': 'font-variant: small-caps;', 'title': 'Fuerzas Armadas Revolucionarias de Colombia'}
Farc
\end{abbr}
.~ Probablemente no fueron las \begin{abbr}% {'style': 'font-variant: small-caps;', 'title': 'Fuerzas Armadas Revolucionarias de Colombia'}
Farc
\end{abbr}
 las que incitaron las protestas, pero sí aprovecharon para aumentar el caos.

\par% p
Y esto ata el tercer elemento con los que inicié este post: las protestas en universidades públicas contra el \begin{abbr}% {'style': 'font-variant: small-caps;', 'title': 'Tratado de libre comercio'}
tlc
\end{abbr}
.

\par% p
Creo firmemente que las acciones más violentas registradas ayer por los protestantes anti-\begin{abbr}% {'style': 'font-variant: small-caps;', 'title': 'Tratado de libre comercio'}
tlc
\end{abbr}
 tuvieron participación de las \begin{abbr}% {'style': 'font-variant: small-caps;', 'title': 'Fuerzas Armadas Revolucionarias de Colombia'}
Farc
\end{abbr}
.~ Repito: no creo que todos los que se oponen al \begin{abbr}% {'style': 'font-variant: small-caps;', 'title': 'Tratado de libre comercio'}
tlc
\end{abbr}
 sean farianos, simpatizantes de las \begin{abbr}% {'style': 'font-variant: small-caps;', 'title': 'Fuerzas Armadas Revolucionarias de Colombia'}
Farc
\end{abbr}
 o idiotas útiles de la insurgencia.
 (Y, desde luego, mucho menos estoy diciendo que los estudiantes de universidades públicas sean guerrilleros.)
 Pero sí creo que las \begin{abbr}% {'style': 'font-variant: small-caps;', 'title': 'Fuerzas Armadas Revolucionarias de Colombia'}
Farc
\end{abbr}
 tenían interés en infiltrar esas protestas como una forma de aumentar el caos.

\par% p
Y si creemos que ayer era una fecha muy especial para el actual gobierno, sin duda la resonancia de todos estos actos juntos hubiera sido un ruidoso (si no muy claro) mensaje.
 Un mensaje de que las \begin{abbr}% {'style': 'font-variant: small-caps;', 'title': 'Fuerzas Armadas Revolucionarias de Colombia'}
Farc
\end{abbr}
 seguirán luchando y que no se sienten derrotadas.

\par% p
Ayer quería escribir sobre el \begin{abbr}% {'style': 'font-variant: small-caps;', 'title': 'Tratado de libre comercio'}
tlc
\end{abbr}
.
 Sobre cómo veo el Tratado de Libre Comercio entre Colombia y los Estados Unidos a la luz de los motivos por los cuales sigo insistiendo en la creación del Partido Pirata Colombiano.
 Pero eso tendrá que esperar.

\chapter{Paz de izquierda o de derecha}
\begin{metadata}
	Published by \anchor[chlewey]{chlewey} on \anchor[http://ewey.co/B1237]{Wed, 16 May 2012 22:07:05 +0000}\\
	\categories{derecha, farc, izquierda, opinion}\\
	Shorthand: \anchor[http://blog.chlewey.net/2012/05/paz-de-izquierda-o-de-derecha/]{paz-de-izquierda-o-de-derecha}
\end{metadata}

\par% p
A veces usar los términos “derecha” e “izquierda” es una simplificación conveniente de la cual abusé bastante en \anchor[http://blog.chlewey.net/2012/05/el-placer-de-especular/]{mi entrada anterior}.
 A grandes rasgos hay todavía ciertas formas de pensar políticamente que pueden catalogarse como la forma de derecha o
 la forma de izquierda.

\anchor[http://blog.chlewey.net/wp-content/uploads/2012/05/swastic.png]{\begin{wrapfigure}{r}{150\px}\centering% {'src': 'http://blog.chlewey.net/wp-content/uploads/2012/05/swastic.png', 'style': 'clear: right;', 'title': u'Esv\xe1stica', 'height': '150', 'width': '150', 'alt': '', 'class': ['alignright', 'size-full', 'wp-image-1244']}
\includegraphics[width=150\px,height=150\px]{blog/swastic.png}
\end{wrapfigure}
}\anchor[http://blog.chlewey.net/wp-content/uploads/2012/05/comunism.png]{\begin{wrapfigure}{r}{150\px}\centering% {'src': 'http://blog.chlewey.net/wp-content/uploads/2012/05/comunism.png', 'style': 'clear: right;', 'title': 'Hoz y martillo', 'height': '140', 'width': '150', 'alt': '', 'class': ['alignright', 'size-full', 'wp-image-1241']}
\includegraphics[width=150\px,height=140\px]{blog/comunism.png}
\end{wrapfigure}
}\anchor[http://blog.chlewey.net/wp-content/uploads/2012/05/fasces.png]{\begin{wrapfigure}{r}{150\px}\centering% {'src': 'http://blog.chlewey.net/wp-content/uploads/2012/05/fasces.png', 'style': 'clear: right;', 'title': 'Fasces', 'height': '302', 'width': '150', 'alt': '', 'class': ['alignright', 'size-full', 'wp-image-1242']}
\includegraphics[width=150\px,height=302\px]{blog/fasces.png}
\end{wrapfigure}
}\anchor[http://blog.chlewey.net/wp-content/uploads/2012/05/anarchy.png]{\begin{wrapfigure}{r}{150\px}\centering% {'src': 'http://blog.chlewey.net/wp-content/uploads/2012/05/anarchy.png', 'style': 'clear: right;', 'title': u'Anarqu\xeda', 'height': '150', 'width': '150', 'alt': '', 'class': ['alignright', 'size-full', 'wp-image-1240']}
\includegraphics[width=150\px,height=150\px]{blog/anarchy.png}
\end{wrapfigure}
}\anchor[http://blog.chlewey.net/wp-content/uploads/2012/05/switch3.png]{\begin{wrapfigure}{r}{150\px}\centering% {'src': 'http://blog.chlewey.net/wp-content/uploads/2012/05/switch3.png', 'style': 'clear: right;', 'title': 'Rock', 'height': '189', 'width': '150', 'alt': '', 'class': ['alignright', 'size-full', 'wp-image-1239']}
\includegraphics[width=150\px,height=189\px]{blog/switch3.png}
\end{wrapfigure}
}\anchor[http://blog.chlewey.net/wp-content/uploads/2012/05/peace.png]{\begin{wrapfigure}{r}{150\px}\centering% {'src': 'http://blog.chlewey.net/wp-content/uploads/2012/05/peace.png', 'style': 'clear: right;', 'title': 'Paz', 'height': '150', 'width': '150', 'alt': '', 'class': ['alignright', 'size-full', 'wp-image-1243']}
\includegraphics[width=150\px,height=150\px]{blog/peace.png}
\end{wrapfigure}
}El ideal tradicional de derecha se basa en tres pilares: Dios, Patria y Familia.
 (Lo de Dios puede ser reemplazado por los dogmas de la religión mayoritaria o por algún tipo de moral “tradicional”.)
 Sugiere una actitud conservadora (respeto a la tradición) y una fuerte referencia a valores considerados tradicionales.

La izquierda, por otra parte, prefiere hacer énfasis en conceptos como “igualdad” y “justicia social”.
 En ese sentido trata de rescatar alguna causa que aun parezca ser una parte oprimida por la tradición (la derecha) y reivindicarla.
 En alguna época fueron los proletarios pero hoy pueden ser los indígenas, las mujeres que quieren abortar, los
 homosexuales que quieren formar familia, los toros de lidia masacrados, etc.

\par% p
Pero este trazado de izquierda y derecha es algo que cambia con el tiempo.
 Los cambios que crea la izquierda se convierten en la tradición de nuevas generaciones.
 El libre mercado, por ejemplo, base del capitalismo actual, fue un concepto que fue revolucionario en su momento: la izquierda que se oponía a la tradición mercantilista y el control del estado sobre la economía.
 Pues bien, esos mercados liberales que se impusieron se han convertido en la nueva derecha.
 Durante la administración de George W. Bush se acuñó incluso el término “Neo conservadores” \emph{(Newcons)} para referirse a los principales defensores de un menor control estatal mientras que las autodenominadas izquierdas
 insistían en un mayor control para evitar que los predadores del libre mercado pasaran por encima de los pobres.

\par% p
Las guerrillas colombianas actuales \anchor[http://blog.chlewey.net/2008/01/injusticia-social-bullshit/]{fueron inspiradas} en la revolución cubana y se adhirieron a los ideales que en su momento emanaban de la Unión Soviética.
 Como tal obtuvieron el remoquete de izquierda, y más adelante para diferenciarlas de los movimientos
 social-demócratas, se les llamó “extrema izquierda”.

Cuando en los años 1980 y 1990 surgieron grupos armados ilegales antiinsurgentes, se los llamó, en contraste, “extrema derecha”.
 Esta denominada extrema derecha incluía a narcotraficantes como Pablo Escobar (quien dejó varios escritos a favor de su interpretación de la “justicia social” en contra del “imperialismo yanqui”), ganaderos (quienes podríamos asumir abrazaban los ideales de Dios, Patria y Familia), militares, etc. a quienes más que una ideología común de cómo manejar el país, los unía un enemigo común.
 No me consta qué tan “derechas” y qué tan “extremas” fueron las autodefensas, más allá de combatir a otro enemigo
 armado.

\par% p
Y tan poca ideología (¿derecha?) había en general en esos grupos que hoy en día las \begin{abbr}% {'style': 'font-variant: small-caps;', 'title': 'Bandas criminales'}
bacrim
\end{abbr}
 que se escindieron del proyecto paramilitar se han aliado a las propias guerrillas para explotar el negocio de la
 droga mientras que se dedican a matar a quienes reclaman la tradición de títulos de propiedad arrebatados durante la
 guerra.

\par% p
Hoy se supone entonces que quien aboga por la solución armada del \anchor[http://blog.chlewey.net/2008/02/no-hay-conflicto/]{no-conflicto} colombiano es una persona de derecha mientras que quien considera que la solución negociada del conflicto debe ser de
 izquierda.

Pero qué pasa si yo prefiero establecer otro tipo de valores, por ejemplo decido que mi ideal máximo es la vida, humana o no.
 Entonces me opongo a las corridas de toros (izquierdista) y al aborto (derechista) y abogo por la solución negociada
 del conflicto (izquierdista) y la prohibición de esos venenos llamados drogas (derechista).

\par% p
O mi ideal máximo es la capacidad de cada ser humano adulto de escoger libremente y como tal no me opongo a las
 corridas de toros (derechista), apoyo el derecho de la mujer a abortar (izquierdista) y de escoger con qué droga
 experimentar o intoxicarse (izquierdista) y dado que en los aspectos del \ins{% {'style': 'color: #aaa;'}
no-}conflicto armado no es cuestión de libertades individuales simplemente tomo partido por, por ejemplo, un gobierno
 democrático sobre una insurgencia de ideales absolutistas y abogo por la solución militar (derecha).

Prefiero que sobre cada tema social cada individuo tome una postura propia y no que se autoetiquete y asuma todas las
 posturas que se supone que tengan que ver con esa etiqueta.

Mi postura frente al hecho de que ciertos colombianos autodenominados guerrilleros estén matando a otros colombianos (sean soldados, ideólogos de derecha o simples transeúntes) es que como sociedad debemos lograr que la matanza pare.
 La solución militar o la solución negociada no son más que medios para lograrlo.

El concepto de la impunidad, esgrimido como sagrado tanto por los ideólogos de la derecha (que no quieren que los guerrilleros salgan impunes por haber combatido al estado) como por los ideólogos
de la izquierda (que no quieren que los militares salgan impunes de haber combatido a las guerrillas), no me parece en
 últimas tan importante.

Sí.
 Prefiero ver a los asesinos presos.
 Pero prefiero ver a un asesino libre con el compromiso de dejar de asesinar que saber que sigue siendo un asesino
 perseguido pero aún activo y asesinando sólo porque aún no lo logramos capturar o matar.

Pero la ley del Marco Legal para la Paz que actualmente propone el gobierno no creo que sirva tampoco de mucho,
 sencillamente porque esos asesinos autodenominados guerrilleros no están realmente interesados.

No veo los demás colombianos qué les podamos ofrecer para que ellos dejen voluntariamente de asesinar colombianos.
 Ni siquiera entregarles el poder porque con seguridad con el poder seguirán conservando las armas y querrán hacer lo
 que han hecho todos los revolucionarios victoriosos: matar a los opositores.

\par% p
Desafortunadamente sólo hay un caso documentado en los cuales un ejército regular venció militarmente a una insurgencia
 irregular moderna: la \anchor[http://es.wikipedia.org/wiki/Emergencia\_Malaya]{Emergencia Malaya} que en 12 años, una relación de fuerzas de 40 a 1 y el desplazamiento forzado de medio millón de campesinos a campos de concentración por parte de los británicos, logró que el imperio británico venciera a las guerrillas comunistas.
 La derrota de los Tigres Tamiles podría también citarse como un caso de derrota por substracción de materia.

\par% p
Esto nos muestra que lograr la derrota militar de las \begin{abbr}% {'style': 'font-variant: small-caps;', 'title': 'Fuerzas Armadas Revolucionarias de Colombia'}
Farc
\end{abbr}
 y el \begin{abbr}% {'style': 'font-variant: small-caps;', 'title': u'Ejercito de Liberaci\xf3n Nacional'}
ELN
\end{abbr}
 será un camino largo y violento.
 Camino para el cual ocho años de doctrina de seguridad democrática no fueron suficientes.

Pero si recordamos que el objetivo no es derrotarlas, ni el objetivo es
apaciguarlas o asimilarlas, sino que el objetivo es neutralizarlas, tal vez se puede buscar alguna otra solución.

\chapter{Piratas del Caribe (ese mar entre Colombia y los Estados Unidos)}
\begin{metadata}
	Published by \anchor[chlewey]{chlewey} on \anchor[http://ewey.co/B1248]{Thu, 17 May 2012 13:09:45 +0000}\\
	\categories{activismo, opinion, partido-pirata, tlc}\\
	Shorthand: \anchor[http://blog.chlewey.net/2012/05/piratas-del-caribe/]{piratas-del-caribe}
\end{metadata}

Uno de los objetivos del Partido Pirata es ser un partido con un propósito principal.
No es nuestro objetivo responder a todos los aspectos de la sociedad y la política y por ello cualquier idea que yo
 haya expresado en este blog que no tenga que ver con nuestros cuatro puntos fundamentales es una opinión que no
 compromete a mi partido.

\anchor[http://blog.chlewey.net/wp-content/uploads/2012/05/ppco-bw.png]{\begin{wrapfigure}{r}{250\px}\centering% {'src': 'http://blog.chlewey.net/wp-content/uploads/2012/05/ppco-bw-300x300.png', 'title': 'Logo PPco', 'height': '250', 'width': '250', 'alt': '', 'class': ['', 'wp-image-1254', 'alignright']}
\includegraphics[width=250\px,height=250\px]{blog/ppco-bw-300x300.png}
\end{wrapfigure}
}Estos puntos fundamentales son:
\begin{enumerate}

\item Promoción y defensa de los derechos y libertades civiles.
\item Revisión de los conceptos de propiedad intelectual en búsqueda de un equilibrio entre productores y consumidores de
 contenidos culturales.
\item Acceso universal y neutralidad en la red.
\item Transparencia en la gestión pública y responsabilidad de los funcionarios.

\end{enumerate}

No somos un partido de izquierda o de derecha ni en lo económico (estatización v/s libre mercado) ni en otros aspectos sociales. No somos pro- ni anti-EE.UU.
Como parte de \anchor[http://pp.interlecto.net/wiki/Manifiesto\#Borrador\_del\_manifiesto]{nuestro manifiesto} reconocemos a Colombia como un estado de derecho donde es posible la deliberación con ideas y con ello rechazamos las
 vías violentas para imponer nuestros puntos de vista (y las vías violentas de otros para imponer sus puntos de vista).

\par% p
Entonces: \textbf{¿Dónde estamos frente al Tratado de Libre Comercio entre Estados Unidos y Colombia o \begin{abbr}% {'style': 'font-variant: small-caps;', 'title': 'tratado de libre comercio'}
tlc
\end{abbr}
?}

\anchor[http://www.youtube.com/watch?v=2W2pgc-6-t0]{\begin{wrapfigure}{r}{300\px}\centering% {'src': 'http://blog.chlewey.net/wp-content/uploads/2012/05/robledo-300x210.jpg', 'title': u'Sobre la entrada en vigencias del TLC - Hay que echar atr\xe1s el TLC', 'height': '210', 'width': '300', 'alt': '', 'class': ['alignright', 'size-medium', 'wp-image-1251']}
\includegraphics[width=300\px,height=210\px]{blog/robledo-300x210.jpg}
\end{wrapfigure}
}Me siento obligado a escribir esto una vez que en un \anchor[http://twitter.com/ppco/status/202415244662935553]{tweet desde la cuenta del Partido} se promocionara la postura del senador Jorge Enrique Robledo del Polo Democrático Alternativo.

\par% p
No voy a hablar de
disensos
al interior del Partido porque este no es uno de los temas del Partido y, aclaro, toda opinión que yo reflejo aquí es
 una opinión personal, precisamente porque no es un tema del Partido.

\section{¿Qué si es tema del Partido?}
Las cláusulas de defensa de la \emph{propiedad intelectual} incluidas en el Tratado y que fueron incluidas a pupitrazo en la ley 201 de 2012 (conocida como \#LeyLleras 2.0) van en contra del ideario de nuestro proyecto pues privilegian el control de contenidos sobre la creación y recreación y amenazan el acceso a Internet de los supuestos infractores.
Adicionalmente la aprobación de leyes por pupitrazo va en contra de nuestro principio de transparencia en la gestión
 pública.

\par% p
Salvo el punto de protección de derechos de copia y patentes y de los problemas de trámite de la ley, los demás puntos
 del \begin{abbr}% {'style': 'font-variant: small-caps;', 'title': 'tratado de libre comercio'}
tlc
\end{abbr}
 son accesorios y en muchos puntos mi opinión se aparta de la de Robledo.

\section{Lo que quedó mal}
Bajo la tesis de que un tratado de libre comercio no es una panacea pero que es mejor tenerlo que quedarse por fuera, el gobierno de Álvaro Uribe Vélez parecía más ansioso en obtener un tratado que en negociar un buen tratado.
Según varios analistas, el~\begin{abbr}% {'style': 'font-variant: small-caps;', 'title': 'tratado de libre comercio'}
tlc
\end{abbr}
~de \begin{abbr}% {'title': 'Estados Unidos'}
EE.UU.
\end{abbr}
 con Colombia es menos ventajoso para nuestro país que los otros~\begin{abbr}% {'style': 'font-variant: small-caps;', 'title': 'tratado de libre comercio'}
tlc
\end{abbr}
 de \begin{abbr}% {'title': 'Estados Unidos'}
EE.UU.
\end{abbr}
 con otros países como Perú o Centroamérica.

\par% p
Desde antes de que se terminara la negociación del tratado en 2006, varias voces advirtieron que la falta de
 infraestructura de transporte en Colombia era un \emph{handicap} para las exportaciones del país. Dentro de este campo estos dos últimos gobiernos no han hecho avances significativos.
 En contraste, gran parte del esfuerzo que el país debió hacer para mejorar esta infraestructura fue destinada a hacer \emph{lobby} en los \begin{abbr}% {'title': 'Estados Unidos'}
EE.UU.
\end{abbr}
 para que se aprobara el Tratado, aun a costa de incluir nuevas cláusulas que favorecían aun más la posición dominante
 del país del norte.

\section{Por otro lado}
Desde mi punto de vista el concepto de libre comercio es completamente compatible con la Filosofía Pirata. Una barrera
 arancelaria, al igual que casi todas las medidas proteccionistas, tiene como objeto principal proteger a ciertas
 élites productoras sobre el grueso de la población consumidora.
¿No sucede lo mismo con las leyes de protección de
 derechos de copia y derechos de autor?

Desde luego que hay que tener un gran cuidado con esta ortodoxia porque en cierta forma dentro de la economía y el comercio todos somos productores y todos somos consumidores.
Si no tengo garantías como productor (o como empleado de un productor), tendré menos recursos para ser consumidor.

Desde el punto de vista de la producción a largo plazo el proteccionismo puede tener uno de dos efectos: 1) al crear un mercado cautivo los productores tienen menos incentivos para innovar y la producción se estanca, aumentando la brecha entre la producción local y extranjera.
2) cuando una demanda exigente no puede ser cubierta por productos extranjeros aumenta la presión para que la
 producción local innove.

Hay ejemplos de ambos casos.
La China pre-contemporánea produjo grandes inventos como el ábaco, la pólvora, la imprenta y sin embargo fue superada por Japón y Europa.
Japón se congeló en el tiempo y cuando lo obligaron a abrir sus mercados se aseguró de aprender rápido.

Dentro de la cultura de compartir, nuestra propia innovación parte de copiar y recrear los modelos de los demás y
 construir sobre ello agregando nuestro aporte, para que este aporte propio sea igualmente tomado por los demás.

Muchos somos productores y consumidores al mismo tiempo y muchos son consumidores netos.

Leyes proteccionistas como las de protección de derechos de autor y derechos de copia en gran medida buscan crear privilegios a unos pocos productores.
Mercados cautivos donde unos pocos se enriquecen y los demás somos tratados como borregos o como delincuentes.

Un acuerdo de libre comercio, en su forma más básica, consiste en eliminar las trabas artificiales que favorecen a ciertas formas de producción.
Aumenta la base sobre la cual podemos comercializar nuestros productos, aumenta la base sobre la cual podemos obtener
 insumos para nuestros productos y aumenta la competencia.

\par% p
Por otro lado, si uno de nuestros principios es la universalización del acceso a la red y no somos productores de
 tecnología de acceso y telecomunicaciones, entonces un esquema económico que facilite la adquisición de tales
 tecnologías parecerían una buena idea.

\section{Concluyendo}
El \begin{abbr}% {'style': 'font-variant: small-caps;', 'title': 'tratado de libre comercio'}
tlc
\end{abbr}
 nos quedó mal hecho y nuestros gobiernos se preocuparon más por hacerlo (así fuese mal hecho) que a prepararnos para su entrada en vigencia.
Una de las cosas que quedaron mal hechas fue dejarnos imponer esquemas de protección de patentes y de derechos de copia
 que desfavorecen la recreación y la innovación local.

Un tratado de libre comercio es perfectamente compatible con los principios piratas.
Los Piratas en el Caribe en gran medida actuaban ante el proteccionismo que el Imperio Español imponía a sus colonias
 para su explotación exclusiva.

Si no estamos listos, en este momento nuestro objetivo debe ser alistarnos.
Debemos pensar que nuestros insumos y nuestros mercados son el mundo. ~Ver la situación como una oportunidad.

Y sí: denunciar los hechos puntuales que nos afectan.
(Yo por ejemplo denuncio a la Ley Lleras 2.0 y a la falta de infraestructura de transporte, los cuales son dos
 problemas colombianos y no extranjeros.)

\chapter{Cuestión de garantías}
\begin{metadata}
	Published by \anchor[chlewey]{chlewey} on \anchor[http://ewey.co/B1263]{Tue, 05 Jun 2012 14:45:40 +0000}\\
	\categories{actualidad, estado, libertad, opinion}\\
	Shorthand: \anchor[http://blog.chlewey.net/2012/06/cuestion-de-garantias/]{cuestion-de-garantias}
\end{metadata}

\par% p
Es a veces frustrante para uno, como ciudadano \emph{% {'title': u'Los que me conocen saben que este t\xe9rmino aqu\xed tiene una funci\xf3n sarc\xe1stica'}
de bien}, saber que el estado ofrece tantas garantías a los delincuentes: dizque derecho a la defensa, dizque derecho a la casa
 por cárcel, dizque el derecho a que la prensa no los trate como culpables hasta no ser vencidos en juicio y una serie
 de derechos más que parecen más destinados a poner a los delincuentes en la calle que a protegernos.

\begin{wrapfigure}{r}{300\px}\centering% {'width': '300', 'align': 'alignright', 'id': '', 'caption': u'Foto de V\xedctor Solano'}
\anchor[http://twitpic.com/9sgesq]{\includegraphics[width=300\px,height=225\px]{blog/591990506.jpg}}
\caption{Foto de Víctor Solano}
\end{wrapfigure}

\par% p
Entonces \anchor[http://blog.chlewey.net/2011/01/escandalitos-e-indignaciones/]{nos indignamos}.~ Sucede un crimen atroz y exigimos que el autor de tal abominación \anchor[http://www.noticiascaracol.com/nacion/articulo-266685-capturado-principal-sospechoso-de-la-muerte-de-rosa-elvira-cely]{se pudra en la cárcel}.
 Exigimos al estado que se encargue de todo aquello que nos produce inquietud o miedo, sea un abusador de niños o una montaña que se nos viene encima.
 Miedo que \anchor[http://www.grijalvo.com/Goebbels/Once\_principios\_de\_la\_propaganda.htm]{es usado} muchas veces por nuestros propios gobernantes para mantener y aumentar su poder.
 Miedo que también tumba gobiernos cuando creemos que no es capaz de copar nuestras temerosas expectativas o
 simplemente nos deja desamparados.

Nuestra posición frente al estado es ambivalente.
 Queremos un estado que nos proteja, pero rechazamos un estado que nos imponga tributos o nos imponga reglas.
 Desde luego que nuestros propios principios y nuestros propios temores nos hacen sacrificar uno de los lados de esta ambivalencia.
 Podemos sacrificar nuestra libertad por un poco de más seguridad, o sacrificar nuestra comodidad amparada por el
 estado por el derecho a que el estado no se entrometa en nuestras vidas.

Pero cuando delegamos en el estado la responsabilidad por nuestra seguridad; ¿sí estamos entregando nuestra confianza a
 una institución en la que realmente confiamos?

\begin{wrapfigure}{r}{300\px}\centering% {'width': '300', 'align': 'alignright', 'id': 'attachment_1264', 'caption': u'Captura de un video de la Polic\xeda'}
\anchor[http://blog.chlewey.net/wp-content/uploads/2012/06/MERLANO.jpg]{\includegraphics[width=300\px,height=184\px]{blog/MERLANO-300x184.jpg}}
\caption{Captura de un video de la Policía}
\end{wrapfigure}

\par% p
Nos indignamos por los senadores que abusan de su posición.
 Nos sentimos legitimados en eludir impuestos porque los políticos y los funcionarios son corruptos y se robarán la plata que como contribuyentes pagamos.
 Hacemos cruzadas en medios de comunicación sociales porque unos \anchor[http://www.rcnradio.com/noticias/policia-quemo-dos-perros-segun-habitantes-del-sur-de-bogota-2774]{policías incineran perros} y recordamos todos los abusos de la policía.
 Nos quejamos de la burocracia inútil.
 Ese estado formado por políticos, funcionarios y fuerza pública nos causa desconfianza.
 Un estado que estorba.~ Que no nos deja trabajar y emprender.~ Que nos roba.~ Que abusa de nosotros.

Un estado en el que no confiamos, pero al que aún así le exigimos que nos ayude.
 Que nos reconstruya la casa tras un desastre ecológico.
 Que encierre a los delincuentes.
 Que acabe con los terroristas.
 Que nos dé educación gratis.~ Que castigue duramente a los que maltratan animalitos.~ Que nos subsidie el desempleo.

Un día criticamos a la policía por sus reiterados abusos contra animales y contra personas desposeídas.
 Al día siguiente exigimos más policías que nos protejan de los predadores sexuales y de los vagabundos que atentan
 contra el disfrute de nuestros parques.

Sí.
 Tal vez no sea una contradicción.
 Queremos policías que se dediquen a atrapar a verdaderos delincuentes y no a maltratar a pobres perritos.

Nos quejamos de todas las garantías que el estado de derecho le otorga a los delincuentes pero esto es porque olvidamos que esas garantías no están allá para proteger a los malvados delincuentes de la justicia punitiva, sino que estas garantías están allá para proteger a todos los ciudadanos de los abusos del estado.
 Sí.
 Para protegernos de esos representantes que se creen con derecho de pasar por encima de nosotros.
 Para protegernos de esos oficiales de policía que no tienen recelos en tratar a las patadas a unos pobres indigentes e incinerar a sus perros.
 Para protegernos de los abusos de los funcionarios estatales.

¿Hasta qué punto queremos que el estado se entrometa en nuestras vidas con el fin de protegernos?
 ¿Qué tipo de estado es el que queremos que se entrometa?
 ¿Ese estado lleno de políticos interesados, funcionarios corruptos y fuerza pública abusadora?
 ¿O un estado dirigido por las personas más capaces y moralmente correctas?
 ¿Creemos realmente que esto últimos es posible?

\begin{wrapfigure}{r}{300\px}\centering% {'width': '300', 'align': 'alignright', 'id': 'attachment_1265', 'caption': '(fuente original requerida)'}
\anchor[http://blog.chlewey.net/wp-content/uploads/2012/06/velasco.jpg]{\includegraphics[width=300\px,height=201\px]{blog/velasco-300x201.jpg}}
\caption{(fuente original requerida)}
\end{wrapfigure}

Y no.
 No quiero ver a Javier Velasco (de comprobarse autor de todo lo que lo acusan) libre sólo porque pobrecito, está enfermo y no sabe lo que hace.
 No quiero ver al confeso asesino serial Luis Alfredo Garavito libre sólo porque haber confesado, haberse portado bien en la cárcel y decir que encontró a Dios en prisión sea algo que se mete en una calculadora de rebaja de penas.
 Esas son personas que desde mi lega opinión no representan garantías a la sociedad.

¿Entonces?

Tampoco quiero un estado que por su afán de encontrar delincuentes se meta en mi conexión de Internet, en mi
 correspondencia, en mis relaciones sociales.

No quiero un estado formado por individuos poco confiables, que roban el erario y abusan de su poder y, para rematar,
 entregarles a ellos la función de vigilarme a mí y a los míos.

\par% p
No quiero a un estado que, respondiendo exclusivamente a la indignación social, entregue más uniformes de policía a
 personas poco capacitadas y \anchor[http://www.legis.com.co/BancoConocimiento/L/la\_inflacion\_legislativa\_penal/la\_inflacion\_legislativa\_penal.asp]{legisle aumentando penas} y llenando las cárceles a límites tales que es imposible pensar que estas tengan un papel resocializador.

Quiero, como ciudadano, garantías frente al estado, así esas garantías también apliquen a mis indeseables conciudadanos.

\chapter{Moviéndonos en la ciudad}
\begin{metadata}
	Published by \anchor[chlewey]{chlewey} on \anchor[http://ewey.co/B1273]{Thu, 14 Jun 2012 16:53:25 +0000}\\
	\categories{movilidad, opinion, restriccion-vehicular, trasnporte-publico}\\
	Shorthand: \anchor[http://blog.chlewey.net/2012/06/moviendonos-en-la-ciudad/]{moviendonos-en-la-ciudad}
\end{metadata}

Un adulto sano, medianamente joven, tiene entre sus posibilidades de transporte por la ciudad un sinnúmero de alternativas.
 Puede caminar, montar en bicicleta o salir en monopatín o patines.
 Puede usar transporte público colectivo (buses), masivo (metro o tránsito rápido de buses) o individual (taxi).
 Puede usar su propio vehículo motorizado bien sea este una moto, un automóvil, una camioneta o un camión.

\anchor[http://iheartinspiration.com/quotes/bike-versus-car-fat-and-money/]{\begin{wrapfigure}{r}{300\px}\centering% {'src': 'http://iheartinspiration.com/wp-content/uploads/2012/03/bike-versus-car-fat-and-money.jpg', 'title': 'Bike v/s Car: fat & money', 'height': '201', 'width': '300', 'alt': '[Bike v/s Car: fat & money] ', 'class': ['alignright']}
\includegraphics[width=300\px,height=201\px]{blog/bike-versus-car-fat-and-money.jpg}
\end{wrapfigure}
}Un adulto sano, medianamente joven, decidirá en cada viaje de acuerdo a sus disponibilidades específicas, la economía y
la comodidad percibida que cada una de estas represente.
 No todos tienen bicicleta, no todos tienen automóvil, no todas las ciudades cuentan con transporte público masivo.
 El automóvil propio suele ser la opción más cómoda pero si sumamos el \emph{stress} de conducir en un trancón y la disponibilidad de parqueaderos en el destino, en ocasiones puede ser preferible el
 transporte público o la bicicleta.

\par% p
Si quiero ir al colegio (escuela) de mi hijo tengo que conducir (según Google Maps) 14,4 km y si es para hacer una vuelta entre semana tengo parqueadero gratis al llegar (si le meto plata eso debe consumirme unos 0,3 galones que valen \$ 2.600).
 Esa distancia fácilmente da para que un taxi me cueste más de \$ 12.000 .
 Puedo también caminar un kilómetro a la estación de Transmilenio (sistema de tránsito rápido de buses de Bogotá), tomar una ruta alimentadora al llegar al portal y caminar 600 m en vías sin andén (\$ 1.750) asumiendo que el bus alimentador funcione a esa hora, de lo contrario tendré que caminar 2 km en vías sin andén o pagar una carrera de taxi
de unos \$ 4.500 .
 La relación comodidad-precio-tiempo sin duda favorecen utilizar el transporte particular en este caso específico.
 (El transporte público colectivo es ligeramente más económico, pero si se suman los tiempos de caminar y de esperar los buses y las condiciones de la caminata la ventaja se diluye.)
 No he hecho el experimento en bicicleta.

\par% div% {'style': 'color: #555; margin: 1ex 1em; font-size: 90%;'}
Nota.
 El viaje diario de mi hijo al colegio se hace usando un bus de la flota de buses que el colegio ofrece a sus alumnos.
 Este servicio cuesta unos \$ 6.500 por trayecto (\$ 12.000 al día).
 Teniendo en cuenta que yo tendría que hacer cuatro trayectos diarios significaría que tendría que invertir más de dos
 horas de mi tiempo y más de \$ 10.000 diarios en gasolina si los llevara y trajera en automóvil, lo que además es un
 ejercicio ecológicamente más irresponsable.

\anchor[http://adhara-black.lacoctelera.net/post/2009/09/05/de-tranv-as-transmilenios-y-otros-vehiculos-carga-era-de]{\begin{wrapfigure}{r}{294\px}\centering% {'src': 'http://www.semana.com/photos/generales/ImgArticulo_T1_60270_200934_184902.jpg', 'title': 'Transmilenio', 'height': '203', 'width': '294', 'alt': '[Transmilenio] ', 'class': ['alignright']}
\includegraphics[width=294\px,height=203\px]{blog/ImgArticulo_T1_60270_200934_184902.jpg}
\end{wrapfigure}
}De mi casa al trabajo de mi esposa son 9,8 km (en un sentido diametralmente opuesto al colegio del niño).
 Si ella usara el carro para un día usual de trabajo éste tendría que gastar cerca de 9 horas de parqueadero, tiempo durante el cual el automóvil no servirá para otras funciones en la casa.
 Si se suma que sus desplazamientos se harían en las horas pico de conmutación esto significa que hay que agregar el stress del trancón.
 Un bus colectivo normal le cuesta \$ 1.450 por trayecto, pero usa vías más congestionadas y el tiempo que tarda supera a la opción que mi esposa prefiere: caminar un kilómetro a Transmilenio, pagar el pasaje de \$ 1.750 y caminar 700 m en subida para llegar a su lugar de trabajo.
 (En total casi kilómetro y medio más de caminada que el bus colectivo.)
 Su relación comodidad-precio-tiempo la hace preferir el transporte público masivo sobre otras opciones.
 La comodidad del automóvil se diluye por los precios de parqueo y el stress de conducir y para ella la bicicleta no parecería una opción.
 El taxi, a pesar de los costos, le parece una alternativa razonable (y más económica que los precios de parqueadero)
 cuando tiene premuras de tiempo o tiene que cargar algún paquete grande.

\par% p
Cuando trabajé para la Regional Bogotá Sur de la Universidad Minuto de Dios, a 20 km de mi casa, descubrí que la
 relación comodidad-precio-tiempo favorecía en mi caso al automóvil sobre Transmilenio, sobre todo teniendo en cuenta
 las horas de desplazamiento (plena hora pico de ida \relax{% {'style': 'color: #999999;'}
[20 minutos de ahorro]} y una hora bastante solitaria de regreso \relax{% {'style': 'color: #999999;'}
[de casi una hora a un viaje de menos de 15 minutos]}), pero para muchas de las vueltas que tengo que hacer cerca de mi casa prefiero la bicicleta o caminar o combinar transporte público colectivo con algo de caminata.
 Tengo dos problemas principales para usar la bicicleta: el primero es que no tengo equipo adecuado sobre todo para usar la bicicleta por la noche (lo cual sería algo que estaría en mis manos solucionar) y la otra es que a muchos de los lugares donde voy en bicicleta no tienen facilidades para dejar este vehículo guardado o amarrado.
 (O a los porteros no se les da la gana permitir que un visitante ocasional use los lugares designados.)

\par% p
Pero hablo aquí de los desplazamientos individuales.
 Salvo la conmutación diaria entre la casa y el trabajo de mi esposa, el \emph{carpooling} (compartir el automóvil) no representa una alternativa real.

\begin{wrapfigure}{r}{300\px}\centering% {'width': '300', 'align': 'alignright', 'id': ''}
\anchor[http://deltamallorca.com/ozonobike/informacion/transporte/movilidad-reducida/]{\includegraphics[width=300\px,height=202\px]{blog/2921426221_333bca691e.jpg}} [Ahora es cuestión de encontrar dónde parquear esto]
\end{wrapfigure}
Pero en mi casa hay otras necesidades.
 Mi suegra, una señora con movilidad reducida, necesita desplazarse de cuando en cuando para asistir a citas médicas o cobrar su pensión.
 Ni Transmilenio (con sus caminatas a las estaciones y dentro de las estaciones ni su congestión de gente), ni el transporte público colectivo (completamente mal diseñado para personas con cierta discapacidad) son buenas alternativas para ella.
 Eso hace que la mejor alternativa para ella, independientemente de los costos, sea el automóvil bien particular o bien
 taxi.

Hacer un mercado grande en la casa implicaría, si no se usa automóvil particular, alquilar un transporte (taxi o
 servicio de envíos) para llevar los paquetes a la casa, y si se hace mercado y compras en más de un sitio este
 alquiler de transporte se complica y encarece.

\anchor[http://cronicasbta.blogspot.com/2010/08/el-fin-de-los-cebolleros-parte-2.html]{\begin{wrapfigure}{r}{288\px}\centering% {'src': 'http://4.bp.blogspot.com/_0VajA1FAiOo/TFnjm7odzHI/AAAAAAAAABQ/_g9jSUgw9m0/s320/bus+cebollero.jpeg', 'title': 'Interior de un bus colectivo', 'height': '146', 'width': '288', 'alt': '[Interior de un bus colectivo]', 'class': ['alignright']}
\includegraphics[width=288\px,height=146\px]{blog/bus+cebollero.jpeg}
\end{wrapfigure}
}Pasear, ir a una cita médica, a una actividad recreativa o a cualquier otro desplazamiento con mi hijo que no esté cubierto por su transporte escolar tiene sus complicaciones, bien en el uso del automóvil particular o bien usando cualquier tipo de transporte público.
 Si sumamos dos hijos, ambos menores de 10 años (pero ambos ya en edades de “todo niño paga”) y más en caso de un
 desplazamiento de toda la familia, suegra incluída, el automóvil particular se convierte en prácticamente una
 obligación.

Desde luego que hay familias más grandes que la mía y con los mismos problemas que no tienen la opción de usar un
 automóvil particular. ¿Cómo hacen ellos?

Por otro lado hay familias que derivan su sustento de explotar un vehículo automotor en formas que no están tan reguladas como un taxi o un camión con placas de servicio público.
 Automóviles particulares utilizados para recoger y repartir por la ciudad pequeños paquetes, visitadores, etc.

\par% p
Las medidas que tome la administración distrital para restringir, desestimular o simplemente prohibir el uso de mi automóvil particular me afectan en mi libertad de movilidad y más cuando incluyo a la familia además de mi interés individual.
 Pero, sin duda, encontraré formas de adaptarme.~ \anchor[http://ntic.uson.mx/wikiseguridad/index.php/Saavedra\_martinez:\_Seguridad\_vial\_Cuidados\_al\_dise\%C3\%B1ar\_espacios\_para\_personas\_capacidades\_diferentes]{\begin{wrapfigure}{r}{210\px}\centering% {'src': 'http://ntic.uson.mx/wikiseguridad/images/7/7d/Minusvalidos.jpg', 'title': u'Aviso para minusv\xe1lido', 'height': '210', 'width': '210', 'alt': u'[Aviso para minusv\xe1lido] ', 'class': ['alignright']}
\includegraphics[width=210\px,height=210\px]{blog/Minusvalidos.jpg}
\end{wrapfigure}
}No me sería difícil, por ejemplo, conseguir un aviso de transporte de discapacitados para cuando acompañe a mi suegra a alguna diligencia porque 1) ella aplica y 2) igual se pueden conseguir.
 Pero la mayor parte de estas medidas que han sugerido las administraciones distritales desde Peñalosa generalmente
 significarían aumentar mis gastos de moverme y de mover a mi familia.

Yo celebraría el día que hubiere un transporte público colectivo que mi suegra pudiera usar.
 Un transporte público colectivo que no fuera amenazante con mis hijos.
 Un transporte público colectivo que pudieran usar mis hijos sin que yo tenga que acompañarlos.
 Que hubiera ciclorrutas completas que pudiera usar un niño de 8 o 10 años.
 Que hubiera andenes cómodos desde la parada del bus público hasta la entrada del colegio de mi hijo.

Cuando esto y muchas más cosas sucedan.
 Cuando al analizar mi relación comodidad-precio-tiempo encuentre en el transporte público mayores beneficios (y no por
 encarecimiento del precio del uso del automóvil particular o su simple prohibición), sin duda me bajaré de mi
 automóvil particular y utilizaré más las alternativas que la ciudad me ofrece.

\anchor[http://www.aleovias.com/noticia\_VuelvePicoyPlacaenBogot\%C3\%A1\_6376]{\begin{wrapfigure}{r}{300\px}\centering% {'src': 'http://www.aleovias.com/aleoAdmin/archivos/files/0noticias_1324654161.jpg', 'title': u'Restricci\xf3n vehicular (mal llamada Pico y Placa) 2011-2012', 'height': '225', 'width': '300', 'alt': '', 'class': ['alignright']}
\includegraphics[width=300\px,height=225\px]{blog/0noticias_1324654161.jpg}
\end{wrapfigure}
}\anchor[http://www.mundonets.com/actualidad/pico-y-placa-bogota-mes-de-julio-del-2012/]{\begin{wrapfigure}{r}{300\px}\centering% {'src': 'http://www.mundonets.com/images/articulos/pico-y-placa-bogota-julio-2012.jpg', 'title': u'Restricci\xf3n vehicular por pico y placa en julio de 2012', 'height': '315', 'width': '300', 'alt': '', 'class': ['alignright']}
\includegraphics[width=300\px,height=315\px]{blog/pico-y-placa-bogota-julio-2012.jpg}
\end{wrapfigure}
}Desde la administración de Enrique Peñalosa, cada vez que nos asustamos por la magnitud del trancón de Bogotá la
 solución \emph{temporal} es prohibir más y más el uso del automóvil.
 A pesar de que parece estar improvisando celebro que el alcalde Gustavo Petro haya reducido la restricción vehicular y
 haya regresado al esquema de \textbf{\emph{Pico}}\emph{ y Placa} (“pico” por restricción en horas pico) sin aumentar la restricción a los días sábado.
 Desde mi punto de vista la restricción debería eliminarse completamente a cambio de políticas que ataquen directamente
 el problema en lugar de esconderlo bajo prohibiciones.

Si las administraciones no parten sus políticas de desestímulo al automóvil particular de preguntarse por qué los ciudadanos (que pueden) usan el automóvil, entonces los ciudadanos (que puedan) seguirán buscando maneras de seguir usando el automóvil.
 Comprarán otro automóvil o no se desharán del viejo cuando renueven o adaptarán sus horarios para tener el carro parqueado durante 13 o 14 horas cerca a su trabajo sacrificando tiempo con sus familias.
 En unos pocos casos algunos ciudadanos entenderán las ventajas de alternativas como la bicicleta o el transporte
 público para sus necesidades particulares, pero estos casos son unas pocas afortunadas excepciones.

Prohibir el automóvil porque sí, atenta contra las libertades ciudadanas y contra la libertad de empresa.
 Afecta, a veces, a poblaciones más vulnerables como las que usan el automóvil para trabajar o parte de la población discapacitada.
 Aumenta la corrupción y la cultura ciudadana del atajo.
 Fomenta la pereza mental de nuestros administradores cuando consideran que toda solución está en prohibir para ocultar
 el problema y no en analizar las causas del mismo para acabarlo o disminuirlo.

Ahora, si en algo este artículo parece que critico a Peñalosa y aplaudo a Petro recordemos que el pico y placa original de Peñalosa era de 4 horas al día para el 40\% de los automóviles particulares mientras que lo que nos ofrece Petro es de 7 horas al día para el 50\% de los mismos.
 (Lo cual simplemente es mejor que las 14 horas al día de Moreno.)
 En 1998 era una medida novedosa que pretendía no cometer los errores que ya se habían probado en Ciudad de México
 (restricción en horas pico sobre restricción todo el día, 40\% de los automóviles vs 50\%) mientras que los
 consiguientes aumentos a la restricción, y particularmente la restricción de todo el día impuesta por Moreno,
 significaron, desde mi punto de vista, no más que la incapacidad de pensar en soluciones de fondo.

Pero es hora de pensar en soluciones de fondo.
 En un transporte público que invite a ser usado.
 En ciclorrutas que puedan ser usadas por los niños.
 En rebajas de impuestos a las empresas que estimulen la bicicleta y el carpooling.
 En un control efectivo para que automovilistas, transporte público y empresas de mensajería y repartición de mercancía
 o de construcción no obstruyan vías públicas parqueando en horas pico.

\chapter{Reformando congresos}
\begin{metadata}
	Published by \anchor[chlewey]{chlewey} on \anchor[http://ewey.co/B1281]{Mon, 25 Jun 2012 14:31:21 +0000}\\
	\categories{information, politica}\\
	Shorthand: \anchor[http://blog.chlewey.net/2012/06/reformando-congresos/]{reformando-congresos}
\end{metadata}

\begin{wrapfigure}{r}{300\px}\centering% {'width': '300', 'align': 'alignright', 'id': 'attachment_1286'}
\anchor[http://blog.chlewey.net/wp-content/uploads/2012/06/plenaria.jpg]{\includegraphics[width=300\px,height=199\px]{blog/plenaria-300x199.jpg}} Sesión plenaria del Senado de Colombia. Foto Angel Vargas.
\end{wrapfigure}

Colombia tiene 268 representantes para 45.000.000 de colombianos. Esto es que tenemos un representante por cada 170.000
 habitantes. Es un número relativamente bajo frente a otras democracias.

\begin{wrapfigure}{r}{273\px}\centering% {'width': '273', 'align': 'alignright', 'id': 'attachment_1284'}
\anchor[http://blog.chlewey.net/wp-content/uploads/2012/06/saque\_cuentas.jpg]{\includegraphics[width=273\px,height=300\px]{blog/saque_cuentas-273x300.jpg}} Poster en formato desmotivaciones que vi circulando hoy por Facebook.
\end{wrapfigure}

Nuestro congreso, por otro lado, es costoso. No tengo datos de costos de funcionamiento de otros congresos y
 parlamentos bien en comparación con el PIB, número de representantes o número de habitantes. Si alguien los tiene le
 agradezco que me los facilite.

Podemos tomar el sentimiento fácil de decir que necesitamos menos representantes para que haya menos pícaros en el
 congreso, pero ¿podremos garantizar que los que saldrán elegidos serán los menos pícaros? O, por el contrario,
 estaremos entregando a menos políticos el poder de repartirse su cuota burocrática (es decir más cuota por afortunado
 representante).

\begin{wrapfigure}{r}{300\px}\centering% {'width': '300', 'align': 'alignright', 'id': 'attachment_1285'}
\anchor[http://blog.chlewey.net/wp-content/uploads/2012/06/senadores.jpg]{\includegraphics[width=300\px,height=195\px]{blog/senadores-300x195.jpg}} Invitación falaz a reducir el número de ``senadores''. Primero, Colombia no tiene 270 senadores sino tan solo 102
 mientras que en EE.UU. hay dos senadores por cada estado.\\
Tenemos 268 congresistas (incluyendo los 166 representantes a la cámara) pero en ese orden de ideas los suecos tienen
 349 parlamentarios y los alemanes 622 (más 69 consejeros).
\end{wrapfigure}

\par% p
¿Qué hay de los movimientos alternativos? Menos congresistas implica mayores trabas para que un movimiento alternativo
 obtenga una representación, pero, tal vez, esto no es problema: muchos de los movimientos alternativos pueden no ser
 suficientemente representativos.

\par% div% {'style': 'opacity: .75; font-size: .9em; margin: 1ex 1em;'}
[Debo aclarar que aquí podría tener yo un conflicto de intereses pues estoy en la construcción de un movimiento
 alternativo con aspiración de poder.]

Entonces tendremos menos congresistas y esto significará menos pícaros. ¿Por qué no llevar el razonamiento al extremo y prescinidir completamente del congreso?
 No más escándalos en el congreso.~ No más pícaros legislando y emitiendo reformas a beneficio propio.

Ese sería un nuevo modelo de estado.~ Seríamos pioneros.~ ¿No?

No.~ Ese modelo de estado ya existe y tiene un nombre: dictadura.

En teoría en una democracia representativa como la letra dice que es la colombiana, existe un congreso o parlamento
 donde el constituyente delegado (parlamentarios, senadores, representantes a la cámara o como querramos llamarlos)
 representan al constituyente primario (nosotros, los ciudadanos).

Si el problema es que nuestros representantes no nos representan.
 Si el problema es que nuestros representantes son costosos y roban.
 Si el problema son las personas que nosotros elegimos porque no responden a nosotros, los ciudadanos, entonces el
 problema no es el número.

Nuestros representantes deben representarnos y no todos lo hacen.
 O tal vez sí.
 En mi caso personal el senador que elegí fue de los que votó no a redacción final de la Reforma a la Justicia (tendría que ver qué participación tuvo en los demás debates).
 Tal vez nuestros senadores y representantes a la cámara sí representen al país que los eligió y que los problemas de
 corrupción y legislación a nombre de intereses propios e intereses privados no sean más que un reflejo de lo que somos
 en conjunto los ciudadanos de nuestro país.

No pretendamos que con sólo disminuir el número de nuestros representantes vamos a estar mejor representados.
 Esa es una solución facilista y retórica.
 Lo que debemos preocuparnos es que nuestros representantes nos respondan por sus actos.
 Tal vez el ajuste en el número sea necesario pero esa no es la solución.

\chapter{No nos crean tan...}
\begin{metadata}
	Published by \anchor[chlewey]{chlewey} on \anchor[http://ewey.co/B1293]{Tue, 21 Aug 2012 19:23:26 +0000}\\
	\categories{alvaro-uribe, opinion, paramilitarismo, politica}\\
	Shorthand: \anchor[http://blog.chlewey.net/2012/08/no-nos-crean-tan/]{no-nos-crean-tan}
\end{metadata}

Ante la declaración de culpabilidad de Santoyo viene Álvaro Uribe Velez a razgarse las vestiduras y mostrarnos como es
 el poder de penetración del narcoparamilitarismo que lograron colar a ese general en el cuerpo de seguridad del
 presidente que más combatió al paramilitarismo y el narcotráfico.

El presidente que hizo del micromanagement su estilo de gobierno.
 Que cada semana iba a un consejo comunal y recordaba el nombre de todos los funcionarios del municipio, desde el alcalde hasta el que servía los tintos.
 Ese presidente no tenía por qué investigar a su jefe de seguridad, ni de las acciones que este ya tenía en la
 Procuraduría antes de ser ascendido a General.

El narcotráfico que untó al después General Santoyo, el paramilitarismo que éste confesó haber apoyado, logró colar esta ficha dentro del propio esquema de seguridad de Álvaro Uribe.
 Pobrecito Presidente.~ El único presidente que ha enfrentado al paramilitarismo.

Pero ¿Cómo enfrentó Álvaro Uribe al paramilitarismo?

Negoció con ellos.
 Les creó una ley a la medida para que pagaran irrisorias penas de arresto y salieran con sus fortunas intactas tras pagar una pequeña indemnización a las víctimas.
 Ley que fue luego endurecida por la Corte Constitucional (que no por el ejecutivo).

Y cuando se dio cuenta que no era posible armar un caso contra los jefes paramilitares sin la propia confesión de
 ellos, y que estas confesiones estarían plagadas de salpicaduras a muchos colombianos de bien; con estos paramilitares
 ya recluídos porque se habían ``demovilizado'' y entregado (porque no fueron capturados), decidió entregarlos a los
 EE.UU.

Porque, eso sí, el gran compromiso que Álvaro Uribe y sus escuderos siempre esgrimen como lucha contra el narcotráfico
 es el récord de extradiciones de colombianos a los EE.UU.

No soy enemigo de la extradición.
 Si un colombiano va a los EE.UU. y comete un delito serio allá, la soberanía colombiana no debe convertirse en un amparo para garantizar la impunidad.
 La extradición es un mecanismo de cooperación judicial y debe ser entendido como tal.

La extradición no es y no debe ser la política criminal de un estado.
 En este sentido lo único que hace es delegar esta política a otro estado.

Uribe ha extraditado a más colombianos a los EE.UU. que cualquier otro presidente sólo porque los EE.UU. han solicitado en extradición a más colombianos durante el mandato de Uribe que durante cualquier otro mandato.
 No porque el compromiso de Uribe haya sido mayor.

Y Uribe extraditó a los cabecillas paramilitares no para luchar contra el paramilitarismo, porque aquí nisiquiera está
 delegando a los EE.UU. los cargos por paramilitarismo y genocidio de estos jefes sino que está ignorando completamente
 los crímenes de sangre que ellos cometieron aquí para que otro país los retenga por un delito accesorio.

Entonces no, señor expresidente Álvaro Uribe.
 No se las venga aquí a andar de indignado.
 Yo no sé si Ud. es culpable de paramilitarismo, o de haber confiado en personas como el exdirector del DAS Jorge Noguera, o haber aceptado como jefe de seguridad a un policía cuyos cuestionamientos existentes de nexos con el paramilitarismo no eran completamente públicos.
 No sé si sea culplable en forma alguna de los desfalcos que algunos empresarios quisieron abusar de una política de su gobierno manejada por Andrés Felipe Arias.
 No sé si sea culpable de que en su afán de mostrar resultados con la guerrilla su Comisionado de Paz haya metido (o se haya dejado meter) falsos desmovilizados.
 No sé si sea culpable de que funcionarios del DAS en afán de congraciarse con Ud. hayan interceptado ilegalmente a
 opositores y magistrados.

Sólo le pido que deje de presentarse como el único colombiano que ha salvado a este país gracias a sus grandes dotes de
 microgerente mientras a su alrededor pasaban todas esas cosas de las que Ud. ahora se muestra indignado.

Porque yo si recuerdo que uno de los grandes puntos en su campaña de 2002 fue acabar con la politiquería y como en ocho
 años de gobierno no sólo olividó esa promesa sino que usó esta misma politiquería para lograr un plan de gobierno que
 fue incapaz de hacer irrelevantes a las autodenominadas Fuerzas Armasdas Revolucionarias de Colombia – Ejercito del
 Pueblo.

Y toda esa corrpupción de la que acusan a su gobierno tiene que ver con eso.

\chapter{Vientos de paz}
\begin{metadata}
	Published by \anchor[chlewey]{chlewey} on \anchor[http://ewey.co/B1302]{Tue, 28 Aug 2012 17:21:50 +0000}\\
	\categories{estado, farc, guerrilla, information, politica}\\
	Shorthand: \anchor[http://blog.chlewey.net/2012/08/vientos-de-paz/]{vientos-de-paz}
\end{metadata}

\par% p
Vuelve y juega el cuento de la paz, de la salida negociada del conflicto o de la máxima de no negociación con el
 terrorismo. El presidente de la República acaba de anunciar que ha habido acercamientos previos con las \begin{abbr}% {'title': 'Fuerzas Armadas Revolucionarias de Colombia'}
Farc
\end{abbr}
 y que espera hacerlos con el \begin{abbr}% {'title': u'Ej\xe9rcito de Liberaci\xf3n Nacional'}
Eln
\end{abbr}
, mientras que el expresidente denuncia que hay acuerdos con el terrorismo.

Tomar una postura debe partir de unos principios, y la postura debe ser consecuencia de esos principios. Los principios
 no son absolutos ni universales y el que otra persona tenga principios diferentes a los míos no lo hace equivocado. En
 mi opinión, más importante que la veracidad de los principios es la coherencia ideológica entre estos y la postura
 tomada. Invito así a quien quiera controvertirme que empiece por enunciarme sus propios principios.

En mi opinión (es decir, mis principios), una sociedad ideal es aquella en la cual se puedan ejercer los derechos y
 libertades civiles tales como la libertad de expresión, el derecho a la privacidad, la libertad de empresa, libertad
 de cultos, etc. con un mínimo de fricción; y que ante los conflictos (porque la ausencia de conflictos la creo
 imposible) existan reglas claras y mecanismos de autoridad reconocida que permitan dirimirlos sin la necesidad de la
 violencia física. Creo en el derecho a la propiedad (incluyendo el derecho a la propiedad colectiva cuando emana de
 una decisión autónoma) y en el derecho a la vida y la dignidad de la vida humana. Creo que tenemos el derecho a tener
 una conciencia propia y a poder expresarla por medios no violentos, incluyendo la libertad de denunciar y sospechar.
 Creo, incluso, que la expresión por medios violentos está amparada por el derecho a la expresión sólo que está
 contrapuesta al derecho a la vida y a la integridad física de otras personas y en mi opinión estos derechos humanos
 priman sobre el derecho civil expuesto. Creo que las sociedades deben tener mecanismos para defender sus derechos
 fundamentales lo cual incluye todo el aparato de policías, fiscales y jueces que prevengan, eviten o castiguen las
 transgresiones a los derechos de los demás.

\par% p
La existencia de grupos armados como las \begin{abbr}
Farc
\end{abbr}
 y el \begin{abbr}
Eln
\end{abbr}
, así como los ejércitos privados al servicio del narcotráfico o de otros intereses particulares, las bandas
 criminales, y otros fenómenos dentro del momento actual de la historia del país atentan contra esa sociedad ideal. En
 otras palabras, lo que algunos llaman amenaza terrorista y otros conflicto armado o guerra, es una amenaza seria al
 desarrollo y la dignidad de los colombianos.

\par% p
Y, como tal, esta amenaza o conflicto debe acabar.

\section{Sobre el concepto de conflicto}
Para mí el concepto de conflicto lo defino a partir de lo que conocemos como conflicto de intereses. Yo quiero algo.
 Tú quieres algo. En ocasiones nuestros algos son distintos y ambos podemos obtenerlo y entonces no hay conflicto, pero
 en otras ocasiones ese algo es lo mismo, o el algo del uno implica el agotamiento del algo del otro. Nuestros
 intereses entran en conflicto y los dos no podemos tener al mismo tiempo nuestros respectivos algos.

El conflicto termina cuando una de las partes, o ambas, renuncian a sus pretensiones o son incapaces de obtenerlas.
 Bien porque una parte se apropió de su interés antes que el otro, o lo tomó por la fuerza despojando a la contraparte,
 bien porque tras una pelea una de las partes se impuso, o bien porque se charló y se llegó a un acuerdo.

En ocasiones el conflicto no se da porque los intereses sean incompatibles sino porque los creemos incompatibles. Esto
 se da, principlamente, cuando se confunde lo que queremos con el método para obtener lo que queremos.

Hay conflictos personales (entre individuos), grupales, internacionales, etc. pero un tipo de conflicto que me parece
 relevante para esta discusión son los conflictos sociales. Esto es cuando dos o más grupos significativos de la
 sociedad se encuentran ante intereses aparentemente incompatibles.

En el caso colombiano, podríamos pensar que existe un estado constitucional que expresa algunos intereses y existe una
 subversión cuyos intereses entran en conflicto con el primero. El estado debe, por mandato constitucional, proteger la
 vida, honra y bienes de los colombianos y como tal no puede aceptar que la subversión atente contra la vida, honra y
 bienes de los demás ciudadanos. Este estado también, tradicionalmente, ha estado al servicio principal de los
 intereses de una clase política dirigente y de quienes financian a esta clase política. (Lo que acabo de decir es una
 sobresimplificación de un fenómeno más complejo.)

Esa subversión tiene como interés expuesto reemplazar al estado por uno bajo los ideales comunistas en el cual no
 existan conflictos de clase y, particularmente, tumbar al estado opresor actual que sirve a los intereses particulares
 de sus clases dirigentes. Para lograr esta lucha debe financiarse y para ello ha recurrido a negocios ilegales y a
 defender estos negocios, y así uno de sus intereses actuales también es proteger y preservar tales negocios. Por otro
 lado, independiente de la perversión que es el narcotráfico, la sola noción de estado comunista niega muchos de los
 derechos y libertades civiles consagrados hoy en la Constitución y los cuales defiendo.

Ahora, dentro del propio estado hay disidencias y conflictos de intereses y el pueblo colombiano está conformado por
 individuos y grupos con sus propios intereses, muchos de los cuales entran en conflicto. Pero en aras de la
 simplificación hablaré sólo de estos dos actores: el estado constitucional y la subversión; particularmente porque en
 la actualidad las diferencias partidistas al interior del estado no se dirimen por las armas.

\par% p
Hay otra definición de conflicto, o más exactamente de conflicto armado sin carácter internacional (conflicto armado
 interno). Una definición legal que aparece en el título de ámbito del \emph{\anchor[http://www.icrc.org/spa/resources/documents/misc/protocolo-ii.htm]{Protocolo II adicional a los Convenios de Ginebra de 1949 relativo a la protección de las víctimas de los conflictos
 armados sin carácter internacional, 1977}}. (Protocolo II en adelante.)

\par% p
Esta definición reza:

\begin{blockquote}
TÍTULO I - ÁMBITO DEL PRESENTE PROTOCOLO
\end{blockquote}

Artículo 1. Ámbito de aplicación material

1. El presente Protocolo, que desarrolla y completa el artículo 3 común a los Convenios de Ginebra del 12 de agosto de
 1949, sin modificar sus actuales condiciones de aplicación, se aplicará a todos los conflictos armados que no estén
 cubiertos por el artículo 1 del Protocolo adicional a los Convenios de Ginebra del 12 de agosto de 1949 relativo a la
 protección de las víctimas de los conflictos armados internacionales (Protocolo I) y que se desarrollen en el
 territorio de una Alta Parte contratante entre sus fuerzas armadas y fuerzas armadas disidentes o grupos armados
 organizados que, bajo la dirección de un mando responsable, ejerzan sobre una parte de dicho territorio un control tal
 que les permita realizar operaciones militares sostenidas y concertadas y aplicar el presente Protocolo.

\par% p
2. El presente Protocolo no se aplicará a las situaciones de tensiones internas y de disturbios interiores, tales como
 los motines, los actos esporádicos y aislados de violencia y otros actos análogos, que no son conflictos armados.
Aquí
 hay una parte clara: hay algo que se está desarrollando en el territorio de una Alta Parte contratante, es decir en el
 territorio colombiano, y las fuerzas armadas constitucionales de la nación. Por más que se quiera torcer el
 significado de las cosas en el momento en el que fuerzas armadas con mando en el estado colombiano constitucional e
 internacionalmente reconocido ejerzan acciones armadas en contra de algo dentro del territorio colombiano, se cumple
 esta parte de la definición.

Donde hay más espacio para la interpretación es en la definición de ese algo contra lo cual las fuerzas armadas
 constitucionales ejercen acciones armadas. El Protocolo II es explícito en determinar algunos puntos donde no aplica
 (p. ej. motines) y se sobreentiende que no aplica al uso de las armas del estado para combatir a la delincuencia común
 (organizada o no).

\par% p
El uribismo insiste en que las \begin{abbr}
Farc
\end{abbr}
 y el \begin{abbr}
Eln
\end{abbr}
 no cumplen las condiciones de esta definición y como tales son asimilables a delincuencia, y por ello no existe
 conflicto armado interno.

\par% p
Hay dos razones prácticas de la negación del conflicto. La una está al interior mismo del Protocolo II, Título IV,
 Artículo 13, numeral 3:

\begin{blockquote}
3. Las personas civiles gozarán de la protección que confiere este Título, salvo si participan directamente en las
 hostilidades y mientras dure tal participación.
\end{blockquote}

Básicamente esto significa que si la población civil colabora con las fuerzas armadas bajo mando del estado
 colombiano, quedarían desamparadas de la protección legal que confiere el Protocolo II frente a las fuerzas
 disidentes. (No es que sirva de mucho porque igual las \begin{abbr}
Farc
\end{abbr}
 no reconocen el Protocolo II ni las protecciones que de ahí emanan.) E, igualmente, si los civiles colaboran con la
 delincuencia no podrían ser procesados como cómplices.

\par% p
La segunda razón es que el Protocolo II niega la intervención de otras naciones. La no intervención, por un lado,
 impide que otro país apoye abiertamente a la disidencia armada y, sobre todo, utilice sus propias fuerzas en apoyo de
 esta disidencia. Pero, por otro lado, el conflicto armado interno debe ser interno y como tal los otros países no
 están obligados a resolver este problema interno. Los miembros de grupos delincuenciales pueden ser apresados en otro
 país y extraditados para que los mecanismos judiciales tengan efecto. Los miembros de las disidencias armadas pueden
 apelar al derecho al asilo.

\section{El concepto de la paz}
Tratemos de no confundir el objetivo con el medio. El artículo 22 de la Constitución Política de Colombia, emitida en
 1991 y en vigor desde entonces, dice que la Paz es un derecho y un deber de obligatorio cumplimiento. Es por lo tanto
 inconstitucional no buscar la Paz. Pero las así llamadas negociaciones de paz no son la Paz. El estado no está
 obligado a buscar una mesa de negociación con las \begin{abbr}
Farc
\end{abbr}
 y el \begin{abbr}
Eln
\end{abbr}
 y menos si considera que tal negociación prolongará la guerra antes que lograr la Paz.

Pero hablo aquí de la paz como si estuviéramos de acuerdo en qué es. La Constitución no la define.

Creo que los conflictos son inevitables. Incluídos los conflictos sociales. La paz no sería la ausencia de conflictos
 sino que esos conflictos no pretendan resolverse mediante el uso de violencia directa. Actualmente la subversión usa
 la violencia directa (ataques terroristas, enfrentamientos con el ejército y la policía, secuestros, amenazas y
 extorsión) como método para lograr sus intereses (altruistas o no) y el estado constitucional a través de sus fuerzas
 armadas usa violencia directa para contener a la subversión.

En su forma más restringida, la paz que queremos muchos colombianos, es el fin de esa violencia directa que produce la
 subversión, entendiendo que la violencia directa del estado contra la subversión es una consecuencia de la primera.

Hay otras formas de violencia, como es la violencia estructural, por ejemplo cuando una clase social dirigente en
 preservación de sus propios intereses desconoce los derechos e intereses básicos de una clase social oprimida. Para
 algunos la paz no sólo debe incluir el cese de la violencia directa de la subversión sino la eliminación de estas
 violencias estructurales.

Pero para continuar con el análisis limitémonos a la paz restringida: el cese de hostilidades de la subversión contra
 el estado y el consecuente cese de hostilidades del estado contra la subversión.

Y regresemos al principio que expuse al comienzo. El estado actual de hostilidades entre la subversión y el estado
 limita mi ejercicio de derechos y libertades civiles e, incluso, amenaza mis derechos humanos (y cuando hablo de mis
 derechos me arrogo hablar por cada uno de los colombianos). La paz, entendida como el cese permanente de hostilidades
 entre estas dos partes, es, por lo tanto, un estado deseable.

Esta paz, sin embargo, no puede obtenerse a cualquier costo. No puede obtenerse al costo de vulnerar nuestros derechos
 humanos ni los derechos y libertades civiles más de lo que ya están siendo vulnerados por el estado actual de
 hostilidades.

\par% p
Esto significa que hay dos escenarios de acabar el conflicto que para mí son peores que el conflicto mismo:

\begin{enumerate}

\item Una victoria militar a cualquier costo.
\item Una negociación donde los principios comunistas que se oponen a mis principios liberales sean impuestos.

\end{enumerate}

\section{Sobre cómo acabar la guerra}
Hay cuatro formas tradicionales de terminar una guerra o un conflicto armado:
\begin{enumerate}

\item Una negociación entre las partes donde estas busquen lograr sus propios objetivos por medio del diálogo y donde los
 intereses cedidos o renunciados no sean interpretados como una derrota de ninguna de las partes. Llamaremos a esto una
 paz negociada.
\item Una negociación donde una de las partes se reconozca derrotada pero condiciona el cese de hostilidades a ciertos
 requisitos y garantías. Llamaremos a esto una rendición condicionada.
\item El reconocimiento de una de las partes a que la continuación de la lucha es más gravosa que someterse a la merced y
 voluntad de la otra parte y, en consecuencia, se rinden sin condiciones. Esto se llama, en consecuencia, rendición
 incondicional.
\item El agotamiento de las partes en la vía armada, bien sea por la eliminación física de los combatientes de un bando, por
 desbandamiento, o porque los combatientes de uno o ambos bandos pierden interés en continuar acciones hostiles. Salvo
 que una de las partes haya sido completamente eliminada, siempre podría haber discusión sobre si la parte más débil al
 final de este conflicto fue realmente derrotada.

\end{enumerate}

En el caso del enfrentamiento del estado colombiano contra las \begin{abbr}
Farc
\end{abbr}
, el primer escenario implica que un estado que no está derrotado busca acuerdos con una subversión que no se admite
 derrotada. En el actual equilibrio de fuerzas que favorece al estado constitucional sobre la subversión, las
 concesiones que el estado haga a los intereses de las \begin{abbr}
Farc
\end{abbr}
 serán considerados por una parte de la opinión pública como una entrega del estado al chantaje de una subversión
 ilegítima.

\par% p
En los demás escenarios asumiré que son las \begin{abbr}
Farc
\end{abbr}
 la parte derrotada porque no veo en un futuro más o menos cercano que la relación de fuerzas sea tal que las \begin{abbr}
Farc
\end{abbr}
 logren una victoria militar.

\par% p
El segundo escenario es muy similar al primero, salvo que las \begin{abbr}
Farc
\end{abbr}
 reconocerían que la vía armada se agotó y que la negociación tiene como objetivo principal lograr garantías en la
 demovilización. Sería una guerrilla que no pide condiciones tales como una reforma agraria o la revisión de la
 política internacional colombiana como condición para entregar las armas, sino que se enfocaría en puntos como
 garantías de debido proceso y disminución o suspensión de penas.

En el caso del M-19 y más adelante del EPL y otros grupos subversivos, hubo una negociación con términos favorables
 para los combatientes que depusieron las armas pero no hubo concesiones directas del estado con respecto a su
 política. La Asamblea Constituyente de 1991 no fue un requisito previo de las guerrillas desmovilizadas sino más bien
 una oportunidad coyuntural. Hasta qué punto fue una paz negociada o una rendición condicionada es debatible.

\par% p
En la actualidad (2012), es difícil pensar que si las \begin{abbr}
Farc
\end{abbr}
 llegan a la negociación como parte derrotada obtenga las mismas garantías que el M-19 en 1990.

\par% p
El tercer escenario: la rendición incondicional, no sería completamente incondicional. Las \begin{abbr}
Farc
\end{abbr}
 no se estarían sometiendo a la merced y voluntad de un estado victorioso sino a las garantías de un estado de derecho
 y unas leyes que limitan las penas. Si esto se lleva a cabo pronto, lo más probable es que sea porque las \begin{abbr}
Farc
\end{abbr}
 han acordado la existencia previa de leyes favorables.

Hay un antecedente reciente en Colombia. La desmovilización de las Autodefensas Unidas de Colombia y otros grupos
 ilegales antiinsurgentes, en las cuales no exigieron garantías más allá de la aplicación de leyes vigentes, incluyendo
 una ley hecha a la medida y que extendía las garantías procesales.

\par% p
Por otro lado, si el estado y sus fuerzas armadas constitucionales se imponen con mayor contundencia sobre las \begin{abbr}
Farc
\end{abbr}
, puede llegar el momento en el que los líderes que entonces sobrevivan decidan entregarse al estado dentro de las
 leyes vigentes.

\par% p
Dentro del cuarto escenario hay varias formas. Una es la eliminación completa y física de las \begin{abbr}
Farc
\end{abbr}
, bien porque el estado libre una guerra sin cuartel (que sería una violación al Protocolo II) hasta que el último
 guerrillero esté preso o muerto, bien porque ningún líder de las \begin{abbr}
Farc
\end{abbr}
 toma la decisión de rendirse. Una escenario más probable es la desbandada: en algún punto los frentes sobrevivientes
 de las \begin{abbr}
Farc
\end{abbr}
 pierden la unidad de mando y renuncian individualmente a proseguir una lucha contra el estado limitándose a sus
 propios intereses criminales. Si esta modalidad de crimen es de relativo bajo impacto eventualmente el estado
 cambiaría la actual estrategia militar a unas acciones más de tipo policivo.

\par% p
La otra manifestación del agotamiento sería que las \begin{abbr}
Farc
\end{abbr}
, todavía con cierto nivel de mando unificado, renuncie unilateralmente a la ofensiva. Sería una guerrilla casi
 derrotada pero que no se atreve a entregarse a la justicia sino simplemente a resistir pasivamente, a esconderse. Si
 logran resistir lo suficiente en esta situación la ofensiva militar del estado será en algún momento insostenible. No
 estoy hablando aquí de un cambio estratégico de las \begin{abbr}
Farc
\end{abbr}
 de esperar el agotamiento de la acción militar del estado para retomar su lucha, sino el abandono permanente de esta
 lucha.

\par% p
Podría darse también (aunque no lo veo políticamente viable), que sea el estado el que abandone primero la lucha, o que
 concentre la actual ofensiva contra las \begin{abbr}
Farc
\end{abbr}
 en una estrategia netamente defensiva y reactiva enfocada a la prevención del terrorismo. Pero teniendo en cuenta la
 historia de las \begin{abbr}
Farc
\end{abbr}
, no veo que estas terminen aceptando una situación de empate virtual sino que deseen retomar la lucha.

\section{Escenario ideal}
Ya había planteado que de acuerdo a mis principios hay escenarios indeseables. Uno es la prolongación indefinida del
 estado de hostilidades. De cuando en cuando las partes cambian de estrategia lo que les puede dar una ventaja puntual.
 Al comenzar la doctrina de seguridad democrática las \begin{abbr}
Farc
\end{abbr}
 se replegaron y tuvieron un golpe grande cuando la doctrina fue reelegida porque esto acabó con sus predicciones de
 resistir un cuatrenio y retomar su rumbo. Pero, y a pesar de los golpes grandes que recibió durante el segundo
 cuatrenio de la seguridad democrática, hacia el final del gobierno de Uribe las \begin{abbr}
Farc
\end{abbr}
 habían retomado cierta ofensiva de carácter terrorista. Personalmente no creo que Santos haya aflojado en el esquema
 de seguridad democrática sino que las \begin{abbr}
Farc
\end{abbr}
 han logrado cambiar su estrategia y parecer más contundentes de lo que fueron entre 2006 y 2009.

\par% p
En una prolongación del conflicto eventualmente una de las partes comete un error grave en la reformulación de la
 estrategia. La situación actual le permite al estado cometer más errores graves sin que se conviertan en derrota. Con
 suerte, para una pronta resolución del conflicto, las \begin{abbr}
Farc
\end{abbr}
 cometerían pronto un error suficientemente grave que las lleve a buscar una rendición o un abandono de la lucha.

Una victoria militar a cualquier costo, de parte del Estado contra la subversión, lo considero también poco deseable
 por lo de “a cualquier costo”. A cualquier costo significaría la suspensión de derechos y libertades civiles de todos
 los ciudadanos y actualmente garantizados en la constitución. Esto puede acelerar la victoria militar del estado, pero
 significaría que el estado se convirtió en algo indeseable desde mi punto de vista.

Una victoria militar del estado dentro de la constitución y las leyes es, desde los principios que he planteado, una
 opción deseable siempre y cuando sea pronta. Buscar la victoria militar sin conseguirla constituye una prolongación
 del conflicto.

La victoria militar puede darse en forma de rendición condicionada, rendición incondicional o simple abandono de la
 lucha o eliminación física de la otra parte, sin entrar a detallar cuál de esas alternativas es la más deseable.

Una victoria de la guerrilla la considero indeseable por el modelo de estado que históricamente ha planteado el
 comunismo y que va en contra de las libertades y derechos civiles que defiendo.

Entonces llegamos al caso de la paz negociada. En principio no considero que ni la paz negociada ni la victoria militar
 sean intrínsecamente malas o buenas. Así como veo buena una victoria militar del estado pronta y encausada dentro de
 la constitución y mala una victoria militar por parte de un estado que viole sistemáticamente los derechos y
 libertades civiles, hay negociaciones buenas y negociaciones malas.

Si las negociaciones de paz no garantizan al menos esa paz restringida vista como el fin de las hostilidades entre las
 fuerzas del estado y la subversión, sino que antes ayudan a prolongar la guerra, entonces no estoy de acuerdo.

Pero incluso una negociación que exitosamente logre la paz sería indeseable si hay un detrimento en los principios que
 defiendo.

Dentro de la paz negociada veo cosas que considero contrarias a mis principios, cosas que considero simplemente feas,
 cosas que me son aceptables y cosas que yo mismo espero.

\par% p
En contra de mis principios estaría que las \begin{abbr}
Farc
\end{abbr}
 impusieran limitaciones a la libertad de empresa o a la libertad de expresión como parte de la negociación.
 Definitivamente inaceptable que las \begin{abbr}
Farc
\end{abbr}
 exigieran la pena de muerte a los opositores de la solución negociada u otras suspensión de los derechos humanos (pero
 se me haría absurdo que el gobierno de Santos siquiera permita que se discuta algo así).

\par% p
Feo vería que las \begin{abbr}
Farc
\end{abbr}
 exijan cambios radicales en la política internacional o que Timochenko sea nombrado senador o ministro sin siquiera
 someter su nombre a las urnas. Sería feo pero no inaceptable. La impunidad frente a crímenes atroces sería fea pero no
 inaceptable. Si estas cosas feas permiten que los colombianos podamos continuar con mejores vidas no me opondré.

Apenas aceptable serían penas cortas y suspendidas a los líderes guerrilleros no directamente involucrados en crímenes
 atroces y la integración de los combatientes rasos a programas de reinserción.

\par% p
Lo que esperaría como deseable: que dentro de las negociaciones que efectivamente lleven a la paz, se dé la oportunidad
 de plantear un modelo de sociedad menos excluyente y que la firma de la paz sirva de pretexto para implementarlo.

\section{Predicciones}
Mi hijo está entrando a primer grado en el colegio. En once años saldrá bachiller a una edad apta para prestar el
 servicio militar obligatorio. Durante este tiempo espero que él pueda tomar una decisión libre y autónoma y que si se
 decide por tomar las armas del estado no sea ante la perspectiva de matar a nombre del estado para defender a unos
 colombianos de la demencia de otros colombianos que pretenden ser insurgentes.

Soy medianamente optimista de que en once años la actual guerra no será determinante en la decisión que tomará mi hijo.

Pero soy mucho menos optimista en creer que la paz se va a lograr dentro de los dos años que le quedan de gobierno al
 Presidente Santos.

Creo que tanto la paz negociada como la victoria militar del estado son posibles, pero ambas son difíciles. Creo que de
 optar el estado por una solución netamente militar estaríamos perdiendo una oportunidad de dar un fin más próximo a la
 guerra actual, como también creo que una negociación de paz sin perspectivas reales ayudará más a prolongar la guerra
 que a acabarla.

\par% p
Las \begin{abbr}
Farc
\end{abbr}
 están bastante golpeadas pero aún no están derrotadas y siguiendo su patrón histórico van a querer mostrar fortaleza
 frente a una negociación. Si esa fortaleza la pretenden a punta de terrorismo harán que sus pretensiones en la mesa de
 negociación sean menos aceptables por la mayoría de ciudadanos votantes de Colombia. Creo que los mandos de las \begin{abbr}
Farc
\end{abbr}
 podrían mostrar mayor fortaleza sosteniendo un alto el fuego unilateral que recurriendo al terrorismo. Pero no lo van
 a hacer.

\par% p
Si los colombianos votantes no apoyan mayoritariamente las concesiones que el gobierno haga a la guerrilla, el gobierno
 no estará en capacidad de conceder. Esto debilita al gobierno quien quedaría atrapado por la opinión pública crítica
 del proceso por un lado y la necesidad de mostrar resultados por el otro. Y las \begin{abbr}
Farc
\end{abbr}
 van a querer aprovechar este debilitamiento para aumentar sus pretensiones.

En conjunto esto significara o bien un nuevo fracaso en la búsqueda de la paz negociada, con la consiguiente
 prolongación de la guerra, o una paz mal hecha, que si bien no será germen de una nueva guerra si desatará una nueva
 ola de violencia.

\par% p
Creo que el modelo de estado que tiene Juan Manuel Santos en la cabeza es bueno, pero también tiene una tendencia de
 querer complacer a casi todo el mundo. Soy algo pesimista de que el gobierno de Santos sea el interlocutor adecuado
 frente a una negociación con las \begin{abbr}
Farc
\end{abbr}
. Me temo que se va a dejar enredar entre el juego de las \begin{abbr}
Farc
\end{abbr}
 de obtener más concesiones usando el terrorismo como muestra de fortalecimiento y el de la derecha política que
 insistirá en ver las pretensiones de las \begin{abbr}
Farc
\end{abbr}
 y su recurso terrorista como muestras del fracaso de la vía negociada.

\par% p
Por otro lado creo que Timochenko en 2012 puede ser más serio de lo que fue Marulanda en 1998. Las \begin{abbr}
Farc
\end{abbr}
 están duramente golpeadas, mientras que en 1998 había una percepción de fortalecimiento. Que sea más serio no quiere
 decir que esta vez sí haya esperanza porque las \begin{abbr}
Farc
\end{abbr}
 siguen mostrando esa política de querer llegar fortalecidas a la mesa de negociación y querer mostrar la fortaleza con
 terrorismo.

Ojalá me equivoque en mi percepción del carácter de Santos, porque un gobierno con un proyecto claro que logre defender
 es la mejor garantía que tenemos los colombianos de llegar a una solución pronta a esta guerra.

Pero no creo que esta sea la última oportunidad tampoco.

\chapter{Conflicto de libertades}
\begin{metadata}
	Published by \anchor[chlewey]{chlewey} on \anchor[http://ewey.co/B1314]{Fri, 31 Aug 2012 19:54:59 +0000}\\
	\categories{libertad-de-expresion, opinion, pussy-riot}\\
	Shorthand: \anchor[http://blog.chlewey.net/2012/08/conflicto-de-libertades/]{conflicto-de-libertades}
\end{metadata}

Supón que estás en la biblioteca aprovechando el silencio para expandir tu mente con la lectura o tus divagaciones.
 De repente aparece un grupo de personas gritando improperios; o puede que no sean improperios: simplemente gritando algo que no te es relevante o intrigante.
 Es muy probable que te ofendas pues irrumpieron en tu concentración y en tu silencio donde esperabas encontrar tal
 silencio.

¿Y si no fuera en una biblioteca donde esperas el silencio para conversar con un libro sino en el sitio donde esperas
 el silencio para hablar con tu concepto de ser superior?

\begin{iframe}% {'src': 'http://www.youtube.com/embed/ALS92big4TY', 'style': 'float: right;', 'height': '169', 'frameborder': '0', 'width': '300'}

\end{iframe}
El \emph{performance} de las integrantes de Pussy Riot en el Templo Catedralicio del Cristo Redentor del Patriarca de Moscú no fue sólo un acto de expresión artística y política: fue un acto deliberadamente ofensivo.
 En el video del acto se ve la cara de consternación de los asistentes a la catedral. (0:12)

\par% p
¿Tiene la libertad de expresión límites?
 Lo contestaría como un sí y no.
 Creo que todos debemos tener la libertad de decir lo que queramos y como queramos, así sea política o factualmente incorrecto.
 Si yo quiero decir que simpatizo con Hitler, o con las FARC, o con las corridas de toros o con la caza de ballenas
 para \emph{fines científicos}, debo poder decirlo.
 O si quiero invocar públicamente a la Virgen María para que saque a Putin del poder, debo poder expresarlo.
 Si quiero masturbarme en la Plaza de Bolívar para demostrar mi desacuerdo con los políticos, eso es parte de la
 libertad de expresión.

Pero poder expresar algo libremente no nos libra de las consecuencias sociales, morales o legales de lo que expresemos.
 Como con toda libertad habrá un momento en el que esta entra en conflicto con las libertades y derechos de los demás.
 Mi expresión puede molestar, ofender o dañar reputaciones.
 Mi expresión puede atentar incluso contra la vida humana bien porque puedo causar pánico en una multitud o puedo
 inspirar un crimen de odio.

Conozco muchas personas, y me incluyo entre ellas, que insisten en que una moralidad religiosa no debe imponerse como moral única de una sociedad laica de corte liberal, y uno de los casos en los que nos expresamos es a defender el derecho a la autonomía sobre nuestros cuerpos.
 La autonomía de decidir si tenemos sexo consensuado con otros adultos.
 La autonomía de decidir si queremos alterar nuestras mentes con substancias psicoactivas como el tabaco, el alcohol o la mariguana.
 La autonomía de decidir si queremos o no una muerte pronta frente a una enfermedad terminal dolorosa y onerosa.

Incluso si estamos convencidos de que la religión es una ficción y un método de control mental de las masas, debemos reconocer que una persona es tan libre de optar por tal ficción y control como lo es de fumarse un porro o de aspirar una línea de cocaína.
 La libertad de cultos es un derecho social tan primario como la libertad de expresión o la libertad de decidir sobre nuestros cuerpos.
 Y una verdadera libertad de culto no se basa en permitir que cada uno de nosotros crea o no, sino en la libertad de
 practicarlo y el respeto de los demás al sentimiento religioso de cada uno.

Las Pussy Riot no sólo expresaron su opinión sobre Vladímir Putin.
 Ellas ofendieron a una comunidad religiosa.
 Ese tipo de ofensas tiene distintas consecuencias en diferentes países y vertientes políticas.
 Si ellas hubieran hecho lo mismo en una mesquita en un país de mayoría musulmana muy probablemente no hubieran llegado vivas al juicio.
 Si lo hubieran hecho en la Catedral Primada de Bogotá, tal vez las hubieran arrestado por desorden público pero no habrían sido condenadas a prisión.
 Probablemente ni les hubieran levantado cargos.

\anchor[http://blog.chlewey.net/wp-content/uploads/2012/08/ASeHrsj1484.jpg]{\begin{wrapfigure}{r}{300\px}\centering% {'src': 'http://blog.chlewey.net/wp-content/uploads/2012/08/ASeHrsj1484-300x241.jpg', 'title': 'ASeHrsj1484', 'height': '241', 'width': '300', 'alt': '', 'class': ['alignright', 'size-medium', 'wp-image-1318']}
\includegraphics[width=300\px,height=241\px]{blog/ASeHrsj1484-300x241.jpg}
\end{wrapfigure}
}Aun recuerdo a mediados de los años 1980 un escándalo porque un fotógrafo tomó unos desnudos en la Catedral de Sal de Zipaquirá.
 Aun bajo el concordato y la constitución de 1886 el fotógrafo Ángel Becassino y la modelo Flor Alba Devia fueron
 arrestados por apenas unos pocos días sin que hubiera una condena real en el juicio subsiguiente.

En este aspecto prefiero vivir en Colombia que en tal país musulmán o en Rusia.
 Porque si bien las Pussy Riot no son inocentes palomitas, su acto de expresión artística y política fue una grave ofensa a un sentimiento religioso pero no fue una incitación al odio.
 A menos que esas chicas lo que hayan incitado fue el odio hacia ellas y, de carambola, el odio de la sociedad
 internacional hacia la Rusia de Putin.

En mi opinión ese acto de expresión no debió haber quedado sin consecuencias, pero dos años en una colonia penal me
 parece exagerado para un acto que no afectó más allá de una susceptibilidad religiosa.

Ahora.
 No nos razguemos las vestiduras frente al tiránico régimen ruso de Vladímir Putin.
 Tal vez aquí no detangan a una punketas por saltar con pasamontañas dentro de una catedral, pero nuestros regimenes democráticos y occidentales con cierta frecuencia exageran ciertas manifestaciones para callar a los otros.
 No en vano el término “sicario moral” es de amplio uso de nuestros políticos para referirse a la prensa.

En últimas lo que estaba en juego para Putin no era proteger el derecho de los cristianos ortodoxos rusos a practicar
 su oración y meditación en paz sino mostrar su fortaleza.

\chapter{La inocencia de los musulmanes}
\begin{metadata}
	Published by \anchor[chlewey]{chlewey} on \anchor[http://ewey.co/B1324]{Wed, 19 Sep 2012 16:16:39 +0000}\\
	\categories{islam, opinion, religion}\\
	Shorthand: \anchor[http://blog.chlewey.net/2012/09/la-inocencia-de-los-musulmanes/]{la-inocencia-de-los-musulmanes}
\end{metadata}

\begin{wrapfigure}{r}{300\px}\centering% {'width': '300', 'align': 'alignright', 'id': 'attachment_1327'}
\anchor[http://blog.chlewey.net/wp-content/uploads/2012/09/innocence-of-muslims-0912-1.jpeg]{\includegraphics[width=300\px,height=200\px]{blog/innocence-of-muslims-0912-1-300x200.jpeg}} (via InterAksyon)
\end{wrapfigure}

Así se titula una supuesta película (y digo supuesta porque no tengo evidencia de que existan más de los 15 minutos filtrados por Internet) cuyo objetivo es mostrar que la sumisión del islam (lo cual es un pleonasmo: islam significa sumisión y musulmán significa sumiso) es lo que convierte a los hombres en fanáticos.
 Como parte del argumento pretende mostrar una vida de Mahoma, el profeta de la sumisión quien tomó elementos del judaísmo y el cristianismo que convivía con el paganismo de la península arábiga cuando vivió.
 Esta representación de Mahoma es en mi opinión ofensiva y de mal gusto, como toda la película en general que dista de la calidad de una cinta hollywoodense.
 (No que el cine de Hollywood sea de alta calidad, pero al menos suele ser de buena factura.)

\par% p
No tengo idea si el Mahoma de \emph{La inocencia de los musulmanes} esté basado en alguna investigación más o menos seria o si no es más que inventos del guionista y el tratamiento del director.
 Si tiene una base seria sin duda el tratamiento fílmico lo convierte en una caricatura lo que añade a la afrenta del
 sólo hecho de mostrar una imagen humana del Profeta (del cual ni siquiera debe haber imágenes).

\begin{wrapfigure}{r}{300\px}\centering% {'width': '300', 'align': 'alignright', 'id': 'attachment_1325'}
\anchor[http://de-avanzada.blogspot.com/2012/09/inocencia-musulmanes.html]{\includegraphics[width=300\px,height=225\px]{blog/inocencia-01-300x225.jpg}} (via de-avanzada.blogspot.com)
\end{wrapfigure}

Yo confieso que no fui capaz de ver siquiera la mitad de esos 15 minutos que han circulado por Internet, simplemente porque como película es mala.
 Lo que me lleva a rechazar aún más la reacción de los fanáticos sumisos en contra de las delegaciones diplomáticas de los EE.UU. y de otros países “occidentales”.
 Claramente ese tipo de películas no corresponden a una política del gobierno estadounidense contra el Islam.
 No es una afrenta del pueblo de los EE.UU. contra la religión de la sumisión.
 Es la obra de un productor independiente, de algo así como un youtubero, con una opinión.
 Opinión que no por ser pésimamente producida puede ser prohibida o censurada dentro de la legislación de los EE.UU.

¿Qué significa en últimas esto?
 Que estas recientes protestas del fundamentalismo musulmán contra los EE.UU. son realmente una protesta del
 fundamentalismo religioso en contra el concepto de la libertad de expresión.

Y lo irónico del asunto es que estas protestas le están dando la razón a lo que los autores de La inocencia de los
 musulmanes acusan: el islam enceguece, impide pensar libremente y reacciona fanática y violentamente contra el libre
 diálogo.

Una obscura película de bajo presupuesto de la cual nadie se hubiera enterado termina siendo promocionada por los
 mismos que pretenden acallarla y termina dándole la razón a sus productores.

\horrule{}

Mientras terminaba de redactar esto me metí en un intercambio twittero sobre la pertinencia de algún medio francés de
 publicar unas caricaturas de Mahoma.

\par% p
Por un lado no conviene torear a los fanáticos y en el estado actual de las cosas publicar caricaturas del Profeta es
 sin duda un acto provocador e irresponsable (\anchor[https://twitter.com/nicolinico/status/248425661906161664]{infantil} dijo mi interlocutor).
 Por otro lado ¿por qué hemos de plegar nuestros principios civiles tales como la libertad de expresión al veto de
 fanáticos?

\begin{wrapfigure}{l}{300\px}\centering% {'width': '300', 'align': 'alignleft', 'id': 'attachment_1326'}
\anchor[http://tumblr.chlewey.net/post/31041910145/es-bueno-tener-ideas-fuertes-y-definidas-por]{\includegraphics[width=300\px,height=120\px]{blog/mas-principios-300x120.png}} (por Chlewey)
\end{wrapfigure}

Creo que mis principios y la defensa de ellos no debe plegarse a quienes lo amenazan sólo porque son más peligrosos que uno.
 Sin duda existe la autocensura.
 Sin duda en muchas ocasiones vale más mi vida y la de los míos que la defensa de unos principios abstractos y que la necesidad de someter esos principios a prueba.
 Pero si tú, yo o cualquier otra persona tenemos algo que decir, tenemos el derecho moral de poderlo hacer.
 Es éticamente correcto hacerlo.
 No somos los culpables ni los responsables éticos o jurídicos de las acciones que nuestras palabras generen en los fanáticos.
 Incluso si no tenemos que decirlo sino que simplemente queremos hacerlo, bien para probar si nuestros gobiernos son
 serios al garantizar nuestra libertad, bien porque queremos evidenciar una reacción.

\chapter{La educación sexual prohibida}
\begin{metadata}
	Published by \anchor[chlewey]{chlewey} on \anchor[http://ewey.co/B1333]{Fri, 21 Sep 2012 05:26:33 +0000}\\
	\categories{aborto, educacion, opinion}\\
	Shorthand: \anchor[http://blog.chlewey.net/2012/09/la-educacion-sexual-prohibida/]{la-educacion-sexual-prohibida}
\end{metadata}

\begin{wrapfigure}{r}{300\px}\centering% {'width': '300', 'align': 'alignright', 'id': 'attachment_1334'}
\anchor[http://borderlessnewsandviews.com/2012/03/im-pro-choice-not-pro-abortion/]{\includegraphics[width=300\px,height=202\px]{blog/pro-life-vs-pro-choice-300x202.png}} Vía “I’m Pro-Choice, Not Pro-Abortion” en Borderless News and Views
\end{wrapfigure}

\par% p
En ocasión de la discusión sobre el aborto, esta nos divide en dos bandos: quienes defienden la vida desde la
 concepción (llamados en inglés \emph{Pro-Life} o pro-vida) y quienes defienden el derecho de la mujer a decidir sobre su útero (en inglés los \emph{Pro-Choice} o pro-escogencia). Hay, desde luego, matices y puntos medios, como habrá personas que no han tomado una postura
 definida bien por apatía bien por el reconocimiento de lo complejo del tema.

Muy probablemente por mi educación cristiana siento que existe algo especial en toda vida humana y por ello no me gusta
 el concepto del aborto como método de control natal. Pero que no me guste no significa que tengo que imponer mi
 criterio sobre los demás.

Cuando escribo esto sigue vigente en Colombia la despenalización de lo que se conoce como aborto terapéutico frente a
 la inviabilidad del producto de la concepción, el peligro a la vida de la madre y el fruto de la violación o la
 inseminación artificial no consentida. Con sus respectivas reservas no considero estos casos como métodos de control
 natal. Por otro lado sí creo en el concepto de control natal cuando esta es una decisión de la mujer o de la mujer
 junto con la pareja que ella libremente escoja. La contracepción previa, o en casos excepcionales la contracepción de
 emergencia son, en mi concepto, métodos válidos de planificación. Como es también válida la abstención libre y
 voluntaria.

Mi educación cristiana y el hecho de que vivo en un país mayoritariamente cristiano (y mayoritariamente católico dentro
 del cristianismo) me hará referirme sin duda a lo que dice la biblia y la Iglesia Católica al respecto. La Iglesia
 tiene una postura de que el sexo sólo es válido dentro del matrimonio y el matrimonio (y el sexo dentro del
 matrimonio) tienen por objeto la procreación. Muchos grupos cristianos, incluyendo varios grupos católicos, promueven
 entonces que la única educación sexual válida es aquella que insiste en la abstención antes del matrimonio. Todo lo
 demás traerá una serie de problemas comenzando por la banalización del sexo, la cosificación de las personas, los
 embarazos no deseados y las enfermedades de transmisión sexual (ETS).

Hay, desde luego, fanáticos que creen que las ETS son un castigo de Dios, como también hay personas muy educadas y que
 viven su fe en comunión con un escepticismo científico y sin fanatismos que no asignan a Dios las ETS sino a las malas
 prácticas sexuales. En el mundo ideal predicado por estas personas, un mundo donde todos nos abstenemos del sexo antes
 de casarnos y luego permanecemos fieles dentro de una institución monogámica, muchos de los problemas asociados con el
 sexo no se presentarían. Una recíprocamente única pareja sexual de por vida nos previene de las ETS y el sexo como
 valoración de la pareja y continuación del diálogo y la intimidad puede ser más enriquecedor si lo reservamos al único
 ser amado.

\anchor[http://blog.chlewey.net/wp-content/uploads/2012/09/condoms\_1.jpg]{\begin{wrapfigure}{r}{300\px}\centering% {'src': 'http://blog.chlewey.net/wp-content/uploads/2012/09/condoms_1-300x300.jpg', 'title': 'condones', 'height': '300', 'width': '300', 'alt': '', 'class': ['alignright', 'size-medium', 'wp-image-1336']}
\includegraphics[width=300\px,height=300\px]{blog/condoms_1-300x300.jpg}
\end{wrapfigure}
}Bueno, lo primero creo que es demostrable, lo segundo no estoy seguro.

De todas formas los métodos anticonceptivos y profilácticos no son 100\% confiables. Depender de ellos bien puede
 generar una falsa sensación de seguridad que no previene en últimas las ETS ni los embarazos no deseados.

Hay dos problemas que tiene insistir en una educación sexual de sólo abstinencia. El primero es de índole práctica: hay
 grupos que se oponen a cualquier tipo de educación sexual que no sea de sólo abstinencia, y esta postura demora la
 implementación de programas de educación sexual. El segundo problema va al concepto mismo de lo que es la educación.
 El modelo de educación actual basado en currículos es defectuoso. A muchos nos sirvió, tal vez, para obtener un título
 universitario y, probablemente, obtener las bases para iniciar un oficio profesional. Pero el modelo también excluye
 porque no todos los alumnos aprehenden el currículo con el mismo nivel de entendimiento y no aprenden más que a
 memorizar para el examen. Por bien intencionado y bien logrado que se haga un currículo de educación sexual basado en
 la abstinencia muchos alumnos no comprenderán este y quedarán sin la protección de tal supuesta abstinencia.

Luego es muy fácil culpar a nuestros jóvenes. Culpar a los medios de comunicación que promueven antivalores familiares
 y la cultura del libertinaje. Culpar a nuestras adolescentes por andar más pendientes de los placeres sexuales que de
 su propia responsabilidad. Entonces, como ellas son culpables de quedar embarazadas, como ellas se lo buscaron, ahora
 que no vengan con el cuento de que el estado debe resolverles el problema matando a un bebé no nacido.

\begin{wrapfigure}{r}{300\px}\centering% {'width': '300', 'align': 'alignright', 'id': 'attachment_1338'}
\anchor[http://www.google.com.co/url?sa=t\&rct=j\&q=\&esrc=s\&source=web\&cd=2\&cad=rja\&ved=0CDAQtwIwAQ\&url=http\%3A\%2F\%2Fwww.youtube.com\%2Fwatch\%3Fv\%3DZ78aaeJR8no\&ei=RfhbUMCLO4je8ATS44GQDA\&usg=AFQjCNE9luWGC3se5mks5JrVHIfh0fcNTQ\&sig2=4YXuLdUkAe7aYtZrJBLbHw]{\includegraphics[width=300\px,height=168\px]{blog/batches-300x168.jpg}} Sir Ken Robinson: “aun educamos a nuestros niños por lotes” en Changing Education Paradigms
\end{wrapfigure}

\begin{wrapfigure}{r}{300\px}\centering% {'width': '300', 'align': 'alignright', 'id': 'attachment_1337'}
\anchor[http://www.educacionprohibida.com/]{\includegraphics[width=300\px,height=168\px]{blog/captura-07-300x168.jpg}} Imagen de “La educación prohibida”
\end{wrapfigure}

Ya sabemos que el modelo educativo basado en currículos es defectuoso. El modelo de educación pública, universal y
 gratuita, con programas curriculares y separación de los niños por edades es un modelo conveniente para quienes
 manejan el estado, conveniente para quienes manejan al proletariado, pero no es conveniente para educar a nuestros
 niños. Ese modelo creado en Prusia a finales del siglo XVIII tiene varias fallas conceptuales y ya muchos educadores
 han venido mostrándolas y replanteando el modelo desde hace décadas. Pero nuestros políticos, nuestros curas, nuestros
 procuradores, asumen todavía que ese es el modelo.

Todos nuestros jóvenes son educados sexualmente. La educación la reciben de sus padres, de sus primos y hermanos
 mayores, de la televisión e Internet, de los maestros en la escuela y de los programas curriculares entre muchas otras
 fuentes. Nuestro aprendizaje del lenguaje o de las ciencias naturales, incluso de la matemática y le historia, es en
 gran medida vivencial. El currículo es una guía para medir logros a distintas edades. Imponer un currículo como fuente
 única del saber académico crea un divorcio entre el aprendizaje vivencial y la escolaridad. Y con la educación sexual
 pasa lo mismo.

Probablemente no debemos llegar al punto de que en los colegios haya instrucciones prácticas y experimentales de cómo
 tener o no sexo. Pero nuestros niños y adolescentes deben comprender su cuerpo y apropiarse de la responsabilidad que
 ellos tienen consigo mismos. Apropiarse de esa responsabilidad bien puede llevarlos a abstenerse antes de estar
 seguros y a protegerse cuando sientan que les llegó el momento y a saber qué hacer cuando esa protección falle.

Ahora, y desde luego, en una educación confesional se les puede insistir que el momento en el que pueden garantizar
 estar seguros para iniciar su vida sexual es cuando se casen. Pero aún bajo esta insistencia algunos alumnos pueden
 equivocarse, pueden ignorar el consejo, pueden perder la fe o reemplazarla por alguna forma de razón escéptica. Si
 esta educación confesional es realmente responsable no debe limitarse a decir que se les advirtió: tiene que preparar
 a los jóvenes para enfrentar las alternativas.

\par% p
Dice una máxima de los movimientos pro-escogencia: \textbf{\emph{\anchor[http://www.sentidoscomunes.cl/diario/2011/11/educacion-para-decidir-anticonceptivos-para-no-abortar-y-aborto-legal-para-no-morir/]{Educación para decidir, anticonceptivos para no abortar y aborto legal para no morir}}}.

Una joven que ha recibido una buena educación sexual (hablando de la educación en el sentido completo del término y no
 del diseño de un programa curricular) puede tomar mejores decisiones sobre su cuerpo y sobre su práctica sexual. Puede
 decidir abstenerse o buscar el placer. Si esta educación ha sido deficiente, por ejemplo, porque hay un divorcio entre
 el currículo y la vivencia, esta joven puede cometer errores más fácilmente.

Esta joven que escoge libre e informadamente tendrá a su disposición una serie de métodos para desarrollar una vida
 sexual activa y protegida. Sabrá protegerse de las ETS y sabrá prevenir los embarazos antes de estar lista para ser
 madre. Entre mejor educadas sean nuestras mujeres y ante la disponibilidad de métodos anticonceptivos, incluyendo la
 contracepción de emergencia (píldora del día después), serán menos los casos en el que la mujer se vea ante el dilema
 de abortar.

\anchor[http://blog.chlewey.net/wp-content/uploads/2012/09/p8hql.png]{\begin{wrapfigure}{r}{300\px}\centering% {'src': 'http://blog.chlewey.net/wp-content/uploads/2012/09/p8hql-300x128.png', 'title': u'Abstenci\xf3n', 'height': '128', 'width': '300', 'alt': '', 'class': ['alignright', 'size-medium', 'wp-image-1345']}
\includegraphics[width=300\px,height=128\px]{blog/p8hql-300x128.png}
\end{wrapfigure}
}Pero llegará el momento en que las prevenciones fallen. Las mujeres que no atendieron bien cuando se les intentó
 enseñar. O que se equivocaron en una prevención. O que el método falló. O que perdieron el juicio frente a una pareja
 dominante. O que fueron forzadas (no sólo frente a amenazas de fuerza letal). O que por cualquier razón, con culpa o
 sin culpa, son empreñadas cuando aún no están listas para ser madres. Muchas de estas mujeres reasumirán su proyecto
 de vida y tal vez hasta sean grandiosas madres. Pero no todas estarán listas. O, tal vez sean sus parejas o sus padres
 quienes no estén listos y obliguen a la mujer a “solucionar el problema”.

A mí no me gusta el aborto. Creo que cada vez que una mujer se enfrenta al dilema de abortar es porque algo malo pasó.
 Una educación fallida. Un violador. Un método anticonceptivo que no funcionó correctamente. Un tamizaje fallido de
 riesgos de enfermedades genéticas incompatibles con la vida. Pero no puedo imponer mi gusto frente al dilema que
 enfrenta la mujer que piensa o es obligada a abortar. Menos cuando yo nunca enfrentaré ese dilema en carne propia.

Un aborto mal practicado es un riesgo para la salud de la mujer y es un riesgo que puede acarrear incluso la muerte. El
 aborto ilegal, como cualquier otra actividad ilegal, carece de una supervisión por el estado lo que aumenta los
 riesgos de ser mal practicado. El aborto mal practicado es un problema existente y grave de salud en Colombia lo que
 significa que es algo que sucede. El aborto es percibido como una necesidad por muchas mujeres y la ley no será un
 impedimento para que en su desesperación lo busquen.

Yo no tengo una convicción fuerte frente al aborto. Simplemente no me gusta. Entiendo también a los hombres y mujeres
 que creen que el aborto es un asesinato. Un embrión de pocas semanas ya es un individuo humano vivo si bien aun no
 tenga el desarrollo neuronal que le permita sentir el ambiente. Por mucho que yo sufra por culpa de una persona
 abusiva, yo no tengo el derecho a asesinar esa persona. Si yo creo que ese embrión es ya una persona, la madre no
 tiene el derecho a asesinarla por el abuso que ese pequeño e inocente ser le infrinja por los próximos 8 meses (o más
 si decide quedárselo). Visto así el aborto sería el asesinato de inocentes sólo porque una persona decidió que no
 coincidía con su proyecto de vida. Es una decisión bastante egoísta.

Si quiero resolver este dilema. Si como sociedad queremos resolver el dilema, no podemos basarnos sólo en creencias. No
 es porque yo crea que un embrión es una persona o no. O porque yo crea o no en la biblia. O porque yo crea o no en el
 concepto de autonomía completa sobre nuestros cuerpos. Lo que sabemos es que antes de las doce semanas de gestación el
 embrión no ha desarrollado un sistema neuronal “humano” y si tengo que escoger entre matar sólo un embrión o por
 proteger al embrión poner en riesgo la vida de la gestante y, por lo tanto, también del embrión, escogeré salvar a la
 mujer.

\anchor[http://douglawrence.wordpress.com/2010/11/07/pro-choice-catholic-politicians-are-being-extremely-short-sighted/]{\begin{wrapfigure}{r}{300\px}\centering% {'src': 'http://blog.chlewey.net/wp-content/uploads/2012/09/choice-cartoon-300x180.jpg', 'title': 'choice-cartoon', 'height': '180', 'width': '300', 'alt': '', 'class': ['alignright', 'size-medium', 'wp-image-1335']}
\includegraphics[width=300\px,height=180\px]{blog/choice-cartoon-300x180.jpg}
\end{wrapfigure}
}Yo iría más allá del aborto terapéutico. Yo diría que el estado debe garantizar que toda mujer que decida abortar pueda
 contar con un medio seguro para hacerlo. No necesariamente un método gratuito pagado por los contribuyentes y
 amparable por tutela pero sí permitir que sea seguro y accesible. Pero el estado también debe velar para que el aborto
 sea una excepción. No prohibiéndolo sino generando las garantías para que el dilema del aborto sea tan raro como sea
 posible.

Y gran parte de esa garantía es la educación sexual. Entendiendo la educación como un conjunto y no sólo como una
 cátedra.

\chapter{Dios te quiere muerto}
\begin{metadata}
	Published by \anchor[chlewey]{chlewey} on \anchor[http://ewey.co/B1348]{Sat, 29 Sep 2012 14:17:44 +0000}\\
	\categories{estado, information, politica, religion, sociedad}\\
	Shorthand: \anchor[http://blog.chlewey.net/2012/09/dios-te-quiere-muerto/]{dios-te-quiere-muerto}
\end{metadata}

\anchor[http://blog.chlewey.net/wp-content/uploads/2012/09/god-wants-you-dead.jpg]{\begin{wrapfigure}{r}{300\px}\centering% {'src': 'http://blog.chlewey.net/wp-content/uploads/2012/09/god-wants-you-dead-300x196.jpg', 'title': 'Portada: God Wants You Dead', 'height': '196', 'width': '300', 'alt': '[portada: God Wants You Dead] ', 'class': ['alignright', 'size-medium', 'wp-image-1349']}
\includegraphics[width=300\px,height=196\px]{blog/god-wants-you-dead-300x196.jpg}
\end{wrapfigure}
}Hagamos un ejercicio mental muy sencillo.
Digamos que tú tienes una idea.
 Me cuentas esa idea y a mí me gusta y la adopto.
 Entonces tú y yo tenemos una misma idea.
 O si vemos la idea como nuestro objeto de estudio la idea existe en dos personas, en dos anfitriones: tú y yo.
 Si compartimos la idea con más personas y a estas le gusta la idea, esa buena idea residirá ahora en muchas más personas.
 Si la idea es suficientemente atractiva la idea incluso podrá sobrevivirnos.
 Las buenas ideas entonces se propagan.
 Pero ¿qué es una buena idea? O más exactamente ¿qué es una idea que se va a propagar y sobrevivir?

No entraré a detallar qué es una idea.
 Por ejemplo “lávate las manos antes de comer” es una idea.
 Con nuestros actuales conocimientos sobre los gérmenes sabemos que es buena idea y sabemos por qué.
 Los que estén familiarizados con la Biblia cristiana saben que Jesús rechazaba esa idea lo que nos dice tres cosas: 1) la idea ya existía en tiempos bíblicos, 2) Jesús no sabía de gérmenes, 3) Jesús sí sabía lo que era la perversión de una idea.
 Los judíos en la época de Jesús no sabían para qué se lavaban las manos antes de comer.
 Lo hacían sólo porque era un dogma de fe.
 Probablemente algunos antepasados de los judíos eran más escrupulosos con respecto a comer la tierra en sus manos junto con sus alimentos y enseñaron este escrúpulo a sus hijos mientras que otros no.
 Los que no vivían más enfermos y sus hijos morían por cualquier infección.
 Así la idea de lavarse las manos prosperó, pues los anfitriones de la idea eran más saludables y se reproducían más.
 La idea quedó así escrita en el código de Hammurabi atribuido luego a Moisés y convirtiéndose en parte de la ley judía.

Algunas ideas nos ayudan a sus anfitriones a sobrevivir.
 Es bueno para el que tiene la idea y para las personas a quien este inspira porque, bueno, sobreviven.
 Es bueno para la idea porque prospera.
 Si pensamos en la idea como un organismo, estas ideas serían organismos simbióticos.
 Otras ideas no favorecen directamente al individuo.
 La idea de “no robarás” puede poner al individuo en desventaja frente a la idea contraria cuando hay un botín apetecible.
 Pero cuando vivimos en sociedad y dentro de la sociedad todos compartimos la idea nos va mejor en conjunto que a cada uno individualmente.
 No obtendré un botín, pero mis bienes no se convertirán tampoco en botín de otros.
 Este beneficio mutuo aún en contra del beneficio individual inmediato es lo que convierte a esta idea en una idea
 altruista.

Las ideas simbióticas y altruistas son buenas ideas.
 Nos ayudan a mantenernos vivos y a convivir en sociedad.
 Y por eso esas ideas se propagan y permanecen.
 Pero no son las únicas ideas.
 En ocasiones una mala idea también se propaga.
 El rechazo de Jesús al dogmatismo lo llevó a descartar el lavado de manos como una buena idea.
 Pero no sólo eso.
 Esas palabras fueron escritas e incluidas en los libros sagrados del cristianismo y tomadas como dogma.
 Cuando los primeros médicos dotados de microscopios descubrieron los gérmenes y su relación con las enfermedades, los demás médicos rechazaban la idea de lavarse las manos cuando abrían pacientes y realizaban operaciones causando la muerte por infecciones de sus pacientes.
 ¿Cuántas personas habrán muerto antes de que se consolidara la teoría del germen por intervenciones quirúrgicas sucias
 y descuidadas?

\par% p
Una mala idea que nos muestran \anchor[http://en.wikipedia.org/wiki/Sean\_Hastings]{Sean Hastings} y Paul Rosemberg en su libro \textbf{\anchor[http://www.amazon.com/s?search-alias=stripbooks\&field-isbn=9780979601118]{\emph{God Wants You Dead}}} ${}^\textrm{\anchor[http://chlewey.net/docs/God\_Wants\_You\_Dead.pdf]{[copia, 4GB]}}$ ${}^\textrm{\anchor[http://www.clearbits.net/get/164-god-wants-you-dead.torrent]{[torrent]}}$ es el de agrupar ideas.
 Digo.~ Si yo tengo una serie de buenas ideas ¿no es buena idea ponerlas juntas y convertirlas en un decálogo?

\begin{blockquote}

\subsection{Decálogo para ser feliz}
\begin{enumerate}

\item Mira el cielo
\item Huele las flores
\item Pasa más tiempo con tus padres, tu pareja o tus hijos
\item Aleja a las personas negativas de tu vida
\item No dependas de cosas externas que deseas
\item Piensa en lo que tienes, no en lo que te falta
\item Si caes once veces levántate doce
\item Comparte siempre una sonrisa
\item Celebra tus triunfos, olvídate de tus derrotas
\item Comparte este decálogo: las personas felices a tu alrededor te harán sentir más feliz.

\end{enumerate}

\end{blockquote}

Son diez ideas que parecen buenas.
 Ponerlas juntas y darle un nombre es también una buena idea porque así podremos referirnos más fácilmente a ellas.
Es más, si tengo varios decálogos (el de ser feliz, el de estar a paz con Dios, el de ser exitoso, el del buen amigo,
 etc.) podemos estar definiendo todo un estilo de vida.

El problema es que al agrupar las cosas no sólo nos ayuda a organizar nuestra mente, sino que al empaquetar las ideas las convertimos en ideas complejas (compuestas) y podemos así seguir hasta tener toda una ideología.
 Y cada una de estas ideas complejas serán aceptadas o rechazadas en bloque.
 Si analizamos por separado cada una de las ideas de mi decálogo para ser feliz podemos ver que algunas parecen muy buenas ideas, otras no tanto, y tal vez alguna parezca, incluso, una mala idea.
 La idea 6 y más en combinación con la 5 tal vez no promuevan la felicidad sino el conformismo.
 Pero no porque haya malas ideas en la lista significa que todas las ideas de la lista sean malas.
~ Convertidas en un decálogo está la tendencia de aceptarlas todas en bloque así haya basura entre las ideas
 individuales, o a rechazarlas todas así haya ideas realmente buenas en la mezcla.

Las ideologías y anti-ideologías, así como sus íconos, junto con algunas ideas simbióticas y altruistas, tienen también una carga de ideas parásitas.
 Ideas que no cumplen un papel directo de preservación del individuo o indirecto de preservación de la sociedad sino que simplemente están ahí para garantizar la perpetuación de la idea y de la ideología.
 Ideas que nos llevan al sacrificio.
 Ideas como que debemos sentirnos culpables por tener sentimientos egoístas.
 Ideas como que es heroico morir por la patria.
 Ideas que nos hacen sentir que sólo somos una parte sacrificable de un colectivo.

No es que los colectivos en sí sean malos.
 Finalmente somos animales sociables y cualquier cosa que nos permita vivir dentro de la sociedad (ideas como “no matarás” o “no robarás”) son aceptables.
 Pero cuando el colectivo se sacraliza y nos demanda sacrificios, entonces esa idea del colectivo, esa ideología, es parásita.
 Jesús (ícono de ese colectivo que son las iglesias cristianas) nos enseñaba
 que no se hizo el hombre para la ley sino la ley para el hombre.
 (Y claramente sabemos ya que la ley es un conjunto de ideas empaquetadas, y que claramente la ley la podemos convertir
 también en un ícono: La Ley.)

La vida y ejemplo de Jesús nos muestra una persona que reivindicaba a las otras personas. “Levántate y anda”: los milagros son ejemplo de la superación del individuo. “Al Cesar lo que es del Cesar y a Dios lo que es de Dios”: cuando colectivos como el estado nos ofrece medios de pago como las monedas pues estas son del colectivo y no debemos mezclarlas con lo que nos piden otros colectivos o nuestra sociedad. “Con la vara que midas serás medido” y “has a otros lo que quieras que te hagan” son llamados a la integridad y a la reflexión. (Alguien dirá que la regla de oro “no hagas a los demás lo que no quieras que te hagan”
 es mucho más antigua que Jesús, pero su forma positiva aparece por primera vez en el cristianismo.) “Da la otra mejilla” y “ama a tu enemigo” nos invitan a detener cualquier ciclo de violencia.
 Ahora bien, cuando una iglesia (o un clérigo) sale a juzgarnos a todos nosotros sin admitir críticas, podemos ver cómo
 el colectivo que pregona a Jesús como ícono se aparta de lo que Jesús (asumiendo su existencia) nos enseñó.

\anchor[http://blog.chlewey.net/wp-content/uploads/2012/09/law-vs-justice.png]{\begin{wrapfigure}{r}{280\px}\centering% {'src': 'http://blog.chlewey.net/wp-content/uploads/2012/09/law-vs-justice-280x300.png', 'title': 'Ley v/s Justicia', 'height': '300', 'width': '280', 'alt': '', 'class': ['alignright', 'size-medium', 'wp-image-1350']}
\includegraphics[width=280\px,height=300\px]{blog/law-vs-justice-280x300.png}
\end{wrapfigure}
}Hastings y Rosemberg nos enumeran una serie de colectivos e íconos, comenzando por la religión y Dios (no el creador cuya existencia es debatible, sino la idea de Dios o dioses que nos imponen las religiones, cuya existencia (de la idea) es verificable) y continuando por el Estado-Nación.
 Íconos y colectivos como La Ley (que se sobrepone al sentido de justicia), las corporaciones, el concepto de El Pueblo, o La Raza, o La Clase.
 Incluso La Familia o La Pareja se convierten en ideas colectivas con elementos parásitos.
 Yo agregaría otros colectivos e íconos como La Universidad, la Educación Pública y La Mujer.
 En el discurso político de ciertos grupos también se evidencia la colectivización del oponente y surgen términos como
 La Oligarquía, El Patriarcado, El Neoliberalismo.

En el proceso de construcción del Partido Pirata Colombiano, una de las cosas que hemos cuidado y discutido es cómo ser un partido político sin hacer lo que criticamos de los partidos políticos.
 Esto incluye cómo vender un discurso positivo de liberación y empoderamiento del individuo cuando es más fácil vender un discurso de miedo para que los votantes nos acojan sin cuestionar.
 El miedo nos dice que una serie de criminales roban a nuestros artistas y que por ello debemos pasar leyes de protección a los derechos de autor, y que una serie de depravados van por nuestros hijos y por ello es necesario vigilar los contenidos que publicamos en Internet para perseguir la pornografía infantil.
 También podemos apelar al miedo y decir que el gobierno quiere meterse en nuestras vidas para controlarnos y usar la excusa de la lucha antipiratería y anti pornografía infantil como un medio para censurarnos.
 Pero esto no sería justo con ustedes.
 Sería traicionarnos a nosotros mismos.
 No es porque un ogro llamado El Gobierno conspire con otros demonios llamados La Industria de Contenidos para someternos a nosotros y maximizar las ganancias de unos pocos y aumentar el control sobre nuestras vidas.
 Simplemente es porque por temor cedemos el control de nuestras vidas a un control colectivo y, por ello mismo,
 irresponsable.

Pero es más fácil vender el miedo.

\chapter{Mi día de la raza}
\begin{metadata}
	Published by \anchor[chlewey]{chlewey} on \anchor[http://ewey.co/B1357]{Fri, 12 Oct 2012 17:13:21 +0000}\\
	\categories{information, raza}\\
	Shorthand: \anchor[http://blog.chlewey.net/2012/10/mi-dia-de-la-raza/]{mi-dia-de-la-raza}
\end{metadata}

\anchor[http://blog.chlewey.net/wp-content/uploads/2012/10/chlewey-cp07.jpg]{\begin{wrapfigure}{r}{248\px}\centering% {'src': 'http://blog.chlewey.net/wp-content/uploads/2012/10/chlewey-cp07-248x300.jpg', 'title': 'chlewey-cp07', 'height': '300', 'width': '248', 'alt': '', 'class': ['alignright', 'size-medium', 'wp-image-1358']}
\includegraphics[width=248\px,height=300\px]{blog/chlewey-cp07-248x300.jpg}
\end{wrapfigure}
}Por mis venas corre sangre amerindia.~ Tal vez amerindias sean mis \anchor[http://es.wikipedia.org/wiki/Genoma\_mitocondrial]{mitocondrias} pero no estoy tan seguro.
 Hace 500 años cuando los europeos llegaron a lo que hoy es Colombia los conquistadores empreñaron a las mujeres de la tierra creando generaciones de mestizos bastardos, pero más adelante, cuando la colonia se asentó, llegaron también mujeres españolas que para defender a sus hijos promovieron leyes que estratificaban las clases sociales por cuánto porcentaje de sangre india tenía cada persona.
 Por pura intuición fenotípica supongo que las mitocondrias de mi padre son amerindias pero las de mi madre son
 españolas.

\par% p
Por mis venas corre sangre africana, pero igual no puedo estar seguro de que hace 500 años mi \anchor[http://es.wikipedia.org/wiki/Haplogrupos\_del\_cromosoma\_Y\_humano]{cromosoma Y} se encontraba en África o en Europa.
 Mi abuelo fue un negro antillano, descendiente de los esclavos que el Imperio Británico importó a las Américas para cultivar caña de azúcar.
 Pero en la historia de la esclavitud negra por los británicos hay muchos casos de hijos mulatos que se asimilaban a la población esclava y negra.
 Tal vez algún inglés de apellido Thompson llamó así a sus hijos mulatos.

\par% p
Por mis venas corre sangre española.~ España como unidad política no existió antes de \anchor[http://es.wikipedia.org/wiki/Decretos\_de\_Nueva\_Planta]{1707} cuando los borbones se hicieron al trono de los reinos asociados a Castilla y a Aragón.
Antes de 1492 España era un genérico para referirse a los reinos cristianos de la península Ibérica en contrasta a los
 reinos musulmanes que conformaban \anchor[http://es.wikipedia.org/wiki/Al-\%C3\%81ndalus]{Al-Ándalus}.
 Estos reinos cristianos se habían convertido en tres coronas principales: la de Portugal, la de Castilla y León y la de Aragón y Navarra.
 Fue Castilla y León la que conquistó y colonizó a la América Hispánica, en unión personal con la corona de Aragón y Navarra cuyo interés principal fue pelear guerras santas en el Mediterraneo.
 Salvo un breve período bajo Felipe II, Portugal se mantuvo al margen.

Entre los años 711 y 715, los musulmanes provenientes del norte de África conquistaron la península Ibérica con excepción de Asturias.
Durante su reinado la península fue parte del mundo musulmán intercambiando conocimientos, artes y personas con el resto de la civilización árabe.
 Por los siguientes 780 años moros y cristianos pelearon por el control de pedacitos de la península (y los cristianos
 peleaban entre sí creando distintos reinos, así el reino de León se escindió de Asturias, y Castilla se escindió de
 León (y finalmente Castilla se hizo al control de León y de Asturias)).

\begin{wrapfigure}{r}{320\px}\centering% {'width': '320', 'align': 'alignright', 'id': ''}
\anchor[http://rosamtristan.com/2012/08/02/los-primeros-tecnologos-los-bosquimanos/]{\includegraphics[width=320\px,height=240\px]{blog/102_02934.jpg}} Mujer Xhosa y su hijo, mostrando razgos claramente distintivos de los negros africanos. Se cree que los Xhosa
 habitaron la mayor parte del África subsahariana antes de ser invadidos por los negros provenientes del Sahel y el
 Noroeste africano.
\end{wrapfigure}

Iberia siempre fue un lugar de paso.
 Probablemente los cromagnon provenientes de África conquistaron a Europa occidental pasando por la península Ibérica.
 Los africanos cartagineses descendientes de los medioorientales fenicios disputaron a Hispania con los romanos.
 España es tan europea como es africana, y es aquí donde debo aclarar: África no siempre fue negra y aun hoy no lo es.
 Cuando hablaba de mis trazas de sangre africana hablaba de sangre negra africana, pero sin duda también tengo sangre
 caucásica africana y sangre caucásica europea.

Hoy hay voces que piden que no celebremos el descubrimiento de América hace 520 años, porque más que un descubrimiento
 fue un exterminio.

Pero no fue a mis ancestros (ni a la de ninguno de nosotros) a los que exterminaron.
 Soy (somos) el producto de ese choque entre pueblos donde muchos murieron pero donde nuestros ancestros sobrevivieron lo suficiente para parirnos.
 Entre mis ancestros estará el hijo bastardo de un español que violó a una india muisca (tal vez más de uno) así como
 el hijo criollo de un matrimonio de castellanos que se asentaron en alguna pequeña ciudad del Nuevo Reino.

Entre mis ancestros estará el producto de la violación de un moro africano a una mujer visigoda, esta, a su vez
 descendiente varias generaciones atrás de un bárbaro godo (germánico) que tomó por la fuerza a una mujer romana,
 descendiente de un soldado itálico que violó a una mujer íbera.

\par% p
Los mismos muiscas fueron producto de la conquista de pueblos chibchas sobre anteriores tribus amerindias, y el áfrica
 negra que aflora en mi piel es el resultado de guerras y conquistas de una población que se originó en el \anchor[http://es.wikipedia.org/wiki/Sahel]{Sahel} y de ahí se extendió hasta dominar casi toda el África subsahariana.

\begin{wrapfigure}{r}{320\px}\centering% {'width': '320', 'align': 'alignright', 'id': ''}
\anchor[http://commons.wikimedia.org/wiki/File:Christopher\_Columbus3.jpg]{\includegraphics[width=320\px,height=205\px]{blog/Christopher_Columbus3.jpg}} Primer desembarco de Cristóbal Colón en América, obra del pintor español Dióscoro Puebla
\end{wrapfigure}

\par% p
El 12 de octubre de 1492 un grupo de europeos pisaron unas islas en las Bahamas.
 Algunos murieron antes de regresar pero los que regresaron contaron sus historias al resto de Europa y esto llevó a que más europeos vinieran, se asentaran y desplazaran una gran parte de la población indígena de América.
 No fue un proceso sistemático de exterminio sino el encuentro entre pueblos con \anchor[http://blog.chlewey.net/2008/08/gunning-germing-and-steeling/]{distintos bagajes tecnológicos} (escritura v/s tradición oral, acero v/s oxidiana, pólvora, etc.) y biológicos (caballos, gérmenes).

Antes de las ideas del humanismo, el resultado de tal encuentro entre pueblos podría pensarse inevitable.
 Ahora, es muy fácil juzgar desde este humanismo del siglo XXI lo que nuestros ancestros hicieron con nuestros ancestros a la luz de los memes del siglo XV, XVI y XVII.
 A veces con un discurso en el que nos identificamos con los hermanos de nuestros ancestros que no sobrevivieron como
 si nuestros ancestros genéticos hubieran sido un pueblo aparte del que somos.

No celebro hoy el descubrimiento de América porque lo que hizo esa expedición que llegó a las Bahamas hace 520 años fue más un hecho fortuito que se consolidaría en los años subsiguientes.
 No celebro el encuentro de dos mundos porque más que un encuentro fue un choque que devastó a casi todos los pueblos que vivían originalmente a este lado del charco.
 No celebro los 520 años de resistencia porque no hubo tal resistencia continuada.
 No condeno tampoco a mis ancestros porque eso sería imponer mi mentalidad liberal y humanista actual a personas que
 tenían pocas opciones diferentes.

Ver el 12 de octubre como día de la raza es, sin embargo, una reflexión sobre lo que es mi raza y, en últimas, sobre el
 absurdo que es etiquetarnos y separarnos por unos pocos rasgos externos como el color de la piel o el tamaño de los
 ojos.

Pero igual no importa porque lo verdaderamente importante en Sudamérica hoy son cuatro partidos de fútbol.

\chapter{El pueblo}
\begin{metadata}
	Published by \anchor[chlewey]{chlewey} on \anchor[http://ewey.co/B1362]{Fri, 19 Oct 2012 15:35:52 +0000}\\
	\categories{information}\\
	Shorthand: \anchor[http://blog.chlewey.net/2012/10/el-pueblo/]{el-pueblo}
\end{metadata}

\begin{wrapfigure}{r}{323\px}\centering% {'width': '323', 'align': 'alignright', 'id': ''}
\anchor[http://commons.wikimedia.org/wiki/File:PeriklisKotzia.JPG]{\includegraphics[width=323\px,height=480\px]{blog/323px-PeriklisKotzia.jpg}} Monumento a Pericles en Atenas (por Gepsimos vía Wikimedia Commons. CC-by-sa 2006)
\end{wrapfigure}

\par% p
En griego existen dos términos que suelen traducirse al español como \emph{pueblo} pero trascienden a nuestro idioma como dos prefijos con significados diferentes de \emph{pueblo}. Está el \emph{\GR{δῆμος}} (\emph{demos}) y el \emph{\GR{έθνος}} (\emph{etnos}) que hoy vemos en palabras como demografía y etnografía. No estoy seguro qué diferencia semántica tenían los griegos
 clásicos. En griego moderno \emph{\GR{δήμος}} se traduce como municipio y \emph{\GR{έθνος}} como nación, al menos en ciertos contextos.

La demografía es el estudio de la población en los términos generales de dimensión, estructura y evolución. La
 etnografía, por su lado, busca estudiar y mostrar la diversidad y diferencias en la cultura. La democracia se define
 como el gobierno del pueblo, entendiéndose históricamente pueblo como las personas en capacidad de decidir.

\par% p
Tal vez una definición más adecuada de democracia, que incluye sus usos históricos en la antigua Grecia y las primeras
 democracias modernas donde sólo podían participar los varones con patrimonio mínimo (p. ej. en EE.UU. y en Colombia)
 sería el gobierno de los ciudadanos, aunque esto nos llevaría a elaborar sobre el término \emph{ciudadano}. Digamos aquí que ciudadano es simplemente la persona que se considera suficientemente libre y responsable para
 decidir sobre el destino de su nación.

\begin{wrapfigure}{l}{322\px}\centering% {'width': '322', 'align': 'alignleft', 'id': ''}
\anchor[http://www.flickr.com/photos/fiore\_barbato/2194886917]{\includegraphics[width=322\px,height=400\px]{blog/2194886917_17d151b92b.jpg}} Romas en procesión de Santa Sara en Saintes-Maries-de-la-Mer, Francia. (por Fiore S. Barbato via Flicker, CC-by-sa
 2000)
\end{wrapfigure}

\par% p
El pueblo es un conjunto. Un substantivo colectivo para designar a las personas, bien las una una generalidad como un
 territorio (\emph{demos}) o un sentido de nación (cuna, nacimiento) común (\emph{etnos}). Un país como Colombia se definió como una nación a partir de un territorio dentro del concepto de Estado-Nación que
 se creó en el siglo XVIII y se consolidó en el siglo XIX. Pero hay naciones sin soberanía (estado) sobre su
 territorio. El pueblo \textbf{wayúu} repartido entre Colombia y Venezuela; la nación \textbf{kurda} que reclama unidad y un territorio que se expande en los actuales Turquía, Iraq e Irán; el pueblo \textbf{roma} (gitanos) que no pretenden territorio propio y sus manifestaciones en distintos países mezclan un grado de asimilación
 con un hilo conductor que los hermana con los demás roma.

Pueblo, como substantivo colectivo, no tiene un término para referirse a los individuos. Otro substantivo colectivo es
 ciudadanía, pero ciudadanía viene de ciudadano y como tal la ciudadanía es el colectivo de individuos denominados
 ciudadanos.

\par% p
Cuando hablamos del pueblo-nación, del \emph{etnos}, el problema de la individualización no es, a mi parecer, crítico. Un individuo del pueblo kurdo es un kurdo (o una
 kurda). Un colombiano es el individuo del pueblo colombiano. Pueblo, en este sentido, es tanto un substantivo
 colectivo como un substantivo contable. Hay muchos pueblos y para cada pueblo podremos individualizar a sus
 constituyentes.

\par% p
¿Cuál es el individuo del otro pueblo, del \emph{demos}?

\par% p
El inglés nos muestra un truco. Pueblo (tanto en el sentido demos como etnos) se traduce como \emph{people}; y si bien alguna vez existió el plural regular de \emph{person} (persona) como \emph{persons}, este ha caído en desuso y \emph{people} se ha convertido en el plural oficial e irregular de \emph{person}. Pero pueblo no es el plural de persona y como colectivo es más un colectivo de seres humanos que un colectivo de
 personas. (ya \anchor[http://blog.chlewey.net/2011/03/de-personas-y-avatares/]{alguna vez discutí} sobre las diferencias entre personas y seres humanos.)

\begin{wrapfigure}{r}{320\px}\centering% {'width': '320', 'align': 'alignright', 'id': ''}
\anchor[http://noticias.terra.com.ar/internacionales/gran-expectativa-por-dialogos-con-las-farc,f4fc61a08d56a310VgnCLD2000000ec6eb0aRCRD.html]{\includegraphics[width=320\px,height=232\px]{blog/ap-6c1752846203571d1e0f6a706700b384.jpg}} Representantes de las FARC en Hurdal (Noruega) 18 de octubre de 2012. (© Scanpix Norway, Audun Braastad) / AP via
 Terra)
\end{wrapfigure}

\par% p
Ayer, durante la instalación de los diálogos de paz entre el estado colombiano en cabeza del presidente Juan Manuel
 Santos y las Fuerzas Armadas Revolucionarias de Colombia – Ejército del Pueblo (\begin{abbr}% {'style': 'font-variant: small-caps;'}
Farc-ep
\end{abbr}
), los voceros de las \begin{abbr}% {'style': 'font-variant: small-caps;'}
Farc
\end{abbr}
 insistieron muchas veces en ser o representar al pueblo. Dijeron muchas veces que ellos no atacan al pueblo y que
 ellos son víctimas de un estado que oprime al pueblo y que como pueblo se vieron obligados a tomar las armas y muchas
 otras frases donde se erigen como representantes del pueblo o simplemente se identifican como el pueblo.

Yo no podía dejar de oírlos sin preguntarme a qué pueblo se referían.

Juan Manuel Santos fue elegido con poco más de 9 millones de votos tras una participación en primera vuelta de más de
 14 millones y medio de votos válidos que, con su participación, validaron las elecciones. En 2010 había casi 30
 millones de ciudadanos habilitados para votar, lo que nos da una abstención algo superior al 50\%.

\par% p
El gobierno de Santos representa así a casi la mitad del pueblo colombiano. Casi la mitad o tal vez más de la mitad. Es
 difícil saber el porcentaje exacto porque el censo electoral parece que está inflado, pero no me atrevo a estimar por
 cuanto. Adicionalmente muchos de los que se abstienen a votar lo hacen por motivos ajenos a desconocer al estado. Pero
 asumamos que Santos, con el 30\% de respaldo directo en las urnas y 49\% de validación ciudadana sólo representa al 49\%
 de los colombianos. Asumamos que tal vez su representación sea menor porque muchos votantes no lo hicieron libremente.
 Esto, en ningún momento significa que las \begin{abbr}% {'style': 'font-variant: small-caps;'}
Farc-ep
\end{abbr}
 representen al 51\% del pueblo colombiano o mucho menos a la totalidad del pueblo colombiano.

\par% p
Incluso, bajo esa interpretación maniquea que contrasta al pueblo con las clases dirigentes excluyéndolas (entiéndase
 por clases dirigentes a políticos, empresarios, terratenientes y cualquier persona que esté ligeramente por encima del
 rasero con el que se quiera definir pueblo), no hay, no existe expresión alguna de que ese pueblo haya depositado su
 representación en las \begin{abbr}% {'style': 'font-variant: small-caps;'}
Farc
\end{abbr}
 o en cualquier otro grupo alzado en armas.

\par% p
Hay individuos que creen en las \begin{abbr}% {'style': 'font-variant: small-caps;'}
Farc
\end{abbr}
 o en su lucha. Hay, supongo, colectivos de mayor o menor tamaño que se identifican con las \begin{abbr}% {'style': 'font-variant: small-caps;'}
Farc
\end{abbr}
 antes que con el estado constitucional. Pero no conozco pronunciamientos que se acerquen en magnitud similar a 9
 millones de individuos más o menos libres que se expresen a favor de las \begin{abbr}% {'style': 'font-variant: small-caps;'}
Farc
\end{abbr}
 y se identifiquen con su supuesta lucha.

Es difícil medir al pueblo y menos si insistimos en una forma vaga de usar el término pueblo. Pero en cualquier tipo de
 medición que se haya llevado a cabo el estado constitucional encabezado hoy por el gobierno de Santos tiene mayor
 respaldo que cualquier identificación que el presunto Ejército del Pueblo pueda mostrar.

\par% p
Pero es muy sencillo. Si yo digo que yo no soy el pueblo al que las \begin{abbr}% {'style': 'font-variant: small-caps;'}
Farc
\end{abbr}
 representan ellos dirán que yo soy, represento o estoy a servicio de la oligarquía que ha oprimido al pueblo. Un
 personaje metafísico llamado oligarquía oprimiendo a un personaje metafísico llamado \emph{\textbf{el pueblo}}.

\chapter{Enclavados en el Caribe Occidental}
\begin{metadata}
	Published by \anchor[chlewey]{chlewey} on \anchor[http://ewey.co/B1368]{Tue, 20 Nov 2012 17:03:54 +0000}\\
	\categories{colombia, information, limites, nicaragua, san-andres}\\
	Shorthand: \anchor[http://blog.chlewey.net/2012/11/enclavados-en-el-caribe-occidental/]{enclavados-en-el-caribe-occidental}
\end{metadata}

En 1928 y a instancias de los Estados Unidos de América quien actuó como observador, los entonces gobiernos de Nicaragua y de Colombia firmaron el tratado Esguerra-Bárcenas (más conocido en Nicaragua como Bárcenas Meneses-Esguerra).
 Los derechos del mar que hoy conocemos, incluyendo el concepto de límites marinos y submarinos no existían, y como
 tales no fueron tratados en el acuerdo de 1928.

¿Qué dirimía el tratado?

Cuando la América Hispana era un territorio de ultramar del Reino de España, con frecuencia la metrópoli cambiaba las distintas jurisdicciones.
 Por ejemplo, en 1803 consideró que San Andrés y Providencia, así como la Costa de los Mosquitos quedaban mejor
 protegidos si se defendían desde Cartagena que desde Guatemala aunque el gobierno civil de las islas dependiera aún de
 la capital centroamericana.

\begin{wrapfigure}{r}{293\px}\centering% {'width': '293', 'align': 'alignright', 'id': ''}
\anchor[http://commons.wikimedia.org/wiki/File:Gran\_Colombia\_map\_1824.jpg]{\includegraphics[width=293\px,height=240\px]{blog/293px-Gran_Colombia_map_1824.jpg}} República de Colombia (Gran Colombia) en 1824
\end{wrapfigure}

Tras la independencia de Hispanoamérica se buscó un principio que nunca se cumplió al pie de la letra: que las nuevas naciones americanas siguieran los límites que existían entre las colonias españolas.
 En la práctica provincia por provincia y ciudad por ciudad se fueron uniendo a una u otra entidad mayor.
 San Andrés y Providencia prefirieron unirse a la Constitución de Cúcuta que al Imperio de Iturbide (México y
 Centroamérica).

La costa de los Mosquitos era una zona selvática de mangles y pantanos.
 Lejos aún de la tecnología del siglo XX, los sucesivos gobiernos de Centroamérica y Nicaragua no tenían forma de reclamar más allá de la boca del Río San Juan mientras su proyección marina estaba realmente en el Pacífico.
 La Nueva Granada tampoco es que ejerciera un control real sobre esas costas.
 Básicamente figuraban en el mapa y no más.
 El poder real en la Mosquitía lo ejercía el Imperio Británico, delegado más tarde en el rey de los Miskitos a modo de
 protectorado.

Los Estados Unidos fueron poco a poco ganando poder en el Caribe definiendo a este mar como su patio trasero a expensas originalmente de la corona Británica.
 Algunos empresarios y aventureros fueron más allá: empezaron a invertir en plantaciones y minas en esos pequeños países de América Central y el Caribe y Colombia y Panamá no se escaparon.
 Al igual que San Andrés, Panamá siempre consideró que su adhesión a la constitución de Cúcuta y a las consiguientes constituciones de la Nueva Granada y de Colombia fueron voluntarias y varias veces amenazó con separarse, pero los intereses de Estados Unidos y su diplomacia de cañonero, impidió que en el siglo XIX estas amenazas se llevaran a cabo.
 Hasta que en 1903 el gobierno de Bogotá dejó de ser favorable al interés estadounidense de obtener la Zona del Canal a
 perpetuidad.

\begin{wrapfigure}{r}{307\px}\centering% {'width': '307', 'align': 'alignright', 'id': ''}
\anchor[http://es.wikipedia.org/wiki/Canal\_de\_Nicaragua]{\includegraphics[width=307\px,height=397\px]{blog/1895NicaraguaCanalCartoon.jpg}} ``Siguiente deber del Tío Sam'' 1895
\end{wrapfigure}

El otro posible canal que pudo haberse construido en Centroamérica por esa época era pasando por Nicaragua, tomando el Río San Juan y pasando por los lagos de Nicaragua y Managua y varios empresarios estadounidenses preferían esta opción.
 Aunque los empresarios de Panamá ganaron frente a los de Nicaragua, estos no perdieron el poder y siguieron influyendo
 en la política nicaragüense.

Desde 1894, Nicaragua había obtenido ya el control de la Mosquitía mientras que San Andrés y Providencia mantenían una unión con Colombia.
 Con la disculpa de que los bolcheviques mexicanos no invadieran a Nicaragua, los EE.UU. en 1927 (quienes ya tenían el
 control del Canal de Panamá) ocupan militarmente a Nicaragua bajo el permiso del presidente Adolfo Díaz.

Colombia alegaba viejos títulos sobre la Costa de los Mosquitos e islas aledañas como las islas Corn y el cayo Miskitos.
 En Nicaragua algunos pensaban que San Andrés y Providencia eran parte de las islas aledañas.
 Para zanjar la disputa, los EE.UU. median un acuerdo firmado por los delegados de Nicaragua y Colombia José Bárcenas Meneses y Manuel Esguerra, respectivamente.
 En el tratado Colombia reconocía que el territorio continental de la Costa de los Mosquitos y las islas aledañas pertenecían a Nicaragua y Nicaragua reconocía que el “Archipiélago de San Andrés y Providencia” pertenecían a Colombia.
 En cuanto a los islotes intermedios se estableció el meridiano 82° occidente de Greenwich para establecer cuales islas
 pertenecían a Mosquitía y cuales al Archipiélago de San Andrés (así Albuquerque quedó del lado colombiano y Corn del
 lado nicaragüense).

Los cayos de Roncador, Quitasueño, Serrana, Serranilla, Bajo Nuevo no estaban cubiertos en el tratado porque se
 encontraban en poder de facto de los EE.UU., reclamados formalmente por Colombia y no por Nicaragua (y, en cualquier
 caso, al estar al oriente del meridiano 82° quedaban excluidos de la reclamación nicaragüense tras el tratado
 Esguerra-Bárcenas).

Frente al avance de la resistencia de Augusto Sandino en Nicaragua y tras las elecciones de 1932, las tropas de EE.UU. se retiran de Nicaragua.
 En ese período el congreso nicaragüense, libre de cualquier injerencia de los EE.UU. ratifica nuevamente el tratado
 Bárcenas Meneses-Esguerra.

En 1980 el entonces presidente Daniel Ortega, tras el triunfo de la revolución Sandinista, desconoce el tratado Bárcenas Meneses-Esguerra, por considerar que Nicaragua no fue libre cuando lo firmó, por estar bajo ocupación de los EE.UU.
 Para entonces el concepto de soberanía del mar ya incluía el concepto de límites marinos y Colombia proclamaba que el meridiano 82° oeste de Greenwich era el límite marino acordado entre Colombia y Nicaragua.
 Con base en este límite, Colombia había resuelto a su favor la disputa con EE.UU. sobre Roncador, Quitasueño, Serrana,
 Serranilla y Bajo Nuevo (estos dos últimos también reclamados por Jamaica), y había establecido tratados limítrofes
 con Costa Rica, Panamá, Honduras y Jamaica, entre otros.

El acuerdo con Jamaica, por ejemplo, establecía una zona de régimen común (mutua explotación) alrededor de los cayos de
 Serranilla y Bajo Nuevo, más una zona exclusiva de Colombia en las inmediaciones de estos cayos.

\par% p
El tratado con Honduras parte de la pretensión colombiana de poseer todo al oriente del meridiano 82° y la pretensión hondureña de tener todo al norte del paralelo 14°59’08” N (paralelo del Cabo Gracias a Dios, hito del límite terrestre entre Honduras y Nicaragua), se establece así el límite con el paralelo 14°59’08” desde el meridiano 82° hasta el meridiano 79°56’ donde comienza la zona común entre Colombia y Jamaica, sube hacia el norte por el 79°56’ hasta el paralelo 15°30’ donde hace un arco para reconocer la soberanía de Colombia sobre los cayos de Serranilla.
\anchor[http://historico.unperiodico.unal.edu.co/ediciones/106/02.html]{\begin{wrapfigure}{r}{286\px}\centering% {'src': 'http://historico.unperiodico.unal.edu.co/ediciones/106/fotos/286/02.jpg', 'title': 'Diferento marino entre Honduras y Nicaragua, fallo de CIJ 2007', 'height': '109', 'width': '286', 'alt': '', 'class': ['alignright']}
\includegraphics[width=286\px,height=109\px]{blog/02.jpg}
\end{wrapfigure}
} Nicaragua también desconocía el paralelo 14°59’08” pretendiendo una línea de bisección que partía de Gracias a Dios hacia el nororiente, quedando los cayos Bobel, South, Savannah y Port Royal (al norte del paralelo 15°) bajo la jurisdicción nicaragüense.
 El fallo de la Corte Internacional de Justicia en 2007 determinó una línea geodésica entre Gracias a Dios y Serranilla
 (un punto medio entre las pretensiones de Honduras y Nicaragua), reconociendo la soberanía de Honduras sobre los cayos
 al norte del paralelo 15° y un mar territorial alrededor de estos cayos.

Este antecedente ya eliminaba el tratado entre Colombia y Honduras (salvo por el arco alrededor de Serranilla).

En 2007 y en respuesta a la demanda que Nicaragua instauró contra Colombia en 2001, la Corte Internacional de Justicia declaró vigente el tratado Esguerra-Bárcenas sólo en lo que respecta a ese tratado: la soberanía de Colombia sobre el archipiélago de San Andrés y Providencia, mientras se declara competente para establecer qué constituye parte del archipiélago, el estatus de los demás territorios insulares que no sean parte del archipiélago y los límites marinos y submarinos.
 En este momento, el pretendido límite marino en el paralelo 82° había dejado de existir.

\anchor[http://www.laprensa.com.ni/files/infografia/1352596144\_colombia\%20nicaragua3.jpg]{\begin{wrapfigure}{r}{300\px}\centering% {'src': 'http://www.laprensa.com.ni/files/infografia/1352596144_colombia%20nicaragua3.jpg', 'alt': '', 'height': '242', 'class': ['alignright'], 'width': '300'}
\includegraphics[width=300\px,height=242\px]{blog/1352596144_colombia_nicaragua3.jpg}
\end{wrapfigure}
}Ante la admisión de estos términos, la posición de Colombia fue considerar que Albuquerque, Roncador, Quitasueño,
 Serrana, Serranilla y Bajo Nuevo eran parte del archipiélago y que se estableciera una línea media entre las islas
 colombianas (p. ej. Albuquerque y Quitasueño) y las islas nicaraguénses (p. ej . Corn u Cayos Miskitos), mientras que
 Nicaragua pretendía una plataforma continental extendida de 350 millas náuticas, llegando a menos de 200 millas de
 Cartagena, y reconociendo un enclave de 12 millas náuticas de mar territorial alrededor de San Andrés, Providencia y
 Santa Catalina (esto hubiera dejado a San Andrés y a Providencia desconectados).

El fallo del 19 de noviembre de 2012 de la Corte Internacional de Justicia en La Haya básicamente reconoce a Albuquerque, Cayos del Sudeste y otros cayos menores como parte del archipiélago.
 Roncador, Quitasueño y Serrana no son parte del archipiélago pero tienen una tradición de presencia soberana de Colombia y se ratifica.
 Serranilla y Bajo Nuevo, según la Corte, nunca tuvieron un reclamo histórico real por Nicaragua (antes de 1980) y por
 lo tanto se reconocen colombianos.

\anchor[http://es.wikipedia.org/wiki/Fallo\_de\_la\_Corte\_de\_La\_Haya\_del\_19\_de\_noviembre\_de\_2012\_sobre\_el\_litigio\_territorial\_entre\_Colombia\_y\_Nicaragua]{\begin{wrapfigure}{l}{300\px}\centering% {'src': 'http://upload.wikimedia.org/wikipedia/commons/6/6d/New_maritime_boundary_between_Colombia_and_Nicaragua.png', 'alt': '', 'height': '391', 'class': ['alignleft'], 'width': '300'}
\includegraphics[width=300\px,height=391\px]{blog/New_maritime_boundary_between_Colombia_and_Nicaragua.png}
\end{wrapfigure}
}Según la Corte, el Archipiélago de San Andrés y Providencia son parte de la plataforma continental centroamericana y por lo tanto tiene derecho a plataforma continental de 200 millas náuticas que se expanden hacia el oriente, mientras que al norte, sur y occidente entran en sobre-posición con la plataforma de Nicaragua, primando “por equidad” la plataforma nicaragüense salvo el mar territorial de 12 millas alrededor de las islas del archipiélago.
 Roncador, Quitasueño y Serrana, por ser cayos y no islas, no tienen plataforma continental.
 (Roncador cae dentro de la plataforma continental de Providencia así que no tiene problema.)
 Se reconoce, sin embargo, el mar territorial alrededor de los cayos de Quitasueño y Serrana.

Finalmente el territorio marino de Nicaragua va hasta 200 millas náuticas de sus costas (no de sus cayos), salvo la sombra que proyecta el archipiélago de San Andrés y Providencia.
 Esta limitación permite que la plataforma de San Andrés conecte con la plataforma continental sudamericana de Colombia
 (igualmente Bajo Nuevo queda conectada a estas plataformas por medio de la zona mixta entre Jamaica y Colombia.

En los análisis y mapas que he visto, no me queda claro qué pasó con la zona de régimen común entre Colombia y Jamaica, cuya mitad occidental estaría dentro de las 200 millas náuticas de Gracias a Dios.
 Ninguno de los mapas que he visto toca esa zona pareciendo indicar que no se concede derechos a Nicaragua sobre esa
 zona de mar.

Personalmente me parece que el fallo de la Corte Internacional de La Haya está en derecho y en concordancia con el antecedente de 2007 entre Honduras y Nicaragua.
 Hubiera preferido otro fallo, particularmente que se hubiera reconocido a Roncador, Quitasueño y a Serrana como parte del archipiélago y por consiguiente se hubiera otorgado carácter de mar interior a la zona entre estos cayos Providencia y San Andrés.
 Creo que es justa la extensión de la soberanía nicaragüense sobre el Caribe salvo por esa porción que es a costa de la
 continuidad del mar entre Providencia y Quitasueño; bajo el supuesto, desde luego, de que sea justo dividir al mar con
 fronteras.

Ya Daniel Ortega declaró que no ve inconveniente en que los pescadores colombianos de San Andrés y Providencia pesquen en las aguas que ahora pertenecen de iure a Nicaragua.
 Creo que es más sabio que nuestro gobierno busque normalizar relaciones y formalice estos acuerdos de paso y de
 explotación conjunta a que se ponga a acusar de errores a la corte y procure prolongar una batalla jurídica.

Pero cuando veo tanta indignación nacional frente a la “pérdida”, se entiende que nuestro gobierno prefiera hacer pataletas frente a la tribuna nacional que a construir el futuro.
 Y seguiremos culpando al apellido Holguín por regalar territorio colombiano (así un Holguín haya logrado el
 reconocimiento de la península guajira como parte indisputada de Colombia).

\chapter{Carta al Niño Dios}
\begin{metadata}
	Published by \anchor[chlewey]{chlewey} on \anchor[http://ewey.co/B1375]{Tue, 04 Dec 2012 14:18:40 +0000}\\
	\categories{navidad, nino-dios, personal}\\
	Shorthand: \anchor[http://blog.chlewey.net/2012/12/carta-al-nino-dios/]{carta-al-nino-dios}
\end{metadata}

Querido Niño Dios

\par% p
Bien sabes que mis creencias están basadas más en los \anchor[http://tumblr.chlewey.net/post/31041910145/es-bueno-tener-ideas-fuertes-y-definidas-por]{principios humanistas que en los dogmas religiosos} y que por ello mismo considero irrelevante tu existencia histórica y que son más los motivos de tradición familiar los que me llevan a escribirte a ti, en lugar de escribirle a los Reyes Magos o a Papá Noel (Santa Claus).
 Un amigo imaginario es tan bueno como cualquier otro así que sigo depositando mi cariño en el que aprendí a querer.

Quiero contarte primero que me he portado bien este año.
 Aunque no lo suficiente he traído un poco de pan a mi familia y comencé un tratamiento orientado a sobreponerme ese indefinible trastorno de la personalidad que se traduce en bloqueos en mis actos.
 Está pendiente una visita al neurólogo para encontrar o descartar lesiones físicas y entenderás que esté planeando la
 cita para el regreso de las festividades de fin de año.

Sé que pedirte cosas no hará que aparezcan debajo del arbolito, pero creo que el hecho de pedírtelas sí logrará que al menos ponga mi mente en claro en conocer mis propios propósitos.
 Por ello mismo no voy a pedir cosas abstractas como la paz en el mundo sino cosas puntuales y materiales que me harán
 un tris más feliz.

Lo primero que te pido es cómo garantizar el seguro médico de mi familia, como un primer paso y la educación de mis hijos como el siguiente.
 Si bien he estado considerando el home schooling, debido a mi condición creo que por el momento es mejor dejar la
 instrucción de mis hijos en manos de profesionales; motivo por el cual debo sobreponerme a mi situación y garantizar
 los costos de la misma.

La casa te la voy a pedir el próximo año (si todo va bien).

\anchor[http://blog.chlewey.net/wp-content/uploads/2012/12/hero3.jpg]{\begin{wrapfigure}{r}{250\px}\centering% {'src': 'http://blog.chlewey.net/wp-content/uploads/2012/12/hero3.jpg', 'title': 'Go Pro Hero3 Black', 'height': '285', 'width': '250', 'alt': '[Go Pro Hero3 Black] ', 'class': ['alignright', 'size-full', 'wp-image-1376']}
\includegraphics[width=250\px,height=285\px]{blog/hero3.jpg}
\end{wrapfigure}
}Te pido también un par de jugueticos.~ (Léase un par como un puñado.)

\par% p
El primero es una buena cámara de video HD.
 Ya sabes que quiero ponerme a experimentar con la narración visual y las bitácoras en video y darle vida a\anchor[https://www.youtube.com/user/chlewey]{ mi canal de Youtube}, y hablar de filosofía y de ciencia, de sociedad y realidad y de tantos otros temas usando formatos diferentes a los
 que exploto en mi blog.

\anchor[http://blog.chlewey.net/wp-content/uploads/2012/12/dlink.jpg]{\begin{wrapfigure}{l}{250\px}\centering% {'src': 'http://blog.chlewey.net/wp-content/uploads/2012/12/dlink.jpg', 'title': 'D-Link DSL-2740B', 'height': '188', 'width': '250', 'alt': '[D-Link DSL-2740B] ', 'class': ['alignleft', 'size-full', 'wp-image-1377']}
\includegraphics[width=250\px,height=188\px]{blog/dlink.jpg}
\end{wrapfigure}
}Lo segundo es un enrutador wifi de SSID múltiple para mi casa.
 Para mí tiene mucho sentido lo del SSID múltiple de tal forma que yo pueda compartir mi conexión a internet con
 cualquier persona que pase por la calle como agradecimiento a todos los que han compartido sus redes wifi conmigo y,
 tal vez incluso, unirme a \anchor[http://www.bogota-mesh.org/]{Bogotá Mesh}.

\anchor[http://blog.chlewey.net/wp-content/uploads/2012/12/s3mini.jpg]{\begin{wrapfigure}{r}{250\px}\centering% {'src': 'http://blog.chlewey.net/wp-content/uploads/2012/12/s3mini.jpg', 'title': 'Samsung Galaxy S-III Mini', 'height': '185', 'width': '250', 'alt': '[Samsung Galaxy S-III Mini] ', 'class': ['alignright', 'size-full', 'wp-image-1378']}
\includegraphics[width=250\px,height=185\px]{blog/s3mini.jpg}
\end{wrapfigure}
}El tercer juguetico sería una herramienta móvil que me permita compartir las experiencias que tengo cuando camino por la ciudad o cuando asisto a un evento.
 Sé que es una necesidad creada y que finalmente estoy compartiendo con personas a las que en su mayoría poco les importa pero a veces me siento algo egoísta de vivir esas experiencias sólo para mí.
 Si eso se puede combinar con una buena agenda fácilmente sincronizable con todo: mucho mejor.

Un disco duro externo portable de uno o dos teras no me vendrían mal.~ Ven lo agrego a la lista.

¿Sí viste el castillo de princesas de Play Mobil?
 Ese no lo pediría para mí, sino para podérselo dar a mi chiquita, aunque no sé si eso implique también que haya que
 sumarle el castillo de caballeros para el hermanito.

Muchas gracias por la atención, mi estimado Niño Dios.~ Sin mayores esperanzas me despido:

Tu incrédulo amigo

\par% p
— \emph{Carlos}

\chapter{Percepción y realidad}
\begin{metadata}
	Published by \anchor[chlewey]{chlewey} on \anchor[http://ewey.co/B1385]{Fri, 07 Dec 2012 19:18:42 +0000}\\
	\categories{desde-youtube, filosofia, information}\\
	Shorthand: \anchor[http://blog.chlewey.net/2012/12/percepcion-y-realidad/]{percepcion-y-realidad}
\end{metadata}

\par% div% {'style': '"text-align: center;'}
\begin{iframe}% {'src': 'http://www.youtube.com/embed/phkppbvGFT4', 'style': 'float: right;', 'height': '324', 'frameborder': '0', 'width': '576'}

\end{iframe}

\chapter{De ciclos y reciclos}
\begin{metadata}
	Published by \anchor[chlewey]{chlewey} on \anchor[http://ewey.co/B1387]{Wed, 12 Dec 2012 16:22:18 +0000}\\
	\categories{basuras, information, reciclaje}\\
	Shorthand: \anchor[http://blog.chlewey.net/2012/12/de-ciclos-y-reciclos/]{de-ciclos-y-reciclos}
\end{metadata}

\par% p
Así funciona el sistema de basuras en el edificio donde vivo: hay un \emph{shut} (un tubo que va del último piso al primero con una compuerta en cada piso, dentro de un pequeño cuarto) Cada que uno
 llena una bolsa de basura en su apartamento lleva la bolsa (cerrada) al shut y la deja caer. Las bolsas caen así en
 una caneca grande. Cada vez que una caneca se llena el portero debe cambiarla por una vacía. Tres días a la semana
 pasa el camión de basuras y el portero tiene que sacar las canecas para que el camión se lleve el contenido. Es un
 proceso manual: los tripulantes del camión (tres o cuatro además del conductor) alzan las canecas y vierten el
 contenido dentro del camión compactador.

\par% p
Antes de que pase el camión pasan los recicladores. Ellos buscan entre las bolsas entre las canecas lo que ellos
 consideran utilizable. Durante este procedimiento producen algo de desorden. Algunos vecinos separamos reciclaje en la
 casa y hay dos formas usuales de sacar las basuras reciclables: la más fácil es dejar la basura reciclable en el
 cuarto de la basura de cada piso, esto es el pequeño cuarto donde está el \emph{shut}. La encargada del aseo del edificio tiene entonces que sacar esa basura reciclable (frascos, periódicos, etc.) y
 bajarlos y no estoy seguro cómo hacen para que llegue finalmente a manos de los recicladores. La otra opción (que es
 la que solemos usar en mi casa) es mantener el material reciclable en bolsas y cajas al lado de la puerta y los días
 de recolección de basuras bajamos nosotros mismos el reciclaje a la vista de los recicladores. Si igual los
 recicladores consideran que no es útil lo que presentamos como reciclaje, lo dejarán ahí, junto con el resto de la
 basura, para que los tripulantes del camión de basuras lo recojan del piso y lo echen al depósito compactador del
 camión.

Hasta el próximo 17 de diciembre, los camiones de basuras son operados por empresas particulares que tienen contrato
 con el Distrito para cubrir ciertas zonas de la ciudad. Estas empresas deben recoger la basura tres días a la semana
 (avisando a las comunidades qué días les corresponde por cada sector) y la llevan a un relleno sanitario, localizado
 en un municipio vecino a Bogotá. Como contraprestación el Distrito les paga por la cantidad de basura recolectada y
 este costo el Distrito lo cobra a los ciudadanos junto con la factura del servicio de Acueducto y Alcantarillado. Este
 pago de los ciudadanos se supone que sería proporcional al volumen de basuras generado y esta cantidad puede ser
 aforada para ajustar las tarifas.

El contrato de los actuales operadores venció hace un par de años, pero la administración distrital anterior no abrió
 licitación a tiempo y los contratos se han venido prorrogando. La administración actual de Gustavo Petro decidió que
 tampoco abriría licitación y que el Distrito asumiría directamente la recolección de basuras por medio de la pública
 Empresa de Acueducto y Alcantarillado de Bogotá, en parte considerando que los operadores privados están abusando en
 las tarifas y no están prestando un servicio eficiente.

El modelo de negocio de los recicladores es diferente. Ellos buscan entre las basuras (y lo que algunos ciudadanos
 sacamos en bolsas aparte como ``reciclable'') lo que les pueda ser de interés y luego lo venden como material recuperado
 a diversos negocios que completan el reciclaje convirtiendo el material en nuevos productos.

Ocasionalmente uno tiene que desechar productos que no caben dentro del sistema normal de basuras y reciclaje: un
 colchón viejo, un televisor inservible, escombros de construcción, etc. Estos tipos de desechos no los recoge el
 camión de la basura normalmente y es responsabilidad de quien genera estos desechos de disponer de ellos, generalmente
 contratando con un reciclador, con la empresa de recolección de basuras o con un tercero para que recojan el desecho y
 lo envíen a un centro de procesamiento o de disposición final. Muchas personas que prestan el servicio de recolección
 de escombros lo que hacen es tirar los escombros en espacios públicos no vigilados, lo cual está por fuera de los
 principios del sistema.

Los hospitales y demás instituciones que generan residuos peligrosos contratan directamente a una empresa que se
 encargue de la recolección de tales productos. Creo, igualmente, que si una empresa produce diariamente más basura de
 la que está programada por el Distrito, puede también contratar con el operador de recolección de basuras para que
 pase con mayor frecuencia (p. ej. diariamente). Para los usuarios domésticos, el costo de la recolección es apenas un
 sobrecosto de la tarifa del agua y, para efectos prácticos, es como si fuera gratis (salvo el caso mencionado de
 desechos extraordinarios).

Conocí, muy por encima, el sistema de recolección de basuras japonés. Ellos establecen diferentes días para recoger
 diferentes tipos de desechos: un día para los desechos no reciclables, otro para recolectar cartón y papel, otro para
 plásticos, otro para vidrios y otro para latas. Todos los desechos reciclables deben ir empacados en bolsas plásticas
 transparentes que se dejan en el lugar de recolección afuera del edificio. Si una bolsa contiene artículos que no
 corresponden al día, los recolectores la dejan ahí con una nota aclaratoria. Si un edificio no tiene un lugar de
 recolección central de basuras no reciclables, cada familia debe guardar la respectiva basura en su hogar hasta que
 sea el día de recolección.

No tengo idea de cómo sea el esquema de negocio del sistema japonés.

\par% p
Esto produce, sin duda, nuevos hábitos y organización dentro del hogar. Si yo no me preocupara por el reciclaje,
 simplemente sacaría todos mis desechos cada vez que una bolsa de basura se llena en el apartamento y lo lanzo por el \emph{shut}, sin tener que preocuparme por los días de recolección. Cuando vivía en casa con mis papás, ellos acumulaban la basura
 de dos o tres días y la sacaban el día de recolección. Aun recuerdo épocas, cuando la recolección la manejaba el EDIS,
 que el camión recolector pasaba con una campana y los vecinos salían con las canecas de basura al escuchar el camión,
 pero con la normalización de los horarios de recolección y el cambio en las costumbres de hogares donde ambos padres
 trabajan y sin servicio doméstico interno, simplemente se sacaba la basura el día acordado y se dejaba abandonada
 frente a la casa.

Si yo tengo que guardar mi basura no reciclable dentro de mi hogar por cuatro o siete días, tendré un incentivo para
 producir menos basura no reciclable. Si, igualmente, los desechos reciclables son recogidos con la misma eficiencia y
 organización que las basuras (y sólo me reciben el tipo de reciclaje programado para el respectivo día), tendré un
 incentivo más para separar basuras de reciclaje y separar los diferentes tipos de reciclaje. La idea es interesante.

¿Pero cómo funcionaría dentro de un esquema de costos y negocios?

Pensemos que el Distrito preste, directamente o subcontratando con particulares, el servicio de recolección de desechos
 domésticos (y oficinas y pequeñas empresas). El costo de este servicio se traslada al usuario (p. ej. como un renglón
 extra en la factura de acueducto) y usando un sistema similar al japonés: un día recolecta basuras no reciclables,
 otro día papeles y cartón, otro día plásticos reciclables, otro vidrios, otro latas, etc. Si el recolector encuentra
 un tipo de desecho que no corresponde al día lo deja ahí y anota para enviarle una carta al residente.

Si un hogar, una comunidad o una empresa necesita que la recolección de basuras no reciclables sea más frecuente,
 tendría que pagar directamente a una empresa de recolección (la empresa pública si la hay, cualquiera de las privadas
 oficiales, o algún tercero). Igual si produce desechos especiales. La recolección de escombros y desechos voluminosos
 extra se contrata también por aparte. Los hogares o empresas que dejen acumular basuras al frente serán multados
 (salvo que el problema sea atribuible a la empresa de recolección oficial en cuyo caso será esta la multada). Los
 operadores de recolección oficiales o no oficiales que no dispongan de los desechos de la forma acordada serán
 multados. Las empresas de recolección oficiales y cualquier particular (p. ej. una empresa de recolección no oficial)
 debe pagar una tarifa al operador del sitio final de disposición de basuras (p. ej. el relleno sanitario o un horno de
 cremación) por el volumen depositado, pero las multas por no depositar los desechos ahí serán mucho más costosas.

El negocio entre los recolectores y los centros de procesamiento de material reciclable es libre: si el centro le paga
 al recolector o el recolector al centro, o no hay transacción. Igual los recolectores oficiales de reciclaje reciben
 un pago del Distrito (de los usuarios en su pago de acueducto) por el servicio de recolección. Se esperaría que si el
 centro de reciclaje cobra por recibir el material cobre menos que el operador de disposición final de basuras. Igual,
 cualquier particular puede llevar basuras o reciclaje al centro respectivo y pagar o recibir un pago por el material
 entregado.

El sistema podría funcionar, pero tiene un inconveniente: no tenemos la disciplina de los japoneses.

Muchos de los actuales recicladores seguirán bajo el esquema actual: recorrer los barrios antes de que pasen los
 camiones recolectores y buscar lo que les parezca útil. Incluso les quedará más fácil porque lo útil ya está separado
 y se convierten así en una competencia ilegal a los operadores oficiales. Otros encontrarán negocio en recorrer
 después de los recolectores oficiales y recoger lo que estos están obligados a dejar, p. ej. recoger las latas que los
 recolectores de vidrios dejan. Pero este sería un problema menor en contraste con el sistema actual.

La indisciplina en los hogares es un mayor riesgo. El hogar que saca cualquier tipo de basuras sin separar y si no se
 la recogen la deja ahí y encima se queja de que la ciudad se está llenando de basuras. Las multas a los hogares que no
 colaboren puede prevenir esto en parte, pero igual muchos preferirán acumular multas sin pagar a organizarse dentro
 del hogar, mientras tanto la basura se acumula para alegría de perros callejeros, ratas y recicladores informales.

No tengo idea qué piense hacer Petro con su nuevo esquema de recolección de basuras y reciclaje, pero pensaría que será
 algo muy diferente a lo que aquí propongo.

\par% p
Un monopolio privado tiende a acortar gastos e inversión, desmejorar el servicio y subir precios al público para
 aumentar las ganancias. Un monopolio público simplemente traslada cualquier costo extra al erario, sin incentivos para
 invertir o mantener un buen servicio. Cualquier esquema de ineficiencia en la prestación de un servicio no se logra
 con sólo cambiar un monopolio privado por uno público o viceversa. Una buena competencia ayuda a optimizar los
 esquemas de prestación de servicios, pero también una regulación que promueva la innovación y multe las malas
 prácticas. Me gusta pensar en términos de incentivos a las buenas prácticas y no una fórmula de que si un gobierno es
 de izquierdas cambie monopolios privados a públicos y viceversa con un gobierno neoliberal.

\par% div% {'id': '_dyhb23rg4374'}

\chapter{Suspensión de la incredulidad y realidad secundaria }
\begin{metadata}
	Published by \anchor[admin]{admin} on \anchor[http://ewey.co/B1398]{Thu, 21 Feb 2013 15:01:24 +0000}\\
	\categories{desde-youtube, ficcion, filosofia, historia, information}\\
	Shorthand: \anchor[http://blog.chlewey.net/2013/02/suspension-incredulidad/]{suspension-incredulidad}
\end{metadata}

\begin{iframe}% {'src': 'http://www.youtube.com/embed/hSkbiFY5_wg', 'allowfullscreen': '', 'width': '560', 'frameborder': '0', 'height': '315'}

\end{iframe}

\par% p% {'style': 'padding-left: 30px;'}
\emph{Hace mucho tiempo, en una galaxia muy, muy lejana…}

Así comienza cada una de las películas de una de las sagas de ficción más populares del cine.
 Por unas dos horas de cada una de las seis películas estamos invitados a olvidarnos de nuestro mundo, nuestros
 problemas e, incluso, de las leyes de la física tal cual la conocemos para sumergirnos en el mundo de los caballeros
 Jedi, los imperios estelares y los viajes interplanetarios.

En esos momentos la realidad no es lo que hemos aprendido en la escuela sino lo que el cineasta nos muestra.
 El cineasta nos redefine el mundo.
 Ese mundo ya no es una mentira sino una realidad secundaria y dentro de esa realidad secundaria sabemos que Obi Wan
 Kenobi le miente a Luke Skywalker cuando afirma que Darth Vader asesinó a su amigo Anakin, padre de Luke.

Sabemos que toda la historia de Luke, Obi Wan y Anakin es ficticia.
 Es una creación.
 Usualmente no la llamaremos una mentira tal vez porque hacemos una distinción entre contar una historia para engañar (mentira) y contar una historia para entretener (ficción).
 Cuando George Lucas nos presenta esa historia no quiere hacernos creer que eso es una verdad fáctica sino que quiere
 que por unos instantes nos adentremos dentro de su creación y establezcamos qué es verdad y qué es mentira dentro de
 ese mundo.

Muchos autores se han puesto a discutir sobre si el Imperio es realmente tan malvado como el sesgo de las películas lo muestran o si no es la resistencia los verdaderos malos de la película.
 Una discusión que sin duda va más allá del universo presentado por Lucas y que en últimas es una discusión sobre algo
 inexistente en nuestra realidad.

La Galaxia de los Sith y los Jedi existe en nuestra imaginación; así como existe la Tierra Media de J. R. R. Tolkien, o existe el Hogwarts de J. K. Rowling.
 Y no sólo en la fantasía y la ciencia ficción, sino en toda obra de ficción, desde una comedia romántica hasta una
 telenovela.

No solo es nuestra mente capaz de suspender nuestra realidad para adentrarnos en la creación de un autor, sino que podemos hacerlo muchas veces durante un mismo período de tiempo y mantener todavía nuestra conexión con el mundo real.
 Podemos seguir varias series de televisión y telenovelas que se desarrollan durante meses y al mismo tiempo leer uno o
 dos libros de ficción y cumplir con las obligaciones de nuestro trabajo o nuestra escuela.

¿Tiene algún beneficio esto?

\par% p
Las especies más inteligentes del reino animal, como los delfines, lobos y chimpancés, son también las que presentan estructuras sociales más complejas, apartándose de un solo modelo de manada.
 Los seres humanos hemos desarrollado múltiples modelos de estructura familiar y de organización social a nivel de aldea, tribu, clan, ciudad, gremio, club y nación.
\relax{% {'style': 'color: #888888;'}
 Gran parte de nuestra capacidad cerebral está dedicada a entender las complejidades sociales que nos rodean.}

\par% p
Pero esta complejidad social la compartimos con ancestros como el \textit{Homo habilis} y un caso muy significativo fue el que ocurrió en Eurasia al final de la penúltima glaciación.

El hombre de Neanderthal apareció hace unos 400.000 años.
 Tenía un cerebro más grande que el nuestro y era físicamente más formidable y adaptado a los inviernos euroasiáticos.
 La población euroasiática de neandertales divergió de los homínidos que poblaban África en esa época y hace unos
 100.000 años hubo una catástrofe climática que devastó a casi la totalidad de la población homínida de África.

Los que sobrevivieron fueron aquellos capaces de inventar mundos mágicos e historias.
 Aquellos que tuvieron esa capacidad adicional de innovar, de recrear, de pensar más allá de lo aprendido y de
 comunicar este pensamiento.

\par% p
Tras esta cercana extinción sobrevivió lo que hoy conocemos como el hombre anatómicamente moderno: el \textit{Homo sapiens}.
 El hombre moderno con su capacidad de crear historias y de mantener activa su imaginación salió de África y llegó no
 sólo al territorio de los neandertales en Eurasia sino que eventualmente llegaría a Australia y a América.

\par% p
El hombre de Cromañón, como hoy conocemos a los primeros hombres modernos que llegaron a Europa, convivió con el hombre
 de Neandertal, llevando finalmente a este último a la extinción justo durante la era climática que era más apta para
 el \textit{Homo neanderthalis}.

Los neandertales eran ya suficientemente inteligentes para sobrevivir en un invierno euroasiático y más aptos físicamente que los hombres africanos que acababan de llegar.
 Pero la nueva concepción del mundo potenciada por el uso de la imaginación y por la capacidad de crear y narrar
 historias, llevó a que fueran estos recién llegados los que sobrevivieran y se convirtieran en nosotros.

Las historias y los mitos no son mentiras cuando podemos reconocerlos como tales.
 La ficción y nuestra capacidad para suspender la incredulidad y retomarla, nuestra capacidad para reconocer realidades
 secundarias y mantenernos en nuestra realidad primaria, ha sido una de las grandes capacidades que el hombre tuvo, no
 sólo con respecto a los otros animales sino con respecto a los demás homínidos.

Pero hay un riesgo.
 Podemos creer tanto en una realidad secundaria que no sólo convertimos los mitos en realidades sino que rechazamos las
 evidencias que la realidad nos presenta en contra de ese mito.

\chapter{Un yo agnóstico}
\begin{metadata}
	Published by \anchor[chlewey]{chlewey} on \anchor[http://ewey.co/B1407]{Wed, 17 Apr 2013 17:04:06 +0000}\\
	\categories{agnosticismo, filosofia, personal, religion, vida}\\
	Shorthand: \anchor[http://blog.chlewey.net/2013/04/un-yo-agnostico/]{un-yo-agnostico}
\end{metadata}

Hace ya mucho tiempo que he tendido a definirme como agnóstico y uno de los grandes temas que surge es si detrás de mi
 agnosticismo lo que hay es un ateísmo que no me atrevo a hacer público mientras que por otro lado mantengo una postura
 ambigua frente a la religión dentro de la cual me crié: el cristianismo católico de rito romano.

\par% p
La otra vez me autodefinía como formalmente cristiano católico, ritualmente católico no practicante y filosóficamente
 agnóstico, pero siento que ya es hora de hacer una introspección y confesarme como no cristiano. No lo soy porque no
 sólo no comparto los dogmas del cristianismo \relax{% {'style': 'opacity: 80%;'}
(posición antidogmática que es compartida por muchos otros cristianos)} y dudo de la historicidad de Jesús \relax{% {'style': 'opacity: 80%;'}
(cosa que incluso algunos clérigos consideran irrelevante)} sino que he dejado de ver a los valores cristianos como guía para mi propia vida.

\par% p
Uno no decide ser creyente o no. Uno va descubriendo sus propias creencias y los fundamentos de las mismas. No es que
 un día yo haya decidido ser agnóstico sino que un día reconocí en mi sistema de creencias lo que \anchor[http://es.wikipedia.org/wiki/Thomas\_Henry\_Huxley]{Thomas Henry Huxley} denominó agnosticismo, mientras que por otro lado reconocía los valores que mi formación cristiana me había conferido
 junto con la falta de sustento empírico sobre los dogmas. Otras personas tendrán otras experiencias que los habrá
 llevado a una postura más gnóstica sobre la existencia de un ser supremo y su alcance.

\par% p
Pero la verdad, incluso en los creyentes en un ser supremo en particular, llamémosle Dios y reconozcámoslo como el dios
 de la Biblia (¿\anchor[http://es.wikipedia.org/wiki/El\_\%28dios\_sem\%C3\%ADtico\%29]{El}? ¿\anchor[http://es.wikipedia.org/wiki/Yahveh]{Yavé}?), o cualquier otra idea de dios o dioses, no aceptan todo ese conjunto de definiciones y normas en el todo de las
 escrituras. La mayoría de los obispos de las grandes religiones cristianas organizadas, comenzando por el catolicismo
 y el Papa consideran que la Biblia es más alegórica que históricamente factual, pero incluso los evangélicos que
 claman una interpretación más literal de la Biblia escogen qué mandamientos seguir y cuales no. Sólo unos pocos
 fanáticos hoy en día considerarían, por ejemplo, que la lapidación es un \relax{% {'style': 'text-decoration: underline;', 'title': u'exagero un poco, la prohibici\xf3n del trabajo en el Sabbat proviene del talmud y no de la biblia, y no se castiga con la muerte, pero el deuteronomio si nombra algunas reglas que hoy no seguir\xedamos.'}
justo castigo} por deshonrar el sábado trabajando.

A lo largo de la biblia, tanto el antiguo como el nuevo testamento, encontramos lecciones de vida positivas y otras
 cuestionables. Un apologista dirá que lo cuestionable puede ser un problema de interpretación: que obedece a otros
 tiempos o que era un mensaje a otras culturas o que, simplemente, no tenemos la capacidad y sabiduría suficiente para
 entender el verdadero significado de esas normas.

Hay otra explicación y es creer que la biblia es un conjunto de escritos, redactado por diferentes seres humanos que
 reflejan sus históricos y culturales puntos de vista. Una obra humana que sería luego traducida, transcrita y
 corregida por otros seres humanos en otros contextos históricos.

\par% p
Sigo creyendo que en la biblia hay buenas lecciones de vida y varios sabios consejos. Pero también sucede esto con \anchor[http://es.wikipedia.org/wiki/El\_conde\_de\_Montecristo]{\emph{El Conde de Montecristo}}, o \anchor[http://es.wikipedia.org/wiki/The\_Call\_of\_the\_Wild]{\emph{El llamado de la selva}}, o \anchor[http://es.wikipedia.org/wiki/Cien\_a\%C3\%B1os\_de\_soledad]{\emph{Cien años de soledad}}, o \anchor[http://html.rincondelvago.com/juventud-en-extasis\_carlos-cuauhtemoc-sanchez\_4.html]{\emph{Juventud en éxtasis}}, o \anchor[http://es.wikipedia.org/wiki/\%C2\%BFQui\%C3\%A9n\_se\_ha\_llevado\_mi\_queso\%3F]{\emph{¿Quién se ha llevado mi queso?}}, o mi blog. Así como en todas estas obras hay elementos cuestionables tanto en la presentación fáctica como en los
 valores presentados.

Hay muchos elementos por los cuales tengo aún un apego emocional con el cristianismo en general y el catolicismo en
 particular. Confesarme no cristiano es un proceso doloroso. Pero, por otro lado, es una realidad si analizo mi sistema
 de creencias y valores. Los valores que comparto con el cristianismo son realmente valores humanistas mientras que no
 puedo esconder mi rechazo a valores bíblicos como la condena al conocimiento y la razón presentados en la fábula de la
 serpiente en el Edén o en el desprecio que se muestra a Tomás por su escepticismo.

\par% p
Pero descubrirme no cristiano no me convierte en ateo. Ni me lleva a abrazar a alguna otra religión. Mi postura sigue
 siendo agnóstica así personajes que admiro como \anchor[http://es.wikipedia.org/wiki/Richard\_Dawkins]{Richard Dawkins} consideren el agnosticismo como \anchor[http://www.youtube.com/watch?v=lQOxvTKKpOg]{una postura pobre}.

No soy agnóstico en materia religiosa porque crea que tal vez exista alguna religión válida, o porque considere
 igualmente probable la existencia o no existencia de un dios en particular o de cualquier tipo de dios… lo que me
 lleva a la siguiente pregunta: ¿qué es un dios?

\par% p
Los deístas creen en un universo creado por un ser superior, pero no creen que ese ser superior sea una persona que
 continúe acompañándonos a lo largo de la historia. ¿Pudo un elemento “consciente”, “racional” o “personal” haber dado
 origen al Big bang? ¿O haber \anchor[http://es.wikipedia.org/wiki/Abiog\%C3\%A9nesis]{creado la primera} forma de vida? ¿O crear (o formar) ese elemento metafísico que reside en nuestras mentes y que llamamos alma?

\par% p
Mi última discusión interna tiene que ver con el concepto del yo y la continuidad del yo. Tal vez estoy demasiado
 contaminado por el \anchor[http://es.wikipedia.org/wiki/Cogito\_ergo\_sum]{concepto cartesiano} pero para mí es clara una cosa: puedo dudar de todo, incluyendo de la misma composición material de mi cuerpo y mi
 mente. Pero existe un yo que se plantea esta duda y que percibe una continuidad entre un pasado que recuerdo y un
 futuro que espero o temo vivir.

A partir de este convencimiento de que yo existo construyo todo lo demás como un sistema de creencias: mi cuerpo existe
 así como las cosas y personas a mi alrededor. Estas cosas que percibo son consistentes o no con lo que otras personas
 me dicen a través de las conversaciones de la vida diaria, los libros, la televisión o Internet. Estas creencias las
 tomo por ciertas y se convierten en lo que sé. Otras de mis creencias permanecen como conjeturas: cosas que creo que
 son así pero estaría abierto a que me demuestren otra cosa. Otras creencias son más bien esperanzas: las cosas
 deberían ser así porque así el mundo sería un mejor lugar para mí.

\par% p
Así como existe este yo y creo firmemente (al punto de decir que sé que es así) que este yo reside dentro de una mente
 físicamente localizada en el cerebro del cuerpo de un animal humano (mi cuerpo tal cual lo percibo), también creo,
 también sé que existes, como una persona que es capaz de tener estos mismos pensamientos bien sea que efectivamente
 así lo pienses o no. Sé que eres una persona que resides en un \relax{% {'style': 'text-decoration: underline;', 'title': u'a menos que seas una Inteligencia artificial ley\xe9ndome en un futuro no muy lejano\u2026'}
cuerpo humano}. Que tienes un punto de vista, una propia continuidad, una propia expectativa y unos propios recuerdos. Eres tu propio
 yo.

Es ese yo (el tuyo, el mío, el de los demás) uno de los desafíos de mi mente. ¿Es algo netamente natural o corresponde
 a una sustancia metafísica?

Si es algo netamente natural cabe la duda de qué elementos naturales llevan a esta condición. ¿Qué tanto yo tiene un
 animal? ¿Podrá un robot alguna vez tener un yo?

Si, por el contrario, existe un yo metafísico. ¿De dónde proviene? ¿qué tanto se parece o no al alma inmortal que nos
 enseñan en el catequismo?

\par% p
Creo que una respuesta completamente naturalista no será completamente satisfactoria. La neurociencia nos podrá mostrar
 el cómo de nuestra mente y nuestra percepción y eventualmente nos dirá el \emph{\textbf{cómo} del yo}, pero ¿puede la neurociencia explicar el \emph{\textbf{qué} del yo}?

No es que crea que un teólogo pueda explicar el yo mejor que un neurocientífico. No es que en mi conjunto de creencias
 de qué es y qué no es yo de igual peso a la posible existencia de lo metafísico que a su probable inexistencia. Mi
 agnosticismo lo que me dicta es la creencia de que no habrá una respuesta completamente satisfactoria.

En cuanto a si creo que existe un ser personal que dio origen al universo y a la vida dentro de la tierra y que nos
 acompaña todos los días de cada una de nuestras vidas y que es supremamente poderoso, lo conoce todo y es toda bondad
 y que se nos manifiesta en tres personas distintas, una de ellas que nació y creció como un ser humano y murió como un
 ser humano pero resucitó y sigue vivo en cuerpo y alma tras casi dos mil años…, lo siento: en ese no creo.

\chapter{A journey}
\begin{metadata}
	Published by \anchor[chlewey]{chlewey} on \anchor[http://ewey.co/B1414]{Thu, 23 May 2013 17:58:27 +0000}\\
	\categories{filosofia, personal, religion}\\
	Shorthand: \anchor[http://blog.chlewey.net/2013/05/a-journey/]{a-journey}
\end{metadata}

\par% div% {'lang': 'en-US'}

\par% p
For many years I was attempting to conciliate my \anchor[http://en.wikipedia.org/wiki/Agnosticism]{agnostic internal philosophy} on the supernatural and my \anchor[http://en.wikipedia.org/wiki/Christianity]{Christian culture}. I was risen by a \anchor[http://en.wikipedia.org/wiki/Catholicism]{Catholic} family, in a Catholic \relax{% {'style': 'border-bottom: dotted #939 1px; padding-bottom: -1px;', 'title': 'Colombia'}
country} and attended to Catholic \relax{% {'style': 'border-bottom: dotted #939 1px; padding-bottom: -1px;', 'title': 'Colegio de La Salle'}
school}. Despite whatever system of believes I hold internally the Christian Catholicism is part of my cultural heritage and
 cultural self-identity.

\par% p
When I was a little kid I had personal questions on the \anchor[http://en.wikipedia.org/wiki/Afterlife]{afterlife}. The both alternatives I could grasp as a kid were unsettling, disturbing: that my being were finite (v. g. eternal
 nothingness after you die) or my being were eternal (regardless of heaven or hell). Now, looking in retrospective, it
 seems I never had a deep believe on \anchor[http://en.wikipedia.org/wiki/Heaven]{heaven}, \anchor[http://en.wikipedia.org/wiki/Limbo]{limbo} or \anchor[http://en.wikipedia.org/wiki/Hell]{hell}.

Yes, I was worried that my acts and thoughts lead me to hell, but I could not actually picture in my mind the eternal
 torment of hell, or the eternal glory of heaven. My thoughts, speculations and fears only lead me to the judgment
 after death, not actually to the eternal suffering or eternal wellbeing.

\par% p
Reincarnation seemed less disturbing but yet. The idea that my being would start over again and again, probably for
 eternity but without memory of an eternal past. There was however a disturbing thought, anyway: if my future \emph{% {'title': 'self'}
me} would not remember my present \emph{% {'title': 'self'}
me} it would mean that my present \emph{% {'title': 'self'}
me} would be finished, dead. My future \emph{% {'title': 'self'}
me} would not be an afterlife. Anyhow, while less disturbing, I did not actually believed in reincarnation.

So that was me as a kid, immersed in a Christian culture, being taught at home, at mass and at school what to believe
 on the nature of humankind, history, morality and afterlife from a Catholic point of view.

\par% p
Unlike other Christian denominations, mainstream Catholicism does not hold a fight against \anchor[http://en.wikipedia.org/wiki/Limbo]{Science} and evidence-based understanding of the natural world and human history. I went to a confessional school, ruled by the
 Brothers of the Christian Schools (\anchor[http://en.wikipedia.org/wiki/Institute\_of\_the\_Brothers\_of\_the\_Christian\_Schools]{Lasallian Brothers}), since fourth grade. (I took third grade in a lay-ruled confessional school; before that, I went to public school in \relax{% {'style': 'border-bottom: dotted #939 1px; padding-bottom: -1px;', 'title': 'Sweden'}
a secular country.)}

\par% p
At school I was taught both the \anchor[http://en.wikipedia.org/wiki/Genesis\_creation\_narrative]{myth of creation} and the \anchor[http://en.wikipedia.org/wiki/Evolution]{theory of evolution}, as well as a few theories on the origin of the Universe. Evolution and creation were not taught as a \anchor[http://en.wikipedia.org/wiki/Creation\%E2\%80\%93evolution\_controversy]{controversy}. Evolution was taught in science class, where we were exposed to the origins of \anchor[http://en.wikipedia.org/wiki/Charles\_Darwin]{Darwin} thesis as opposed to \anchor[http://en.wikipedia.org/wiki/Jean-Baptiste\_Lamarck]{Lamarck}, and how \anchor[http://en.wikipedia.org/wiki/Spontaneous\_generation]{spontaneous generation} was discarded. We learned on \anchor[http://en.wikipedia.org/wiki/Mendelian\_inheritance]{Mendelian inheritance} and \anchor[http://en.wikipedia.org/wiki/DNA]{DNA}, and the possible \anchor[http://en.wikipedia.org/wiki/Abiogenesis]{origin of life from natural processes}. We were taught on the origin of Earth and the solar system. On how to prove that the Earth is round and orbits around
 the Sun. We were exposed to some thesis on the existence of the Universe including the thesis of the \anchor[http://en.wikipedia.org/wiki/Static\_universe]{static Universe} and the \anchor[http://en.wikipedia.org/wiki/Metric\_expansion\_of\_space]{expanding Universe}, and how probably it all begun in an event called the \anchor[http://en.wikipedia.org/wiki/Big\_Bang]{Big Bang}.

\par% p
I was taught the biblical creation story in religion class in third grade, before any scientific theory on the origin
 of the universe or the diversity of life. But when the conflict came between the biblical story and what we were
 exposed in science class we were taught the official position of the Catholic Church after Vatican II: the Bible is
 not a book of science and history, but \anchor[http://en.wikipedia.org/wiki/Biblical\_infallibility]{the Bible is infallible} in theological matters. The creation story in the Bible is allegorical.

There was no controversy. I never heard a priest negating the scientific theses. I never heard a science professor
 claiming that the Bible was wrong.

\par% p
Religion class in La Salle school was not only \anchor[http://en.wikipedia.org/wiki/Catechesis]{catechesis}. We were taught on the origin and existence of religions, including some of the theses of other religions. We were
 taught on the question of \anchor[http://en.wikipedia.org/wiki/Historicity\_of\_Jesus]{historicity of Jesus} (of course, it was a confessional school so the conclusion was that \anchor[http://en.wikipedia.org/wiki/Jesus]{Jesus} indeed existed as a human being).

\par% p
There was no controversy between science and religion in my mind either. I accepted most of what I learned in school on
 scientific matters, probably because that was also consisting with one of my favorite TV shows: \anchor[http://en.wikipedia.org/wiki/Carl\_Sagan]{Carl Sagan}'s \anchor[http://en.wikipedia.org/wiki/Cosmos:\_A\_Personal\_Voyage]{\emph{Cosmos}}. I did question a little more what I was taught in religion class. I didn't question religion for the magic. I had
 already conciliated that part. I didn't question the \anchor[http://en.wikipedia.org/wiki/History\_of\_ancient\_Israel\_and\_Judah]{history of ancient Israel}, probably because there was no much to compare it against. What I questioned was the theology.

\par% p
I remember, when I was 14, our religion teacher challenged us to express what we believed and what we didn't believe. I
 don't recall what exactly I believed back then but I do remember what I claimed and why. I said I was an atheist and
 my main supporting thesis was the \anchor[http://en.wikipedia.org/wiki/Problem\_of\_evil]{problem of evil}. \relax{% {'style': 'color: #060; font-style: italic;'}
How could an omnipotent, omniscient, omnibenevolent god allow evil?} Another question was the scope of the revelation. \relax{% {'style': 'color: #060; font-style: italic;'}
Why isn't the revelation universal?} \relax{% {'style': 'color: #060; font-style: italic;'}
Why are there other religions?} It seemed to me that God were rather a human construct rather than humans a creation from God.

\par% p
Religions have a \textbf{mythical} aspect and I accepted part of the Christian myth (v. g. the history of the Kingdom of Israel) and rejected other (the
 literal biblical creation). But according to post-Vatican II, the believe in the myth is not fundamental. Religions
 have \textbf{theological} aspect, v. g. what's the nature of God. I'm not sure what I accepted or what I rejected, in theological terms, back
 when I was 14. Religions have a \textbf{ritual} aspect, and when I was 14 I was a reluctantly practicing Roman Catholic. Religions have a \textbf{cultural} aspect, and I deeply identified myself a Catholic and a Christian back then, and probably I currently do. Religions
 have a \textbf{moral} aspect, however I am not sure how separated is this aspect from the cultural one. Religions have a \textbf{spiritual} aspect, the personal feelings a human being experiments as interpreted by the theology, the practice and the culture
 of a religion. I guess I held that spiritual aspect back then when I claimed to be and atheist, and those feelings
 have been one of the main reasons I had rejected the label of “atheist” later in my adult life.

\par% p
But, as I said, my school was a confessional Catholic school, and one of the requirements for taking my high school
 degree was that I were \anchor[http://en.wikipedia.org/wiki/Confirmation\_\%28Catholic\_Church\%29]{confirmed} as a Catholic. So when I was 16, I took the confirmation catechesis and confirmation ceremony with a mixed feeling. I
 rationalized that if I was indeed an atheist, the confirmation would not hurt me. On the other hand I would put my
 best to sincerely commit as a good Catholic.

\par% p
So I took the confirmation. Later that year I took my high-school degree, and the next year my family moved back to
 Sweden. I had lived in Sweden when I was 6–7, and previously in this post I described it as a \anchor[http://en.wikipedia.org/wiki/Secular\_state]{secular country}. I'm not sure if Sweden pass a strict definition of secular as the \anchor[http://en.wikipedia.org/wiki/Church\_of\_Sweden]{Church of Sweden} was an integral part of the Swedish state and hold the registration of every Swedish citizen, but for practical
 matters it behaved as a secular country with freedom of religion and freedom from religion.

Of all religions' aspects (mythical, theological, ritual, cultural, moral, spiritual) the cultural aspect was a key
 element when I was 18–19 years old living in Sweden. I became a less-reluctant practicing Catholic, and when I was 19
 I joined a youth group at church. I committed to renew my theology and my morality. I wanted to be a good Catholic
 Christian.

I had friends from different religious backgrounds and different commitment on their own believes, and different cultural expectations.
 One thing is to know from textbooks other religions exists, another one is to share with them.

\par% p
I eventually came back and began college in a \relax{% {'style': 'border-bottom: dotted #939 1px; padding-bottom: -1px;', 'title': 'Pontificia Universidad Javeriana'}
Jesuit ruled University}. While at the beginning I wanted to continue my religious renewal and to join some religious student group in College
 (there were plenty) I didn't (I later joined a student group that had no religious purpose in their chart).

\par% p
However I kept realizing that I didn't hold a deep believe in God. No matter my attempts to be a good Christian, I was
 relegating the myth on God (father) to the same drawer I had relegated the myth of Creation, the myth of Abraham, the
 myth of Moses, and the myth of Jesus. (If you are offended by my use of the word “\anchor[http://en.wikipedia.org/wiki/Myth\_\%28disambiguation\%29]{myth}” I am not claiming that a myth is a falsehood, but rather a story, a narrative.) I guess I never internalized the
 mystery of the \anchor[http://en.wikipedia.org/wiki/Trinity]{Holy Trinity} or understood what exactly the \anchor[http://en.wikipedia.org/wiki/Holy\_Spirit\_\%28Christianity\%29]{Holy Spirit} was.

\par% p
For many Christian theologians (including many Catholic theologians) no part of the biblical myth is sacred. Even
 Episcopal Bishop \anchor[http://en.wikipedia.org/wiki/John\_Shelby\_Spong]{John Shelby Spong} claims that the myth, including the God's myth, is a burden to Christianity (but I hadn't read Spong's thesis back
 then). Just as the creation myth was proved false any other myth could be proven false and the Christian faith should
 not be invalidated.

I still praised the Christian morality based on the (alleged) teachings of Jesus. I still felt the cultural communion
 with my fellow Christians, and while my reason was telling me that God was not necessary, my feelings kept me thinking
 that there should still be something out there.

\par% p
So I rediscovered the term “\anchor[http://en.wikipedia.org/wiki/Agnosticism]{agnosticism}”. I didn't have a strong believe in the supernatural, including the Christian myths but I couldn't prove them false,
 either. I still felt identified as a Catholic. I still felt that there might be something out there. I would not
 embrace the term “atheist” as it would have meant a rejection, rather than an incredulity of the myths, and would have
 meant resigning as a Catholic.

\par% p
I found many meanings of what agnosticism was, and somehow I adhered (or reinterpreted) \anchor[http://en.wikipedia.org/wiki/Thomas\_Henry\_Huxley]{Thomas H. Huxley}'s original definition of the term: I lacked an enough feeling of certainty on the trueness or falsehood of the thesis
 of a supreme being.

I was 27 or 28 when I reached this conclusion. I didn't need a supreme being to understand the universe. I didn't need
 a supreme being to explain my feelings on spirituality. I didn't need a supreme being to identify myself as a Catholic
 Christian. I didn't need a supreme being to justify morality. A supreme being has not been proved by science, and a
 supreme being has not been disproved by science (and it seemed to be unknowable by science). So I was still a
 non-practicing Roman Catholic Christian and I was an agnostic.

\par% p
I knew about \anchor[http://en.wikipedia.org/wiki/Richard\_Dawkins]{Richard Dawkins} by his work \anchor[http://en.wikipedia.org/wiki/The\_Selfish\_Gene]{\emph{The Selfish Gen}}. I read that book when I was younger, and found a compelling case on how Evolution works. I didn't knew and didn't
 care that Dawkins was an atheist. I also read \anchor[http://en.wikipedia.org/wiki/The\_Universe\_in\_a\_Nutshell]{\emph{The Universe in a Nutshell}} by \anchor[http://en.wikipedia.org/wiki/Stephen\_Hawking]{Stephen Hawking}, and I found intriguing the concept of a Universe without a clear beginning in the singularity as a plausible thesis
 that left no room for a creator.

\par% p
While I didn't came to \anchor[http://en.wikipedia.org/wiki/The\_God\_Delusion]{\emph{The God Delusion}} or the later works by Dawkins, when I first learned he advocated that religious belief is incompatible with science my
 reaction was that Dawkins failed. After all, there are many people who hold a religious belief and do good science. I
 also respected \anchor[http://en.wikipedia.org/wiki/James\_Randi]{the Amazing Randi} for his works against pseudoscience and debunking charlatans, so I felt somehow betrayed when I realized Randi
 included organized religion into the things he was against.

\par% p
The first time I found a \anchor[http://en.wikipedia.org/wiki/Young\_Earth\_creationism]{Young Earth Creationism} website my reaction was one of incredulity. How can people still believe on the creation myth and claim evidence-based
 prof on that? Wasn't the whole Creation/Evolution polemic solved in the US back in late 19th century or early 20th
 century?

\par% p
Of course, I knew there was people who believed literally in the Bible, people who refused medicine preferring praying
 instead. I knew Ned Flanders in \emph{The Simpsons} reflected a reality. What I didn't knew is that some of these people were attempting to fight science in scientific
 matters and posing as scientists.

As a young Catholic I grew up without that dichotomy between science and religion. Even if I chose a less religious
 path in my life many of my classmates became later more religious people (inside and outside the Catholic Church). I
 know of priests and pastors doing hard science. So why should these (mostly evangelical) Christians need to undermine
 science to promote their religiousness?

During the last year I joined the debate. What is the motivation of Dawkins and Randi to fight religion? What is the
 motivation of evangelical Christianity to undermine science? What does that mean for my understanding of the World.

\par% p
I still disagree with Dawkins: a person can hold a religious belief and engage in hard evidence-based science. I still
 disagree with fundamental skepticism that claim religious based believes incompatible with \anchor[http://en.wikipedia.org/wiki/Skepticism]{skepticism} (you can believe whatever you want, religious or non-religious, as longer as you are willing to question your
 believes). I cannot respect the position of some religious leaders and apologetics who engage in undermining
 evidence-based science (I would respect a position that science is mundane and therefore unimportant, even if you
 consume the fruits of science). I find fascinating the thesis by Spong on a non-theistic Christianity. I don't find
 contradictory my last years in which I tried to conciliate a Christian identity and an agnostic philosophy on the
 supernatural and the supreme beings.

\relax{% {'style': 'color: #006600;'}
But that's not me anymore}.

While I still appreciate and hold some Christian values, I cannot relate to the whole Christian morality (whatever
 version) in a way I cannot honestly claim to be Christian any more. I cannot relate to some of the teachings from
 Jesus (or alleged to Jesus the Christ).

\par% p
I now recognize that my spiritual feeling that there might be something out there is just an internal feeling. I
 realize that finding an spiritual meaning of my life might be more comfortable and might help me overcome some
 personal problems. \relax{% {'style': 'color: #006600;'}
But I cannot trust a comfortable lie}. A comfortable idea in which I cannot believe.

\par% p
I'm still culturally Christian. I'm still culturally Roman Catholic. I relate to the myths and stories. I relate to the
 celebration of \anchor[http://en.wikipedia.org/wiki/Christmas]{Christmas} and I love the myth of \anchor[http://en.wikipedia.org/wiki/Nativity\_scene]{the manger}. But cultural identity in absence of any other religious element is not enough for me to call myself a Christian.

I'm not a Christian. I'm not a theist. Neither hold I a pantheistic world view or any other non-theistic religious
 world view.

\par% p
I have realized that when I claim I don't need a superior being to understand my world, my epistemology, my ethic and
 moral code, etc. while not having a personal connection to the idea of a supreme being, that's quite much the
 definition of an \anchor[http://en.wikipedia.org/wiki/Atheism]{\textbf{atheist}}: someone who lacks a believe in a god or gods.

\par% p
I am still an agnostic, as I described myself some 13 years ago. I have no comfortable amount of certainty on the
 trueness or falsehood of the thesis of existence of supreme beings. But for any practical purpose that also mean that
 I lack believe in a supreme being. I embrace the term “atheist” without the need \relax{% {'style': 'border-bottom: dotted #939 1px; padding-bottom: -1px;', 'title': 'agnostic-atheist'}
to dash it} with me being an agnostic.

\relax{% {'style': 'color: #006600;'}
Is this the end of the road in my personal search for my world view?} Probably not. From that little Catholic boy who found uneasy the idea of an eternal afterlife (even in Heaven) to the
 40 year old who still struggles to accept he being an atheist rather than a Christian, my life has brought me several
 ways to look at the religious question and probably will bring me some other challenges in my worldview.

\chapter{Materialismo apático}
\begin{metadata}
	Published by \anchor[chlewey]{chlewey} on \anchor[http://ewey.co/B1426]{Sun, 16 Jun 2013 00:14:51 +0000}\\
	\categories{agnosticismo, ateismo, filosofia, information, religion}\\
	Shorthand: \anchor[http://blog.chlewey.net/2013/06/materialismo-apatico/]{materialismo-apatico}
\end{metadata}

\anchor[https://en.wikipedia.org/wiki/File:The\_Creation\_of\_Adam.jpg]{\begin{wrapfigure}{r}{300\px}\centering% {'src': 'https://upload.wikimedia.org/wikipedia/commons/6/63/The_Creation_of_Adam.jpg', 'alt': u'[La creaci\xf3n de Ad\xe1n] ', 'height': '150', 'class': ['alignright'], 'width': '300'}
\includegraphics[width=300\px,height=150\px]{blog/The_Creation_of_Adam.jpg}
\end{wrapfigure}
}En \anchor[http://blog.chlewey.net/2013/05/a-journey/]{mi anterior post sobre mi viaje personal} en cuestiones de fe explicaba el papel que el \anchor[http://en.wikipedia.org/wiki/Young\_Earth\_creationism]{creacionismo de tierra joven} jugó en el descubrimiento de mi posición filosófica frente a la religión. Paradójico en mi caso es que la versión de
 cristianismo en la cual fui criado es una versión abierta a la ciencia, motivo por el cual nunca hubo una crisis
 personal en mí entre tener que aceptar la realidad científica por un lado y la fe cristiana por el otro. Pero la sola
 exposición de esa otra versión de cristianismo me hizo concentrarme en un debate en el cual mis conclusiones
 personales no fueron favorables a la catequesis.

\anchor[https://es.wikipedia.org/wiki/Bible\_Belt]{\begin{wrapfigure}{r}{300\px}\centering% {'src': 'https://upload.wikimedia.org/wikipedia/commons/c/c2/BibleBelt.png', 'alt': 'Bible Belt', 'height': '195', 'class': ['alignright'], 'width': '300'}
\includegraphics[width=300\px,height=195\px]{blog/BibleBelt.png}
\end{wrapfigure}
}Desde los años 1960 en el cinturón bíblico de los EE.UU. hubo un resurgimiento del cristianismo evangélico, una versión
 del cristianismo de origen protestante basada en la experiencia personal de aceptación de Jesús, la lectura de la
 biblia, y la guía de figuras carismáticas que sirven como pastores o evangelistas. Hay un par de versículos en la
 biblia, cuya cita no recuerdo ni me interesa, que dicen que si la escritura (la biblia), o la palabra (Jesús) no es
 confiable en términos del mundo real, no sería confiable en términos de fe.

La ciencia moderna, sin embargo, contradice muchas de las cosas que figuran en la biblia tales como una tierra plana en
 el centro del universo y un firmamento de agua bajo el cual giran el sol, la luna y las estrellas. Muchas de las
 grandes iglesias cristianas, incluida la Iglesia Católica, la Iglesia Anglicana y varias de las grandes denominaciones
 protestantes, han concluido que la base de la revelación cristiana no está en una lectura literal de la biblia. Que la
 biblia es inefable como documento teológico y no debe ser tomado como un texto de ciencia o de historia.

\anchor[https://commons.wikimedia.org/wiki/File:Editorial\_cartoon\_depicting\_Charles\_Darwin\_as\_an\_ape\_\%281871\%29.jpg]{\begin{wrapfigure}{r}{223\px}\centering% {'src': 'https://upload.wikimedia.org/wikipedia/commons/6/6f/Editorial_cartoon_depicting_Charles_Darwin_as_an_ape_%281871%29.jpg', 'alt': '', 'height': '300', 'class': ['alignright'], 'width': '223'}
\includegraphics[width=223\px,height=300\px]{blog/Editorial_cartoon_depicting_Charles_Darwin_as_an_ape__1871_.jpg}
\end{wrapfigure}
}Dentro del cristianismo evangélico (que muchas veces rechaza el término “evangélico” para denominarse simplemente
 “cristianismo” como si las demás corrientes cristianas no fueran lo verdadero) surgen ideólogos que comparten otra
 visión. La aparente contradicción entre la biblia y la ciencia es el resultado de una comunidad científica que se ha
 alejado de Dios por culpa de la arrogancia de sus miembros y la guía de figuras satánicas como \anchor[https://es.wikipedia.org/wiki/Charles\_Darwin]{Charles Darwin}.

\par% p
En conclusión, el creacionismo de tierra joven, de origen evangélico pero que ha trascendido a otras denominaciones
 cristianas, incluidos sectores “conservadores” del catolicismo, y del judaísmo se ha puesto a la tarea de crear su
 propia ciencia, teniendo entre sus más notables exponentes a \anchor[https://en.wikipedia.org/wiki/Ken\_Ham]{Ken Ham} de \anchor[https://en.wikipedia.org/wiki/Answers\_in\_Genesis]{\emph{Answers in Genesis}} y \anchor[https://en.wikipedia.org/wiki/Kent\_Hovind]{Kent Hovind}.

\begin{wrapfigure}{r}{170\px}\centering% {'width': '170', 'align': 'alignright', 'id': ''}
\anchor[http://ololo.fm/artist/photos/Kent+Hovind]{\includegraphics[width=170\px,height=200\px]{blog/Kent+Hovind+Hovind2.jpg}} Kent Hovind
\end{wrapfigure}

\begin{wrapfigure}{r}{148\px}\centering% {'width': '148', 'align': 'alignright', 'id': ''}
\anchor[http://livinglifewithoutanet.wordpress.com/2011/05/07/ken-ham-christian-schools-arent-christian-enough/]{\includegraphics[width=148\px,height=200\px]{blog/kenhamgraybeard.jpg}} Ken Ham
\end{wrapfigure}

\par% p
La ciencia promocionada por Ham, Hovind y demás líderes de la tierra joven, es contraria al consenso científico en \anchor[https://es.wikipedia.org/wiki/F\%C3\%ADsica]{física}, \anchor[https://es.wikipedia.org/wiki/Astrof\%C3\%ADsica]{astrofísica}, \anchor[https://es.wikipedia.org/wiki/Cosmolog\%C3\%ADa]{cosmología}, \anchor[https://es.wikipedia.org/wiki/Geolog\%C3\%ADa]{geología}, \anchor[https://es.wikipedia.org/wiki/Paleontolog\%C3\%ADa]{paleontología}, \anchor[https://es.wikipedia.org/wiki/Antropolog\%C3\%ADa]{antropología}, \anchor[https://es.wikipedia.org/wiki/Biolog\%C3\%ADa]{biología} y casi cualquier otra rama de la ciencia; aunque, desde su punto de vista todo a lo que se oponen lo enmarcan dentro
 del nombre de \emph{evolucionismo} o \emph{darwinismo}. Darwin y su legado es el principal escollo a vencer, y por ello el debate que pretenden entablar se suele titular
 como “\textbf{creación v/s evolución}”.

\par% p
La comunidad científica no guarda mayor respeto por lo que la ciencia creacionista propone; así como tampoco considera
 científicas las hipotesis de la \anchor[https://es.wikipedia.org/wiki/Astrolog\%C3\%ADa]{astrología}, el \anchor[https://es.wikipedia.org/wiki/Espiritismo]{espiritismo}, la \anchor[https://es.wikipedia.org/wiki/Ufolog\%C3\%ADa]{ufología}, la \anchor[https://es.wikipedia.org/wiki/Dian\%C3\%A9tica]{dianética}, la \anchor[https://es.wikipedia.org/wiki/Homeopat\%C3\%ADa]{homeopatía} y muchas otras propuestas que normalmente enmarcan bajo el título de pseudociencia.

\par% p
Pero \textbf{¿qué es la ciencia?}

\anchor[http://www.wondercafe.ca/discussion/religion-and-faith/richard-dawkins-slammed-peter-higgs]{\begin{wrapfigure}{r}{300\px}\centering% {'src': 'http://4.bp.blogspot.com/_VRO1mvfM3Cc/TAVWhvOKbcI/AAAAAAAAAAM/-6SeVeT7M0Y/S724/method.jpg', 'alt': '', 'height': '225', 'class': ['alignright'], 'width': '300'}
\includegraphics[width=300\px,height=225\px]{blog/method.jpg}
\end{wrapfigure}
}La ciencia es, principalmente, un método. La ciencia consiste en formular hipótesis que tratan de describir cómo
 funciona el mundo. Estas hipótesis deben estar basadas en observaciones. Estas hipótesis deben permitir hacer
 predicciones. Estas hipótesis deben ser \anchor[https://es.wikipedia.org/wiki/Falsabilidad]{falsables}, esto es que sobre las predicciones, existen resultados sujetos a prueba que pueden corroborar falsa la hipótesis. Si
 los resultados de las pruebas de laboratorio u observaciones posteriores para probar las predicciones no contradicen
 la hipótesis y no hay otras hipótesis falsables y probadas que expliquen mejor el fenómeno, la hipótesis es reconocida
 por la comunidad científica como una teoría. La teoría está abierta, sin embargo, a que más adelante nuevos datos,
 nuevas observaciones y nuevas hipótesis contradigan o mejoren la teoría.

\par% p
Existen mecanismos tales como la \anchor[https://es.wikipedia.org/wiki/Revisi\%C3\%B3n\_por\_pares]{revisión por pares} y requisitos de \anchor[https://es.wikipedia.org/wiki/Publicaci\%C3\%B3n\_acad\%C3\%A9mica]{publicación} a las que deben someterse las teorías científicas antes de ser aceptadas como tales.

\par% p
Las pseudociencias se escapan de este método, estableciendo hipótesis no falsables y evitando el escrutinio de la
 comunidad científica, casi siempre bajo la excusa de que la comunidad científica es una comunidad cerrada a ideas poco
 convencionales, contaminada por ideas fijas (como el \emph{darwinismo}, según los creacionistas de tierra joven), arrogante frente a la disidencia.

\par% p
Las ciencias, y particularmente las \textbf{\anchor[https://es.wikipedia.org/wiki/Ciencias\_naturales]{ciencia naturales}}, tienen bajo su base de trabajo lo que se conoce como naturalismo o \textbf{materialismo científico}. El materialismo científico es una limitación del alcance de las ciencias naturales. Básicamente significa que las
 ciencias sólo se ocupan del mundo material ofreciendo respuestas sobre el mundo material.

\par% p
Las \textbf{\anchor[https://es.wikipedia.org/wiki/Ciencias\_sociales]{ciencias sociales}} tratan sobre el hombre y dada las complejidades de las interacciones humanas y la dificultad de predicciones, algunos
 científicos catalogan a la \anchor[https://es.wikipedia.org/wiki/Econom\%C3\%ADa]{economía}, la \anchor[https://es.wikipedia.org/wiki/Psicolog\%C3\%ADa]{psicología} y otras ciencias sociales como pseudociencias.

\par% p
Otro tipo de ciencias, como las \textbf{\anchor[https://es.wikipedia.org/wiki/Ciencias\_formales]{ciencias formales}} tales como la \anchor[https://es.wikipedia.org/wiki/Matem\%C3\%A1tica]{matemática} y la \anchor[https://es.wikipedia.org/wiki/L\%C3\%B3gica]{lógica}, por ejemplo no nos dicen mucho sobre el mundo material, pero ofrecen un lenguaje con el cual se puede describir,
 entre otras cosas, las ciencias naturales.

\par% p
La \anchor[https://es.wikipedia.org/wiki/Filosof\%C3\%ADa]{filosofía} no es una ciencia. Parte de la filosofía es una metaciencia que valida lo que es el conocimiento y a la ciencia y su
 método como una respuesta a qué es la verdad sobre el mundo material, pero abarca más elementos de la existencia
 humana como la validación o no de la política, la ética y la práctica humana.

\anchor[http://blog.chlewey.net/wp-content/uploads/2013/06/conocimiento.png]{\begin{wrapfigure}{r}{300\px}\centering% {'src': 'http://blog.chlewey.net/wp-content/uploads/2013/06/conocimiento-300x300.png', 'alt': 'conocimiento', 'height': '300', 'class': ['alignright', 'size-medium', 'wp-image-1429'], 'width': '300'}
\includegraphics[width=300\px,height=300\px]{blog/conocimiento-300x300.png}
\end{wrapfigure}
}En un diagrama que vi en la escuela, mostraban a la \anchor[https://es.wikipedia.org/wiki/Teolog\%C3\%ADa]{teología} como una capa superior a la filosofía, una justificación trascendental de que el conocimiento humano: filosofía,
 ciencias naturales y formales y las prácticas humanas son válidos por provenir de Dios, pero esta es una visión
 claramente teísta y como tal no sería compartible con personas que tienen otra idea sobre el sentir, práctica y
 filosofía religiosos.

Regresando a las ciencias naturales, el materialismo científico considera que todo lo no material, por ejemplo lo
 sobrenatural, es irrelevante para la ciencia pues no agrega nada al conocimiento científico. Una explicación no
 falsable sobre la causa o propósito último de un fenómeno no permite una mejor comprensión sobre el cómo, que es de lo
 que trata la ciencia.

\anchor[https://commons.wikimedia.org/wiki/File:Sir\_Isaac\_Newton\_\%281643-1727\%29.jpg]{\begin{wrapfigure}{r}{245\px}\centering% {'src': 'https://upload.wikimedia.org/wikipedia/commons/8/83/Sir_Isaac_Newton_%281643-1727%29.jpg', 'alt': '', 'height': '300', 'class': ['alignright'], 'width': '245'}
\includegraphics[width=245\px,height=300\px]{blog/Sir_Isaac_Newton__1643-1727_.jpg}
\end{wrapfigure}
}Hace 400 años, cuando el método científico empezaba a desarrollarse, aun eran muchas las cosas que la ciencia no podía
 explicar y esto dejaba un gran espacio a la teología para justificar una intervención directa de Dios como causa
 próxima de muchos fenómenos, pero a medida que el conocimiento científico se ha desarrollado gran parte de ese espacio
 se ha venido cerrando. \anchor[https://es.wikipedia.org/wiki/Isaac\_Newton]{Newton}, uno de los más grandes científicos de todos los tiempos, no pudo resolver el problema de la estabilidad del sistema
 solar y apeló a Dios como la mano invisible que permitía que todo funcionara y a los 36 años dejó de pensar en el
 problema para dedicarse a la alquimia y la meditación metafísica. Cien años después y usando las mismas fórmulas y la
 misma matemática que desarrolló Newton, \anchor[https://es.wikipedia.org/wiki/Pierre\_Simon\_Laplace]{Laplace} completó el problema sin necesidad de recurrir a la hipótesis de Dios.

\par% p
La cada vez menor intervención de una causa divina como explicación del mundo material ha desarrollado una visión que
 podemos llamar naturalismo o \textbf{materialismo filosófico} o materialismo metafísico.

\par% p
El materialismo científico nos dice que la ciencia sólo establece verdades sobre el mundo natural apelando a
 explicaciones naturales. El materialismo filosófico nos dice \textbf{que no existe nada más} que el mundo natural. Como la hipótesis de Dios no tiene poder explicativo, entonces se descarta la existencia de Dios.

\anchor[https://commons.wikimedia.org/wiki/File:Dawkins\_at\_UT\_Austin.jpg]{\begin{wrapfigure}{r}{200\px}\centering% {'src': 'https://upload.wikimedia.org/wikipedia/commons/7/77/Dawkins_at_UT_Austin.jpg', 'alt': '', 'height': '300', 'class': ['alignright'], 'width': '200'}
\includegraphics[width=200\px,height=300\px]{blog/Dawkins_at_UT_Austin.jpg}
\end{wrapfigure}
}Hasta donde tengo entendido Charles Darwin no se adhirió al materialismo filosófico y adoptó el termino acuñado por su
 amigo Thomas Henry Huxley de agnosticismo para describir su visión personal. El principal exponente de la evolución
 darwiniana en la actualidad: \anchor[https://es.wikipedia.org/wiki/Richard\_Dawkins]{Richard Dawkins}, sí es un adherente del materialismo filosófico. Claramente el materialismo filosófico lleva a una conclusión: Dios no
 existe. Todo lo que esté por fuera del mundo natural, todo lo que no tenga un efecto cuantificable sobre el mundo
 natural no existe y un concepto sobre un dios o un conjunto de deidades que no sean falsables ni medibles no tienen
 efectos cuantificables sobre el mundo material.

Hay, sin embargo, dos falacias a evitar. La primera es obvia: el materialismo científico no es equivalente al
 materialismo filosófico. El materialismo científico habla sobre los alcances y los límites de las ciencias naturales
 (y probablemente de las ciencias sociales), pero no nos dice nada sobre lo trascendente, y ello da lugar a que
 personas de distintos credos religiosos puedan hacer ciencia de verdad, salvo que sus propios principios religiosos se
 lo impidan. El materialismo filosófico niega la trascendencia.

La otra falacia consiste en creer que si bien el materialismo filosófico lleva al ateísmo, el ateísmo se base en el
 materialismo filosófico.

\par% p
El ateísmo es simplemente la falta de creencia en la existencia de deidades en general y de Dios en particular. Una de
 las manifestaciones de la falta de creencia es la \textbf{negación de la existencia}. La afirmación de que Dios y las deidades no existen. Y una de las razones para llegar a esta negación es la
 adherencia al materialismo filosófico. Sin embargo podría negarse la existencia de Dios sin necesitad de adoptar el
 materialismo; bien sea por rebeldía, creencia en algo distinto a deidades teístas como el \anchor[https://es.wikipedia.org/wiki/Pante\%C3\%ADsmo]{panteísmo}, falta de exposición a la hipótesis de deidades, etc.

Otra manifestación de la falta de creencia en Dios o en deidades es simplemente falta de creencia. No se niega a Dios,
 simplemente no se requiere asumir su existencia. No necesariamente el materialismo filosófico lleva a una negación de
 Dios, porque si bien declara que lo que está por fuera del mundo material no existe, podría aceptarse que
 eventualmente logre demostrarse por medio de hipótesis explicativas y falsables que hay atributos de Dios con efectos
 predictivos y cuantificables en el mundo material. Simplemente que mientras tal evidencia aparezca, la posición más
 razonable, para los adherentes de esta doctrina, es no creer en la existencia de deidades o de Dios.

\par% p
Pero también hay muchas otras razones para carecer de una creencia en la existencia de Dios, además del materialismo.
 Se puede ser simplemente \textbf{irreligioso}. Una persona que no ha sido criada dentro de dogmas religiosos podría no adoptar una idea sobre la existencia de seres
 supremos. Estoy pendiente de confirmar estudios que sugieren una predisposición a crecer con la idea de un dios, pero
 la observación no científica a la que he tenido acceso es que hijos de padres que no inculcan una idea de un dios
 parece que dejan hijos sin idea de un dios. También existe una \textbf{apatía pasiva}: personas que si bien han sido expuestas a la idea de un dios, no piensan en su vida diaria en ello y en la práctica
 no forman o han perdido la creencia en la existencia de dioses. Y está la \textbf{apatía activa}, entre otras muchas razones para simplemente no creer.

\anchor[http://www.bornagainpagan.com/other/049-off-is-not-a-channel.html]{\begin{wrapfigure}{r}{245\px}\centering% {'src': 'http://www.bornagainpagan.com/other/049-off-is-a-tv-channel.jpg', 'alt': '', 'height': '300', 'class': ['alignright'], 'width': '245'}
\includegraphics[width=245\px,height=300\px]{blog/049-off-is-a-tv-channel.jpg}
\end{wrapfigure}
}Es claro que la existencia del materialismo filosófico pueda ser visto por los creyentes en alguna religión como \textbf{una creencia en algo}. Muchos creyentes acusan al ateísmo de ser una fe religiosa. Que así como existen \textbf{religiones politeístas} (con varios dioses) y \textbf{religiones monoteístas} (con un solo dios), existen \textbf{religiones ateístas} (con cero dioses). Un ejemplo de una religión ateísta son ciertas vertientes del \anchor[https://es.wikipedia.org/wiki/Budismo]{budismo}. Otro ejemplo, aseguran algunos creyentes, es el \textbf{materialismo}. El ateísmo derivado de la adherencia al materialismo filosófico sería una fe religiosa, y muchos religiosos
 monoteístas atacan al ateísmo como una creencia religiosa.

\par% p
Pero al no distinguir el materialismo científico y el materialismo filosófico, para un grupo de creyentes la ciencia es
 también una manifestación de esa \textbf{falsa religión} que es el materialismo. El creacionismo de tierra joven es muy dado a este tipo de ataques: atacar a la ciencia
 materialista para tratar de imponer su propia versión de qué es la ciencia: una serie de conjeturas sobre cómo
 funciona el mundo ajustada a cierto tipo de interpretación literal de la biblia.

Otros muchos apologistas del cristianismo (así como de otros teísmos) [que no se unen a la doctrina de tierra joven]
 atacan al materialismo filosófico, en parte porque su labor es defender al cristianismo de los ataques que los
 materialistas filosóficos hacen a la religión. Si bien he visto a apologistas con una buena comprensión del método
 científico y que defienden el materialismo científico, muchos de estos apologistas son más filósofos que científicos y
 son dados a considerar que toda crítica a la fe desde la ciencia parte de un materialismo filosófico y no de otro tipo
 de objeciones.

\begin{wrapfigure}{r}{191\px}\centering% {'width': '191', 'align': 'alignright', 'id': ''}
\anchor[https://commons.wikimedia.org/wiki/File:Williamlanecraig.jpg]{\includegraphics[width=191\px,height=300\px]{blog/Williamlanecraig.jpg}} Hasta donde entiendo William Lane Craig no se adhiere a la doctrina de tierra joven.
\end{wrapfigure}

Muchas veces me he visto tentado a considerar a los apologistas y a los creacionistas como una misma clase de
 individuos y en muchas de las discusiones y debates que he observado pareciera que se confunden. Y pareciera que
 muchos ateos también los confunden y los cuentan en un mismo saco y por extensión a cualquier versión del cristianismo.

Personalmente creo que es una visión válida observar un método materialista en la práctica científica manteniendo una
 concepción religiosa sea esta teísta o no teísta. Muchas personas religiosas que hacen ciencia ven a la ciencia como
 el cómo, como la explicación de las causas próximas, mientras que relegan su visión religiosa a la teleología o
 explicación de las causas últimas. Muchos ateos reconocen que el materialismo científico no implica un materialismo
 filosófico y no tienen inconveniente con compartir la ciencia con colegas religiosos.

Es dentro de ciertas filosofías, como la filosofía religiosa de los creacionistas o la filosofía antireligiosa de los
 materialistas que la ciencia y la religión no combinan.

\par% p
Pero en todas estas. \textbf{¿Dónde estoy yo?}

\par% p
Arriba mencionaba a la apatía activa como una de las causas de la falta de creencia en la existencia de dioses. Creo
 que la \textbf{apatía activa} es la mejor descripción de mi sentir sobre el tema.

\par% p
En la ciencia Dios carece de \textbf{valor explicativo} como \textbf{causa próxima}. Si Dios o algo similar tiene sentido como \textbf{causa última} no es una cuestión científica. Por mucho tiempo traté de compaginar mi cristianismo con mi agnosticismo y dentro de
 ese diálogo mental he llegado a varias conclusiones, muchas de las cuales han venido tomando forma aun después de que \anchor[http://blog.chlewey.net/2013/04/un-yo-agnostico/]{abandoné mi pretensión} de ser cristiano.

La idea de un dios necesario como causa última requiere que yo crea en la existencia o necesidad de una causa última.
 Tampoco requiero de un dios como causa de una moral objetiva, ni requiero de un dios como causa epistemológica. Para
 mí tiene más sentido considerar a las ciencias formales como una creación humana que como una realidad trascendente
 que sólo podemos conocer a partir de un dios. No hay un solo motivo en mi sistema de creencias que me lleven a
 necesitar a un dios.

\par% p
No porque yo me adhiera a la filosofía materialista. Creo que el materialismo científico es un límite y que hay muchas
 cosas que la ciencia no puede descartar como falsas, sólo como innecesarias. Y un dios es parte de eso. Si nuestra
 auto-conciencia es tan sólo un producto de nuestras mentes o se trata de algo más que opera sobre nuestras mentes,
 creo que no lo puede resolver la ciencia. La paradoja del \anchor[https://es.wikipedia.org/wiki/Habitaci\%C3\%B3n\_china]{cuarto chino} nos dice por qué la hipótesis no es falsable. Lo que sí parece claro es que no existe un alma que opere independiente
 a nuestra mente y que tenga efectos sobre el mundo material.

\par% p
Todo mi sistema de creencias me lleva entonces a una conclusión: la existencia de Dios \textbf{me es} irrelevante. No es que no haya pensado el problema. No es que no entienda el concepto. Creo incluso que hoy tengo una
 mejor explicación de qué es la trinidad de la que pude haber tenido cuando aún tenía una fe vaga. Y no rechazo a Dios
 por se trascendente o por ser trinitario, o porque me decepcioné de la teología o me decepcioné de la biblia.
 Simplemente no creo. No tiene sentido en mi sistema de creencias. No tiene lugar en mi concepción del mundo.

\par% p
Más que un materialismo filosófico lo mío es una \textbf{filosofía de materialismo apático}. Lo sobrenatural me es irrelevante. Lo sobrenatural para mí son conjeturas interesantes y aun tengo un interés en los
 dogmas de las religiones. Pero, para mí, esos dogmas, esas mitologías, son casi indistinguibles de la ciencia ficción
 y de la fantasía como \textbf{géneros literarios}. Me interesa \textbf{\emph{conjeturar}} lo posible. Pero separo lo posible de lo que creo que es la realidad, y la hipótesis de Dios, para mí, está en la
 primera parte.

\chapter{El dios de la lógica}
\begin{metadata}
	Published by \anchor[admin]{admin} on \anchor[http://ewey.co/B1435]{Sun, 02 Jun 2013 06:11:23 +0000}\\
	\categories{apologetica, ateismo, desde-youtube, dios, filosofia, information}\\
	Shorthand: \anchor[http://blog.chlewey.net/2013/06/dios-logica/]{dios-logica}
\end{metadata}

\begin{iframe}% {'src': 'http://www.youtube.com/embed/TrEBrgUYp7Y', 'allowfullscreen': '', 'width': '560', 'frameborder': '0', 'height': '315'}

\end{iframe}

\chapter{Dos lenguajes - entre asumir y conocer}
\begin{metadata}
	Published by \anchor[admin]{admin} on \anchor[http://ewey.co/B1450]{Fri, 26 Apr 2013 13:19:07 +0000}\\
	\categories{ateismo, desde-youtube, filosofia, opinion, religion}\\
	Shorthand: \anchor[http://blog.chlewey.net/2013/04/dos-lenguajes-entre-asumir-y-conocer/]{dos-lenguajes-entre-asumir-y-conocer}
\end{metadata}

\begin{iframe}% {'src': 'http://www.youtube.com/embed/o5Umt36bCxI', 'allowfullscreen': '', 'width': '560', 'frameborder': '0', 'height': '315'}

\end{iframe}

\chapter{Self-consciousness or a conjecture for a soul}
\begin{metadata}
	Published by \anchor[admin]{admin} on \anchor[http://ewey.co/B1452]{Wed, 06 Mar 2013 19:47:32 +0000}\\
	\categories{alma, conciencia, desde-youtube, opinion, trascendencia}\\
	Shorthand: \anchor[http://blog.chlewey.net/2013/03/self-consciousness/]{self-consciousness}
\end{metadata}

\begin{iframe}% {'src': 'http://www.youtube.com/embed/yLYnSjTyg1Q', 'allowfullscreen': '', 'width': '560', 'frameborder': '0', 'height': '315'}

\end{iframe}

\chapter{Una reflexión sobre lo que quiero lograr}
\begin{metadata}
	Published by \anchor[admin]{admin} on \anchor[http://ewey.co/B1455]{Wed, 06 Mar 2013 03:20:38 +0000}\\
	\categories{ciencia, desde-youtube, filosofia, personal, vocacion}\\
	Shorthand: \anchor[http://blog.chlewey.net/2013/03/una-reflexion-sobre-lo-que-quiero-lograr/]{una-reflexion-sobre-lo-que-quiero-lograr}
\end{metadata}

\begin{iframe}% {'src': 'http://www.youtube.com/embed/ECVX_PBealg', 'allowfullscreen': '', 'width': '560', 'frameborder': '0', 'height': '315'}

\end{iframe}

\chapter{Pasando el testimonio}
\begin{metadata}
	Published by \anchor[chlewey]{chlewey} on \anchor[http://ewey.co/B1467]{Fri, 23 Aug 2013 10:00:45 +0000}\\
	\categories{ateismo, educacion, personal, religion}\\
	Shorthand: \anchor[http://blog.chlewey.net/2013/08/pasando-el-testimonio/]{pasando-el-testimonio}
\end{metadata}

\par% p
Como lo recordé en mi pasado post \anchor[http://blog.chlewey.net/2013/05/a-journey/]{\textit{A journey}}, estudié en una escuela confesional y creo que gran parte de mi pensamiento humanista lo heredé de ese catolicismo que
 recién salía de Vaticano II, que no temía a la ciencia y por el contrario abrazaba el diálogo ecuménico y el
 entendimiento entre los pueblos. El cristianismo de <<da la otra mejilla>> y <<ama a tu prójimo>>, el de <<trata a los demás como quieras que te traten>> como una forma positiva de la regla de oro que trasciende culturas: <<no hagas a los demás lo que no quieres que te hagan>>. El cristianismo dirigido por un papa carismático que llamaba por la paz mundial y se oponía, como guía espiritual, a
 la pena de muerte.

En gran medida consideré que el núcleo de mis valores como persona venían de esa formación cristiana, aun cuando mis
 creencias sobre la metafísica del mundo se hubieran apartado de la metafísica judeocristiana. El Dios padre y creador
 y Jesús podrían ser sólo fábulas mientras siguiera apreciando los valores del cristianismo como el ideal de los
 valores humanos.

Mis estudios universitarios los hice en una universidad confesional, regida por la Compañía de Jesús y la filosofía
 ignaciana. La misma Compañía de Jesús que se acercó a la Teología de la Liberación sin adoptar la parte más extrema de
 la misma y que se consagró como el ala liberal y social del catolicismo frente al conservadurismo del Opus Dei. Si
 bien mi fe religiosa se fue diluyendo, tenía frente a mí varios ejemplos de como los valores cristianos pueden ser la
 base del mundo ideal al que todos aspiramos.

A pesar de mi agnosticismo, tomé la decisión de que mis hijos fueran bautizados en la iglesia donde me crié, en la fe
 de mi esposa y de la mayor parte de mi familia y conocidos. Y tomé la decisión de que estudiaran en una escuela
 confesional. No fui la única persona tomando esa decisión: mis padres sugirieron, mi esposa tuvo mucho que ver en la
 decisión, pero mi papel no fue sólo pasivo de aceptar la decisión de los demás sino que la apoyé por esa convicción de
 que la escuela confesional podría aportarle a mis hijos esos valores y principios que yo valoraba. Finalmente ellos,
 luego como adultos, podrían definir su propia fe.

\par% p
Pero algo cambió con \anchor[http://blog.chlewey.net/2013/04/un-yo-agnostico/]{mi apostasía formal} y con los motivos que me llevaron a la misma.

Parte ha sido entender que mis valores no son los valores del cristianismo. Son valores humanistas que bien comparten
 tanto las corrientes humanistas del cristianismo como el humanismo secular, mientras que existe todo un cristianismo
 no humanista, dentro y fuera del catolicismo, con el que no puedo identificarme.

\par% p
No puedo identificarme con los creacionistas de tierra joven que prefieren boicotear los fundamentos de la ciencia con
 tal de que su escritura sea literalmente correcta. No puedo identificarme con los cristianos que restriegan el
 deuteronomio para justificar su odio a la homosexualidad, pero luego hablan de la nueva alianza para desligarse de las
 partes incómodas del antiguo testamento. No puedo identificarme con la fábula de \emph{La Caída} y el mensaje implícito de que buscar la verdad (tomar el fruto del árbol del conocimiento del bien y del mal) sea el
 origen de los males del hombre. Que la colaboración entre todos los hombres sea tan amenazadora para Dios que tenga
 que confundir nuestras lenguas tal cual lo refleja la fábula de la torre de Babel. No puedo identificarme con Pablo
 escribiendo en sus cartas que buscar la razón es de necios.

En estos momentos me pregunto si realmente quiero que sean los valores cristianos, que pueden estar o no alineados con
 mi pensamiento humanista, los que sean enseñados a mis hijos. Me preocupa, sobre todo, que sean indoctrinados en una
 ideología que rechace el pensamiento crítico en aras de preservar un dogma. Por experiencia sé que no todo en la
 religión es una fe ciega que rechaza el pensamiento crítico pero lo hay.

Sé que particularmente en la escuela donde están mis hijos se desarrollan muchas destrezas intelectuales lo que se
 refleja en muy buenos resultados en las pruebas de estado de sus bachilleres y su aceptación en las universidades y
 eso no se logra con un rechazo total al pensamiento crítico.

\par% p
Podría pensar en pasar a mis hijos a una educación secular, pero el solo hecho de ser secular no garantiza todas las
 destrezas intelectuales que espero que mis hijos aprendan. Podría dejarlos seguir en la escuela confesional donde
 están y yo preocuparme por que aprendan lo que yo considero importante y que la escuela no les puede dar. Está siempre
 la alternativa de la educación en casa, el \textit{home schooling}, donde yo asumo la totalidad del proceso de aprendizaje. Pero en cualquier decisión yo no soy el único padre que toma
 las decisiones sobre cómo mis hijos enfrentarán la cuestión religiosa y sé que mi esposa se opondrá a lo que ella
 perciba como indoctrinación al ateismo.

\par% p
No será mi objetivo criar hijos que rechacen el concepto de Dios, en parte porque yo no soy un cruzado antiteísta. En
 parte porque espero que ellos tengan el criterio suficiente para pensar por sí mismos. En parte porque el cristianismo
 en el que me crié y en el cual creo que mis hijos se están criando es preferible a muchas otras sectas y religiones
 donde se subvierte toda la concepción del mundo a lo que el pastor o guía religioso decida o interprete: desde los
 fundamentalismos cristianos, hasta los cultos extraterrestres suicidas, la cienciología o el marxismo dogmático o el \anchor[http://blog.chlewey.net/2008/03/uribista-acritico/]{uribismo acrítico}.

\chapter{¿Paro o no paro?}
\begin{metadata}
	Published by \anchor[Carlos Th]{Carlos Th} on \anchor[http://ewey.co/B1470]{Wed, 28 Aug 2013 21:06:58 +0000}\\
	\categories{activismo, opinion, paro, paro-agrario}\\
	Shorthand: \anchor[http://blog.chlewey.net/2013/08/paro-o-no-paro/]{paro-o-no-paro}
\end{metadata}

Últimamente ando un poco desactualizado de fuentes oficiales de noticias y por ello mismo es posible que me perdí
 cuando anunciaron cuales son las reivindicaciones que los campesinos colombianos piden y cual es el motivo por el cuál
 una protesta con paro de actividades, bloqueos a vías y manifestaciones que han desembocado en violencia es la única
 alternativa que tienen.

\par% p
He escuchado una voz últimamente: “hay que apoyar el paro” y sin embargo me siento impedido a apoyarlo. La primera
 razón es que no sé que es lo que piden los organizadores del paro. Hay una que otra cosa clara: el regreso del \begin{abbr}% {'style': 'font-variant: small-caps;'}
Idema
\end{abbr}
 con sus precios de sustentación o la derogación de la resolución 970 del \begin{abbr}% {'style': 'font-variant: small-caps;', 'title': 'Instituto Colombiano Agropecuario'}
Ica
\end{abbr}
 (por cierto, los arroceros que son protagonistas del \anchor[http://www.youtube.com/watch?v=kZWAqS-El\_g]{Documental 9.70 de Victoria Solano} contra esa resolución del \begin{abbr}% {'style': 'font-variant: small-caps;', 'title': 'Instituto Colombiano Agropecuario'}
Ica
\end{abbr}
 no están siendo parte del paro).~ ¿O lo que piden es revertir el \anchor[http://blog.chlewey.net/2012/05/piratas-del-caribe/]{\begin{abbr}% {'style': 'font-variant: small-caps;', 'title': 'tratado de libre comercio'}
tlc
\end{abbr}
 con los \begin{abbr}% {'title': 'Estados Unidos'}
EE.UU.
\end{abbr}
} y no importar más comida?

\par% p
No estoy seguro de estar de acuerdo con todos esos puntos.
 Aunque digan que no piden subsidios un precio de sustentación y un mercado cautivo (sin competencia por productos importados) son formas de subsidios.
 Si un sector económico no evoluciona \anchor[http://blog.chlewey.net/2011/07/con-patente-de-corso/]{no debe exigir que se legisle} para que lo protejan.
 Por otro lado desproteger al campo en aras de unos principios generales y fríos es un atentado contra nosotros mismos.
 Creo que hay algo en el fondo en el modelo económico que debe replantearse porque no pareciera lógico que producir en el exterior e importar (incluyendo costos de transporte) sea monetariamente menos costoso que producir local.
 Pero este tipo de replanteamientos del modelo económico global no es algo que se resuelva en una mesa de concertación
 entre los organizadores de un paro y un gobierno.

\par% p
Pero aún si estuviere de acuerdo con los puntos y las causas (repito, no los conozco más allá de \anchor[http://www.youtube.com/watch?v=Fo50C8iw6kc]{un par de videos}), esto no me lleva necesariamente a apoyar los métodos.

Una manifestación violenta (y el bloqueo de vías es una forma de violencia pues afecta derechos civiles de otras
 personas, como el derecho a la movilidad y al trabajo, así no necesariamente sea violencia física y letal) debería
 legitimarse como última opción para una reivindicación legítima cuando se han agotado las opciones no violentas y el
 asunto a promover supera ampliamente los derechos suspendidos a las demás personas.

\par% p
Si quiero pedir que me dejen trabajar, como \anchor[http://www.youtube.com/watch?v=Fo50C8iw6kc]{César Pachón expone en su intervención en el Senado}, impedir que alguien se desplace para ir a su trabajo (o porque transportar personas y carga es su trabajo) tiene cierto halo de incoherencia.
 Este y otros paros que incluyen bloqueos, me han hecho temer por mis hijos.
 ¿Exponer la vida de ellos o de los otros menores de edad que son convidados a participar de los bloqueos se justifica
 frente a la reivindicación?

¿Qué pasa cuando no son sólo bloqueos sino amenazas directas a la integridad física y a la propiedad de quienes no se
 unen al paro? ¿Cuando se queman camiones y cargas para asegurar el bloqueo de una vía? ¿Cuando se lanzan piedras
 contra los vehículos de otros transportadores que prefieren trabajar a participar de la huelga?

\par% p
Si bien no soy partidario de la lucha armada puedo en teoría apoyar el alzamiento en armas (con todas las consecuencias letales que eso implica) cuando las condiciones no se prestan para reclamar derechos de otra forma.
 Finalmente así se fundaron naciones como los~\begin{abbr}% {'title': 'Estados Unidos'}
EE.UU.
\end{abbr}
 y Colombia.~ Simplemente \anchor[http://blog.chlewey.net/2008/01/injusticia-social-bullshit/]{no creo que la lucha de las \begin{abbr}% {'style': 'font-variant: small-caps;', 'title': 'Fuerzas Armadas Revolucionarias de Colombia'}
Farc
\end{abbr}
 se incluyan} entre ese tipo de luchas y como tal rechazo su pretensión de que su causa y sus métodos son necesarios.
 Las condiciones para justificar el grado extremo de violencia son condiciones extremas.
 Pero incluso una violencia pequeña debe estar justificada, y aún no estoy convencido de que la violencia de un bloqueo
 por vías de hecho sea justificada en este caso.

Pero incluso si creyera que estas vías de hecho están justificadas, eso no implica que apoyaría el paro.

Apoyar el paro es apoyar el medio, no el fin.

\par% p
Hacer actos simbólicos como salir a marchar con una ruana y una cacerola (o peor aún, con \anchor[https://play.google.com/store/apps/details?id=com.kinkastudio.icacerolazo]{una aplicación de cacerola para mi smartphone}) como apoyo al paro es un mensaje equívoco.

\par% p
Por favor: \textbf{% {'style': 'font-variant: small-caps;'}
apoyen} primero \textbf{% {'style': 'font-variant: small-caps;'}
la causa} que motivó al paro.
 Y si en serio están convencidos de que el paro es la única forma de promocionar esa causa entonces \textbf{no apoyen} el paro: \textbf{únanse} al paro.

\relax{% {'style': 'color: rgba(0,0,0,.6); font-size: .8em;'}
(Y, desde luego, uds. pueden unirse al paro y rechazar las formas más extremas de violencia dentro de la manifestación.
 Uds. no necesitan mi autorización para eso, es simple lógica de qué quieren expresar.)}

\chapter{Presuponiendo morales}
\begin{metadata}
	Published by \anchor[Carlos Th]{Carlos Th} on \anchor[http://ewey.co/B1477]{Sun, 08 Sep 2013 08:58:03 +0000}\\
	\categories{apologetica, ateismo, cristianismo, information, moral, religion}\\
	Shorthand: \anchor[http://blog.chlewey.net/2013/09/presuponiendo-morales/]{presuponiendo-morales}
\end{metadata}

\par% p
En \anchor[http://www.youtube.com/watch?v=IZeWPScnolo\&lc=nQONpa1ZyaUOnBSKT2J16K1PpzTdgObjdcEc8gF3bAw]{una de las discusiones} que encontré por Internet alguien hacía la siguiente pregunta: <<¿Hay alguna razón por la cuál es moralmente malo levantar a un bebé de su cuna, violarlo y luego estamparlo contra el
 piso hasta reventarle la cabeza?>> Quien proponía la pregunta luego argumentaba que si la respuesta es <<sí>> eso probaba la existencia de Dios, si la respuesta es <<no>> quien así respondiera necesitaba ser internado en un manicomio.

Existe toda una línea de la apologética cristiana que clama que todo lo existente, material o inmaterial, es creado por
 Dios. El universo, es decir el espacio, el tiempo y todo lo material, es creado por Dios, pero también el
 conocimiento, la moral (absoluta), el amor y, en general, las cosas inmateriales. La gran mayoría de seres humanos que
 vivimos en nuestras sociedades modernas sentiremos repulsión por la idea presentada de violar y matar brutalmente a un
 bebé. Los pocos que no sienten esa repulsión son los psicópatas, aunque muchos psicópatas sabrán reconocer que si bien
 ellos no sienten que la situación está mal, la pedofilia y el infanticidio no es socialmente aceptado. La mayor parte
 de los psicópatas no se convierten en asesinos en serie, pedófilos o infanticidas.

La pregunta, sin embargo, no es si sentimos repulsión sino si podemos establecer una línea lógica de pensamiento para
 decir porqué violar y asesinar brutalmente a un infante está mal.

Si dentro de nuestra visión del mundo, si dentro de nuestra filosofía de las cosas, no podemos establecer una línea
 lógica para condenar la pedofilia y el infanticidio entonces eso implica, según el apologista, que nuestra visión del
 mundo está equivocada y que la razón de que (casi) todos sienten repulsión por la imagen mental descrita en la premisa
 es porque tenemos una moral innata que es parte de la moral objetiva que sólo tiene un origen: el creador de todo lo
 que existe, incluyendo la moral.

\par% p
No conozco mucho sobre las visiones del mundo de las filosofías orientales como el budismo zen, por lo tanto no podré
 establecer qué línea de pensamiento pueda llevar a un seguidor de la mística oriental a rechazar este tipo de
 pedofilia e infanticidio descritos. Me enfocaré en lo que estoy más familiarizado: el humanismo cristiano v/s el
 humanismo secular. Algún apologista podría aquí hablar de cristianismo v/s ateísmo/agnosticismo, pero eso es una
 simplificación: hay corrientes dentro del cristianismo que se apartan del humanismo como la Iglesia Bautista de
 Westboro \emph{(Westboro Babtist Church)} que claramente están divorciadas del humanismo, pero si bien ellos son un caso extremo no son un caso único. Por otro
 lado el humanismo secular ni es seguido por todos los no creyentes, ni es restringido a no creyentes.

El humanismo cristiano se enfoca en el ejemplo y las enseñanzas de Jesucristo como guía moral para respetar y ayudar a
 los demás seres humanos. Entiende que de las tres virtudes teologales: fé, esperanza y caridad, la caridad es una
 pieza clave en cómo debemos relacionarnos con las demás personas. Debemos desprendernos de lo material con el objetivo
 de ayudar a otros, y parábolas como la del buen samaritano o la orden de Jesús de que hagamos a los demás como
 queramos que nos traten son fundamentales en esta concepción del mundo. Muchas versiones del cristianismo evangélico
 prefiere enfocarse en la virtud teologal de la fe y en la salvación personal por medio de una relación personal con
 Jesús, lo que los aparta de un sentido más humanista. El cristianismo en el que me crié y el que aprendí en la escuela
 hace, sin embargo, mayor énfasis en los valores humanistas pero, clamaba, el verdadero humanismo sólo puede existir
 desde el cristianismo.

El humanismo cristiano, si bien parte de la caridad cristiana, no se queda ahí. El humanismo tiene sus orígenes en el
 renacimiento y en gran medida como oposición al papel de la iglesia. El humanismo pone al ser humano como primer punto
 de la agenda ideológica, pero al dársele una lectura cristiana se encuentra que las enseñanzas de Jesús sustentan el
 amor a nuestros semejantes. El cristianismo, al menos algunas vertientes del cristianismo, fue adoptando lo que los
 pensadores humanistas fueron produciendo y lo leyeron desde la biblia y lo adaptaron.

\par% p
El humanismo secular, por su parte, pone al ser humano como primer punto de la agenda ideológica~\textbf{independiente} de las iglesias y confesiones religiosas. La política debe hacerse pensando en las personas, no en los dioses.

Entonces, ahora, desde el humanismo secular debo explicar porqué violar a un bebé y asesinarlo brutalmente está mal.
 Hay algo que parece casi tautológico: si lo que el humanismo quiere es el bienestar de cada ser humano, claramente
 causar sufrimiento y muerte a un pequeño ser humano está mal. Pero “claramente” no es un argumento lógico.

Es aquí a dónde recurro a tres pensadores, a tres líneas filosóficas, que pueden dar luz sobre que es justificable o no.

\anchor[http://es.wikipedia.org/wiki/Immanuel\_Kant]{Emanuel Kant} pensando sobre el problema de la moral, llega a la conclusión que es un imperativo categórico que los seres humanos no
 debemos ser medios sino fines. Está mal utilizar a otras personas como medios para un fin, y desde esta perspectiva no
 hay nada que justifique la cruel violación y brutal asesinato del bebé de nuestro ejemplo. La violación y el asesinato
 son ofensas máximas contra la integridad de un ser humano, en este caso un bebé, y categóricamente está mal. La línea
 de pensamiento que elabora Kant no requiere como punto de partida que Dios nos haya dictado que ofender a otros seres
 humanos esté mal.

\par% p
De los tiempos de Kant pasamos a nuestros días a una de las figuras que fue incluída entre los así llamados \anchor[http://www.youtube.com/watch?v=tS\_MT79m4Vw]{los cuatro jinetes} del \anchor[http://es.wikipedia.org/wiki/Nuevo\_ateísmo]{nuevo ateísmo}: \anchor[http://es.wikipedia.org/wiki/Sam\_Harris]{Sam Harris}. Harris pretende establecer las bases de la existencia de una moral objetiva que no requiera a Dios como punto de
 partida. Si observamos la naturaleza podemos ver cosas crueles estrellas que explotan o \anchor[http://blog.chlewey.net/2007/09/no-me-envien/]{una manada de leones dando muerte a un búfalo}, pero no podemos decir que eso es moralmente reprobable porque ni las estrellas, ni los leones ni los búfalos tienen
 conciencia. Sólo cuando las interacciones involucran personas es que podemos hablar de bien o de mal en términos
 morales y esto nos da una base lógica para construir una moralidad que Harris resume en buscar el mayor bienestar a
 los seres racionales. Nuestro bebé de marras como ser racional o miembro de la única especie de seres racionales que
 conocemos debe ser preservado del daño (falta de bienestar) que nuestro pedófilo infanticida le infligiría.

\anchor[http://es.wikipedia.org/wiki/Peter\_Singer]{Peter Singer}, otro contemporáneo nuestro, subscribe una escuela llamada personalismo. Extiende el concepto de persona a todo ser
 que es capaz de sentir y tener aunque sea una rudimentaria conciencia de sí mismo, aunque admite grados de
 personeidad. Un bebé es menos persona (aunque sí una persona) que un adulto. Los leones y los búfalos tienen algún
 rango de personeidad. A diferencia de Harris, Singer no se enfoca en maximizar el bienestar sino en reducir el
 sufrimiento en la capacidad en la que podamos entender el sufrimiento de los demás. Por ello está mal que los seres
 humanos matemos a un toro para divertirnos así el toro sufra lo mismo que el búfalo atacado por las leonas: el torero
 puede entender el sufrimiento del toro, las leonas no. El perpetrador el ejemplo puede entender el sufrimiento del
 bebé por sus actos crueles (si es un psicópata puede entender el sufrimiento, así no sienta empatía con el mismo) y
 eso nos da una base lógica para concluir que ese comportamiento es moralmente inaceptable.

Tres puntos de vista, completamente seculares, nos llevan a una conclusión lógica de porqué está mal la situación
 planteada en la pregunta. Ninguna de estas argumentaciones lógicas requieren a Dios. Desde luego, son puntos de vista
 que parten de premisas (los humanos no somos medios sino fines, hay que maximizar bienestar de los seres racionales o
 minimizar el sufrimiento de las personas) que no son necesariamente autoevidentes y aquí podrá saltar nuestro
 apologista para indicar que las premisas son creación de Dios.

Entonces entro a analizar el caso de Dios.

Dice nuestro apologista: si existe una moral objetiva, eso significa que hay un estándar y que hay reglas; si hay un
 estándar implica un plan; y las reglas un plan no surgen espontáneamente sino que requieren de un hacedor de reglas,
 de un planeador, y ese planeador es Dios.

Si consideramos, por ello, que la respuesta a la pregunta planteada es sí, estamos aceptando la línea de pensamiento
 que nos lleva a Dios. Pero esta línea sólo sucede en la cabeza de nuesto apologista y de las personas que piensan como
 él. Kant, Harris y Singer proponen estándares que no implican un plan. La premisa de que un estándar implica un plan
 es una proposición que necesita ser probada o verificada. En cuanto a las reglas, éstas pueden ser bien como las leyes
 de la física: proposiciones que resumen una observación en términos matemáticos y nos permiten predecir eventos y
 observaciones futuras, o bien corolarios del estandar, deducidos, más que dictados.

\par% p
Considerar que una moral objetiva implica a Dios es partir de la premisa de que una moral objetiva es el resultado de
 Dios: una \anchor[http://es.wikipedia.org/wiki/Petici\%C3\%B3n\_de\_principio]{petición de principio}.

Pero dentro del humanismo secular, las posiciones de Kant, Harris o Singer (y probablemente muchos otros pensadores que
 ignoro) no son las únicas. Muchos afirman que no hay tal cosa como una moral objetiva.

\par% p
La palabra moral viene del latín \emph{mores} que significa costumbre. La moral no es más que un acuerdo, una construcción social. Lo que consideramos moralmente
 bueno o malo es lo que aprendemos como tal inmersos en una sociedad donde nuestros padres y otros adultos nos dicen
 qué es y qué no es aceptable y dónde deducimos por nuestra cuenta del comportamientos y la costumbres de los otros qué
 es permitido y qué no. Luego podemos estudiar y adquirir una filosofía de la vida que nos lleve a establecer nuestros
 propios juicios sobre qué está mal y qué esta bien.

Un elemento importante en nuestra construcción de una moral personal es la empatía. El origen de la empatía en el 99\%
 de los seres humanos (y de la falta de empatía de ese 1\% que son los psicópatas), puede explicarse desde la teoría de
 la evolución en caso de que alguien quiera argumentar que la empatía tiene origen divino.

La empatía es lo que nos permite reconocer el sufrimiento de otros manifestándose como un sufrimiento propio y
 reconocer la felicidad de otros y poder compartir esa felicidad así no seamos beneficiarios de lo que la haya causado.

En nuestra sociedad actual la gente normal no viola niños y mucho menos bebés. La violación en sí misma es anormal pero
 aún dentro de esa anormalidad la violación de infantes es más anormal aún. Aun donde no sea tan anormal es ilegal. El
 asesinato, y más el asesinato de infantes, es anormal e ilegal en nuestras sociedades. Eso nos lleva a todos los que
 no somos psicópatas e incluso a la mayoría de los no psicópatas, a responder sí a la pregunta inicial: está mal. Los
 no psicópatas tenemos una razón más para decir que está mal, la razón por la cual sentimos repulsión sólo ante la
 imagen mental que plantea la pregunta: nuestra empatía con la víctima.

Pero si esta tesis es cierta: si es cierto que la moral es relativa a la sociedad, deberíamos ver ejemplos de
 sociedades donde la pedofilia y el infanticidio sean aceptados o incluso considerados moralmente bien. Para la
 violación de bebés no se me ocurre ni conozco una justificación pero sí para el infanticidio: en sociedades donde la
 lucha por la supervivencia es cruel, un pequeño infante puede llegar a ser más una carga que una ayuda y por lo tanto
 es sacrificable en ciertas condiciones. Los espartanos que abandonaban a sus bebés si estos nacían con algún defecto,
 los esquimales que mataban al primer nacido si se trataba de una hembra, las familias que huyen de las sequías en el
 Cuerno de África dejando atrás a los hijos menores para aumentar la probabilidad de que el mayor sobreviva (la
 alternativa no es salvar al menor sino asegurar que todos mueren). Tan espantosos como nos puedan sonar los casos,
 dentro de la moral y las condiciones de supervivencia de esas sociedades ese sacrificio es no sólo aceptable sino que
 es lo correcto.

Ahora, el reconocimiento de la existencia de morales relativas no implica que no pueda existir una moral objetiva.
 Quienes creen en la existencia de una moral objetiva ven a las morales relativas como aproximaciones a la moral real y
 objetiva, y las variaciones y desviaciones son o bien malas costumbres que deben extirparse o el reconocimiento a
 casos extremos (p. ej. el infanticidio como sacrificio para la supervivencia).

El principal inconveniente con considerar a Dios como fuente de la única y verdadera moral objetiva es que no podemos
 saber cual es el estándar, cuáles son las reglas. Si la moral es innata y plantada por el Creador en cada uno de
 nosotros no veríamos tantas morales relativas. Si todos tenemos la verdadera y única moral y las morales relativas son
 perversiones (la idea de Rousseau de que todos nacemos buenos y la sociedad nos corrompe), nos enfrentamos al problema
 de lo único que podemos deducir se deriva de todas esas morales relativas corruptas.

¿Está esa única y verdadera moral objetiva en la Biblia?

Pues respecto a la pregunta que inicia esta disertación la Biblia dice absolutamente nada. La Biblia no dice nada a
 favor o en contra de la pedofilia. No condena a la violación de mujeres solteras (sólo ordena que el violador repare a
 la víctima desposándola) mientras que la violación de mujeres casadas cae dentro de la prohibición del adulterio, pero
 nada específico con respecto a violar bebés. Fuera del mandamiento de “no matarás” la Biblia no prohíbe asesinar bebés
 mientras que por otro lado en el libro de Josué ordena matar a los infantes de las ciudades enemigas caídas.

Tal vez sí exista un dios creador de todo lo visible y lo invisible, de lo material y lo inmaterial, y dentro de lo
 inmaterial haya creado el libro de reglas de la única y verdadera moral objetiva. Pero si no conocemos esa tal moral
 objetiva es lo mismo que si no existiera y tuviéramos que describir nuestro propio estándar para definirla (como Kant,
 Harris y Singer), o creer que sólo las morales relativas existen.

\par% p
Para cualquier efecto práctico, responder <<sí>> a la pregunta planteada no implica un dios, y responder <<no>> no nos convierte en psicópatas peligrosos sino en el reconocimiento que así el escenario nos cause repulsión no
 debemos responder desde nuestros propios prejuicios.

\chapter{Octubre 2013}
\begin{metadata}
	Published by \anchor[chlewey]{chlewey} on \anchor[http://ewey.co/B1484]{Thu, 31 Oct 2013 20:58:52 +0000}\\
	\categories{personal}\\
	Shorthand: \anchor[http://blog.chlewey.net/2013/10/octubre-2013/]{octubre-2013}
\end{metadata}

\begin{wrapfigure}{r}{300\px}\centering% {'width': '300', 'align': 'alignright', 'id': 'attachment_1485'}
\anchor[http://blog.chlewey.net/wp-content/uploads/2013/10/DSC0369.jpg]{\includegraphics[width=300\px,height=199\px]{blog/DSC0369-300x199.jpg}} Ana García vda. de Thompson. 12 de julio de 1922 - 31 de octubre de 2013
\end{wrapfigure}

Se acaba octubre de 2013 con una noticia triste. Inició octubre de 2013 con una noticia triste.

El accidente de mi hermana fue un pequeño abrebocas: el 1º de octubre frenó en seco en su bicicleta para evitar chocar a un peatón que cruzó sin mirar.
Fractura de rótula que todo parece indicar no necesitará de cirugía.

\begin{wrapfigure}{r}{300\px}\centering% {'width': '300', 'align': 'alignright', 'id': 'attachment_1487'}
\anchor[http://blog.chlewey.net/wp-content/uploads/2013/10/beatriz-r.jpg]{\includegraphics[width=300\px,height=300\px]{blog/beatriz-r-300x300.jpg}} Ana Beatriz Cerón de Baquero, 29 de noviembre de 1933 - 2 de octubre de 2013
\end{wrapfigure}

El 2 de octubre fallece mi suegra, la mujer que en gran medida crió a mis hijos cuando estos eran más pequeños y a quien en abril del presente año le detectaron unas masas en el cerebro.
A veces no sé si el rápido deterioro fue por culpa de la enfermedad o si las hospitalizaciones influyeron, pero
 finalmente el miércoles 2 de octubre su cuerpo no dio más, tras varios días de apenas responder a estímulos físicos.

\anchor[http://blog.chlewey.net/wp-content/uploads/2013/10/anita.jpg]{\begin{wrapfigure}{r}{300\px}\centering% {'src': 'http://blog.chlewey.net/wp-content/uploads/2013/10/anita-300x300.jpg', 'alt': 'Abuelita Ana', 'height': '300', 'class': ['alignright', 'size-medium', 'wp-image-1489'], 'width': '300'}
\includegraphics[width=300\px,height=300\px]{blog/anita-300x300.jpg}
\end{wrapfigure}
}Mi abuelita Ana también venía enferma, y el pasado 29 de octubre, cuando pasé a saludarla por última vez sentí que en
 cierta forma ya se había ido y que apenas quedaba su cuerpo respondiendo a estímulos físicos básicos como el reflejo
 de comer cuando le mojaban sus labios, y quejándose del dolor.

Esta mañana falleció.

Y a veces no sé si desde mi distimia este sentimiento de tristeza sea por todo lo que ha venido pasando o por que se
 espera que este sea mi estado de ánimo.

Pero no ha sido un buen octubre.

\chapter{La campaña del señor Blanco}
\begin{metadata}
	Published by \anchor[chlewey]{chlewey} on \anchor[http://ewey.co/B1494]{Tue, 12 Nov 2013 13:55:09 +0000}\\
	\categories{activismo, elecciones, opinion, partido-pirata, politica}\\
	Shorthand: \anchor[http://blog.chlewey.net/2013/11/la-campana-del-senor-blanco/]{la-campana-del-senor-blanco}
\end{metadata}

\par% p
Veo que desde ya se está destapando la campaña a favor del voto en blanco en las próximas elecciones presidenciales que
 tendrán lugar en mayo de 2014.

\par% p
Ya en otras ocasiones me he referido \anchor[http://blog.chlewey.net/2007/10/voto-en-blanco/]{al voto en blanco}, de cómo este tuvo sus orígenes como \textit{carta blanca}, es decir como un voto de no compromiso, una opción de aceptar que otros decidan por uno, y se ha convertido en una
 especia de voto de protesta.

\begin{wrapfigure}{r}{293\px}\centering% {'width': '293', 'align': 'alignright', 'id': 'attachment_1495'}
\anchor[http://www.votoenblanco.com/El-Voto-en-Blanco-solucion-contra-la-partitocracia-y-la-degeneracion-democratica\_a2060.html]{\includegraphics[width=293\px,height=300\px]{blog/795175-973433-293x300.jpg}} Vía \anchor[http://www.votoenblanco.com/El-Voto-en-Blanco-solucion-contra-la-partitocracia-y-la-degeneracion-democratica\_a2060.html]{votoenblanco.com}
\end{wrapfigure}

Hay tres razones principales para votar en blanco: la primera es de la persona que acepta de antemano el resultado de
 la elección pero no quiere comprometerse con un nombre o una postura en particular. La segunda es de quien considera
 que los candidatos o las posturas presentadas no reflejan su propia posición y por lo tanto las rechaza todas. La
 tercera es de quien simplemente se reusa a pensar y toma la salida fácil: depositar un voto sin valor.

Como todos estos votos en blanco se cuentan con una sola cifra, discernir las razones y, sobre todo darle un
 significado a esos votos, es prejuicioso, particularmente cuando la votación en blanco se encuentra dentro de los
 niveles históricos. Ahí hay de los tres tipos de votantes en blanco: los que protestan y los que aceptan, los que
 pensaron su voto en blanco y los que no quisieron pensar.

Las razones personales por las que una persona puede votar en blanco son muchas. Dentro de las tres básicas mencionadas
 hay muchos matices y la decantación por el voto en blanco frente a las alternativas de candidatos o propuestas a votar
 puede ser más o menos razonada. Por ejemplo, alguien puede pensar que ninguno de los candidatos que puntea las
 encuestas convence y que ninguno de los demás candidatos merece ser considerado. O ninguno de los candidatos que
 conoce (o cree conocer por referencias o afiliación política) es su candidato y los que no conoce no hay tiempo de
 conocerlos (o si son desconocidos es porque en principio no convocan y por lo tanto no serán buenos). O repasó las
 hojas de vidas de todos y ninguno convenció. O, sencillamente, todos tienen el vicio de ser políticos y por lo tanto
 ninguno merece el voto. O porque siempre vota en blanco.

Si bien es deseable, no podemos exigir que todos nos tomemos el tiempo debido para evaluar todos y cada uno de los
 candidatos, sus pros y sus contras y decidirnos por el mejor o, en su defecto, por el voto en blanco.

Por lo tanto, si tienes razones personales para votar en blanco, aún cuando no todos los candidatos se han lanzado y
 han expuesto sus propuestas, no entraré a juzgarlas. Tu voto es tu voto.

Lo que sí desconfío a esta hora de la contienda es de una campaña a favor del voto en blanco.

Para empezar no seamos ingenuos de pensar que toda campaña de voto en blanco está descontaminada de la política
 tradicional. El Concejo Electoral ofrece espacios de participación política a cargo del erario para las campañas a
 favor del voto en blanco así como para las campañas a favor de votar por un candidato. No sólo eso sino que las
 campañas a favor del voto en blanco reciben, al igual que las campañas a favor de candidato, una plata por reposición
 de votos. Con los solos promedios históricos de votación en blanco, esa reposición está garantizada mucho más fácil
 que saliendo a competir con propuestas reales.

Pero incluso pensando que las campañas por el voto en blanco son sinceras y no una forma de obtener plata del estado
 (de los contribuyentes que pagamos IVA y 4‰ y nos descuentan retefuente, etc.) una campaña por el voto en blanco
 cuando no se han definido la totalidad de candidatos y propuestas es prematura.

La consecuencia política del voto en blanco es la siguiente.

Si el voto en blanco se mantiene dentro del promedio histórico, ese voto no significa mucho en términos políticos. No
 significa nada. El estado seguirá como siempre y no pasará nada.

Si el voto en blanco sube substancialmente en contra de los candidatos menores pero los candidatos punteros y de la
 maquinaria política tradicional se mantienen con cifras importantes, entonces el voto en blanco es un castigo a las
 alternativas. Eso significa que la única forma de ser tenido en cuenta en política no es presentar propuestas
 novedosas sino pertenecer a la maquinaria. Un castigo a quienes se atreven a presentar propuestas alternativas.

Si el voto en blanco sube substancialmente a costa de la votación de los candidatos punteros (pero no los sobrepasa),
 es claramente un voto de castigo. El estado seguirá funcionando como siempre pero la importancia de la votación en
 blanco se incrementa. Significa algo en contra del ganador, significa que este no tiene un apoyo total de parte del
 electorado.

Si el voto en blanco gana (no recuerdo es si es necesario que el voto en blanco sea mayoría o sólo baste con ser la
 mayor pluralidad), las elecciones son invalidadas. He aquí el verdadero poder del voto en blanco de acuerdo a la
 legislación colombiana: obliga a unas segundas elecciones en las que no participen ninguno de los candidatos que no
 pudieron vencer al voto en blanco. Claramente, cuando ningún candidato ofrece suficientes garantías al electorado (por
 ejemplo en casos que se han dado a nivel regional en los que hay un candidato único y este no representa a sus
 ciudadanos) la consecuencia política del voto en blanco como voto mayoritario importa.

\begin{wrapfigure}{r}{275\px}\centering% {'width': '275', 'align': 'alignright', 'id': 'attachment_1496'}
\anchor[http://blog.chlewey.net/wp-content/uploads/2013/11/Voto\_.jpg]{\includegraphics[width=275\px,height=300\px]{blog/Voto_-275x300.jpg}} Vía \anchor[http://www.movimientopirata.co/2013/10/votoblanco.html]{movimientopirata.co}
\end{wrapfigure}

Pero, repito, ese no es el escenario aún.

\par% p
Aun no podemos decir si los candidatos punteros (p. ej. \anchor[http://www.caracol.com.co/opinion/bloggers/blogs/los-secretos-de-darcy/santos-anuncia-nuevo-miembro-de-comite-para-reeleccion-y-mas-secretos-de-darcy-quinn/20130910/blog/1991189.aspx]{Juan Manuel Santos}, \anchor[http://www.eluniversal.com.co/politica/oscar-ivan-zuluaga-candidato-presidencial-del-uribismo-139860]{Óscar Iván Zuluaga}, \anchor[http://www.elcolombiano.com/BancoConocimiento/C/clara\_lopez\_fue\_elegida\_candidata\_presidencial\_por\_el\_polo\_democratico/clara\_lopez\_fue\_elegida\_candidata\_presidencial\_por\_el\_polo\_democratico.asp]{Clara López}) o los \anchor[http://es.wikipedia.org/wiki/Elecciones\_presidenciales\_de\_Colombia\_de\_2014]{candidatos alternativos} (p. ej. Camilo Romero, Óscar Naranjo, Antonio Navarro) no merecen ocupar el cargo de Presidente de la República de
 Colombia.

Adicionalmente, cuando el voto en blanco se convierte en una opción permanente (y no sólo coyuntural), cuando la
 campaña es a que votemos en blanco esta vez, pero también la próxima, lo que se está es proponiendo otro modelo de
 política. Se crea una crisis institucional dentro del acontecer político tradicional pues haría imposible la toma de
 decisiones de acuerdo con la constitución.

No estoy diciendo que eso sea malo. (No creo que sea bueno, pero esto es tan solo una opinión.) Pero esto requiere que
 el promotor del voto en blanco sea claro en decirnos cual es la propuesta real que hay detrás del voto en blanco:
 castigar una coyuntura o generar un cambio profundo en la forma de hacer política. ¿Qué es lo que propone si gana el
 voto en blanco?

Mientras una campaña de promoción del voto en blanco no responda esta pregunta, consideraré que esa campaña es poco
 seria y su propuesta es por moda o por plata.

\par% p
Y eso incluye una campaña a favor del voto en blanco \anchor[http://www.movimientopirata.co/2013/10/votoblanco.html]{por el movimiento político que ayudé a fundar}.

\chapter{¿A qué está jugando una indignada tuitera?}
\begin{metadata}
	Published by \anchor[chlewey]{chlewey} on \anchor[http://ewey.co/B1498]{Wed, 13 Nov 2013 16:20:47 +0000}\\
	\categories{alcohol, feminismo, machismo, misoginia, opinion}\\
	Shorthand: \anchor[http://blog.chlewey.net/2013/11/a-que-esta-jugando-una-indignada-tuitera/]{a-que-esta-jugando-una-indignada-tuitera}
\end{metadata}

\par% p
A veces tengo la impresión de que cosas que dije ayer en Twitter sobre el debate de \emph{\textbf{Andrés Carne de Res}} pueden hacerme quedar como un macho misógino sólo porque no me adhiero completamente a la postura de la indignación
 oficial de tachar a Andrés Jaramillo de bestia por sugerir que una minifalda invita a una violación.

\begin{wrapfigure}{r}{300\px}\centering% {'width': '300', 'align': 'alignright', 'id': 'attachment_1499'}
\anchor[http://blog.chlewey.net/wp-content/uploads/2013/11/andres\_jaramillo\_-\_cromos\_0.jpg]{\includegraphics[width=300\px,height=199\px]{blog/andres_jaramillo_-_cromos_0-300x199.jpg}} Andrés Jaramillo - foto: \textbf{Cromos} (tomado de \textbf{\anchor[http://www.bluradio.com/48044/que-esta-jugando-una-nina-que-llega-en-minifalda-andres-jaramillo]{Blu Radio}})
\end{wrapfigure}

\par% p
He repasado las palabras que Andrés Jaramillo expresó en su entrevista en \textbf{\anchor[http://www.bluradio.com/48044/que-esta-jugando-una-nina-que-llega-en-minifalda-andres-jaramill]{Blu Radio}} y el título que Blu Radio coloca a su clip: \emph{«A qué está jugando una niña que llega en minifalda: Andrés Jaramillo»} y me parece que, en el mejor de los casos, el título es amarillista y en el peor malintencionado. Sí, Jaramillo \relax{% {'style': 'margin-bottom: -1px; border-bottom: 1px dotted red;', 'title': u'Aparentemente el \xe9nfasis era a que la joven ocult\xf3 la minifalda portando un gab\xe1n para que el pap\xe1 no la viera, no a que la joven vistiera minifalda'}
expresó esas palabras}. No: ese no fue el punto central de lo que Jaramillo expresó en la entrevista. Es más, el propio Andrés Jaramillo
 retractó esas palabras. Hay muchas pifias y bestialidades que Jaramillo expresó en esa entrevista: puntos en los que
 se sostuvo, y por los cuales vale la pena cuestionar a este intelectual empresario, pero reducir todo lo que dijo a
 que Jaramillo cree que una minifalda justifica una violación es simplificar absurdamente el debate, simplificar
 absurdamente los hechos y dar pie a que no se planteen los asuntos de fondo.

Andrés Jaramillo discute que se haya tratado de una violación. No, no justifica la violación sino que pone en duda el
 hecho. Más adelante hablaré de si tiene méritos o no. Su punto es que de acuerdo a las cámaras de vigilancia y al
 testimonio de un empleado del parqueadero, el sexo que tuvo la joven denunciante con otro cliente mayor fue
 consensuado. Más adelante hablaré de si se puede hablar de consentimiento o no. Que tras este acto de sexo casual la
 joven se quedó inconsciente (aparentemente durmiendo la borrachera) y fue encontrada por personal del establecimiento
 y llamaron al padre para que la recogiera y que la versión de la violación surgió cuando la joven tuvo que responderle
 al padre por lo sucedido. Jaramillo acusa a los medios de amarillismo y de haber destacado un caso sólo porque la
 joven es estudiante de Los Andes y los hechos ocurrieron en su establecimiento, mientras que muchos otros hechos de
 violaciones reales ocurren sin ser visibilizados. Jaramillo también cuestiona a las autoridades insinuando que quieren
 sacar provecho político de estos hechos.

\par% p
Lo de la minifalda es apenas una parte del punto de Jaramillo, tras hablar de las muestras de afecto que mostraba la
 joven con otros clientes de su establecimiento, particularmente con un súbdito español y con el presunto agresor, hace
 una mención sobre el atuendo (como una forma de apoyar la versión ya presentada), y luego insiste en retirar esas
 palabras. Mal por \textbf{Blu Radio} de enfocarse en sólo esas palabras en el título del clip y mal por la masa indignada de quedarse en el titular de \textbf{Blu Radio} sin indagar la verdadera razón por la cual las palabras de Andrés Jaramillo son un insulto a nuestra inteligencia.

El primer punto es tratar de minimizar los hechos: defender al establecimiento sembrando una duda sobre la acusación y
 difamando públicamente el carácter de la joven denunciante en este intento de defensa. Pretender acusar al padre y a
 la situación familiar de los hechos e insistir en que los medios y las autoridades tienen un interés particular sobre
 el asunto.

\par% p
Palabras más sabias pudo haber sido que \emph{«es prematuro hablar de violación mientras no se establezcan plenamente los hechos, si bien hay indicios de que pudo
 haber sido una relación consensuada, estamos colaborando plenamente con las autoridades, hemos entregado las
 grabaciones de las cámaras de seguridad y la identificación del sospechoso.»} Punto. Sin necesidad de estar ventilando por radio el carácter de la joven ni su relación familiar, ni insinuar
 ensañamiento de la prensa con su local, ni interés político de las autoridades.

Ahora. Dadas las explicaciones no pedidas ¿es posible hablar de una relación consensuada o necesariamente eso fue una
 violación?

Aclaro que no soy abogado, ni tengo un amplio conocimiento de la jurisprudencia existente. Mi visión es la de un
 ingeniero que intenta ponerle sentido común a las relaciones entre personas y de estas frente al estado, así que si
 afirmo algo que va en contra de una sentencia me disculparán los que saben mejor, pero sólo me disculparé cuando me
 expliquen por qué ahí las leyes tienen más sentido común que yo.

\par% p
El concepto de consentimiento no aplica sólo para relaciones sexuales (o, según establece el código penal: \emph{% {'style': 'margin-bottom: -1px; border-bottom: dotted 1px red;', 'title': u'Ser\xeda interesante analizar el significado literal de esta frase'}
acceso carnal}). Nosotros consentimos o no procedimientos médicos para nosotros o para nuestros hijos. Consentimos o no que nos
 envíen por correo electrónico información comercial de un servicio al que nos suscribimos. Consentimos o no participar
 en actividades tales como dar un paseo por el monte con unos muchachos armados. Consentimos o no entregarle nuestra
 billetera a un desconocido en la calle. Consentimos o no conducir bajo efectos del alcohol. \emph{Et cétera}.

\begin{wrapfigure}{r}{279\px}\centering% {'width': '279', 'align': 'alignright', 'id': 'attachment_1503'}
\anchor[http://blog.chlewey.net/wp-content/uploads/2013/11/Informed\_consent\_of\_patients-279x300.png]{\includegraphics[width=279\px,height=300\px]{blog/Informed_consent_of_patients-279x300.png}} \emph{Do I need to ask you? Mrs. Brown.} vía \anchor[http://blog.medicallaw.in/what-can-a-doctor-do-and-what-cannot-be-done-without-the-consent-of-a-patient/]{Medical Law}
\end{wrapfigure}

\par% p
En la práctica médica se habla del \textbf{consentimiento libre informado}. He traducido libros de bioética que tratan del tema y por eso podría estar mejor familiarizado con esto que con el
 consentimiento al \emph{acceso carnal}. Cuando un paciente va a ser sometido a un procedimiento debe, en principio, aceptar que se le realice ese
 procedimiento. En el mejor de los casos el médico informa al paciente de qué trata el procedimiento, cuales son los
 riesgos y posibles beneficios tanto del procedimiento como de no llevarlo a cabo y, una vez plenamente informado y
 libre de dudas, el paciente libremente acepta o rechaza el procedimiento. Sin el consentimiento del paciente el médico
 puede verse expuesto a una demanda por mal-praxis. Desde luego, hay muchas razones por las cuales este consentimiento
 libre informado no puede darse: el paciente sufre considerablemente por una dolencia y no puede pensar con claridad, o
 el paciente está inconsciente, o el paciente es menor de edad o sufre de una condición por la cual no puede legalmente
 expresar consentimiento. En muchos casos el personal médico no puede esperar a que el paciente esté en condiciones de
 ser informado y tenga la libertad de consentir o no un procedimiento, motivo por el cual el consentimiento se delega a
 un familiar, un juez, un comité médico o, en casos de vida y muerte inminente, al criterio del médico tratante. Un
 médico no debería ser acusado de mal-praxis si salva a un paciente a punto de morir si morir era el deseo del
 paciente, salvo que el paciente haya advertido previamente al médico su deseo (y, aún así, hoy en día tal deseo sería
 invalidado).

En muchos casos el médico tiene que inferir por declaraciones previas o por su criterio médico si ejecuta un procedimiento agresivo y no explícitamente consentido como amputarle una pierna gangrenada a un paciente inconsciente.
 (Bueno, muchos dirán que no es lo mismo salvar una vida a un a costa de una mutilación, que penetrar un falo en una
 vagina.)

\begin{wrapfigure}{r}{300\px}\centering% {'width': '300', 'align': 'alignright', 'id': 'attachment_1500'}
\anchor[http://blog.chlewey.net/wp-content/uploads/2013/11/grandeserrores.png]{\includegraphics[width=300\px,height=82\px]{blog/grandeserrores-300x82.png}} Vía \anchor[http://catalinapordios.com/2009/03/27/los-marihuaneros-trabajan/]{\emph{Los marihuaneros trabajan}}
\end{wrapfigure}

El alcohol produce varios efectos en la mente humana. El alcohol adormece, pero suele adormecer primero ciertas áreas
 del cerebro que nos retienen, que nos hacen ser más cautos de lo que nos sería beneficioso, antes de adormecer por
 completo el juicio, la capacidad de pensar y coordinar y finalmente adormecer todo el cuerpo. El alcohol desinhibe y
 por ello muchas personas lo usan para atreverse a hacer cosas que sí quieren hacer pero no se atreven estando sobrios.
 Por ejemplo para ir a hablarle a una chica desconocida en un bar sin temer al rechazo. En otras dosis no nos permite
 ser conscientes de nuestros propios actos y en cantidades aún mayores no podremos siquiera defendernos de lo que nos
 pasa hasta que somos simples pedazos de carne a merced de la intemperie.

\begin{wrapfigure}{r}{300\px}\centering% {'width': '300', 'align': 'alignright', 'id': 'attachment_1265'}
\anchor[http://blog.chlewey.net/wp-content/uploads/2012/06/velasco.jpg]{\includegraphics[width=300\px,height=201\px]{blog/velasco-300x201.jpg}} Javier Velasco (fuente original requerida)
\end{wrapfigure}

Legalmente no podría, no debería poder usar el alcohol como defensa si cometo un delito bajo su influencia. Cuando
 Javier Velasco asesinó por primera vez (o lo atraparon por primera vez) el juez consideró la defensa de que su juicio
 estaba nublado por el alcohol y que por lo tanto no era responsable de sus actos. Lo liberó con una orden de que fuera
 a rehabilitación. Un par de años después, también bajo influencia del alcohol, asesina a Rosa Elvira Cely. Son
 incontables las veces que los medios sociales de comunicación se indignan por un nuevo conductor borracho, bien porque
 es un político o bien porque causó una tragedia. Independientemente de lo que los jueces digan como sociedad no
 aceptamos que el alcohol sea una excusa para evadir responsabilidades penales. Siempre y cuando el caso nos indigne,
 porque por otro lado seguimos recurriendo al alcohol para desinhibirnos y pasarla bien en una rumba.

Podría ser una defensa si puedo argumentar que ingerí alcohol sin mi consentimiento, pero, salvo que me hayan sometido
 a la fuerza e inyectado alcohol directamente en la sangre, ¿podremos aceptar como excusa que ingerí alcohol por
 presión social pero yo no quería? ¿Y que si yo hubiera estado en mi sano juicio no hubiera aceptado el reto de mis
 amigos de subirme al carro borracho causando un accidente fatal?

\begin{wrapfigure}{r}{300\px}\centering% {'width': '300', 'align': 'alignright', 'id': 'attachment_1501'}
\anchor[http://blog.chlewey.net/wp-content/uploads/2013/11/tumblr\_lb4t0fxjrb1qcyfqeo1\_500\_large.jpg]{\includegraphics[width=300\px,height=199\px]{blog/tumblr_lb4t0fxjrb1qcyfqeo1_500_large-300x199.jpg}} Vía \anchor[thatbananas.blogspot.com/2011/05/hoje-e-dia-de-gala-de-finalistas.html]{\emph{That's Bananas!}}
\end{wrapfigure}

Es un hecho, hay hombres y mujeres que (por presión de grupo o por sus propias razones) van a bares y otros sitios de
 rumba buscando pareja y relaciones casuales. Es un hecho que muchos de ellos usan el alcohol para desinhibirse y
 disfrutar mejor del momento y, en ocasiones, para atreverse a establecer contacto con las potenciales parejas.

Es también un hecho que algunas personas inducen a otras a consumir alcohol para dominarlas más fácil.

¿Cómo podemos establecer entonces la diferencia entre una joven que va a un bar con el objetivo de divertirse y tener
 sexo casual con un apuesto desconocido y obtiene lo que busca pero en el proceso tomó más de la cuenta y termina
 inconsciente y otra joven que sólo quiere bailar, posiblemente encontrar a alguien interesante con quien establecer
 una relación seria, pero es inducida a seguir tomando hasta perder el juicio y, cuando esto sucede, es usada y
 abandonada?

Ni Andrés Jaramillo, ni el padre de la joven mayor de edad, son las personas más idóneas para establecer la diferencia.
 Muchos de los comportamientos que muestra una cámara de seguridad, o lo que puede ver el vigilante de un parqueadero,
 son ambiguos.

Mucho menos somos idóneos nosotros de inferir lo que pasó por testimonios de terceros.

\par% p
Salvo que otros testigos corroboren que escucharon a la víctima decirle que \textbf{no} al tipo, o que corroboren que la denunciante había previamente declarado su intención de tener sexo con cualquiera no
 sabemos qué pasó.

Podemos tomar una actitud tajante y mecánica con respecto a lo sucedido: si en el momento en el que ocurrió el acto
 sexual la joven estaba tan borracha como para que su aceptación pudiera considerarse como consentimiento libre
 informado entonces no hubo consentimiento, entonces fue acceso carnal no consentido, entonces fue violación. Creo que
 sí hay una sentencia de la corte en el sentido de que ni la forma de vestir ni el comportamiento previo pueden ser
 tomados como consentimiento si en el momento del acto la mujer no estaba plenamente consciente.

Pero esto sería una diferencia importante a cómo tratamos el alicoramiento en otros casos. Convertimos a la mujer
 automáticamente en víctima despojándola de cualquier tipo de voluntad (¿responsabilidad?) que haya podido tener en el
 transcurso de los hechos.

\par% p
Nos dice \anchor[http://www.elcolombiano.com/BancoConocimiento/P/preguntas\_incomodas/preguntas\_incomodas.asp]{Ana Cristina Restrepo} que \emph{«según cifras oficiales, a octubre de 2013, \textbf{178 mujeres} han sido asesinadas en Antioquia.»} La cifra se ve impresionante así presentada, pero contrastando \anchor[http://diarioadn.co/medell\%C3\%ADn/mi-ciudad/reducci\%C3\%B3n-de-homicidios-en-medell\%C3\%ADn-durante-octubre-de-2013-1.83679]{otras cifras que consulté} en sólo el área metropolitana del valle de Aburrá, en el mismo período han sido asesinadas \textbf{1.137 personas}. No me malinterpreten: una sola mujer asesinada está mal. Una sola persona asesinada está mal. Lo que las cifras solas
 no llevan a deducir es que el homicidio de mujeres sea un problema preocupante frente al problema del homicidio en
 general. ¿Es un problema de género? ¿Es el asesinato de mujeres un preocupante problema social que merece una atención
 especial frente al asesinato en general?

\par% p
Hay cierta visión feminista de los hechos que debería ser debatible como cualquier otra visión. Es posible que haya
 razones por las cuales sea necesario resaltar que de las más de un millar de víctimas mortales en Antioquia en lo
 transcurrido del año algo menos de doscientas hayan sido mujeres; o que si el alcohol no es excusa para pegarle a la
 mujer sí es excusa de esta para denunciar como \textbf{no consensuado} un acto sexual. Y yo no debería ser acusado de ser misógino o machista por disentir de o cuestionar esta visión.

Y en cuanto al intelectual empresario Andrés Jaramillo, por favor vean más allá de la parte de la minifalda para darse
 cuenta por qué sí es un imbécil.

\chapter{La ley de la papaya}
\begin{metadata}
	Published by \anchor[chlewey]{chlewey} on \anchor[http://ewey.co/B1507]{Fri, 15 Nov 2013 15:26:56 +0000}\\
	\categories{familia, futuro, opinion, papaya, politica, sociedad}\\
	Shorthand: \anchor[http://blog.chlewey.net/2013/11/la-ley-de-la-papaya/]{la-ley-de-la-papaya}
\end{metadata}

A veces termina uno metido en unos estrambóticos debates en Twitter, probablemente porque la tiranía de los 140
 caracteres nos lleva a fraccionar o simplificar las ideas y estas no son tan claras como uno quisiera expresarlas, o
 porque sencillamente tenemos una tendencia a observar el mundo de tal forma que confirmemos el juicio que ya hicimos
 del mismo y no con la mente abierta de pensar si encontramos nueva información para corregir nuestra preconcepción.

Uno de los temas álgidos del debate que se está dando es de la relación entre lo que algunos percibimos como el sentido
 común de la prevención y el deseo de poder disfrutar de la vida sin tener que preocuparnos de lo que no debería ser
 problema.

\par% p
A mí me gusta \anchor[http://blog.chlewey.net/2011/08/caminando-ciudades/]{caminar por la ciudad} y he caminado a diferentes horas en diferentes partes de esta y otras ciudades, en ocasiones con mayor o menor temor de diferentes amenazas como toparme con asaltantes armados en Bogotá o con bandas de muchachos xenófobos en Yokohama.
 A veces los temores pudieron haber estado injustificados y a veces mi falta de prevención pudo haber rayado en lo insensato.
 Afortunadamente no tengo nada que lamentar, como desafortunadamente muchas otras personas más precavidas que yo sí
 tienen episodios trágicamente lamentables en sus vidas.

Como padre de un par de muchachos (niño y niña) que estarán entrando a la adolescencia en los próximos diez años
 entiendo que mi responsabilidad va mucho más allá de las precauciones individuales que tomo o no tomo cuando salgo a
 caminar, sino que incluyen el poder que tengo como ciudadano de formar la sociedad y más cuando entre las posibles
 formas que he previsto he considerado la participación directa como hacedor de leyes.

Quiero que mis hijos crezcan y disfruten de la vida con muchos menos temores de los que yo tuve.
 Y que disfruten más.
 No quiero tenerlos resguardados en una jaula ni en la protección de cuatro paredes sólo para que no les pase nada, sino que salgan y se la gocen.
 Pero tampoco quiero que sean insensatamente temerarios y que crean que pueden hacer lo que quieran sin asumir
 responsabilidades por sus decisiones.

Pero si quiero que disfruten más, teman menos y no les pase nada, hay muchas cosas que puedo hacer como padre y ciudadano.
 Debo darles libertad para que vayan y disfruten, pero debo darles límites que por un lado refuercen su confianza y por otro su responsabilidad.
 Debo procurar una sociedad que no los trate como delincuentes por ser adolescentes, ni los desampare.
 Debo buscar que la sociedad no les sea hostil cuando ellos estén en lo correcto ni cuando ellos se equivoquen.

No quiero que mi hijo el día de mañana sea acusado, falsa o correctamente, por una violación, ni que mi hija sea víctima de una.
 Ni al contrario.
 Que ninguno de ellos, al calor del alcohol y sus hormonas, no sepa controlarse; ni que actuando correctamente de pie a una falsa acusación; ni que sea víctima de quien no pudo controlarse o de quien creyendo actuar correctamente no lo hizo.
 Debo enseñarles donde están sus propios límites porque a partir de ahí están los derechos de las otras personas.
 Y debo enseñarles a que establezcan sus límites frente a los demás para que no sean ellos los abusados.
 Y debo enseñarles a que sean precavidos sin vivir asustados.
 A que puedan explorar y disfrutar más allá de la zona segura que yo pueda construirles.
 Quiero que mi hija pueda seguir luciendo sus minifaldas que hoy disfruta en su inocencia infantil sin que eso sea una
 invitación a que la traten como ella no merece.

Quiero que los otros muchachos y muchachas (y hombres y mujeres más maduros) que mis chicos puedan encontrar no sean una amenaza para ellos.
 Que los demás sepan respetar la voluntad de mis hijos.
 Que no abusen de mis hijos ni les hagan daño.~ Que sepan que si mi hija o mi hijo dicen no, entonces es no.

Quiero que los demás respeten a mis hijos porque lo correcto es respetar a los demás.
 Porque, así como espero enseñarles a mis hijos el respeto al otro, a ellos otros también les hayan enseñado a respetar.
 Que este mutuo respeto a nuestros mutuos derechos sea por convicción de vivir en una sociedad y no sólo por temor a la policía y los jueces que los condenarán, porque finalmente si la única razón de actuar bien es el temor al castigo, la otra solución es actuar mal y ocultar el hecho.
 Pero no soy ingenuo de pensar que todos los demás (o mis propios hijos) se portarán bien sólo por convicción.

Entonces también quiero una legislación que proteja a las víctimas y un estado capaz de hacer cumplir esa protección, tanto preventiva como punitivamente.
 Que el potencial agresor de mis hijos se restrinja porque sabe que el riesgo de que lo atrapen es alto.
 Que el potencial agresor de mis hijos se restrinja porque sabe que si lo atrapan no tendrá excusas.

Pero esto también es ingenuo.
 Es ingenuo en un país donde los ciudadanos normalmente respetan la ley porque siempre hay casos de predadores humanos que creen que pueden salirse con la suya y de predadores humanos a quienes no les importa las consecuencias.
 Lo decía arriba.
 Si la única razón para no hacer algo es el temor al castigo, muchos interpretarán que el verdadero problema para sí mismos no es cometer el acto prohibido sino dejarse atrapar.
 Con suficiente legislación puedo proteger a mis hijos de los ciudadanos temerosos de la ley, pero no los puedo
 proteger de quienes carecen de ese temor.

\par% p
Y en la Colombia donde mis hijos viven y probablemente vivirán cuando sean adolescentes y adultos jóvenes, los ciudadanos no nos caracterizamos por nuestro respeto a la ley.
 Esto es algo que va mucho \anchor[http://tumblr.chlewey.net/post/66964149373/reflexiones-sobre-el-machismo]{más allá del machismo} o de \anchor[https://twitter.com/Oyerista/status/401202383759237120]{una visión machista} de la sociedad sino que se ha cimentado en años de guerra y de un estado que por años ha servido más al interés de los
 agentes de poder que al interés del ciudadano común.

\par% p
Así yo logre \relax{% {'style': 'margin-bottom: -1px; border-bottom: 1px dotted red;', 'title': u'No pasar\xe1 en 2014.  Mi movimiento pol\xedtico decidi\xf3 que no participar\xedamos en 2014.'}
entrar al congreso} y desde allí impulsar y lograr aprobar las leyes correctas para que Colombia no sea un país machista donde impere la ley de la papaya no voy a lograr generar el cambio a tiempo para que mis hijos estén 100\% seguros.
 O 98\% seguros.

\par% p
Para mí no es solamente ingenuo \textbf{sino que es irresponsable} pretender que porque la culpa moral y penal recaiga en el potencial abusador de mis hijos, eso signifique que yo no
 tenga el deber de enseñarles a ser cuidadosos; porque sé que la culpa moral y penal del potencial abusador no es
 suficiente para que estén a salvo.

\par% p
Cerca del \anchor[http://www.telegraph.co.uk/health/children\_shealth/9510937/One-in-100-children-are-psychopaths-experts-believe.html]{1\% de la población humana} carece de empatía: el cimiento del comportamiento moral y de que hacer daño a los demás está mal independientemente del posible castigo, y el 1\% de 47 millones de colombianos son 470.000 psicópatas que si bien no todos serán violentos, su número no es despreciable.
 Sumado a esto una de las más probadas tácticas de reclutamiento de menores para la guerra (también aplicable a adultos jóvenes) es borrar la empatía.
 Esto es algo que hace la guerrilla.
 Es algo que han hecho los paramilitares.
 Es algo que también hace el estado cuando se enfrenta a una guerra, y es algo que hacen las pandillas en las calles.
 Demasiadas personas para ser controladas sólo por leyes.

Sumemos la esquizofrenia y su capacidad de ocultar la realidad.
 Sumemos la depresión clínica (que podría afectar hasta un 20\% de la población) y la capacidad que tiene ésta de que a
 una persona normal no le importen en algún momento las consecuencias de sus actos.

Y sumemos todos los posibles peligros que no tienen como origen a otra persona como salir a acampar a un sitio seguro
 (libre de delincuentes humanos) pero perderse en el camino, caer por un barranco o toparse con un animal de presa o
 una alimaña ponzoñosa.

Ningún esquema de seguridad será 100\% efectivo.
 A un vecino se le puede escapar la boa que guarda como mascota y asfixiar a nuestro hijo en la seguridad de su habitación.
 Puedo vivir en un edificio que resiste temblores de 7,5 pero estar justo el día del terremoto haciendo una vuelta en un edificio que no es sismorresistente.
 Puedo prohibirle a mi hija ir a fiestas en minifalda pero justo está viajando en un bus que secuestran delincuentes
 altamente armados.

Por más que la prevención no sea 100\% efectiva.
 Por más que la sociedad sea 99\% segura frente a amenazas originadas por otras personas, eso no significa que no debo
 enseñarle a mis hijos normas básicas de prevención.

Y por más que yo les enseñe prevención, ellos también pueden decidir no seguir mis consejos.

\par% p
A mí no me gusta, \emph{\textbf{aborrezco}}, la \relax{% {'style': 'margin-bottom: -1px; border-bottom: 1px dotted red;', 'title': u'En Colombia se dice \u201cdar papaya\u201d a asumir una conducta que permita ser v\xedctima.  Puede ser, por ejemplo, decir algo que propicie que los dem\xe1s se burlen, o descuidar un bien haci\xe9ndolo bot\xedn f\xe1cil para el robo, o, en casos extremos, incurrir en conductas de riesgo que faciliten la acci\xf3n de hampones y delincuentes peligrosos.  Por \u201ccultura de la papaya\u201c se entiende la aceptaci\xf3n moral de que tomar la papaya, es decir burlarse, robar, violar o asesinar est\xe1 permitido si fue por error o descuido de la v\xedctima.'}
cultura de la papaya} en Colombia.
 Me parece que un alcalde, como jefe de la policía, no debe limitar su acción frente al crimen al consejo de no dar papaya.
 Para mí \textbf{\emph{es inaceptable}} que un juez absuelva a un victimario porque la víctima dio papaya.
 (Si la ley existe que autoriza al juez a hacer esto, díganme, por favor, como demandarla o apoyar la demanda.)

\par% p
Pero que no me guste la existencia de una ley de la papaya, no me exime de ser cauto y enseñarle a mis hijos precaución.
 Y no acepto, \relax{% {'style': 'margin-bottom: -1px; border-bottom: 1px dotted red;', 'title': u'Estoy abierto al di\xe1logo, a la conversaci\xf3n y al debate.  Yo puedo estar equivocado pero la \xfanica forma de no estarlo no es que me tilden de una cosa o la otra, o me tachen de ignorante por no darme cuenta, sino que me muestren el porqu\xe9 del punto de vista que estoy ignorando.'}
salvo razones,} que esta forma de pensar sea tachada de \anchor[http://tumblr.chlewey.net/post/66964149373/reflexiones-sobre-el-machismo]{machismo}.

\backmatter

\end{document}
