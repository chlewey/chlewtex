\documentclass{icontex}
\usepackage{iconman}
\title{The \IconTeX~class}
\author{Carlos Eugenio Thompson Pinz\'on}
\purpose{Manual for explaining how to use the \IconTeX~class}
\responsible{The author}
\responsiblepost{Electronic Engineer}
\institution{Chlewey.ORG}
\date{2004}
\makeindex
\begin{document}
\selectlanguage{english}
\makeblankpage
\frontmatter
\makecover
\maketitle
\begin{pageblank}
  ~\vfill
  \begin{dedicatory}
    To my family, specially my parents and my wife
  \end{dedicatory}
\end{pageblank}
\chapter*{Thanks}
\tableofcontents
\listoftables
\listofframes
\listoffigures
\listofappendex
\begin{abstract}
  The \IconTeX~class is a \LaTeXe~class that allows document
  preparation under the rules of \Icontec, as described by the
  standard NTC\,1486.
\end{abstract}
\chapter{Glossary}
\begin{description}
\item[\Icontec] Stands for Instituto Colombiano de Normas T�cnicas
  (Colombian Institute of Technical Standards).
\item[\IconTeX] This is an independently developed \LaTeX\ class that
  allows to write documents following the \Icontec\ NTC\,1486
  standard.
\item[\LaTeX] Stands for Lamptorp's \TeX.
\item[\LaTeXe] Is the current version of \LaTeX.
\item[\TeX] Is a program writen by Knuth, that allows to ...
\end{description}
\mainmatter
\chapter*{Introduction}
\chapter{Why \IconTeX}
Here I explain why \IconTeX\ was developed.
\section{Presedents}
Here I describe what is \TeX, \LaTeX, and \LaTeXe.
I present other \LaTeX\ clases and styles oriented to the presentation
of Thesis and degree projects, including \LaTeX\ report class.
\section{Goals}
\subsection{Main goal}
To write a \LaTeX\ class that easily allows the user to prepare
documents following \Icontec\ NTC\,1486 standard for Thesis and Degree
projects.
\subsection{Other goals}
\begin{enumerate}
\item To allow stylistic costumization of the class, and provide a few
  samples of those stylistic costumizations.
\item To get those stylistic costumization compatible with \LaTeX\
  standard classes.
\item To write the manual for this class.
\end{enumerate}
\section{Theorical Frame}
This is a more technical exposition of what is \TeX, etc., and what is
\Icontec\ and the NTC\,1486 standard.
\chapter{How to Use \IconTeX}
This is the first part of the manual and is focused on how to write an
NTC\,1486 compliant document using \IconTeX.
\chapter{Advanced \IconTeX}
This chapter focuses on how to typeset specialities such as writing
mathematics, chemistry, etc. using the \IconTeX\ class and other
\LaTeX\ packages.

\section{Mathematics}
\section{Indexes}
And a quick reference to makeindex
\section{Glossary}
\section{Bibliography}
And a quick reference to \textsc{Bib\TeX}.

\chapter{Costumizing \IconTeX}
This chapter points on how to use style packages provided by \IconTeX\
to get stylistic variations, like those used by different Colombian
universities.

Also points on how to change manually the style of the document and
how to boundle all those changes into a new style package.

\chapter{Defining new elements in \IconTeX}
This chapter describes a little \LaTeX\ (and \TeX) programming
techniques, and how to get things done beyond what the designers of
\IconTeX\ allowed to costumize.

\backmatter
\begin{thebibliography}{99}
\bibitem{lshort} \emph{The (Not So) Short Introduction to \LaTeXe}
\end{thebibliography}

\appendix

\chapter{Examples}
To get a printed version of a degree project for faculty of Science at
Universidad Nacional de Colombia, use the following heading:
\begin{verbatim}
\documentclass[final,latin1]{icontex}
\usepackage[ciencias]{unal}
\begin{document}
. . .
\end{document}
\end{verbatim}

The unal package makes style costumizations to the class, and defines
institutional strings (as well as it adds the proper
\textbackslash{institution} command).

The latin1 parameter sets input encoding as latin1, allowing the use
of ISO-8859-1 (Latin-1) characters.

If you want a version for publishing using \LaTeX-like style (as in
\LaTeX\ standard report class), add a line to the heading:
\begin{verbatim}
\documentclass[final,latin1]{icontex}
\usepackage[ciencias]{unal}
\usepackage{iclatex}
\begin{document}
. . .
\end{document}
\end{verbatim}

The iclatex package redefines the style to make it close to the
\LaTeX\ report class (actually a book class).

The iclatex package changes the margins, font sizes, heading styles,
etc.  Some of this changes might be prevented with parameters like
\texttt{keepmargins}.

The university-specific packages should also be tested for the
following \LaTeX\ classes:
\begin{itemize}
\item \LaTeX\ standard book class
\item \LaTeX\ standard report class
\item \LaTeX\ standard article class
\item AMS-\LaTeX\ amsbook class
\item AMS-\LaTeX\ amsart class
\item memoir class
\end{itemize}

\chapter{The NTC\,1486 Standard}
\chapter{To do\ldots}
\begin{enumerate}
\item To get my  NTC\,1486 document, and check style formats.
\item To ensure proper handling of headings acording to NTC\,1486
  rules.
  \begin{enumerate}
  \item Chapters will have no ``\chaptername'' heading, will be
    centered begining a new anverse page (odd page in double side).
  \item Section and chapters will be in bold uppercase font, of the
    same size as the body font.  Subsections would be bold.
    Subsubsections, paragraphs and subparagraphs would have the same
    font as the body.
  \item Spacing as mandated by NTC\,1486.
  \end{enumerate}
\item To ensure proper handling of document parts according to
  NTC\,1486 rules.
  \begin{enumerate}
  \item Get the abstract look correct and allow Spanish and English
    versions of the abstract in the same page.
  \item Allow for letters and non-standard pages in the frontmatter.
  \item Get correctly sections like ``Thanks'', ``Glosary'', and the
    lists of figures, tables, frames and appendixes.
  \end{enumerate}
\item To ensure proper handling of the table of contents and lists of
  frames, tables, figures and appendixes.
  \begin{enumerate}
  \item To allow a third floating environment: frames as opposed to
    tables.
  \item To allow definition of new floating environments with their
    respective listings.
  \item To list appendixes appart from mainmatter chapters.
  \item To get bibliography and index listed in the table of contents.
  \item To prevent inclusion of frontmatter elements in the table of
    contents.
  \end{enumerate}
\item To ensure proper style of body, lists, tables and frames.
  \begin{enumerate}
  \item Double space in body text (overwritable)
  \item Single space in tables and frames (overwritable)
  \item If posible: automated environments for tables and frames.
  \end{enumerate}
\item To ensure compatibility with main packages.
\item To make the style elements overwritable for other Colombian
  standards and even for an style close to \LaTeX\ report class.
\item To write this manual following the NTC\,1486 guidelines as an
  example of how to use this class.
\item To ensure that the manual, while focused on people somehow
  familiar with \LaTeX, can be used by people not familiar with \TeX\
  or \LaTeX.
\end{enumerate}
\end{document}
